%-*- coding: UTF-8 -*-

\documentclass[UTF8]{ctexart}
\usepackage{graphicx}
\usepackage{float}
\usepackage{CJKpunct}
\usepackage{amsmath}
\usepackage{geometry}
\geometry{a5paper,centering,scale=0.8}
\usepackage[format=hang,font=small,textfont=it]{caption}
\usepackage[nottoc]{tocbibind}
\setromanfont{SourceHanSerifSC-Medium} %设置中文字体
\punctstyle{quanjiao} %使用全角标点	

\title{呼兰河传}
\author{萧\ 红}
\date{}
\begin{document}
\maketitle

\newpage

\section{第一章}

\subsection{一}

  严冬一封锁了大地的时候,则大地满地裂着口。从南到北,从东到西,几尺长的,一丈长的,还有好几丈长的,它们毫无方向地,便随时随地,只要严冬一到,大地就裂开口了。

  严寒把大地冻裂了。

  年老的人,一进屋用扫帚扫着胡子上的冰溜,一面说:「今天好冷啊!地冻裂了。」

  赶车的车夫,顶着三星,绕着大鞭子走了六七十里,天刚一蒙亮,进了大车店,第一句话就向客栈掌柜的说:「好厉害的天啊!小刀子一样。」

  等进了栈房,摘下狗皮帽子来,抽一袋烟之后,伸手去拿热馒头的时候,那伸出来的手在手背上有无数的裂口。

  人的手被冻裂了。

  卖豆腐的人清早起来沿着人家去叫卖,偶一不慎,就把盛豆腐的方木盘贴在地上拿不起来了。被冻在地上了。

  卖馒头的老头,背着木箱子,里边装着热馒头,太阳一出来,就在街上

  叫唤。他刚一从家里出来的时候,他走的快,他喊的声音也大。可是过不了一会,他的脚上挂了掌子了,在脚心上好像踏着一个鸡蛋似的,圆滚滚的。

  原来冰雪封满了他的脚底了。他走起来十分的不得力,若不是十分的加着小心,他就要跌倒了。就是这样,也还是跌倒的。跌倒了是不很好的,把馒头箱子跌翻了,馒头从箱底一个一个的滚了出来。旁边若有人看见,趁着这机会,趁着老头子倒下一时还爬不起来的时候,就拾了几个一边吃着就走了。

  等老头子挣扎起来,连馒头带冰雪一起拣到箱子去,一数,不对数。他明白了。他向着那走不太远的吃他馒头的人说:「好冷的天,地皮冻裂了,吞了我的馒头了。」

  行路人听了这话都笑了。他背起箱子来再往前走,那脚下的冰溜,似乎是越结越高,使他越走越困难,于是背上出了汗,眼睛上了霜,胡子上的冰溜越挂越多,而且因为呼吸的关系,把破皮帽子的帽耳朵和帽前遮都挂了霜了。这老头越走越慢,担心受怕,颤颤惊惊,好象初次穿上滑冰鞋,被朋友推上了溜冰场似的。

  小狗冻得夜夜的叫唤,哽哽的,好像它的脚爪被火烧着一样。

  天再冷下去:水缸被冻裂了;井被冻住了;大风雪的夜里,竟会把人家的房子封住,睡了一夜,早晨起来,一推门,竟推不开门了。

  大地一到了这严寒的季节,一切都变了样,天空是灰色的,好像刮了大风之后,呈着一种混沌沌的气象,而且整天飞着清雪。人们走起路来是快的,嘴里边的呼吸,一遇到了严寒好像冒着烟似的。七匹马拉着一辆大车,在旷野上成串的一辆挨着一辆地跑,打着灯笼,甩着大鞭子,天空挂着三星。跑了两里路之后,马就冒汗了。再跑下去,这一批人马在冰天雪地里边竟热气腾腾的了。一直到太阳出来,进了栈房,那些马才停止了出汗。但是一停止了出汗,马毛立刻就上了霜。

  人和马吃饱了之后,他们再跑。这寒带的地方,人家很少,不像南方,走了一村,不远又来了一村,过了一镇,不远又来了一镇。这里是什么也看不见,远望出去是一片白。从这一村到那一村,根本是看不见的。只有凭了认路的人的记忆才知道是走向了什么方向。拉着粮食的七匹马的大车,是到他们附近的城里去。载来大豆的卖了大豆,载来高粱的卖了高粱。等回去的时候,他们带了油、盐和布匹。

  呼兰河就是这样的小城,这小城并不怎样繁华,只有两条大街,一条从南到北,一条从东到西,而最有名的算是十字街了。十字街口集中了全城的精华。十字街上有金银首饰店、布庄、油盐店、茶庄、药店,也有拔牙的洋医生。那医生的门前,挂着很大的招牌,那招牌上画着特别大的有量米的斗那么大的一排牙齿。这广告在这小城里边无乃太不相当,使人们看了竟不知道那是什么东西,因为油店、布店和盐店,他们都没有什么广告,也不过是盐店门前写个盐字,布店门前挂了两张怕是自古亦有之的两张布幌子。

  其余的如药店的招牌,也不过是:把那戴着花镜的伸出手去在小枕头上号着妇女们的脉管的医生的名字挂在门外就是了。比方那医生的名字叫李永春,那药店也就叫李永春。人们凭着记忆,哪怕就是李永春摘掉了他的招牌,人们也都知李永春是在那里。不但城里的人这样,就是从乡下来的人也多少都把这城里的街道,和街道上尽是些什么都记熟了。用不着什么广告,用不着什么招引的方式,要买的比如油盐、布匹之类,自己走进去就会买。不需要的,你就是挂了多大的牌子,人们也是不去买。那牙医生就是一个例子,那从乡下来的人们看了这么大的牙齿,真是觉得希奇古怪,所以那大牌子前边,停了许多人在看,看也看不出是什么道理来。假若他是正在牙痛,他也绝对的不去让那用洋法子的医生给他拔掉,也还是走到李永春药店去,买二两黄连,回家去含着算了吧!因为那牌子上的牙齿太大了,有点莫名其妙,怪害怕的。

  所以那牙医生,挂了两三年招牌,到那里去拔牙的却是寥寥无几。

  后来那女医生没有办法,大概是生活没法维持,她兼做了收生婆。

  城里除了十字街之外,还有两条街,一条叫做东二道街,一条叫做西二道街。这两条街是从南到北的,大概五六里长。这两条街上没有什么好记载的,有几座庙,有几家烧饼铺,有几家粮栈。

  东二道街上有一家火磨,那火磨的院子很大,用红色的好砖砌起来的大烟筒是非常高的,听说那火磨里边进去不得,那里边的消信可多了,是碰不得的。一碰就会把人用火烧死,不然为什么叫火磨呢?就是因为有火,听说那里边不用马,或是毛驴拉磨,用的是火。一般人以为尽是用火,岂不把火磨烧着了吗?想来想去,想不明白,越想也就越糊涂。偏偏那火磨又是不准参观的。听说门口站着守卫。

  东二道街上还有两家学堂,一个在南头,一个在北头。都是在庙里边,一个在龙王庙里,一个在祖师庙里。两个都是小学:龙王庙里的那个学的是养蚕,叫做农业学校。祖师庙里的那个,是个普通的小学,还有高级班,所以又叫做高等小学。

  这两个学校,名目上虽然不同,实际上是没有什么分别的。也不过那叫做农业学校的,到了秋天把蚕用油炒起来,教员们大吃几顿就是了。

  那叫做高等小学的,没有蚕吃,那里边的学生的确比农业学校的学生长的高,农业学生开头是念「人、手、足、刀、尺」,顶大的也不过十六七岁。

  那高等小学的学生却不同了,吹着洋号,竟有二十四岁的,在乡下私学馆里已经教了四五年的书了,现在才来上高等小学。也有在粮栈里当了二年的管帐先生的现在也来上学了。

  这小学的学生写起家信来,竟有写到:「小秃子闹眼睛好了没有?」小秃子就是他的八岁的长公子的小名。次公子,女公子还都没有写上,若都写上怕是把信写得太长了。因为他已经子女成群,已经是一家之主了,写起信来总是多谈一些个家政:姓王的地户的地租送来没有?大豆卖了没有?行情如何之类。

  这样的学生,在课堂里边也是极有地位的,教师也得尊敬他,一不留心,他这样的学生就站起来了,手里拿着康熙字典,常常会把先生指问住的。

  万里乾坤的「乾」和乾菜的「乾」,据这学生说是不同的。乾菜的「乾」应该这样写:「乾」,而不是那样写:「乾」。

  西二道街上不但没有火磨,学堂也就只有一个。是个清真学校,设在城隍庙里边。

  其余的也和东二道街一样,灰秃秃的,若有车马走过,则烟尘滚滚,下了雨满地是泥。而且东二道街上有大泥坑一个,五六尺深。不下雨那泥浆好

  像粥一样,下了雨,这泥坑就变成河了,附近的人家,就要吃它的苦头,冲了人家里满满是泥,等坑水一落了去,天一晴了,被太阳一晒,出来很多蚊子飞到附近的人家去。同时那泥坑也就越晒越纯净,好像在提炼什么来似的,好像要从那泥坑里边提炼出点什么来似的。若是一个月以上不下雨,那大泥坑的质度更纯了,水份完全被蒸发走了,那里边的泥,又粘又黑,比粥锅糊,比浆糊还粘。好像炼胶的大锅似的,黑糊糊的,油亮亮的,那怕苍蝇蚊子从那里一飞也要粘住的。

  小燕子是很喜欢水的,有时误飞到这泥坑上来,用翅子点着水,看起来很危险,差一点没有被泥坑陷害了它,差一点没有被粘住,赶快地头也不回地飞跑了。

  若是一匹马,那就不然了,非粘住不可。不仅仅是粘住,而且把它陷进去,马在那里边滚着,挣扎着,挣扎了一会,没有了力气那马就躺下了。一躺下那就很危险,很有致命的可能。但是这种时候不很多,很少有人牵着马或是拉着车子来冒这种险。

  这大泥坑出乱子的时候,多半是在旱年,若两三个月不下雨这泥坑子才到了真正危险的时候。在表面上看来,似乎是越下雨越坏,一下了雨好像小河似的了,该多么危险,有一丈来深,人掉下去也要没顶的。其实不然,呼兰河这城里的人没有这么傻,他们都晓得这个坑是很厉害的,没有一个人敢有这样大的胆子牵着马从这泥坑上过。

  可是若三个月不下雨,这泥坑子就一天一天地干下去,到后来也不过是二三尺深,有些勇敢者就试探着冒险的赶着车从上边过去了,还有些次勇敢者,看着别人过去,也就跟着过去了,一来二去的,这坑子的两岸,就压成车轮经过的车辙了。那再后来者,一看,前边已经有人走在先了,这懦怯者比之勇敢的人更勇敢,赶着车子走上去了。

  谁知这泥坑子的底是高低不平的,人家过去了,可是他却翻了车了。

  车夫从泥坑爬出来,弄得和个小鬼似的,满脸泥污,而后再从泥中往外挖掘他的马,不料那马已经倒在泥污之中了,这时候有些过路的人,也就走上前来,帮忙施救。

  这过路的人分成两种,一种是穿着长袍短褂的,非常清洁。看那样子也伸不出手来,因为他的手也是很洁净的。不用说那就是绅士一流的人物了,他们是站在一旁参观的。

  看那马要站起来了,他们就喝彩,「噢!噢!」地喊叫着,看那马又站不起来,又倒下去了,这时他们又是喝彩,噢噢地又叫了几声。不过这喝的是倒彩。

  就这样的马要站起来,而又站不起来的闹了一阵之后,仍然没有站起来,仍是照原样可怜地躺在那里。这时候,那些看热闹的觉得也不过如此,也没有什么新花样了。于是星散开去,各自回家去了。

  现在再来说那马还是在那里躺着,那些帮忙救马的过路人,都是些普通的老百姓,是这城里的担葱的、卖菜的、瓦匠、车夫之流。他们卷卷裤脚,脱了鞋子,看看没有什么办法,走下泥坑去,想用几个人的力量把那马抬起来。

  结果抬不起来了,那马的呼吸不大多了。于是人们着了慌,赶快解了马套。从车子把马解下来,以为这回那马毫无担负的就可以站起来了。

  不料那马还是站不起来。马的脑袋露在泥浆的外边,两个耳朵哆嗦着,

  眼睛闭着,鼻子往外喷着突突的气。

  看了这样可怜的景象,附近的人们跑回家去,取了绳索,拿了绞锥。用绳子把马捆了起来,用绞锥从下边掘着。人们喊着号令,好像造房子或是架桥梁似的。把马抬出来了。

  马是没有死,躺在道旁。人们给马浇了一些水,还给马洗了一个脸。

  看热闹的也有来的,也有去的。

  第二天大家都说:「那大水泡子又淹死了一匹马。」

  虽然马没有死,一哄起来就说马死了。若不这样说,觉得那大泥坑也太没有什么威严了。

  在这大泥坑上翻车的事情不知有多少。一年除了被冬天冻住105的季节之外,其余的时间,这大泥坑子像它被赋给生命了似的,它是活的。水涨了,水落了,过些日子大了,过些日子又小了。大家对它都起着无限的关切。

  水大的时间,不但阻碍了车马,且也阻碍了行人,老头走在泥坑子的沿上,两条腿打颤,小孩子在泥坑子的沿上吓得狼哭鬼叫。

  一下起雨来这大泥坑子白亮亮地涨得溜溜地满,涨到两边的人家的墙根上去了,把人家的墙根给淹没了。来往过路的人,一走到这里,就像在人生的路上碰到了打击。是要奋斗的,卷起袖子来,咬紧了牙根,全身的精力集中起来,手抓着人家的板墙,心脏扑通扑通地跳,头不要晕,眼睛不要花,要沉着迎战。

  偏偏那人家的板墙造得又非常地平滑整齐,好像有意在危难的时候不帮人家的忙似的,使那行路人不管怎样巧妙地伸出手来,也得不到那板墙的怜悯,东抓抓不着什么,西摸也摸不到什么,平滑得连一个疤拉节子也没有,这可不知道是什么山上长的木头,长得这样完好无缺。

  挣扎了五六分钟之后,总算是过去了。弄得满头流汗,满身发烧,那都不说。再说那后来的人,依法炮制,那花样也不多,也只是东抓抓,西摸摸。

  弄了五六分钟之后,又过去了。

  一过去了可就精神饱满,哈哈大笑着,回头向那后来的人,向那正在艰苦阶段上奋斗着的人说:「这算什么,一辈子不走几回险路那不算英雄。」

  可也不然,也不一定都是精神饱满的,而大半是被吓得脸色发白。有的虽然已经过去了多时,还是不能够很快地抬起腿来走路,因为那腿还在打颤。

  这一类胆小的人,虽然是险路已经过去了,但是心里边无由地生起来一种感伤的情绪,心里颤抖抖的,好像被这大泥坑子所感动了似的,总要回过头来望一望,打量一会,似乎要有些话说。终于也没有说什么,还是走了。

  有一天,下大雨的时候,一个小孩子掉下去,让一个卖豆腐的救了上来。

  救上来一看,那孩子是农业学校校长的儿子。

  于是议论纷纷了,有的说是因为农业学堂设在庙里边,冲了龙王爷了,龙王爷要降大雨淹死这孩子。

  有的说不然,完全不是这样,都是因为这孩子的父亲的关系,他父亲在讲堂上指手画脚的讲,讲给学生们说,说这天下雨不是在天的龙王爷下的雨,他说没有龙王爷。你看这不把龙王爷活活地气死,他这口气那能不出呢?所以就抓住了他的儿子来实行因果报应了。

  有的说,那学堂里的学生也太不像样了,有的爬上了老龙王的头顶,给老龙王去戴了一个草帽。这是什么年头,一个毛孩子就敢惹这么大的祸,老龙王怎么会不报应呢?看着吧,这还不能算了事,你想龙王爷并不是白人呵!

  你若惹了他,他可能够饶了你?那不像对付一个拉车的、卖菜的,随便的踢他们一脚就让他们去。那是龙王爷呀!龙王爷还是惹得的吗?

  有的说,那学堂的学生都太不像样了,他说他亲眼看见过,学生们拿了蚕放在大殿上老龙王的手上。你想老龙王那能够受得了。

  有的说,现在的学堂太不好了,有孩子是千万上不得学堂的。一上了学堂就天地人鬼神不分了。

  有的说他要到学堂把他的儿子领回来,不让他念书了。

  有的说孩子在学堂里念书,是越念越坏,比方吓掉了魂,他娘给他叫魂的时候,你听他说什么?他说这叫迷信。你说再念下去那还了得吗?

  说来说去,越说越远了。

  过了几天,大泥坑子又落下去了,泥坑两岸的行人通行无阻。

  再过些日子不下雨,泥坑子就又有点像要干了。这时候,又有车马开始在上面走,又有车子翻在上面,又有马倒在泥中打滚,又是绳索棍棒之类的,往外抬马,被抬出去的赶着车子走了,后来的,陷进去,再抬。

  一年之中抬车抬马,在这泥坑子上不知抬了多少次,可没有一个人说把泥坑子用土填起来不就好了吗?没有一个。

  有一次一个老绅士在泥坑涨水时掉在里边了。一爬出来,他就说:「这街道太窄了,去了这水泡子连走路的地方都没有了。这两边的院子,怎么不把院墙拆了让出一块来?」

  他正说着,板墙里边,就是那院中的老太太搭了言。她说院墙是拆不得的,她说最好种树,若是沿着墙根种上一排树,下起雨来人就可以攀着树过去了。

  说拆墙的有,说种树的有,若说用土把泥坑来填平的,一个人也没有。

  这泥坑子里边淹死过小猪,用泥浆闷死过狗,闷死过猫,鸡和鸭也常常死在这泥坑里边。

  原因是这泥坑上边结了一层硬壳,动物们不认识那硬壳下面就是陷阱,等晓得了可也就晚了。它们跑着或是飞着,等往那硬壳上一落可就再也站不起来了。白天还好,或者有人又要来施救。夜晚可就没有办法了。它们自己挣扎,挣扎到没有力量的时候就很自然的沉下去了,其实也或者越挣扎越沉下去的快。有时至死也还不沉下去的事也有。若是那泥浆的密度过高的时候,就有这样的事。

  比方肉上市,忽然卖便宜猪肉了,于是大家就想起那泥坑子来了,说:「可不是那泥坑子里边又淹死了猪了?」

  说着若是腿快的,就赶快跑到邻人的家去,告诉邻居:「快去买便宜肉吧,快去吧,快去吧,一会没有了。」

  等买回家来才细看一番,似乎有点不大对,怎么这肉又紫又青的!可不要是瘟猪肉。

  但是又一想,那能是瘟猪肉呢,一定是那泥坑子淹死的。

  于是煎、炒、蒸、煮,家家吃起便宜猪肉来。虽然吃起来了,但就总觉得不大香,怕还是瘟猪肉。

  可是又一想,瘟猪肉怎么可以吃得,那么还是泥坑子淹死的吧!

  本来这泥坑子一年只淹死一两只猪,或两三口猪,有几年还连一个猪也没有淹死。至于居民们常吃淹死的猪肉,这可不知是怎么一回事,真是龙王爷晓得。

  虽然吃的自己说是泥坑子淹死的猪肉,但也有吃了病的,那吃病了的就大发议论说:「就是淹死的猪肉也不应该抬到市上去卖,死猪肉终究是不新鲜的,税局子是干什么的,让大街上,在光天化日之下就卖起死猪肉来?」

  那也是吃了死猪肉的,但是尚且没有病的人说:「话可也不能是那么说,一定是你疑心,你三心二意的吃下去还会好。

  你看我们也一样的吃了,可怎么没病?「

  间或也有小孩子太不知时务,他说他妈不让他吃,说那是瘟猪肉。

  这样的孩子,大家都不喜欢。大家都用眼睛瞪着他,说他:「瞎说,瞎说!」

  有一次一个孩子说那猪肉一定是瘟猪肉,并且是当着母亲的面向邻人说的。那邻人听了倒并没有坚决的表示什么,可是他的母亲的脸立刻就红了。

  伸出手去就打了那孩子。

  那孩子很固执,仍是说:「是瘟猪肉吗!是瘟猪肉吗!」

  母亲实在难为情起来,就拾起门旁的烧火的叉子,向着那孩子的肩膀就打了过去。于是孩子一边哭着一边跑回家里去了。

  一进门,炕沿上坐着外祖母,那孩子一边哭着一边扑到外祖母的怀里说:「姥姥,你吃的不是瘟猪肉吗?我妈打我。」

  外祖母对这打得可怜的孩子本想安慰一番,但是一抬头看见了同院的老李家的奶妈站在门口往里看。

  于是外祖母就掀起孩子后衣襟来,用力地在孩子的屁股上哐哐地打起来,嘴里还说着:「谁让你这么一点你就胡说八道!」

  一直打到李家的奶妈抱着孩子走了才算完事。

  那孩子哭得一塌糊涂,什么「瘟猪肉」不「瘟猪肉」的,哭得也说不清了。

  总共这泥坑子施给当地居民的福利有两条:第一条:常常抬车抬马,淹鸡淹鸭,闹得非常热闹,可使居民说长道短,得以消遣。

  第二条就是这猪肉的问题了,若没有这泥坑子,可怎么吃瘟猪肉呢?\underline{吃是可以吃的,但是可怎么说法呢}?真正说是吃的瘟猪肉,岂不太不讲卫生了吗?有这泥坑子可就好办,可以使瘟猪变成淹猪,居民们买起肉来,第一经济,第二也不算什么不卫生。

\subsection{二}

  东二道街除了大泥坑子这番盛举之外,再就没有什么了。也不过是几家碾磨房,几家豆腐店,也有一两家机房,也许有一两家染布匹的染缸房,这个也不过是自己默默地在那里做着自己的工作,没有什么可以使别人开心的,也不能招来什么议论。那里边的人都是天黑了就睡觉,天亮了就起来工作。一年四季,春暖花开、秋雨、冬雪,也不过是随着季节穿起棉衣来,脱下单衣去地过着。生老病死也都是一声不响地默默地办理。

  比方就是东二道街南头,那卖豆芽菜的王寡妇吧:她在房脊上插了一个很高的杆子,杆子头上挑着一个破筐。因为那杆子很高,差不多和龙王庙的铁马铃子一般高了。来了风,庙上的铃子格棱格棱地响。王寡妇的破筐子虽是它不会响,但是它也会东摇西摆地作着态。

  就这样一年一年地过去,王寡妇一年一年地卖着豆芽菜,平静无事,过着安祥的日子,忽然有一年夏天,她的独子到河边去洗澡,掉河淹死了。

  这事情似乎轰动了一时,家传户晓,可是不久也就平静下去了。不但邻人、街坊,就是她的亲戚朋友也都把这回事情忘记了。

  再说那王寡妇,虽然她从此以后就疯了,但她到底还晓得卖豆芽菜,她仍还是静静地活着,虽然偶尔她的菜被偷了,在大街上或是在庙台上狂哭一场,但一哭过了之后,她还是平平静静地活着。

  至于邻人街坊们,或是过路人看见了她在庙台上哭,也会引起一点恻隐之心来的,不过为时甚短罢了。

  还有人们常常喜欢把一些不幸者归划在一起,比如疯子傻子之类,都一律去看待。

  哪个乡、哪个县、哪个村都有些个不幸者,瘸子啦、瞎子啦、疯子或是傻子。

  呼兰河这城里,就有许多这一类的人。人们关于他们都似乎听得多、看得多,也就不以为奇了。偶尔在庙台上或是大门洞里不幸遇到了一个,刚想多少加一点恻隐之心在那人身上,但是一转念,人间这样的人多着哩!于是转过眼睛去,三步两步地就走过去了。即或有人停下来,也不过是和那些毫没有记性的小孩子似的向那疯子投一个石子,或是做着把瞎子故意领到水沟里边去的事情。

  一切不幸者,就都是叫化子,至少在呼兰河这城里边是这样。

  人们对待叫化子们是很平凡的。

  门前聚了一群狗在咬,主人问:「咬什么?」

  仆人答:「咬一个讨饭的。」

  说完了也就完了。

  可见这讨饭人的活着是\underline{一钱不值}了。

  卖豆芽菜的女疯子,虽然她疯了还忘不了自己的悲哀,隔三差五的还到庙台上去哭一场,但是一哭完了,仍是得回家去吃饭、睡觉、卖豆芽菜。

  她仍是平平静静地活着。

\subsection{三}

  再说那染缸房里边,也发生过不幸,两个年青的学徒,为了争一个街头上的妇人,其中的一个把另一个按进染缸子给淹死了。死了的不说,就说那活着的也下了监狱,判了个无期徒刑。

  但这也是不声不响地把事就解决了,过了三年二载,若有人提起那件事来,差不多就像人们讲着岳飞、秦桧似的,久远得不知多少年前的事情似的。

  同时发生这件事情的染缸房,仍旧是在原址,甚或连那淹死人的大缸也许至今还在那儿使用着。从那染缸房发卖出来的布匹,仍旧是远近的乡镇都流通着。蓝色的布匹男人们做起棉裤棉袄,冬天穿它来抵御严寒。红色的布匹,则做成大红袍子,给十八九岁的姑娘穿上,让她去做新娘子。

  总之,除了染缸房子在某年某月某日死了一个人外,其余的世界,并没有因此而改动了一点。

  再说那豆腐房里边也发生过不幸:两个伙计打仗,竟把拉磨的小驴的腿打断了。

  因为它是驴子,不谈它也就罢了。只因为这驴子哭瞎了一个妇人的眼睛(即打了驴子那人的母亲),所以不能不记上。

  再说那造纸的纸房里边,把一个私生子活活饿死了。因为他是一个初生的孩子,算不了什么。也就不说他了。

\subsection{四}

  其余的东二道街上,还有几家扎彩铺。这是为死人而预备的。

  人死了,魂灵就要到地狱里边去,地狱里边怕是他没有房子住、没有衣裳穿、没有马骑。活着的人就为他做了这么一套,用火烧了,据说是到阴间就样样都有了。

  大至喷钱兽、聚宝盆、大金山、大银山,小至丫鬟使女、厨房里的厨子、喂猪的猪倌,再小至花盆、茶壶茶杯、鸡鸭鹅犬,以至窗前的鹦鹉。

  看起来真是万分的好看,大院子也有院墙,墙头上是金色的琉璃瓦。一进了院,正房五间,厢房三间,一律是青红砖瓦房,窗明几净,空气特别新鲜。花盆一盆一盆的摆在花架子上,石柱子、全百合、马蛇菜、九月菊都一齐的开了。看起使人不知道是什么季节,是夏天还是秋天,居然那马蛇菜也和菊花同时站在一起。也许阴间是不分什么春夏秋冬的。这且不说。

  再说那厨房里的厨子,真是活神活现,比真的厨子真是干净到一千倍,头戴白帽子、身扎白围裙,手里边在做拉面条,似乎午饭的时候就要到了,煮了面就要开饭了似的。

  院子里的牵马童,站在一匹大白马的旁边,那马好像是阿拉伯马,特别高大,英姿挺立,假若有人骑上,看样子一定比火车跑得更快。就是呼兰河这城里的将军,相信他也没有骑过这样的马。

  小车子、大骡子,都排在一边。骡子是油黑的、闪亮的,用鸡蛋壳做的眼睛,所以眼珠是不会转的。

  大骡子旁边还站着一匹小骡子,那小骡子是特别好看,睛珠是和大骡子一般的大。

  小车子装潢得特别漂亮,车轮子都是银色的。车前边的帘子是半掩半卷的,使人得以看到里边去。车里边是红堂堂地铺着大红的褥子。赶车的坐在车沿上,满脸是笑,得意洋洋,装饰得特别漂亮,扎着紫色的腰带,穿着蓝色花丝葛的大袍,黑缎鞋,雪白的鞋底。大概穿起这鞋来还没有走路就赶过车来了。他头上戴着黑帽头,红帽顶,把脸扬着,他蔑视着一切,越看他越不像一个车夫,好像一位新郎。

  公鸡三两只,母鸡七八只,都是在院子里边静静地啄食,一声不响,鸭子也并不呱呱地直叫,叫得烦人。狗蹲在上房的门旁,非常的守职,一动不动。

  看热闹的人,人人说好,个个称赞。穷人们看了这个竟觉得活着还没有死了好。

  正房里,窗帘、被格、桌椅板凳,一切齐全。

  还有一个管家的,手里拿着一个算盘在打着,旁边还摆着一个帐本,上边写着:「北烧锅欠酒二十二斤东乡老王家昨借米二十担白旗屯泥人子昨送地租四百三十吊白旗屯二个子共欠地租两千吊」

  这以下写了个:四月二十八日以上的是四月二十七日的流水帐,大概二十八日的还没有写吧!

  看这帐目也就知道阴间欠了帐也是马虎不得的,也设了专门人才,即管帐先生一流的人物来管。同时也可以看出来,这大宅子的主人不用说就是个地主了。

  这院子里边,一切齐全,一切都好,就是看不见这院子的主人在什么地方,未免地使人疑心这么好的院子而没有主人了。这一点似乎使人感到空虚,无着无落的。

  再一回头看,就觉得这院子终归是有点两样,怎么丫鬟、使女、车夫、马童的胸前都挂着一张纸条,那纸条上写着他们每个人的名字:那漂亮得和新郎似的车夫的名字叫:「长鞭」

  马童的名字叫:「快腿」

  左手拿着水烟袋,右手抡着花手巾的小丫鬟叫:「德顺」

  另外一个叫:「顺平」

  管帐的先生叫:「妙算」

  提着喷壶在浇花的使女叫:「花姐」

  再一细看才知道那匹大白马也是有名字的,那名字是贴在马屁股上的,叫:「千里驹」

  其余的如骡子、狗、鸡、鸭之类没有名字。

  那在厨房里拉着面条的「老王」,他身上写着他名字的纸条,来风一吹,还忽咧忽咧地跳着。

  这可真有点奇怪,自家的仆人,自己都不认识了,还要挂上个名签。

  这一点未免地使人迷离恍惚,似乎阴间究竟没有阳间好。

  虽然这么说,羡慕这座宅子的人还是不知多少。因为的确这座宅子是好:清悠、闲静、鸦雀无声,一切规整,绝不紊乱。丫鬟、使女,照着阳间的一样,鸡犬猪马,也都和阳间一样,阳间有什么,到了阴间也有,阳间吃面条,到了阴间也吃面条,阳间有车子坐,到了阴间也一样的有车子坐,阴间是完全和阳间一样,一模一样的。

  只不过没有东二道街上那大泥坑子就是了。是凡好的一律都有,坏的不必有。

\subsection{五}

  东二道街上的扎彩铺,就扎的是这一些。一摆起来又威风、又好看,但那作坊里边是乱七八糟的,满地碎纸,秫杆棍子一大堆,破盒子、乱罐子、颜料瓶子、浆糊盆、细麻绳、粗麻绳………走起路来,会使人跌倒。那里边砍的砍、绑的绑,苍蝇也来回地飞着。

  要做人,先做一个脸孔,糊好了,挂在墙上,男的女的,到用的时候,摘下一个来就用。给一个用秫杆捆好的人架子,穿上衣服,装上一个头就像人了。把一个瘦骨伶仃的用纸糊好的马架子,上边贴上用纸剪成的白毛,那就是一匹很漂亮的马了。

  做这样的活计的,也不过是几个极粗糙极丑陋的人,他们虽懂得怎样打扮一个马童或是打扮一个车夫,怎样打扮一个妇人女子,但他们对他们自己是毫不加修饰的,长头发的、毛头发的、歪嘴的、歪眼的、赤足裸膝的,似乎使人不能相信,这么漂亮炫眼耀目,好像要活了的人似的,是出于他们之手。

  他们吃的是粗菜、粗饭,穿的是破烂的衣服,睡觉则睡在车马、人、头之中。

  他们这种生活,似乎也很苦的。但是一天一天的,也就糊里糊涂地过去了,也就过着春夏秋冬,脱下单衣去,穿起棉衣来地过去了。

  生、老、病、死,都没有什么表示。生了就任其自然的长去;长大就长大,长不大也就\underline{算了}。

  老,老了也没有什么关系,眼花了,就不看;耳聋了,就不听;牙掉了,就整吞;走不动了,就瘫着。这有什么办法,谁老谁活该。

  病,人吃五谷杂粮,谁不生病呢?

  死,这回可是悲哀的事情了,父亲死了儿子哭;儿子死了母亲哭;哥哥死了一家全哭;嫂子死了,她的娘家人来哭。

  哭了一朝或是三日,就总得到城外去,挖一个坑把这人埋起来。

  埋了之后,那活着的仍旧得回家照旧地过着日子。该吃饭,吃饭。该睡觉,睡觉。外人绝对看不出来是他家已经没有了父亲或是失掉了哥哥,就连他们自己也不是关起门来,每天哭上一场。他们心中的悲哀,也不过是随着当地的风俗的大流逢年过节的到坟上去观望一回。二月过清明,家家户户都提着香火去上坟茔,有的坟头上塌了一块土,有的坟头上陷了几个洞,相观之下,感慨唏嘘,烧香点酒。若有近亲的人如子女父母之类,往往且哭上一场;那哭的语句,数数落落,无异是在做一篇文章或者是在诵一篇长诗。歌诵完了之后,站起来拍拍屁股上的土,也就随着上坟的人们回城的大流,回城去了。

  回到城中的家里,又得照旧的过着日子,一年柴米油盐,浆洗缝补。从早晨到晚上忙了个不休。夜里疲乏之极,躺在炕上就睡了。在夜梦中并梦不到什么悲哀的或是欣喜的景况,只不过咬着牙、打着哼,一夜一夜地就都这样地过去了。

  假若有人问他们,人生是为了什么?他们并不会茫然无所对答的,他们会直截了当地不加思索地说了出来:「人活着是为吃饭穿衣。」

  再问他,人死了呢?他们会说:「人死了就完了。」

  所以没有人看见过做扎彩匠的活着的时候为他自己糊一座阴宅,大概他不怎么相信阴间。假如有了阴间,到那时候他再开扎彩铺,怕又要租人家的房子了。

\subsection{六}

  呼兰河城里,除了东二道街、西二道街、十字街之外,再就都是些个小胡同了。

  小胡同里边更没有什么了,就连打烧饼麻花的店铺也不大有,就连卖红绿糖球的小床子,也都是摆在街口上去,很少有摆在小胡同里边的。那些住在小街上的人家,一天到晚看不见多少闲散杂人。耳听的眼看的,都比较的少,所以整天寂寂寞寞的,关起门来在过着生活。破草房有上半间,买上二斗豆子,煮一点盐豆下饭吃,就是一年。

  在小街上住着,又冷清、又寂寞。

  一个提篮子卖烧饼的,从胡同的东头喊,胡同向西头都听到了。虽然不买,若走谁家的门口,谁家的人都是把头探出来看看,间或有问一问价钱的,问一问糖麻花和油麻花现在是不是还卖着前些日子的价钱。

  间或有人走过去掀开了筐子上盖着的那张布,好像要买似的,拿起一个来摸一摸是否还是热的。

  摸完了也就放下了,卖麻花的也绝对的不生气。

  于是又提到第二家的门口去。

  第二家的老太婆也是在闲着,于是就又伸出手来,打开筐子,摸了一回。

  摸完了也是没有买。

  等到了第三家,这第三家可要买了。

  一个三十多岁的女人,刚刚睡午觉起来,她的头顶上梳着一个卷,大概头发不怎样整齐,发卷上罩着一个用大黑珠线织的网子,网子上还插了不少的疙瘩针。可是因为这一睡觉,不但头发乱了,就是那些疙瘩针也都跳出来了,好像这女人的发卷上被射了不少的小箭头。

  她一开门就很爽快,把门扇刮打的往两边一分,她就从门里闪出来了。

  随后就跟出来五个孩子。这五个孩子也都个个爽快。像一个小连队似的,一排就排好了。

  第一个是女孩子,十二三岁,伸出手来就拿了一个五吊钱一只的一竹筷子长的大麻花。她的眼光很迅速,这麻花在这筐子里的确是最大的,而且就只有这一个。

  第二个是男孩子,拿了一个两吊钱一只的。

  第三个也是拿了个两吊钱一只的。也是个男孩子。

  第四个看了看,没有办法,也只得拿了一个两吊钱的。也是个男孩子。

  轮到第五个了,这个可分不出来是男孩子,还是女孩子。头是秃的,一只耳朵上挂着钳子,瘦得好像个干柳条,肚子可特别大。看样子也不过五岁。

  一伸手,他的手就比其余的四个的都黑得更厉害,其余的四个,虽然他们的手也黑得够厉害的,但总还认得出来那是手,而不是别的什么,唯有他的手是连认也认不出来了,说是手吗,说是什么呢,说什么都行。完全起着黑的灰的、深的浅的,各种的云层。看上去,好像看隔山照似的,有无穷的趣味。

  他就用这手在筐子里边挑选,几乎是每个都让他摸过了,不一会工夫,全个的筐子都让他翻遍了。本来这筐子虽大,麻花也并没有几只。除了一个顶大的之外,其余小的也不过十来只,经了他这一翻,可就完全遍了。弄了他满手是油,把那小黑手染得油亮油亮的,黑亮黑亮的。

  而后他说:「我要大的。」

  于是就在门口打了起来。

  他跑得非常之快,他去追着他的姐姐。他的第二个哥哥,他的第三个哥哥,也都跑了上去,都比他跑得更快。再说他的大姐,那个拿着大麻花的女孩,她跑得更快到不能想像了。已经找到一块墙的缺口的地方,跳了出去,后边的也就跟着一溜烟地跳过去。等他们刚一追着跳过去,那大孩子又跳回来了。在院子里跑成了一阵旋风。

  那个最小的,不知是男孩子还是女孩子的,早已追不上了。落在后边,在号陶大哭。间或也想拣一点便宜,那就是当他的两个哥哥,把他的姐姐已经扭住的时候,他就趁机会想要从中抢他姐姐手里的麻花。可是几次都没有做到,于是又落在后边号陶大哭。

  他们的母亲,虽然是很有威风的样子,但是不动手是招呼不住他们的。

  母亲看了这样子也还没有个完了,就进屋去,拿起烧火的铁叉子来,向着她的孩子就奔去了。不料院子里有一个小泥坑,是猪在里打腻的地方。她恰好就跌在泥坑那儿了。把叉子跌出去五尺多远。

  于是这场戏才算达到了高潮,看热闹的人没有不笑的,没有不称心愉快的。

  就连那卖麻花的人也看出神了,当那女人坐到泥坑中把泥花四边溅起来的时候,那卖麻花的差一点没把筐子掉了地下。他高兴极了,他早已经忘了他手里的筐子了。

  至于那几个孩子,则早就不见了。

  等母亲起来去把他们追回来的时候,那做母亲的这回可发了威风,让他们一个一个的向着太阳跪下。在院子里排起一小队来,把麻花一律的解除。

  顶大的孩子的麻花没有多少了,完全被撞碎了。

  第三个孩子的已经吃完了。

  第二个的还剩了一点点。

  只有第四个的还拿在手上没有动。

  第五个,不用说,根本没有拿在手里。

  闹到结果,卖麻花的和那女人吵了一阵之后提着筐子又到另一家去叫卖去了。他和那女人所吵的是关于那第四个孩子手上拿了半天的麻花又退回了的问题,卖麻花的坚持着不让退,那女人又非退回不可。结果是付了三个麻花的钱,就把那提篮子的人赶了出来了。

  为着麻花而下跪的五个孩子不提了。再说那一进胡同口就被挨家摸索过来的麻花,被提到另外的胡同里去,倒底也卖掉了。一个已经脱完了牙齿的老太太买了其中的一个,用纸裹着拿到屋子去了。她一边走着一边说:「这麻花真干净,油亮亮的。」

  而后招呼了她的小孙子,快来吧。

  那卖麻花的人看了老太太很喜欢这麻花,于是就又说:

  「是刚出锅的,还热忽着哩!」

\subsection{七}

  过去了卖麻花的,后半天,也许又来了卖凉粉的,也是一在胡同口的这头喊,那头就听到了。

  要买的拿着小瓦盆出去了。不买的坐在屋子一听这卖凉粉的一招呼,就知道是应烧晚饭的时候了。因为这凉粉一个整个的夏天都是在太阳偏西,他就来的,来得那么准,就像时钟一样,到了四五点钟他必来的。就象他卖凉粉专门到这一条胡同来卖似的。似乎在别的胡同里就没有为着多卖几家而耽误了这一定的时间。

  卖凉粉的一过去了。一天也就快黑了。

  打着拨浪鼓的货郎,一到太阳偏西,就再不进到小巷子里来,就连僻静的街他也不去了,他担着担子从大街口走回家去。

  卖瓦盆的,也早都收市了。

  拣绳头的,换破烂的也都回家去了。

  只有卖豆腐的则又出来了。

  晚饭时节,吃了小葱蘸大酱就已经很可口了,若外加上一块豆腐,那真是锦上添花,一定要多浪费两碗包米大云豆粥的。一吃就吃多了,那是很自然的,豆腐加上点辣椒油,再拌上点大酱,那是多么可口的东西;用筷子触了一点点豆腐,就能够吃下去半碗饭,再到豆腐上去触了一下,一碗饭就完了。因为豆腐而多吃两碗饭,并不算吃得多,没有吃过的人,不能够晓得其中的滋味的。

  所以卖豆腐的人来了,男女老幼,全都欢迎。打开门来,笑盈盈的,虽然不说什么,但是彼此有一种融洽的感情,默默生了起来。

  似乎卖豆腐的在说:「我的豆腐真好!」

  似乎买豆腐的回答:「你的豆腐果然不错。」

  买不起豆腐的人对那卖豆腐的,就非常的羡慕,一听了那从街口越招呼越近的声音就特别地感到诱惑,假若能吃一块豆腐可不错,切上一点青辣椒,拌上一点小葱子。

  但是天天这样想,天天就没有买成,卖豆腐的一来,就把这等人白白地引诱一场。于是那被诱惑的人,仍然逗不起决心,就多吃几口辣椒,辣得满头是汗。他想假若一个人开了一个豆腐房可不错,那就可以自由随便地吃豆腐了。

  果然,他的儿子长到五岁的时候,问他:「你长大了干什么?」

  五岁的孩子说:「开豆腐房。」

  这显然要继承他父亲未遂的志愿。

  关于豆腐这美妙的一盘菜的爱好,竟还有甚于此的,竟有想要倾家荡产的。传说上,有这样的一个家长,他下了决心,他说:「不过了,买一块豆腐吃去!」这「不过了」的三个字,用旧的语言来翻译,就是毁家纾难的意思;用现代的话来说,就是:「我破产了!」

\subsection{八}

  卖豆腐的一收了市,一天的事情都完了。

  家家户户都把晚饭吃过了。吃过了晚饭,看晚霞的看晚霞,不看晚霞的躺到炕上去睡觉的也有。

  这地方的晚霞是很好看的,有一个土名,叫火烧云。说「晚霞」人们不懂,若一说「火烧云」就连三岁的孩子也会呀呀地往西天空里指给你看。

  晚饭一过,火烧云就上来了。照得小孩子的脸是红的。把大白狗变成红色的狗了。红公鸡就变成金的了。黑母鸡变成紫檀色的了。喂猪的老头子,往墙根上靠,他笑盈盈地看着他的两匹小白猪,变成小金猪了,他刚想说:「他妈的,你们也变了……」

  他的旁边走来了一个乘凉的人,那人说:「你老人家必要高寿,你老是金胡子了。」

  天空的云,从西边一直烧到东边,红堂堂的,好像是天着了火。

  这地方的火烧云变化极多,一会红堂堂的了,一会金洞洞的了,一会半紫半黄的,一会半灰半百合色。葡萄灰、大黄梨、紫茄子,这些颜色天空上边都有。还有些说也说不出来的,见也未曾见过的,诸多种的颜色。

  五秒钟之内,天空里有一匹马,马头向南,马尾向西,那马是跪着的,像是在等着有人骑到它的背上,它才站起来。再过一秒钟,没有什么变化。

  再过两三秒钟,那匹马加大了,马腿也伸开了,马脖子也长了,但是一条马尾巴却不见了。

  看的人,正在寻找马尾巴的时候,那马就变靡了。

  忽然又来了一条大狗,这条狗十分凶猛,它在前边跑着,它的后面似乎还跟了好几条小狗仔。跑着跑着,小狗就不知跑到哪里去了,大狗也不见了。

  又找到了一个大狮子,和娘娘庙门前的大石头狮子一模一样的,也是那么大,也是那样的蹲着,很威武的,很镇静地蹲着,它表示着蔑视一切的样子,似乎眼睛连什么也不睬,看着看着地,一不谨慎,同时又看到了别一个什么。这时候,可就麻烦了,人的眼睛不能同时又看东,又看西。这样子会活活把那个大狮子糟蹋了。一转眼,一低头,那天空的东西就变了。若是再找,怕是看瞎了眼睛也找不到了。

  大狮子既然找不到,另外的那什么,比方就是一个猴子吧,猴子虽不如大狮子,可同时也没有了。

  一时恍恍惚惚的,满天空里又像这个,又像那个,其实是什么也不像,什么也没有了。

  必须是低下头去,把眼睛揉一揉,或者是沉静一会再来看。

  可是天空偏偏又不常常等待着那些爱好它的孩子。一会工夫火烧云下去了。

  于是孩子们困倦了,回屋去睡觉了。竟有还没能来得及进屋的,就靠在姐姐的腿上,或者是依在祖母的怀里就睡着了。

  祖母的手里,拿着白马鬃的蝇甩子,就用蝇甩子给他驱逐着蚊虫。

  祖母还不知道这孩子是已经睡了,还以为他在那里玩着呢!

  「下去玩一会去吧!把奶奶的腿压麻了。」

  用手一推,这孩子已经睡得摇摇晃晃的了。

  这时候,火烧云已经完全下去了。

  于是家家户户都进屋去睡觉,关起窗门来。

  呼兰河这地方,就是在六月里也是不十分热的,夜里总要盖着薄棉被睡觉。等黄昏之后的乌鸦飞过时,只能够隔着窗子听到那很少的尚未睡的孩子在嚷叫:「乌鸦乌鸦你打场,给你二斗粮……」

  那漫天盖地的一群黑乌鸦,呱呱地大叫着,在整个的县城的头顶上飞过去了。

  据说飞过了呼兰河的南岸,就在一个大树林子里边住下了。明天早晨起来再飞。

  夏秋之间每夜要过乌鸦,究竟这些成百成千的乌鸦过到哪里去,孩子们是不大晓得的,大人们也不大讲给他们听。只晓得念这套歌,「乌鸦乌鸦你打场,给你二斗粮。」

  究竟给乌鸦二斗粮做什么,似乎不大有道理。

\subsection{九}

  乌鸦一飞过,这一天才真正地过去了。

  因为大昂星升起来了,大昂星好像铜球似的亮晶晶的了。天河和月亮也都上来了。

  蝙蝠也飞起来了。

  是凡跟着太阳一起来的,现在都回去了。人睡了,猪、马、牛、羊也都睡了,燕子和蝴蝶也都不飞了。就连房根底下的牵牛花,也一朵没有开的。

  含苞的含苞,卷缩的卷缩。含苞的准备着欢迎那早晨又要来的太阳,那卷缩的,因为它已经在昨天欢迎过了,它要落去了。

  随着月亮上来的星夜,大昴星也不过是月亮的一个马前卒,让它先跑到一步就是了。

  夜一来蛤蟆就叫,在河沟里叫,在洼地里叫。虫子也叫,在院心草棵子里,在城外的大田上,有的叫在人家的花盆里,有的叫在人家的坟头上。

  夏夜若无风无雨就这样地过去了,一夜又一夜。

  很快地夏天就过完了,秋天就来了。秋天和夏天的分别不太大,也不过天凉了,夜里非盖着被子睡觉不可。种田的人白天忙着收割,夜里多做几个割高粱的梦就是了。

  女人一到了八月也不过就是浆衣裳,拆被子,捶棒棰,捶得街街巷巷早晚地叮叮地乱响。

  「棒棰」一捶完,做起被子来,就是冬天。

  冬天下雪了。

  人们四季里,风、霜、雨、雪的过着,霜打了,雨淋了。大风来时是飞沙走石。似乎是很了不起的样子。冬天,大地被冻裂了,江河被冻住了。再冷起来,江河也被冻得锵锵地响着裂开了纹。冬天,冻掉了人的耳朵,破了人的鼻子,裂了人的手和脚。

  但这是大自然的威风,与小民们无关。

  呼兰河的人们就是这样,冬天来了就穿棉衣裳,夏天来了就穿单衣裳。

  就好像太阳出来了就起来,太阳落了就睡觉似的。

  被冬天冻裂了手指的,到了夏天也自然就好了。好不了的,「李永春」药铺,去买二两红花,泡一点红花酒来擦一擦,擦得手指通红也不见消,也许就越来越肿起来。那么再到「李永春」药铺去,这回可不买红花了,是买了一贴膏药来。回到家里,用火一烤,粘粘糊糊地就贴在冻疮上了。这膏药是真好,贴上了一点也不碍事。该赶车的去赶车,该切菜的去切菜。粘粘糊糊地是真好,见了水也不掉,该洗衣裳的洗衣裳去好了。就是掉了,拿在火上再一烤,就还贴得上的。一贴,贴了半个月。

  呼兰河这地方的人,什么都讲结实、耐用,这膏药这样的耐用,实在是合乎这地方的人情。虽然是贴了半个月,手也还没有见好,但这膏药总算是耐用,没有白花钱。

  于是再买一贴去,贴来贴去,这手可就越肿越大了。还有些买不起膏药的,就拣人家贴乏了的来贴。

  到后来,那结果,谁晓得是怎样呢,反正\underline{一塌糊涂去了}吧。

  春夏秋冬,一年四季来回循环地走,那是自古也就这样的了。风霜雨雪,受得住的就过去了,受不住的,就寻求着自然的结果。

  那自然的结果不大好,把一个人默默地一声不响地就拉着离开了这人间的世界了。

  至于那还没有被拉去的,就风霜雨雪,仍旧在人间被吹打着。

\section{第二章}

\subsection{一}

  呼兰河除了这些卑琐平凡的实际生活之外,在精神上,也还有不少的盛举,如跳大神;唱秧歌;放河灯;野台子戏;四月十八娘娘庙大会……

  先说大神。大神是会治病的,她穿着奇怪的衣裳,那衣裳平常的人不穿;红的,是一张裙子,那裙子一围在她的腰上,她的人就变样了。开初,她并不打鼓,只是一围起那红花裙子就哆嗦。从头到脚,无处不哆嗦,哆嗦了一阵之后,又开始打颤。她闭着眼睛,嘴里边叽咕的。每一打颤,就装出来要倒的样子。把四边的人都吓得一跳,可是她又坐住了。

  大神坐的是凳子,她的对面摆着一块牌位,牌位上贴着红纸,写着黑字。

  那牌位越旧越好,好显得她一年之中跳神的次数不少,越跳多了就越好,她的信用就远近皆知。她的生意就会兴隆起来。那牌前,点着香,香烟慢慢地旋着。

  那女大神多半在香点了一半的时候神就下来了。那神一下来,可就威风不同,好像有万马千军让她领导似的,她全身是劲,她站起来乱跳。

  大神的旁边,还有一个二神,当二神的都是男人。他并不昏乱,他是清晰如常的,他赶快把一张圆鼓交到大神的手里,大神拿了这鼓,站起来就乱跳,先诉说那附在她身上的神灵的下山的经历,是乘着云,是随着风,或者是驾雾而来,说得非常之雄壮。二神站在一边,大神问他什么,他回答什么。

  好的二神是对答如流的,坏的二神,一不加小心说冲着了大神的一字,大神就要闹起来的。大神一闹起来的时候,她也没有别的办法,只是打着鼓,乱骂一阵,说这病人,不出今夜就必得死的,死了之后,还会游魂不散,家族、亲戚、乡里都要招灾的。这时吓得那请神的人家赶快烧香点酒,烧香点酒之后,若再不行,就得赶送上红布来,把红布挂在牌位上,若再不行,就得杀鸡,若闹到了杀鸡这个阶段,就多半不能再闹了。因为再闹就没有什么想头了。

  这鸡、这布,一律都归大神所有,跳过了神之后,她把鸡拿回家去自己煮上吃了。把红布用蓝靛染了之后,做起裤子穿了。

  有的大神,一上手就百般的下不来神。请神的人家就得赶快的杀鸡来,若一杀慢了,等一会跳到半道就要骂的,谁家请神都是为了治病,请大神骂,是非常不吉利的。所以对大神是非常尊敬的,又非常怕。

  跳大神,大半是天黑跳起,只要一打起鼓来,就男女老幼,都往这跳神的人家跑,若是夏天,就屋里屋外都挤满了人。还有些女人,拉着孩子,抱着孩子,哭天叫地地从墙头上跳过来,跳过来看跳神的。

  跳到半夜时分,要送神归山了,那时候,那鼓打得分外地响,大神也唱得分外地好听;邻居左右,十家二十家的人家都听得到,使人听了起着一种悲凉的情绪,二神嘴里唱:「大仙家回山了,要慢慢地走,要慢慢地行。」

  大神说:「我的二仙家,青龙山,白虎山……夜行三千里,乘着风儿不算难……」

  这唱着的词调,混合着鼓声,从几十丈远的地方传来,实在是冷森森的,越听就越悲凉。听了这种鼓声,往往终夜而不能眠的人也有。

  请神的人家为了治病,可不知那家的病人好了没有?却使邻居街坊感慨兴叹,终夜而不能已的也常常有。

  满天星光,满屋月亮,人生何如,为什么这么悲凉。

  过了十天半月的,又是跳神的鼓,地响。于是人们又都着了慌,爬墙的爬墙,登门的登门,看看这一家的大神,显的是什么本领,穿的是什么衣裳。听听她唱的是什么腔调,看看她的衣裳漂亮不漂亮。

  跳到了夜静时分,又是送神回山。送神回山的鼓,个个都打得漂亮。

  若赶上一个下雨的夜,就特别凄凉,寡妇可以落泪,鳏夫就要起来彷徨。

  那鼓声就好像故意招惹那般不幸的人,打得有急有慢,好像一个迷路的人在夜里诉说着他的迷惘,又好像不幸的老人在回想着他幸福的短短的幼年。又好像慈爱的母亲送着她的儿子远行。又好像是生离死别,万分地难舍。

  \underline{人生为了什么,才有这样凄凉的夜。}

  似乎下回再有打鼓的连听也不要听了。其实不然,鼓一响就又是上墙头的上墙头,侧着耳朵听的侧着耳朵在听,比西洋人赴音乐会更热心。

\subsection{二}

  七月十五盂兰会,呼兰河上放河灯了。

  河灯有白菜灯、西瓜灯,还有莲花灯。

  和尚、道士吹着笙、管、笛、箫,穿着拼金大红缎子的褊衫。在河沿上打起场子来在做道场。那乐器的声音离开河沿二里路就听到了。

  一到了黄昏,天还没有完全黑下来,奔着去看河灯的人就络绎不绝了。

  小街大巷,那怕终年不出门的人,也要随着人群奔到河沿去。先到了河沿的就蹲在那里。沿着河岸蹲满了人,可是从大街小巷往外出发的人仍是不绝,瞎子、瘸子都来看河灯(这里说错了,唯独瞎子是不来看河灯的),把街道跑得冒了烟了。

  姑娘、媳妇,三个一群,两个一伙,一出了大门,不用问,到哪里去。

  就都是看河灯去。

  黄昏时候的七月,火烧云刚刚落下去,街道上发着显微的白光,嘁嘁喳喳,把往日的寂静都冲散了,个个街道都活了起来,好像这城里发生了大火,人们都赶去救火的样子。非常忙迫,踢踢踏踏地向前跑。

  先跑到了河沿的就蹲在那里,后跑到的,也就挤上去蹲在那里。

  大家一齐等候着,等候着月亮高起来,河灯就要从水上放下来七月十五日是个鬼节,死了的冤魂怨鬼,不得脱生,缠绵在地狱里边是非常苦的,想脱生,又找不着路。这一天若是每个鬼托着一个河灯,就可得以脱生。大概从阴间到阳间的这一条路,非常之黑,若没有灯是看不见路的。所以放河灯这件事情是件善举。可见活着的正人君子们,对着那些已死的冤魂怨鬼还没有忘记。

  但是这其间也有一个矛盾,就是七月十五这夜生的孩子,怕是都不大好,多半都是野鬼托着个莲花灯投生而来的。这个孩子长大了将不被父母所喜欢,长到结婚的年龄,男女两家必要先对过生日时辰,才能够结亲。若是女家生在七月十五,这女子就很难出嫁,必须改了生日,欺骗男家。若是男家七月十五的生日,也不大好,不过若是财产丰富的,也就没有多大关系,嫁是可以嫁过去的,虽然就是一个恶鬼,有了钱大概怕也不怎样恶了。但在女子这方面可就万万不可,绝对的不可以;若是有钱的寡妇的独养女,又当别论,因为娶了这姑娘可以有一份财产在那里晃来晃去,就是娶了而带不过财产来,先说那一份妆奁也是少不了的。假说女子就是一个恶鬼的化身,但那也不要紧。

  平常的人说:「有钱能使鬼推磨。」似乎人们相信鬼是假的,有点不十分真。

  但是当河灯一放下来的时候,和尚为着庆祝鬼们更生,打着鼓,叮地响;念着经,好像紧急符咒似的,表示着,这一工夫可是千金一刻,且莫匆匆地让过,诸位男鬼女鬼,赶快托着灯去投生吧。

  念完了经,就吹笙管笛箫,那声音实在好听,远近皆闻。

  同时那河灯从上流拥拥挤挤,往下浮来了。浮得很慢,又镇静、又稳当,绝对的看不出来水里边会有鬼们来捉了它们去。

  这灯一下来的时候,金呼呼的,亮通通的,又加上有千万人的观众,这举动实在是不小的。河灯之多,有数不过来的数目,大概是几千百只。两岸上的孩子们,拍手叫绝,跳脚欢迎。大人则都看出了神了,一声不响,陶醉在灯光河色之中。灯光照得河水幽幽地发亮。水上跳跃着天空的月亮。真是人生何世,会有这样好的景况。

  一直闹到月亮来到了中天,大昴星,二昴星,三昴星都出齐了的时候,才算渐渐地从繁华的景况,走向了冷静的路去。

  河灯从几里路长的上流,流了很久很久才流过来了。再流了很久很久才流过去了。在这过程中,有的流到半路就灭了。有的被冲到了岸边,在岸边

  生了野草的地方就被挂住了。还有每当河灯一流到了下流,就有些孩子拿着竿子去抓它,有些渔船也顺手取了一两只。到后来河灯越来越稀疏了。

  到往下流去,就显出荒凉孤寂的样子来了。因为越流越少了。

  流到极远处去的,似乎那里的河水也发了黑。而且是流着流着地就少了一个。

  河灯从上流过来的时候,虽然路上也有许多落伍的,也有许多淹灭了的,但始终没有觉得河灯是被鬼们托着走了的感觉。

  可是当这河灯,从上流的远处流来,人们是满心欢喜的,等流过了自己,也还没有什么,唯独到了最后,那河灯流到了极远的下流去的时候,使看河灯的人们,内心里无由地来了空虚。

  「那河灯,到底是要漂到哪里去呢?」

  多半的人们,看到了这样的景况,就抬起身来离开了河沿回家去了。

  于是不但河里冷落,岸上也冷落了起来。

  这时再往远处的下流看去,看着,看着,那灯就灭了一个。再看着看着,又灭了一个,还有两个一块灭的。于是就真像被鬼一个一个地托着走了。

  打过了三更,河沿上一个人也没有了,河里边一个灯也没有了。

  河水是寂静如常的,小风把河水皱着极细的波浪。月光在河水上边并不像在海水上边闪着一片一片的金光,而是月亮落到河底里去了。似乎那渔船上的人,伸手可以把月亮拿到船上来似的。

  河的南岸,尽是柳条丛,河的北岸就是呼兰河城。

  那看河灯回去的人们,也许都睡着了。不过月亮还是在河上照着。

\subsection{三}

  野台子戏也是在河边上唱的。也是秋天,比方这一年秋收好,就要唱一台子戏,感谢天地。若是夏天大旱,人们戴起柳条圈来求雨,在街上几十人,跑了几天,唱着,打着鼓。求雨的人不准穿鞋,龙王爷可怜他们在太阳下边把脚烫得很痛,就因此下了雨了。一下了雨,到秋天就得唱戏的,因为求雨的时候许下了愿。许愿就得还愿,若是还愿的戏就更非唱不可了。

  一唱就是三天。

  在河岸的沙滩上搭起了台子来。这台子是用杆子绑起来的,上边搭上了席棚,下了一点小雨也不要紧,太阳则完全可以遮住的。

  戏台搭好了之后,两边就搭看台。看台还有楼座。坐在那楼座上是很好的,又风凉,又可以远眺。不过,楼座是不大容易坐得到的,除非当地的官、绅,别人是不大坐得到的。既不卖票,哪怕你就有钱,也没有办法。

  只搭戏台,就搭三五天。

  台子的架一竖起来,城里的人就说:「戏台竖起架子来了。」

  一上了棚,人就说:「戏台上棚了。」

  戏台搭完了就搭看台,看台是顺着戏台的左边搭一排,右边搭一排,所以是两排平行而相对的。一搭要搭出十几丈远去。

  眼看台子就要搭好了,这时候,接亲戚的接亲戚,唤朋友的唤朋友。

  比方嫁了的女儿,回来住娘家,临走(回婆家)的时候,做母亲的送到大门外,摆着手还说:「秋天唱戏的时候,再接你来看戏。」

  坐着女儿的车子远了,母亲含着眼泪还说:「看戏的时候接你回来。」

  所以一到了唱戏的时候,可并不是简单地看戏,而是接姑娘唤女婿,热闹得很。

  东家的女儿长大了,西家的男孩子也该成亲了,说媒的这个时候,就走上门来。约定两家的父母在戏台底下,第一天或是第二天,彼此相看。也有只通知男家而不通知女家的,这叫做「偷看」,这样的看法,成与不成,没有关系,比较的自由,反正那家的姑娘也不知道。

  所以看戏去的姑娘,个个都打扮得漂亮。都穿了新衣裳,擦了胭脂涂了粉,刘海剪得并排齐。头辫梳得一丝不乱,扎了红辫根,绿辫梢。也有扎了水红的,也有扎了蛋青的。走起路来象客人,吃起瓜子来,头不歪眼不斜的,温文尔雅,都变成了大家闺秀。有的着蛋青色布长衫,有的穿了藕荷色的,有的银灰的。有的还把衣服的边上压了条,有的蛋青色的衣裳压了黑条,有的水红洋纱的衣裳压了蓝条,脚上穿了蓝缎鞋,或是黑缎绣花鞋。

  鞋上有的绣着蝴蝶,有的绣着蜻蜓,有的绣着莲花,绣着牡丹的,各样的都有。

  手里边拿着花手巾。耳朵上戴了长钳子,土名叫做「带穗钳子」。这带穗钳子有两种,一种是金的、翠的;一种是铜的、琉璃的。有钱一点的戴金的,少微差一点的带琉璃的。反正都很好看,在耳朵上摇来晃去。黄忽忽,绿森森的。再加上满脸矜持的微笑,真不知这都是谁家的闺秀。

  那些已嫁的妇女,也是照样地打扮起来,在戏台下边,东邻西舍的姊妹们相遇了,好互相的品评。

  谁的模样俊,谁的鬓角黑。谁的手镯是福泰银楼的新花样,谁的压头簪又小巧又玲珑。谁的一双绛紫缎鞋,真是绣得漂亮。

  老太太虽然不穿什么带颜色的衣裳,但也个个整齐,人人利落,手拿长烟袋,头上撇着大扁方。慈祥,温静。

  戏还没有开台,呼兰河城就热闹不得了了,接姑娘的,唤女婿的,有一个很好的童谣:「拉大锯,扯大锯,老爷(外公)门口唱大戏。接姑娘,唤女婿,小外孙也要去。………」

  于是乎不但小外甥,三姨二姑也都聚在了一起。

  每家如此,杀鸡买酒,笑语迎门,彼此谈着家常,说着趣事,每夜必到三更,灯油不知浪费了多少。

  某村某村,婆婆虐待媳妇。哪家哪家的公公喝了酒就耍酒疯。又是谁家的姑娘出嫁了刚过一年就生了一对双生。又是谁的儿子十三岁就定了一家十八岁的姑娘做妻子。

  烛火灯光之下,一谈谈个半夜,真是非常的温暖而亲切。

  一家若有几个女儿,这几个女儿都出嫁了,亲姊妹,两三年不能相遇的也有。平常是一个住东,一个住西。不是隔水的就是离山,而且每人有一大群孩子,也各自有自己的家务,若想彼此过访,那是不可能的事情。

  若是做母亲的同时把几个女儿都接来了,那她们的相遇,真仿佛已经隔了三十年了。相见之下,真是不知从何说起,羞羞惭惭,欲言又止,刚一开口又觉得不好意思,过了一刻工夫,耳脸都发起烧来,于是相对无语,心中又喜又悲。过了一袋烟的工夫,等那往上冲的血流落了下去,彼此都逃出了那种昏昏恍恍的境界,这才来找几句不相干的话来开头;或是:「你多咱来的?」

  或是:「孩子们都带来了?」

  关于别离了几年的事情,连一个字也不敢提。

  从表面上看来,她们并不是像姊妹,丝毫没有亲热的表现。面面相对的,不知道她们两个人是什么关系,似乎连认识也不认识,似乎从前她们两个并没有见过,而今天是第一次的相见,所以异常的冷落。

  但是这只是外表,她们的心里,就早已沟通着了。甚至于在十天或半月之前,她们的心里就早已开始很远地牵动起来,那就是当着她们彼此都接到了母亲的信的时候。

  那信上写着迎接她们姊妹回来看戏的。

  从那时候起,她们就把要送给姐姐或妹妹的礼物规定好了。

  一双黑大绒的云子卷,是亲手做的。或者就在她们的本城和本乡里,有一个出名的染缸房,那染缸房会染出来很好的麻花布来。于是送了两匹白布去,嘱咐他好好地加细地染着。一匹是白地染蓝花,一匹是蓝地染白花。蓝地的染的是刘海戏金蟾,白地的染的是蝴蝶闹莲花。

  一匹送给大姐姐,一匹送给三妹妹。

  现在这东西,就都带在箱子里边。等过了一天二日的,寻个夜深人静的时候,轻轻地从自己的箱底把这等东西取出来,摆在姐姐的面前,说:「这麻花布被面,你带回去吧!」

  只说了这么一句,看样子并不像是送礼物,并不像今人似的,送一点礼物很怕邻居左右看不见,是大嚷大吵着的,说这东西是从什么山上,或是什么海里得来的,那怕是小河沟子的出品,也必要连那小河沟子的身份也提高,说河沟子是怎样地不凡,是怎样地与众不同,可不同别的河沟子。

  这等乡下人,糊里糊涂的,要表现的,无法表现,什么也说不出来,只能把东西递过去就算了事。

  至于那受了东西的,也是不会说什么,连声道谢也不说,就收下了。也有的稍微推辞了一下,也就收下了。

  「留着你自己用吧!」

  当然那送礼物的是加以拒绝。一拒绝,也就收下了。

  每个回娘家看戏的姑娘,都零零碎碎的带来一大批东西。送父母的,送兄嫂的,送姪女的,送三亲六故的。带了东西最多的,是凡见了长辈或晚辈都多少有点东西拿得出来,那就是谁的人情最周到。

  这一类的事情,等野台子唱完,拆了台子的时候,家家户户才慢慢的传诵。

  每个从娘家回婆家的姑娘,也都带着很丰富的东西,这些都是人家送给她的礼品。东西丰富得很,不但有用的,也有吃的,母亲亲手装的咸肉,姐姐亲手晒的干鱼,哥哥上山打猎打了一只雁来腌上,至今还有一只雁大腿,这个也给看戏小姑娘带回去,带回去给公公去喝洒吧。

  于是乌三八四的,离走的前一天晚上,真是忙了个不休,就要分散的姊妹们连说个话儿的工夫都没有了。大包小包一大堆。

  再说在这看戏的时间,除了看亲戚,会朋友,还成了许多好事,那就是谁家的女儿和谁家公子订婚了,说是明年二月,或是三月就要娶亲。订婚酒,已经吃过了,眼前就要过「小礼」的,所谓「小礼」就是在法律上的订婚形式,一经过了这番手续,东家的女儿,终归就要成了西家的媳妇了。

  也有男女两家都是外乡赶来看戏的,男家的公子也并不在,女家的小姐也并不在。只是两家的双亲有媒人从中勾通着,就把亲事给定了。也有的喝酒作乐的随便的把自己的女儿许给了人家。也有的男女两家的公子、小姐都还没有生出来,就给定下亲了。这叫做「指腹为亲」。这指腹为亲的,多半都是相当有点资财的人家才有这样的事。

  两家都很有钱,一家是本地的烧锅掌柜的,一家是白旗屯的大窝堡,两家是一家种高粱,是一家开烧锅。开烧锅的需要高粱,种高粱的需要烧锅买他的高粱,烧锅非高粱不可,高粱非烧锅不行。恰巧又赶上这两家的妇人,都要将近生产,所以就「指腹为亲」了。

  无管是谁家生了男孩子,谁家生了女孩子,只要是一男一女就规定他们是夫妇。假若两家都生了男孩,都就不能勉强规定了。两家都生了女孩也是不能够规定的。

  但是这指腹为亲,好处不太多,坏处是很多的。半路上当中的一家穷了,不开烧锅了,或者没有窝堡了。其余的一家,就不愿意娶他家的姑娘,或是把女儿嫁给一家穷人。假若女家穷了,那还好办,若实在不娶,他也没有什么办法。若是男家穷了,男家就一定要娶,若一定不让娶,那姑娘的名誉就很坏,说她把谁家谁给「妨」穷了,又不嫁了。「妨」字在迷信上说就是因为她命硬,因为她某家某家穷了。以后她就不大容易找婆家,会给她起一个名叫做「望门妨」。无法,只得嫁过去,嫁过去之后,妯娌之间又要说她嫌贫爱富,百般地侮辱她。丈夫因此也不喜欢她了,公公婆婆也虐待她,她一个年轻的未出过家门的女子,受不住这许多攻击,回到娘家去,娘家也无甚办法,就是那当年指腹为亲的母亲说:\underline{「这都是你的命(命运),你好好地耐着吧!」}

  年轻的女子,莫名其妙的,不知道自己为什么要有这样的命,于是往往演出悲剧来,跳井的跳井,上吊的上吊。

  古语说,「女子上不了战场。」

  其实不对的,这井多么深,平白地你问一个男子,问他这井敢跳不敢跳,怕他也不敢的。而一个年轻的女子竟敢了,上战场不一定死,也许回来闹个一官半职的。可是跳井就很难不死,一跳就多半跳死了。

  那么节妇坊上为什么没写着赞美女子跳井跳得勇敢的赞词?那是修节妇坊的人故意给删去的。因为修节妇坊的,多半是男人。他家里也有一个女人。

  他怕是写上了,将来他打他女人的时候,他的女人也去跳井。女人也跳下井,留下来一大群孩子可怎么办?于是一律不写。只写,温文尔雅,孝顺公婆……

  大戏还没有开台,就来了这许多事情。等大戏一开了台,那戏台下边,真是人山人海,拥挤不堪。搭戏台的人,也真是会搭,正选了一块平平坦坦的大沙滩,又光滑、又干净,使人就是倒在上边,也不会把衣裳沾一丝儿的土星。这沙滩有半里路长。

  人们笑语连天,哪里是在看戏,闹得比锣鼓好像更响,那戏台上出来一个穿红的,进去一个穿绿的,只看见摇摇摆摆地走出走进,别的什么也不知道了,不用说唱得好不好,就连听也听不到。离着近的还看得见不挂胡子的戏子在张嘴,离得远的就连戏台那个穿红衣裳的究竟是一个坤角,还是一个男角也都不大看得清楚。简直是还不如看木偶戏。

  但是若有一个唱木偶戏的这时候来在台下,唱起来,问他们看不看,那他们一定不看的,哪怕就连戏台子的边也看不见了,哪怕是站在二里路之外,他们也不看那木偶戏的。因为在大戏台底下,哪怕就是睡了一觉回去,也总算是从大戏台子底下回来的,而不是从什么别的地方回来的。

  一年没有什么别的好看,就这一场大戏还能够轻易地放过吗?所以无论看不看,戏台底下是不能不来。

  所以一些乡下的人也都来了,赶着几套马的大车,赶着老牛车,赶着花轮子,赶着小车子。小车子上边驾着大骡子。总之家里有什么车就驾了什么车来。也有的似乎他们家里并不养马,也不养别的牲口,就只用了一匹小毛驴,拉着一个花轮子也就来了。

  来了之后,这些车马,就一齐停在沙滩上,马匹在草包上吃着草,骡子到河里去喝水。车子上都搭席棚,好像小看台似的,排列在戏台的远处。那车子带来了他们的全家,从祖母到孙子媳,老少三辈,他们离着戏台二三十丈远,听是什么也听不见的,看也很难看到什么,也不过是五红大绿的,在戏台上跑着圈子,头上戴着奇怪的帽子,身上穿着奇怪的衣裳。谁知道那些人都是干什么的,有的看了三天大戏子台,而连一场的戏名字也都叫不出来。

  回到乡下去,他也跟着人家说长道短的,偶尔人家问了他说的是哪出戏,他竟瞪了眼睛,说不出来了。

  至于一些孩子们在戏台底下,就更什么也不知道了,只记住一个大胡子,一个花脸的,谁知道那些都是在做什么,比比划划,刀枪棍棒的乱闹一阵。

  反正戏台底下有些卖凉粉的,有些卖糖球的,随便吃去好了。什么粘糕,油炸馒头,豆腐脑都有,这些东西吃了又不饱,吃了这样再去吃那样。卖西瓜的,卖香瓜的,戏台底下都有,招得苍蝇一大堆,嗡嗡地飞。

  戏台下敲锣打鼓震天地响。

  那唱戏的人,也似乎怕远处的人听不见,也在拚命地喊,喊破了喉咙也压不住台的。那在台下的早已忘记了是在看戏,都在那里说长道短,男男女女的谈起家常来。还有些个远亲,平常一年也看不到,今天在这里看到了,哪能下打招呼。所以三姨二婶子的,就在人多的地方大叫起来,假若是在看台的凉棚里坐着,忽然有一个老太太站了起来,大叫着说:「他二舅母,你可多咱来的?」

  于是那一方也就应声而起。原来坐在看台的楼座上的,离着戏台比较近,听唱是听得到的,所以那看台上比较安静。姑娘媳妇都吃着爪子,喝着茶。

  对这大嚷大叫的人,别人虽然讨厌,但也不敢去禁止,你若让她小一点声讲话,她会骂了出来:「这野台子戏,也不是你家的,你愿听戏,你请一台子到你家里去唱……」

  另外的一个也说:「哟哟,我没见过,看起戏来,都六亲不认了,说个话儿也不让……」

  这还是比较好的,还有更不客气的,一开口就说:「小养汉老婆……你奶奶,一辈子家里外头靡受过谁的大声小气,今天来到戏台底下受你的管教来啦,你娘的……」

  被骂的人若是不搭言,过一回也就了事了,若一搭言,自然也没有好听的。于是两边就打了起来啦,西瓜皮之类就飞了过去。

  这一来在戏台下看戏的,不料自己竟演起戏来,于是人们一窝蜂似的,都聚在这个真打真骂的活戏的方面来了。也有一些流氓混子之类,故意地叫着好,惹得全场的人哄哄大笑。假若打仗的还是个年轻的女子,那些讨厌的流氓们还会说着各样的俏皮话,使她火上加油越骂就越凶猛。

  自然那老太太无理,她一开口就骂了人。但是一闹到后来,谁是谁非也就看不出来了。

  幸而戏台上的戏子总算沉着,不为所动,还在那里阿拉阿拉地唱。过了一个时候,那打得热闹的也究竟平静了。

  再说戏台下边也有一些个调情的,那都是南街豆腐房里的嫂嫂,或是碾磨房的碾倌磨倌的老婆。碾官的老婆看上了一个赶马车的车夫。或是豆腐匠看上了开粮米铺那家的小姑娘。有的是两方面都眉来眼去,有的是一方面殷勤,他一方面则表示要拒之千里之外。这样的多半是一边低,一边高,两方面的资财不对。

  绅士之流,也有调情的,彼此都坐在看台之上,东张张,西望望。三亲六故,姐夫小姨之间,未免地就要多看几眼,何况又都打扮得漂亮,非常好看。

  绅士们平常到别人家的客厅去拜访的时候,绝不能够看上了人家的小姐就不住地看,那该多么不绅士,那该多么不讲道德。那小姐若一告诉了她的父母,她的父母立刻就和这样的朋友绝交。绝交了,倒不要紧,要紧的是一传出去名誉该多坏。绅士是高雅的,哪能够不清不白的,哪能够不分长幼地去存心朋友的女儿,像那般下等人似的。

  绅士彼此一拜访的时候,都是先让到客厅里去,端端庄庄地坐在那里,而后倒茶装烟。规矩礼法,彼此都尊为是上等人。朋友的妻子儿女,也都出来拜见,尊为长者。在这种时候,只能问问大少爷的书读了多少,或是又写了多少字了。连朋友的太太也不可以过多的谈话,何况朋友的女儿呢?那就连头也不能够抬的,哪里还敢细看。

  现在在戏台上看看怕不要紧,假设有人问道,就说是东看西看,瞧一瞧是否有朋友在别的看台上。何况这地方又人多眼杂,也许没有人留意。

  三看两看的,朋友的小姐倒没有看上,可看上了一个不知道在什么地方见到过的一位妇人,那妇人拿着小小的鹅翎扇子,从扇子梢上往这边转着眼珠,虽说是一位妇人,可是又年轻,又漂亮。

  这时候,这绅士就应该站起来打着口哨,好表示他是开心的,可是我们中国上一辈的老绅士不会这一套。他另外也有一套,就是他的眼睛似睁非睁的迷离恍惚的望了出去,表示他对她有无限的情意。可惜离得太远,怕不会看得清楚,也许是枉费了心思了。

  也有的在戏台下边,不听父母之命,不听媒妁之言,自己就结了终生不解之缘。这多半是表哥表妹等等,稍有点出身来历的公子小姐的行为。他们一言为定,终生合好。间或也有被父母所阻拦,生出来许多波折。但那波折都是非常美丽的,使人一讲起来,真是比看《红楼梦》更有趣味。来年再唱大戏的时候,姊妹们一讲起这佳话来,真是增添了不少的回想……

  赶着车进城来看戏的乡下人,他们就在河边沙滩上,扎了营了。夜里大戏散了,人们都回家了,只有这等连车带马的,他们就在沙滩上过夜。好像出征的军人似的,露天为营。有的住了一夜,第二夜就回去了。有的住了三夜,一直到大戏唱完,才赶着车子回乡。不用说这沙滩上是很雄壮的,夜里,他们每家燃了火,煮茶的煮茶,谈天的谈天,但终归是人数太少,也不过二三十辆车子。所燃起来的火,也不会火光冲天,所以多少有一些凄凉之感。

  夜深了,住在河边上,被河水吸着又特别的凉,人家睡起觉来都觉得冷森森的。尤其是车夫马信之类,他们不能够睡觉,怕是有土匪来抢劫他们马匹,所以就坐以待旦。

  于是在纸灯笼下边,三个两个的赌钱。赌到天色发白了,该牵着马到河边去饮水去了。在河上,遇到了捉蟹的蟹船。蟹船上的老头说:「昨天的《打渔杀家》唱得不错,听说今天有《汾河湾》。」

  那牵着牲口饮水的人,是一点大戏常识也没有的。他只听到牲口喝水的声音呵呵的,其他的则不知所答了。

\subsection{四}

  四月十八娘娘庙大会,这也是为着神鬼,而不是为着人的。

  这庙会的土名叫做「逛庙」,也是无分男女老幼都来逛的,但其中以女子最多。

  女子们早晨起来,吃了早饭,就开始梳洗打扮。打扮好了,就约了东家姐姐,西家妹妹的去逛庙去了。竟有一起来就先梳洗打扮的,打扮好了,才吃饭,一吃了饭就走了。总之一到逛庙这天,各不后人,到不了半晌午,就车水马龙,拥挤得气息不通了。

  挤丢了孩子的站在那儿喊,找不到妈的孩子在人群里边哭,三岁的、五岁的,还有两岁的刚刚会走,竟也被挤丢了。

  所以每年庙会上必得有几个警察在收这些孩子。收了站在庙台上,等着他的家人来领。偏偏这些孩子都很胆小,张着嘴大哭,哭得实在可怜,满头满脸是汗。有的十二三岁了,也被丢了,问他家住在哪里?他竟说不出所以然来,东指指,西划划,说是他家门口有一条小河沟,那河沟里边出虾米,就叫做「虾沟子」,也许他家那地名就叫「虾沟子」,听了使人莫名其妙。

  再问他这虾沟子离城多远,他便说:骑马要一顿饭的工夫可到,坐车要三顿饭的工夫可到。究竟离城多远,他没有说。问他姓什么,他说他祖父叫史二,他父亲叫史成……这样你就再也不敢问他了。要问他吃饭没有?他就说:「睡觉了」。这是没有办法的,任他去吧。于是却连大带小的一齐站在庙门口,他们哭的哭,叫的叫。好像小兽似的,警察在看守他们。

  娘娘庙是在北大街上,老爷庙和娘娘庙离不了好远。那些烧香的人,虽然说是求子求孙,是先该向娘娘来烧香的,但是人们都以为阴间也是一样的重男轻女,所以不敢倒反天干。所以都是先到老爷庙去,打过钟,磕过头,好像跪到那里报个到似的,而后才上娘娘庙去。

  老爷庙有大泥像十多尊,不知道哪个是老爷,都是威风凛凛,气概盖世的样子。有的泥像的手指尖都被攀了去,举着没有手指的手在那里站着,有的眼睛被挖了,像是个瞎子似的。有的泥像的脚趾是被写了一大堆的字,那字不太高雅,不怎么合乎神的身份。似乎是说泥像也该娶个老婆,不然他看了和尚去找小尼姑,他是要忌妒的。这字现在没有了,传说是这样。

  为了这个,县官下了手令,不到初一十五,一律的把庙门锁起来,不准闲人进去。

  当地的县官是很讲仁义道德的。传说他第五个姨太太,就是从尼姑庵接来的。所以他始终相信尼姑绝不会找和尚。自古就把尼姑列在和尚一起,其实是世人不查,人云亦云。好比县官的第五房姨太太,就是个尼姑。难道她也被和尚找过了吗?这是不可能的。

  所以下令一律的把庙门关了。

  娘娘庙里比较的清静,泥像也有一些个,以女子为多,多半都没有横眉竖眼,近乎普通人,使人走进了大殿不必害怕。不用说是娘娘了,那自然是很好的温顺的女性。就说女鬼吧,也都不怎样恶,至多也不过披头散发的就完了,也决没有像老爷庙里那般泥像似的,眼睛冒了火,或像老虎似的张着嘴。

  不但孩子进了老爷庙有的吓得大哭,就连壮年的男人进去也要肃然起敬,好像说虽然他在壮年,那泥像若走过来和他打打,他也决打不过那泥像的。

  所以在老爷庙上磕头的人,心里比较虔诚,因为那泥像,身子高、力气大。

  到了娘娘庙,虽然也磕头,但就总觉得那娘娘没有什么出奇之处。

  塑泥像的人是男人,他把女人塑得很温顺,似乎对女人很尊敬。他把男人塑得很凶猛,似乎男性很不好。其实不对的,世界上的男人,无论多凶猛,眼睛冒火的似乎还未曾见过。就说西洋人吧,虽然与中国人的眼睛不同,但也不过是蓝瓦瓦地有点类似猫头的眼睛而已,居然间冒了火的也没有。眼睛会冒火的民族,目前的世界还未发现。那么塑泥像的人为什么把他塑成那个样子呢?那就是让你一见生畏,不但磕头,而且要心服。就是磕完了头站起再看着,也绝不会后悔,不会后悔这头是向一个平庸无奇的人白白磕了。至于塑像的人塑起女子来为什么要那么温顺,那就告诉人,温顺的就是老实的,老实的就是好欺侮的,告诉人快来欺侮她们吧。

  人若老实了,不但异类要来欺侮,就是同类也不同情。

  比方女子去拜过了娘娘庙,也不过向娘娘讨子讨孙。讨完了就出来了,其余的并没有什么尊敬的意思。觉得子孙娘娘也不过是个普通的女子而已,只是她的孩子多了一些。

  所以男人打老婆的时候便说:「娘娘还得怕老爷打呢?何况你一个长舌妇!」

  可见男人打女人是天理应该,神鬼齐一。怪不得那娘娘庙里的娘娘特别温顺,原来是常常挨打的缘故。可见温顺也不是怎么优良的天性,而是被打的结果。甚或是招打的原由。

  两个庙都拜过了的人,就出来了,拥挤在街上。街上卖什么玩具的都有,多半玩具都是适于几岁的小孩子玩的。泥做的泥公鸡,鸡尾巴上插着两根红鸡毛,一点也不像,可是使人看去,就比活的更好看。家里有小孩子的不能不买。何况拿在嘴上一吹又会呜呜地响。买了泥公鸡,又看见了小泥人,小泥人的背上也有一个洞,这洞里边插着一根芦苇,一吹就响。那声音好像是诉怨似的,不太好听,但是孩子们都喜欢,做母亲的也一定要买。其余的如卖哨子的,卖小笛子的,卖线蝴蝶的,卖不倒翁的,其中尤以不倒翁最著名,也最上讲究,家家都买,有钱的买大的,没有钱的,买个小的。大的有一尺多高,二尺来高。小的有小得像个鸭蛋似的。无论大小,都非常灵活,按倒了就起来,起得很快,是随手就起来的。买不倒翁要当场试验,间或有生手的工匠所做出来的不倒翁,因屁股太大了,他不愿意倒下,也有的倒下了他就不起来。所以买不倒翁的人就把手伸出去,一律把他们按倒,看哪个先站起来就买哪个,当那一倒一起的时候真是可笑,摊子旁边围了些孩子,专在那里笑。不倒翁长得很好看,又白又胖。并不是老翁的样子,也不过他的名字叫不倒翁就是了。其实他是一个胖孩子。做得讲究一点的,头顶上还贴了一簇毛算是头发。有头发的比没有头发的要贵二百钱。有的孩子买的时候力争要戴头发的,做母亲的舍不得那二百钱,就说到家给他剪点狗毛贴。孩子非要戴毛的不可,选了一个戴毛的抱在怀里不放。没有法只得买了。这孩子抱着欢喜了一路,等到家一看,那簇毛不知什么时候已经飞了。于是孩子大哭。虽然母亲已经给剪了簇狗毛贴上了,但那孩子就总觉得这狗毛不是真的,不如原来的好看。也许那原来也贴的是狗毛,或许还不如现在的这个好看。

  但那孩子就总不开心,忧愁了一个下半天。

  庙会到下半天就散了。虽然庙会是散了,可是庙门还开着,烧香的人,拜佛的人继续的还有。有些没有儿子的妇女,仍旧在娘娘庙上捉弄着娘娘。

  给子孙娘娘的背后钉一个钮扣,给她的脚上绑一条带子,耳朵上挂一只耳环,给她带一副眼镜,把她旁边的泥娃娃给偷着抱走了一个。据说这样做,来年就都会生儿子的。

  娘娘庙的门口,卖带子的特别多,妇人们都争着去买,她们相信买了带子,就会把儿子给带来了。

  若是未出嫁的女儿,也误买了这东西,那就将成为大家的笑柄了。

  庙会一过,家家户户就都有一个不倒翁,离城远至十八里路的,也都买了一个回去。回到家里,摆在迎门的向口,使别人一过眼就看见了,他家的确有一个不倒翁。不差,这证明逛庙会的时节他家并没有落伍,的确是去逛过了。

  歌谣上说:「小大姐,去逛庙,扭扭搭搭走的俏,回来买个搬不倒。」

\subsection{五}

  这些盛举,都是为鬼而做的,并非为人而做的。至于人去看戏、逛庙,也不过是揩油借光的意思。

  跳大神有鬼,唱大戏是唱给龙王爷看的,七月十五放河灯,是把灯放给鬼,让他顶着个灯去脱生。四月十八也是烧香磕头的祭鬼。

  只是跳秧歌,是为活人而不是为鬼预备的。跳秧歌是在正月十五,正是农闲的时候,趁着新年而化起装来,男人装女人,装得滑稽可笑。

  狮子、龙灯、旱船……等等,似乎也跟祭鬼似的,花样复杂,一时说不清楚。
\end{document}