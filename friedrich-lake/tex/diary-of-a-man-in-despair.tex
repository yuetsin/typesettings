%-*- coding: UTF-8 -*-

\documentclass[UTF8]{ctexart}
\usepackage{graphicx}
\usepackage{float}
\usepackage{CJKpunct}
\usepackage{amsmath}
\usepackage{geometry}
\geometry{a5paper,centering,scale=0.8}
\usepackage[format=hang,font=small,textfont=it]{caption}
\usepackage[nottoc]{tocbibind}
\setromanfont{SourceHanSerifSC-Medium} %设置中文字体
\punctstyle{quanjiao} %使用全角标点	


\title{绝望者日记:纳粹德国的政治与社会生活}
\author{[德] 弗里德里希·莱克\ 著}
\date{何为宁\ 译}
\begin{document}

\maketitle


\newpage

\tableofcontents

\newpage

\section{1936年5月\ 斯宾格勒死了}

斯宾格勒\footnote{Oswald Spengler,1880—1936,德国著名哲学家、文学家,著有《西方的没落》。}死了。他的死,就跟古代皇帝的死一样,要仆人陪葬。几天后,阿伯斯(Albers)也跟着死了。阿伯斯在出版社负责为斯宾格勒出书。阿伯斯的死法很恐怖,他跳到了一列开往施塔恩贝格(Starnberg)的火车的轮下,尸体留在铁轨上,大腿被碾断了。

就在几周前,我和斯宾格勒在慕尼黑的拜尔大街(Bayerstrasse)见了最后一面。像往常一样,他穿着昂贵的斜纹软呢西服。像往常一样,他表情严肃,满嘴气话。他做出了一系列令人吃惊的预言,这反映出他所受到的伤害和他的报复欲。跟他聊天颇有收获。

我还记得我们第一次见面的情况,当时是阿伯斯把他带到我家里来的。一辆小马车把他从火车站接来,那辆马车实在是太小了,根本不是为他设计的。他的体型十分庞大,又穿着一件厚厚的外套,就更显得庞大了。他看上去就像一块巨大的石头一样稳定:低沉的声音;穿着斜纹软呢西装——他那时就已经有了这样的穿衣习惯;晚餐的胃口极大;夜里睡觉打呼噜,声音惊人地大,就好像锯木头一样。那天晚上,住在我位于基姆高(Chiemgau)乡下别墅的还有几位客人,他们被吓得一夜没睡好。

他在这个时期不是很成功,还没有彻底投奔大财团——他直到投奔了大财团后命运才有所改变。此时,他仍然能过愉快闲散的生活,有时甚至在别人的劝诱下,敢于放下尊严去河里游泳。当然,后来就无法想象他敢在种地的农民面前,穿着游泳衣,气喘吁吁地像个河神一样爬上岸。

他是最奇怪的混合物,身上既拥有真正伟大的人性,也有或大或小的缺点。现在回想起来,我离开他,是害怕他日后伤害到我。他是个喜欢独自吃饭的人——在大吃大喝中,眼睛里仍然流露出忧郁的神情。有件事回忆起来很有意思。当时距离第一次世界大战爆发只有几周时间了,他来找我和阿伯斯一起吃一顿轻便的晚餐,那时惹客人生气不是什么大不了的事,整个晚餐时间,他一边滔滔不绝地说,一边吃饭,吃掉了整整一只鹅,没让别人尝一口。

他喜欢大吃大喝(费用由支持他的工业大亨们负担)不是他唯一的趣闻。在他成名前,我俩见过一面,他要求我不要去他住的地方(我记得是在慕尼黑的阿哥尼大街)。他解释说自己的住处太小,而且他很想让我去看看他在附近的书房,那里有丰富的藏书。

到了1926年,他获得了一些重工业巨头的赏识,把家搬到了伊萨河畔瓦登梅尔大街上的一套豪宅中。他邀请我去看他家里的几个大房间。他让我看他的地毯和绘画,甚至让我看他的床——这张床很值得一看,因为它的样子就像一个棺材。当我提出还然想看他的书房时,他显得有点惊慌。他犹豫了一下,但还是让我看了。我发现他的书房很小,书房里有一个胡桃木的书架,书架上除了有一排乌尔施泰因出版社印制的书和几本侦探小说之外,还有一些公认的“色情书籍”。

我从来没有见到过像他那样极度缺少幽默感的人,而且别人稍微批评他一下,他就受不了。他最憎恶骗子;尽管他那部《西方的没落》推演出了许多宏大的结论,但他始终不去更正自己书中的错误。比如,他说陀思妥耶夫斯基是在圣彼得堡成名了,而不是在莫斯科。魏玛共和国的伯恩哈德公爵(Duke Bernhard)死的时候,华伦斯坦\footnote{Wallenstein,1618—1648,神圣罗马帝国的军事统帅。}还没有遇刺。他有几个重要结论是建立在这些错误之上的。谁都有可能犯这样的错误;但悲哀的是没有人敢告诉斯宾格勒!

我记得在我家里曾发生过很好玩的一幕。他有个习惯,吃完了晚饭就抓住在场的人,用盘问的方式硬让他们听他说教。他有一位听众,刚从非洲回来,患了疟疾,坐在扶手椅子上边听讲边睡着了,且呼噜声特别响,但每次那个主讲的大人物问他问题的时候,他却总能醒过来,而且能非常自如地立即回答问题,所用的词汇竟然是斯宾格勒常用的。这件事本来可以逗他一乐,但他却感到受到了沉重的伤害,后来也不理人家了。

再重申一遍,他是我见到过的最没有幽默感的人;在这方面,只有希特勒先生和纳粹党羽能超过他。希特勒之流,卑鄙得要死,从骨子里就缺乏幽默感,在公众场合的表现极为沉闷单调,在他们统治下的四年里,生活简直像死尸一样僵硬,我们快要被憋死了。然而,如果你认为我回忆了斯宾格勒的这么多缺点是为了贬低他,那就错了。我既不必提及他研究赫拉克利特(Heraclitus)的独立工作,也不必提及他对整整一代人做出的前瞻性预言。无论谁见到他,都能看出他出众的才气,即使是临时发言,他的才气也不会受到任何影响;他还代表了人类所能接受的最高教养;他一副沉着镇定的样子,这副样子只有在罗马帝国后期的那些表现淡泊主义的雕塑中才能看到。

我不知他是否真地预言了这个世界将会出现非理性的潮流,但这股潮流确实出现了。我也不知道他所说的“西方的没落”,其实真正没落的只是文艺复兴以来的这400年来所创造的世界。然而,在他追求自己的学问的过程中,他陷入了对重工业寡头的依赖,而这依赖已经开始对他的思想产生了影响,这是他命中注定的。我至少有一点不如他,我不知道如何把接近于陀思妥耶夫斯基在1922年出版的《衰落》第二卷中所表现出的宏大的基督世界的预见力,与他在最后一部著作中表现出来的技术治国的理念混合在一起。他是个悲剧,虽然他有极高的智慧,要我说就是那种喜欢恶作剧的老师的坏智慧,但他的智慧让他不相信众神,且更加不相信上帝。大约在1926年的时候,他的信徒开始离他而去,因为他与当代的德国同流合污——不是纳粹,没有人比他更恨纳粹,无论是躺下,或梦中,或醒来,他都在恨纳粹——而是鲁尔那边的军队商人,这些人在帝国垮台后成为了国家的真正主宰者,他们很高兴满足斯宾格勒对生活格调的追求,他的生活格调,虽然有一部分贵族的成分,但不知何故也包含了享乐主义的成分。他的早期工作所表现出的不断上升中的思维力,在他的饭桌上堆满了工业寡头提供给他的法国勃艮第产的佳酿后就突然消失了——我说的不是圣安东尼,而是蒂森\footnote{Thyssen,钢铁寡头。}和\footnote{霍希(Hosch,汽车寡头。}。

所以,斯宾格勒是被自己的享乐主义倾向给欺骗了,被美味的佐料给欺骗了,被他姐姐的厨艺给欺骗了(他姐姐为他把持家务)。给纳粹编辑报纸的人,过去是有特殊经历的小学教师,或是第一次世界大战时的中尉,这些军人在战后什么事都没有做。纳粹的这些编辑们都兴高采烈地认为斯宾格勒转变为跟他们有一样的思想了;他们还认为剩下的反对派人士会一个接着一个地转变。斯宾格勒的《决胜的年代》有两部,第一部使他成了一名殉教者,他的第二部目前存在瑞士一家银行的地下室,正在等着复苏的时刻,我们都盼望着那个时刻的到来。

\section{1936年7月\ 兴登堡的迂腐}

慕尼黑,这座被普鲁士人占领且让人感到陌生的城市,传出了一个有趣的故事。故事的主角是交通部长埃塞尔先生,从他的所作所为看,他应该被称为放荡交通部长。埃塞尔与一家小旅馆老板的女儿私通,被那女孩的爸爸狠狠揍了一顿,他既不敢外出、也不敢留在慕尼黑。这个政权的风格,就是把道德当作累赘。他在不久之后被调到了柏林,而且职位还升了。升了官后,他宣布德国人今后不能单独出国,而必须一起组团出国,这样就能释放“组织给人带来的愉快”。因此,我们失去了仅有的个人自由,完全变成了这支游牧部落里的囚犯,统治这支部落的是几只恶毒的“猴子”\footnote{指纳粹党——编者注。},他们在三年前获得了统治我们的权力。

我最近与一个很有见识的人谈起了纳粹夺权的事。他说这件被人们称为“德国革命”的事,其本质就是勒索。他的故事如下:

老兴登堡是个穷光蛋。他想在离职前改变这种状况,于是让自己的儿子奥斯卡接管自己的生意。奥斯卡把钱投在股票市场,但股票市场突然崩溃,他欠了1300万马克的债。为了还钱,他参与了“东部免除债务运作”——我相信他父亲不知道这些情况——纳粹在1932年发现了这个秘密(布鲁宁内阁倒台可能与此有关)。希特勒的人得到罪证的复本,这等于是鞭子在手了。

兴登堡一直不想见希特勒。据报道,他可能真的说过那句话,“我甚至不想让那个下士去做邮政局长,就更不用说让他去做总理了。”\footnote{第一次世界大战期间,希特勒在军队里做信使。}但到了1932年夏季,兴登堡已经变得身不由己了。这样说并非言过其实。当普特姆帕(Potempa)发生残杀共产党人的案件时,希特勒厚颜无耻地发电报祝贺,但兴登堡作为国家元首岂能一言不发?

到了1932年底,德国国会开始调查东部股票免除债务的事,兴登堡在纽德克的地产浮出水面。兴登堡集团感到十分忧虑。这时又爆发了柏林大游行,冯·巴本内阁变得更加顺从纳粹的“解决方案”。希特勒判断现在可以施压去获得总理提名了。

这个故事与我从其他人那里听到的消息是吻合的。格里哥·斯特拉瑟(Gregor Strasser)这个人,在罗姆暴动中被杀,他在1932年11月曾经向我暗示过同样的事。这也解释了为什么兴登堡要与纳粹在冯·巴本的别墅中召开秘密会议。冯·施罗特夫人(Schrǒter)担任双方交谈的调解人。自从运输工人罢工之后,冯·巴本一想到自己妻子和财产的安全,就害怕得发抖,他在这些人面前中扮演了一个奇怪的角色。

这最终还能解释另一个被否认的幕后事件。冯·施莱歇(Schleicher)是整个事件中的另一个阴谋人物,他在与老总统分裂后,在弗里德里希大街的火车站逮捕了奥斯卡·冯·兴登堡,并囚禁了他一晚上。据说,冯·布雷多(Bredow)将军是指挥这次逮捕行动的军官。一年半后,他与冯·施莱歇一起在罗姆暴动中被杀。

因此,我们之所以陷入目前的极度悲惨境界,似乎是因为兴登堡在股市投机失败后被人勒索的结果。

我无法对一个死人进行审判。但我相信,当德皇在1918年11月9日受到威胁时,他的犹豫不决是对德皇的背叛。他在临终之际与希特勒见面的故事,总能让我浮想联翩。

兴登堡拒绝希特勒来探望病情。但这阻止不了希特勒:不去将会影响他的声誉。希特勒强行去了,并得到了兴登堡的祝福。兴登堡一直无法原谅自己在16年前背叛德皇的事。他显然错把希特勒当成了德皇,他拍着希特勒的手,请希特勒给予原谅。

如果这些情况中有一小部分是真的,当真相暴露出来后,这个国家就会大乱。我不担心那老头的名声:他根本无法应付这个局面。我不相信他有能力做出什么错事,即使他用尽自己全部智慧做出来也一样。他在一战中的表现极为迟缓,这才没有让鲁登道夫的大胆迂回作战遭受失败。

霍夫曼(Hoffmann)将军曾经是兴登堡的助手,他的遗孀最近向我展示了他丈夫在1914年秋季写的信,当时德军正在向波兰的北部挺近。信是这样写的:“他(兴登堡)把大部分时间用于打猎,晚上才回到指挥部,让我宣读明天要发出的命令,然后说,‘天哪,小伙子,我做也不过如此!’贝特曼·霍尔威格\footnote{Bethmann Hollweg,当时的德国首相。}要来听取战略形势汇报。我们需要告诉兴登堡将军怎样说。他甚至不知道我军部队的位置。”

我要再次说明,我不想做死者的判官。兴登堡对自己的处境认识不清。他太老了,很可能是不愿意去克服困难。然而,整个国家竟然认同他的迂腐领导,这就另当别论了。德国的议会制度也有责任:只要这个国家认同这个政治体制,我们就只能容忍这个体制制造出的混淆、动乱、政治迫害。确实,德国现在需要一个主宰者。但我不认为我们应该受那个额头留着一缕马鬃的流浪汉来领导。

\section{1936年8月11日\ 希特勒印象}

在慕尼黑见到了弗兰肯贝格(Frankenberg),我们谈起了罗姆暴动的事。罗姆死的时候很勇敢,在对监狱的咖啡质量提出一番抱怨之后,他像一个战士那样慷慨赴死了。戈培尔\footnote{Joseph Paul Goebbels,纳粹的教育与宣传部部长。}和他手下那帮人散布的那个版本说罗姆躲在床下,这只不过是他们的另一个谎言而已,这是恶毒的、怯懦的,因为死人是无法站出来反驳他们的诽谤的。他们在撒谎方面颇有天分。总有一天他们会受报应的,谁也跑不了。

威利·施密德(Willi Schmid)也在罗姆暴动中被杀了,他是《慕尼黑日报》的音乐批评家——你可以说,他是因为疏忽而被杀的,因为杀手不幸认错了人。似乎纳粹是在电话簿里找他们想找的施密德,他们在误杀了一串“施密德”后,才杀死他们真正想杀的那个施密德。这就是他们常说的“宁可我负天下人,不可天下人负我”。72岁高龄的古斯塔夫·冯·卡尔(Gustav von Kahr)也被杀了,不是被枪毙的,而是被党卫军队员在马里昂巴德旅馆的院子里踩死了。

罗姆暴动这件事,非常奇怪,其意义也深远得难以预测;一旦真相有机会披露出来,肯定能让大家感到战栗……我听说希特勒本人在巴特威斯浴场发动了印第安人式的奇袭,亲手杀死了自己的政敌,而且其中有一个对手还进行反击。那人被气得大喊大叫,挥舞着手枪,沿着楼梯往楼下追希特勒,希特勒最后跑到地下室,躲到一扇铁门后面才幸免于难。我们这个新政权曾经有过如此悲剧的开始,真让人感到亲切,日后肯定有“好结果”!

我正在写一本有关16世纪再洗礼派教徒在明斯特(Munster)建立一座城市国家的书。同时,我也阅读了一些与我同时代的人对“天国”的描述,这些人的描述让我感到震惊。在所有方面,甚至于极其荒谬的细节方面,那个城市国家与现在的“天国”都很相似,以致让我感觉建设城市国家是我们持之以恒的目标。与现代德国一样,明斯特城市国家把自己与文明世界隔离开来;像纳粹德国一样,它在相当长的历史跨度里是极为成功的,似乎战无不胜。然后,突然有一天,出乎意料地,一件不起眼的小事就能让它崩溃……

就我们而言,那贫民窟里生出来的私生子,却变成了先知,他的反对派被瓦解了,世界对我们既惊奇又不解。在我们中(最近在贝希特斯加登发生了一件事,疯狂的妇女抓起那位漂亮的流浪汉脚下的沙土,吞下肚里),过度狂躁的妇女、教师、叛变的牧师、社会渣滓、外来人口竟然变成了这个政权的主要支持者。我不得不删除一些类比,否则我会变得更加危险。在明斯特,被一层薄薄的意识形态佐料覆盖着的是邪恶、虐待狂、不可理喻的权力欲,无论是谁,只要不接受新的教条,马上就会被处死。希特勒在罗姆暴动中扮演官方刽子手的角色,就如同明斯特的博克尔松\footnote{Bockelson,明斯特城市国家的国王,实行共妻制度,他一人就娶了15位妻子。他于1536年1月26日被处死,尸体放于笼中示众。}一样。与我们一样,博克尔松颁布了苛刻的法律,用以控制可怜的平民,但他和追随者们却不必遵守。博克尔松周围都是保镖,没有任何刺客能接近他。与我们一样,明斯特的街上有群众大会,人们必须“自发地做贡献”,反对的人就会被放逐。与我们一样,明斯特的群众被灌输了毒品:举办民间节日,建造毫无用途的建筑物,目的就是为了不让街上的群众有时间去思考。

跟纳粹的所作所为一样,明斯特派遣自己的“第五纵队”和说客去周围的国家搞破坏。事实上,明斯特主管宣传的杜森施努尔(Dusentschnur)走路就跟戈培尔一样地拐,历史是个笑话,花费了四百年的时间才培养出两个一模一样的的人:许多与我一样熟悉我们“撒谎部长”诡辩能力的人,都劝我不要把这点写进我的书里。于是,在中世纪和现代世界之间出现了一个强盗国家,他们用谎言作为建国的基础,威胁所有的现实秩序——我指的是德皇、贵族以及各种早就建立好的关系。设计这个国家的人是一群具有疯狂权力欲的暴徒。还有几个类似的例子有待观察。在1534年围攻明斯特的战役中,人们被迫吞自己的大便、吃自家的孩子。这样的事有可能再次发生。希特勒和他周围那些马屁精将不可避免地遇到相同的结局,就跟明斯特的博克尔松和克尼佩尔多林(Knipperdolling)一样。于是,在我们的大教堂上,哥特式的怪兽猥亵的后背伸到空中,这怪兽长着鹰钩鼻和有爪子的脚,代表着所有的丑恶;还有举着鞭子抽打基督的家伙,为的是维持法律,这不可避免地让人感到可怜……

现在假定所有这些隐藏在我们的潜意识中的东西,在清除疮疖的脓血时喷发出来会怎样?假定这底层世界里的东西被撒旦释放而喷发出来,那潘多拉盒子里的恶毒的精灵逃逸出来又将怎样?这难道不正是明斯特所发生的事吗?实际上,明斯特人在事发前和事发后都是很保守的。这是不是能解释德国目前发生的所有情况?从希特勒的统治开始的那一天起,不仅天气变得恶劣,夏天大雨不断,庄稼被毁,奇怪的爬行动物折磨着这个古老的地球,而且一些诸如你我、对错、好坏、上帝魔鬼这样重要的概念都出现了混淆,这是一场与明斯特的变故非常类似的严酷的、巨大得难以估量的宇宙大动乱,但平时守秩序、勤恳劳动、热爱美好生活的德国人,竟然接受了眼前的这一切。

我最近恰巧去了一趟明斯特,参加了官方举办的一次庆祝活动,如今这类活动天天都有。庆祝活动中,有吹大喇叭的,也有敲大鼓的。平时我都是住在火车站旁边的旅馆里,但这次那里没有空房间了,于是我在老城区找到一处可以睡觉的地方,对面就是一所学校,一队希特勒青年团员正好住在里面。

我看到他们中有个男孩子,放下背包,走近空旷的教室。他在教室里四处张望,最后看到了讲台背后墙上挂着的耶稣受难像。这时,我看到他稚嫩的脸因气愤而变了形。他上前把这个德意志大教堂和巴赫《马太受难曲》供奉着的神圣标志从墙上扯下来,扔到了窗户外面去了……

他嘴中还大声叫喊道:“滚吧,你这肮脏的犹太人!”

这是我亲眼看到的。在我认识的人中间,我听说了多起孩子在政治上谴责父母,并用斧子砍父母的事件。我不相信这些孩子天生邪恶:就拿昨天破坏耶稣像的孩子来说,昨天他可能还着迷于杜松树的神话传说或忠实的海因里希的故事,但到了今天,出于对他的那位蛊惑人心的主人的忠诚和担心,他竟然有了一副铁石心肠。

我沉陷在这个深渊中已经快五年的时间了。在这42个月中,我满腔仇恨,睡觉前内心里充满了仇恨,睡醒了心里仍然是仇恨。我感到自己是被一群歹毒的猴子包围的囚徒,这种感觉让我窒息。我绞尽脑汁也无法解答一个永恒的困惑,这个民族在几年前还专横地守护着自己的权利,可现在却能糊里糊涂地追随那个昨天的流浪汉,而且是不知羞耻地追随。

我最近见到了希特勒,地点是在泽布鲁克(Seebruck),他坐在一辆有装甲保护的轿车上悄悄地缓慢驶过,轿车的前面有骑摩托车的武装保镖提供进一步的保护:他脸上的肉像果冻,脸色是矿渣灰色,圆面孔长着一对忧郁的黑眼睛,就像两粒葡萄干一样。他的这张悲伤的、无聊的、跟偷来一样的脸庞,如果放在30年前最黑暗的威廉皇帝时代,根本无法当官。如果这张脸坐在大臣的椅子上,开口发布命令,不仅首相府里的高官不会听他的,就是看门人和清洁女工也不会听他的!

如今怎样?我听说希特勒最近否决了一份报告——这份报告是德国陆军司令官凯特尔(Keitel)提交的,但希特勒不满意,于是就把一个铜花瓶向将军的头部砸了过去。这难道不是一个落入令人感到耻辱的污水坑里的民族会做的事吗?“他们只能那样做,因为这是上帝的意志。”这是我在明斯特16世纪编年史中看到的。

我既不是个超自然主义者,也不是一个神秘主义者。我虽然有许多预感,但仍然是时代的小人物,但我坚信我看到的东西。然而,我心里有一个可怕的困惑,我不断回顾我所看到的,但结论都是一样的:

我所看到那个由一群马穆鲁克\footnote{Mameluke,1250—1517年间统治埃及的军事阶层,原为土耳其奴隶,1811年该阶层被摧毁。}守护着的人,跟撒旦一样的,有人形,但不是人。

他是个鬼怪故事中的人物。

我见过他几次——但不是参加他召开的会议。第一次是在1920年,地点在我朋友克莱门斯·冯·弗兰肯斯坦(Clemens von Franckenstein)的家里,他的家后来成为伦巴赫别墅。根据男管家的说法,来的人中有一个人强行到处乱闯,闹了足有一个小时。这个人就是希特勒。希特勒设法获得了一份邀请函,他假装对舞台设计感兴趣(克莱门斯是大戏院的总监)。希特勒很可能觉得舞台设计跟他的从前的职业室内装修是差不多的。

这是他第一次进这栋房子,他背着吉他,带着一顶邋遢的宽边帽子,手拿着马鞭。他身旁还有一只牧羊犬。从戏剧效果看,在冰冷的大理石墙壁和哥白林挂毯的映衬下,他就像一个牛仔,穿着皮马裤,带着马刺,坐在巴洛克式的祭坛的台阶上,旁边站着一匹小马。他坐着,好像是个服务生领班——那时他比较瘦,饥肠辘辘的样子——看到一位真男爵在面前,他似乎很有兴趣,但显得拘谨;有敬畏感,只敢把半个屁股坐在椅子上,但他的腰是直立的;无论主人怎样优雅地嘲笑他,他似乎并不介意,却像一只狗在啃生肉一样,贪婪地侧耳倾听着每个字。

最后,他开口说话了。他滔滔不绝地说开。他在说教。他对我们讲话就像一个军队里的牧师一样。我们一点都没有反驳他,甚至连一点不同意见都没有,但他对我们却怒吼起来。仆人以为我们受到了攻击,赶紧进来保护我们。

他走了后,我们默默地坐着,感到很困惑,一点都不高兴。大家的情绪是沮丧的,就好像乘坐火车时包厢里坐着一个神经病人一样。我们默默地坐了很长时间。最后,克莱门斯站了起来,把一扇巨大的窗户打开,让温暖的春风吹进屋里。并不是我们这位严厉的客人身上不干净,让屋里充满了异味。实际上,巴伐利亚乡下的房子里总是有异味。那股春风,实际上是赶走人们心中的压迫感。不是屋里的那个人身上肮脏让人们有压迫感,而是某种其他的东西:一头怪物携带的那种肮脏的本性。

我经常去慕尼黑军械库骑马,然后去罗温布劳凯乐吃饭:在那里,我第二次见到了希特勒。他这时不再担心会有谁来打扰他,所以没有像他在弗兰肯斯坦家那样连续地用马鞭抽打他的靴子。我一眼就看出他上次那种紧张情绪没有了,这使得他立即便能开始长篇的讲演。我骑马已经很累了,所以饿得想立即独自大吃一顿。然而,他却把他的那本政治书的所有陈词滥调都浇到我的头上。他把自己的教条和盘托出,我在这里就不累述了,未来的读者肯定会感谢我为你们节约了时间。这是因为这个小男人的狡诈本性,德国外交政策变成了一系列的合法行窃和一系列有领导的盗用、伪造、违约,而所有这一些都是为了让教师、官僚、速记员满意,这些人构成了这个政体的真正支持者和堡垒。在这些人眼里,这个小男人成了真正的“成吉思汗式”的政治家。

当他激昂地说话的时候,一律油乎乎的头发会垂落在他的脸上,看上去就跟骗子一样。我感觉他在本质是愚蠢的,与他的亲信巴本的愚蠢是一样的——这就如同在买卖马匹时使用政治手腕一样愚蠢。

但这不是我对希特勒的全部印象,他还给我留下过更深刻的印象。每当我想起这次见面,我都感到越发回味无穷,他是在我吃香肠这道菜和小牛排这道菜之间停止了布道,离开了我,他走的时候那副样子就像一个服务员拿了小费一样,而且跟他与兴登堡握手的照片里的样子是一样的——就跟服务员领班手抓着小费一样。

第三次见到希特勒是在法庭上,罪名是在一次政治集会上制造混乱:那时他的名声已经超出了慕尼黑的范围……我看到他走进旅馆,此时他是个名人了。在法庭上,他似乎在向一名负责审讯但级别很低的官员求情:那副样子就像入狱过几次的人一样。另有一次,他估计自己会被赶出旅馆,于是随着一名背影僵硬的看门人一起去找旅馆经理要求赊欠。

自从第一次见到他之后,他在20年的时间里飞黄腾达了,但我对他的印象却一直都没有改变过。事实上,无论过去还是现在,他丝毫没有一点自知之明,没有快乐可言,他恨自己。他的投机主义、他渴望成名的巨大野心、心比天高的虚荣心,都有一个共同的基础——他有一种强烈的欲望想掩盖内心因巨大的精神创伤而产生的痛苦。

还有一些细节可以披露。厄纳·汉夫施滕格尔(Erna Hanfstaengl)比我更了解希特勒,她说希特勒越来越怕鬼魂。她认为他是害怕被他谋杀的那些人的鬼魂,所以他不敢在一个地方住很长时间……与此很类似的还有一件事,希特勒每天晚上都要去他的私人电影室,要一部接着一部看6部电影……

这可能是真的。这些事加起来印证了我对希特勒的判断。我甚至认为不应该从道德角度去评判他——他连“大罪犯”的称号都不配。除了在报纸上,如果德国政府再建立起一个大广播电台,宣布他是最大的艺术家,设法满足他的无止境的虚荣,我相信他会变成一个无害的追梦者,根本不会想到要去放火烧毁整个世界。

不,我不认为他是个波吉亚式的人物(Borgia,曾经担任过瓦伦西亚大主教和枢机主教,是个令人恐惧的野心家、阴谋家,尼可罗·马基雅维利以他为原型写下传世名作《君主论》)。我认为希特勒是一个具有乱七八糟欲望但又受到压制的窝囊废,被历史所愚弄,被给予了像希腊民主领袖克里昂\footnote{Cleon,早年反对伯利克里受挫,伯利克里死后,他掌了大权,竭力纠集古希腊城邦反对斯巴达,却以失败告终。}一样的权力,在一段时间里让他掌控一下社会这台巨大的机器。我认为这个可怜的魔鬼,来自史特林柏\footnote{Strindberg,1849—1912,瑞典作家。}笔下的肮脏地狱,他与历史上的博克尔松一样,趁着民族的脓肿爆裂之际,代表着大众被禁锢着的黑暗欲望——他就像明斯特的前辈一样,是从德国神鬼故事中跑出来的人物。

后来,我有一次近距离见到了希特勒,时间是在1932年秋天,当时德国已经陷入狂热之中。我和弗里德里希·冯·慕克(Friedrich von Mucke)一起在慕尼黑的巴伐利亚饭店吃饭,这时希特勒走来进来,他走过餐厅,坐在我们旁边的桌子上——就他一个人,没像往常那样带着保镖。这位德国的大人物,坐在那里……他感到我们正在用挑剔的眼光观察他,他感到不舒服。他的脸色阴沉,就好像一个小官僚走进了一处他不该进的地方,但既然进来了,他就要获得公正的服务,他“要得到那些绅士们得到的服务……”

他坐在那里,样子像是个新涌现出来的成吉思汗,或是个滴酒不沾的亚历山大,或是个没有女人陪伴的拿破仑,或是个雕像般的俾斯麦。如果他吃一顿俾斯麦的早餐的话,他肯定要在床上躺四周的时间……

我是开车去镇子上的,因为在1932年9月的时候大街上已经安静得不太安全了。我身上有一只装着子弹的左轮枪。在那间几乎是空荡荡的餐厅里,我本可以轻松地射杀希特勒。如果能预见到他未来会发挥丑恶的作用,让我们受难那么多年,我肯定会二话不说就杀死他。但我认定他是个喜剧人物,所以没有开枪。

但我其实帮不了太多的忙:我们的最高殉难委员会已经做出了决定。如果希特勒在那个时刻乘坐火车,火车将会被出轨。当他死期到的时候,在任何地方的所有方向都会有取他性命的行动,甚至是他意想不到的地方。有许多要刺杀他的谣言。已经有许多企图失败了,而且未来还会有失败。这些年来上帝似乎睡着了(这是一片魔鬼猖獗的土地)。俄罗斯有一句格言:

\begin{verse}
如果上帝愿意,扫帚也能当枪用。
\end{verse}

\section{1937年5月\ 喧哗与躁动的柏林}

又有一桩政治丑闻传遍了整个德国。普奇·汉夫施滕格尔(Putzi Hanfstaengl)是慕尼黑著名出版家族的继承人,在失宠前是纳粹对外新闻宣传的主任。事情发生得很快。在2月的一个早晨,他登上一架去往西班牙的飞机,飞机在天上转了好几个大圈子,显然是想把他抛出机外,但没成功,于是迫降,迫降的地点在图林根州的森林里,当时是暴风雪天气,温度零下10多度,他本人仅穿着西装。回到柏林后,他发现自己的办公室——德国对外新闻办公室——被关闭了。英国驻德国大使埃里克·菲普斯(Eric Phipps)爵士,帮助他逃亡到了英格兰。这位英国大使在罗姆暴乱中曾为布伦宁(Brunning)和特雷维拉努斯(Treviranus)求情。

据说之所以要用如此不寻常的办法迫使汉夫施滕格尔辞职,是因为他大肆批评了德国对西班牙的外交政策。此外,他的电影公司侵犯了戈培尔的地盘。另有一个故事说他在巴黎的一间咖啡厅里喝醉了,有人听到他说图哈切夫斯基\footnote{Tukhachevsky,1936年6月,苏联进行“肃反”期间,图哈切夫斯基因被指控犯有“间谍和叛国罪”而遭处决。}与希姆莱(Himmler)之间的联系,这最终导致阴谋败露。如今他已经去了英格兰。无论详情到底如何,反正几周前,我和他在慕尼黑的女王餐厅一起聊过天,我认为他是个有礼貌、有教养的人。由于他知道的秘密比较多,特别是有关德国国会纵火案的事,柏林非常害怕他。汉夫施滕格尔的母亲已经有80岁了,但仍然被送到伦敦去把他带回来。她随身携带着德国政府的保证书,以及戈林的不追究责任的特别保证。

不追究责任?汉夫施滕格尔与德国的经济联系密切,他的财产都在德国,德国政府可以采取任何行动处置他的财产。因此他母亲去了伦敦,但儿子不愿继续玩游戏,并说他知道希特勒和戈林承诺的真正价值。这件有教育意义的小事仍然悬而未决。

我和安若·雷希贝格(Anro Rechberg)去普奇的姐姐家吃早餐。他姐姐厄纳救过希特勒,当年希特勒在统帅堂搞政变失败,就是藏在厄纳家里,所以被称为“第三帝国的女施主”。此时,这位女士正在生戈培尔的气,她跟他有私人恩怨。让她气愤的是一件旧事,人们只是知道个大概。

1933年秋天,她住在慕尼黑东部的一栋孤零零的别墅里,其位置在博根豪森(Bogenhausen)郊区的边缘。当她不在家的时候,似乎有人私自进入了这栋房子。她找希姆莱投诉这件事,但希姆莱告诉她,下命令的人地位极高,她无法获得赔偿。他还告诉她,行动的目的不是要拿走她的信件,而是她的性命。他拒绝继续介入这件事,催促她快点搬进城里住。她接受了建议。如今,她告诉我,下命令的高官是戈培尔先生。戈培尔下令搜查她的别墅,是为了找到希特勒写给她的信件,这些信件如果流传到海外,不利于戈培尔的主子希特勒本人。

这是个有趣的故事,因为这个故事说明我们的帝国大主管正努力向这位慷慨好施的妇人求爱:厄纳·汉夫施滕格尔,与希特勒不同,她是个典型的慕尼黑的巴伐利亚人。

所以,这就是我们今日的德国。

我们在厄纳·汉夫施滕格尔家,见到了一个年轻的英国女人,她是一个介于天使与肥皂广告模特之间的人物。她的名字叫尤妮蒂·米特福德(Unity Mitford),她在希特勒的度假地奥柏萨尔斯堡有固定的住处。她的使命是成为德国女王,以便在德国和英国之间进行媾和。

预祝这位伟大的妇人和希特勒一帆风顺。

就在这个时期,我来到了柏林——按照官方的说法,柏林是勤奋、运动、完美的中心。以我卑微的见识,我倒觉得柏林是一台巨大的机器,不停运作着,但什么都没有生产。

我不相信柏林的一切。我听说过柏林人能“手脚并用打电话”,还听说过他们未来三个月的“会议日程表”精确到每分钟的时间。我了解他们所谓的“不惜任何代价进行生产”的方式,那实在是一种绝望,因为他们追求的并非是真正的美国精神。他们把生活看作是一个大兵营,这让整个世界感到厌恶。只要这个国家仍然把这座城市看作是国家的标志,德国的外交政策就会是从一个失败走下另一个失败。

我不相信柏林人比其他人工作更努力。他们歇斯底里地制造动静,这也许反映了他们对自己内心的空虚缺乏认识。我认为他们就是在造假欺骗,把女招待监工变成总监,把后院的小房子变成花园的亭子,把一次如何欺骗顾客买汤粉的交谈变成一次“会议”。

我相信在柏林真正具有效率的是柏林东部有轨电车的司机、邮递员、货车司机。我信任出租车司机,当我告诉他我要去的地方,他会通知你大约的车费,并建议你去乘坐高架铁路,显露出普鲁士人传统的节俭精神……我不仅相信柏林脾气暴躁看门人的话,还相信写在挥舞长剑的冯·布吕歇尔\footnote{von Blucher,1742—1819,普鲁士元帅,曾经指挥过数次重大战役,以积极进攻的指挥风格著称。}的雕像基座上的那句幽默的话:

\begin{verse}
这个地窖里只能躺一个人。
\end{verse}

我不能接受的是这里在过去90年来形成的干腐……这些带着太阳镜的女人、她们像船帆一样宽大的后背、她们巨大的乳房、她们宁愿做男人玩物的态度……这些手拿着会议预约簿的总监们——总之,我无法接受的是这里的过于殷勤而显得繁忙的气氛,你看,无论是簿记员,或是专利员,或是彩票销售员,他们都拿着安装着大使馆专员用的三个锁的公文包,而手里却举着三片干瘪瘪的奶酪夹心面包。

柏林最有特点的东西就是欺诈:只求实现功能,不求质量有保障,不求运作平稳;机械学徒工们,刚刚学会精确的制造,就马上宣布已经成为了全面发展的发明家或建筑师:流线型的小汽车内配着假皮,交通信号灯都能亮,但相互之间没有联系,按照“新功能主义”的理念用高标混凝土建造桌子和床铺,这些新东西与过去“浪漫的”旧东西相比,其实极为不实用。再看看其他几样东西:

“发展中的经济”,垃圾被称为“半成品”;人造羊毛西装,既不暖和,也无法清洗;法本公司依靠可怕的生产工艺用硫磺和糖制造出的蛇毒,放在玻璃瓶中销售,就跟柏林西区随处可见餐厅里卖酒一样:这种东西看上去和闻起来,都跟真酒一样,有酒的黏性,有味道,但极为便宜,在一场酩酊大醉后很难清醒。

不,我不相信有多少城市会像柏林这样把时间花费在无用的官僚机构重组上。我不喜欢他们在反驳你的时候,总是先说一句开始于“但是,在另一方面”这个前缀。我不喜毫无目的的聊天,不喜欢他们的武断。

最近,我遇到一位名气好、能量大的电影编剧,他说,“我受邀去巴贝尔堡(Babelsberg)展示我的电影脚本,我看到一张绿色的桌子周围有七个老绅士,显然血压都高,面前的桌子上都摆着药丸瓶。这几个绅士对我的脚本都感着迷。当我们就要达成协议的时候,阴影里跳出来一个戴着牛角眼镜框的助理剧作家。这家伙非常清楚自己无足轻重,便鸡蛋里挑骨头,借以证明他配拿每个月300马克的工资。”

“那家伙说,脚本固然不错,但有几个场景有问题,比如有一个可能冒犯德国壁纸制造商协会,而另一个场景的对话,让没有上过中学的居民、公务员、速记员都无法理解。无论怎样解释本意,都说服不了对方。那几个毫无生气的老绅士也受到启发,开始证明自己也配拿更高的工资,于是每个人都开始绞尽脑汁说一段自己的‘但是,在另一方面……’在此后的几周时间里,又开了好几次烟雾缭绕的会议,还打了许多通电话,还进行好几次早餐会,接着又是几次会议。至此,巴贝尔堡摄影棚的所有作者都知道了几个脚本。最后,这个脚本简直变成了一个垃圾堆。新版的脚本,小心翼翼地避免所有自然的联想,偏好一种超级聪明的人工制品。根据那个‘不能把简单事办复杂’的原则,剧情变成是要飞向月亮。”

“最后,参与讨论的各方觉得应该去休假了,可脚本此时臃肿得快要崩溃了。于是,大家赶紧反思,终于发现了一个“简单、可信、完全令人满意的”方案——其实那就是原来的那个脚本。”

“事已至此,大家都相互道歉起来,还温柔地拍拍后背,或许还微微带着些尴尬。不幸的是时间被浪费在无用的讨论中了——供拍电影的时间一共是三个月,现在四周没有了,剩下几周必须疯狂地工作才行。”

这就是柏林吗?柏林在过去60年里难道就是按照这个原则运作的吗?柏林的工业、艺术、政治难道也是这个原则的产物吗?

最近有一位参谋官告诉了我1917年夏季他们在巴尔干前线的经历。“那是七月的一天,”他说,“我们面临非常紧张的形势,有的时候都不能肯定我们能坚守阵地。那天在早餐的几分钟时间里,我被叫去接电话:总参谋长来电话了。我听出是鲁登道夫的声音。话筒里的声音特别清晰,这让我很吃惊,因为距离很远,途径孚日山脉、多瑙河、莱茵河、巴尔干地区的山脉。尽管如此,我仍然听见那话筒里的声音问:‘你那边有草莓吗?’”

我确实不知道我们的上级这话意思。我有些疑惑,不知道他是在问我们节俭的早餐的菜单。或者他有什么其他意思。最后,在经历了一阵痛苦的困惑之后,我终于知道了他的意思。

他听说我们这块地方非常适合种植草莓。由于他很担心德国的经济,同时不让德国士兵闲着没有事情做,他想到可以让我们种草莓,而销售草莓的收入能增加德国的外汇收入。由于我们的作战压力很大,所以提出了反对意见,但没有用——他就是想要草莓。

他得到了他想要的。我们赶忙从前线撤下部队,去做种植工作。我们做这项工作时,内心充满了疑虑。为了弥补前线出现的缺口,我们花费了巨大的代价。我们按照他要求的面积种植草莓,第二年获得了好收成,他计划把草莓保存在柏林,然后卖到国外去。草莓的质量是一流的,但运到柏林后全腐烂了,发酵了,发霉了。草莓是通过繁忙的铁路运输的,铁路的运输负荷很大。最后,所有草莓都被丢弃了。

今天,我与告诉我上述消息的人一起在安哈耳特大街的意大利餐厅聊天,看到四个救世军的高级军官对餐厅的老板和侍者大吵大闹,这位老板就像威尔第歌剧中的一个人物不敢说话。由于刚签订的德意协定,所以他们的吼叫中不断出现“合作”这个意大利词,他们唯一知道的意大利词——与此同时,我身后又发生了另一起在声学效果上丝毫不弱于第一起的事件。两个柏林的资产阶级妇女吵起架来,原因是一个女人放在椅子背上的大衣给蹭掉了。就在她指责大衣是另一个女人故意给蹭掉大衣的时候,那几个勒斯侍者只是咧嘴笑着看。另一个女人尖叫道,“请原谅,夫人!我是个德国女人!”

这样的事柏林也出现了。

眼前的这一切就跟一架空转的磨盘无休无止地转动着,我精疲力竭地回到安哈耳特大街火车站附近的旅馆里。房子是战前大批量营造出来的,简直就是垃圾,墙壁跟我的手指一样薄。我住四层,夏天的高温烤得我难受,如果我在自己的房间里用比平时说话大一点的声音说,“太热了,”我敢肯定一层会有一个操着巴尔干口音的男中音用油滑的腔调说,“我这里很凉快。”

这就是发生在柏林的事。这座城市是按照公式和模型建造的。能在这里繁荣起来的只是数字、队列、公式、图案。可这样的繁荣,实际上就是令人厌恶的匮乏,它与形式简单无关,而是掩盖了低劣和愚蠢。弄虚作假是这片土地的座右铭。据说腓特烈大帝的精锐部队的士兵穿的不是真马甲,而是缝在衣服上的三角形红布。无论这个故事真假,我看类似的三角形红布到处都有,有大的,也有小的。好做表面文章在这个国家是根深蒂固的。为什么?因为我们有掠夺人家的强烈欲望——这才是贱人的根本特征。

“德国从来不知满足,既不注意形式,也没有品位,对幸福生活缺乏理解,德国人只有一个野心:再多得到一点。等获得的东西用都用不完时,就把东西放在旁边,谁来碰一碰,我们就大呼小叫!德国是陆地上的海盗,而且是一边掠夺,一边大唱感恩赞美歌。在这片土地上,永远不缺少能写在旗帜上的铭文。”

这是一拥抱莱茵联盟的知识分子的说法吧?这是巴伐利亚主义的宣言?不,这是特奥多尔·冯塔纳(Theodor Fontane)的《普鲁士之歌》,他被认为是柏林市民之一。我把他的歌引用在此。我也是个老普鲁士人,但我的母亲是奥地利人。

我想到了我家的历史。我爷爷(汉姆生说,“他是你爸爸可能像的人。”)是一个保守的、有文化的人,过着冥思苦想的生活。他阅读德国哲学家加尔夫\footnote{Garve,1742—1798,他曾给民意下的定义是:“民意,是一个国家的大多数公民,每人反省或实际了解某件事所得到的判断后,许多人的共识”。}、地理学家洪堡\footnote{Humboldt,1769—1859。}的书,在50岁时退休了,悠闲地把余生用于打猎和钓鱼。他是真正的保守派,一个真正的容克贵族——相当有教养,到各地去旅游,对所有夸大之词都怀有疑心——就是霍亨索伦家族人的嘴,他也会表示怀疑,他会像东普鲁士人一样嘲笑那个家族的人是“纽伦堡人”。

巨变发生普法战争那一代人中,他们取得了这场战争的伟大胜利,但不幸也跟着来了。普鲁士人跟工业和金融寡头形成了强大的联姻,于是这些寡头就能对政府施加从来没用过的强大影响。历史上,英格兰在维多利亚时代出现过这种情况,而法国发生在王朝复辟时期。这味药,被英格兰吸收了,没有出现什么副作用;但法国吃下这味药却受到了伤害;对德国来说,所受的伤害是致命的,因为德国的经济基础是田园诗一般的农业。1853年,俾斯麦站在1840年普法战争阵亡将士墓前,他甚至“无法原谅这些死去的人。”然而,他在凡尔赛镜厅胜利后的18年里,把民族自由主义变成德国的主流意识形态。所以,他在推动国家走向繁荣的过程中,也破坏了他亲手创立的这个国家的基础。

我最近读了比洛(Bulow)写的回忆录。在这本写得很粗犷的回忆录中,他指责俾斯麦的政策没有考虑地理局限,从而给德国带来了悲剧。德国的本质(斯宾格勒的概念)要求必须防止工业和资本的无限制扩张。自从普鲁士寡头娶了工业资本做姘妇后,德国便开始倒霉了,因为那些对健康的德国来说是必备的社会规矩全都被破坏了,德国因此变成了一个在政治上没有定式的国家。

从这时起,德国把政策的地理重心放在了海外,外交政策越来越强调出口市场。结果就是第一次世界大战,这场战争就是为“地理”而战……在此之前,人们对德国的状况的怀疑就没有缓和过,比如,在1840年前后,那一代人是在学生俱乐部和扬恩体育俱乐部(Turnverein Jahn)培养大的,把德国的所有精神传统都抛弃了……人们沉迷于梦想之中,特别是日耳曼人繁荣昌盛的梦想,并把梦想都寄托在一代人身上实现,从而造成自然资源无法挽回的破坏,其破坏程度前所未有,同时德国传统的文化和道德基础也被破坏了——取而代之的是一种股票经纪人的哲学,这种哲学在1860年代和1870年代变得相对流行,其结果是阻碍任何对未来的思考……

在德皇威廉的领导下,我们的社会灾难性地变成一个没有定式的东西:有学识的人去当赛车驾驶员,银行家去培养纯种牲口,骑兵上尉热衷于投资股票……混在大众之中,变得跟大众一样没脸没皮,而想要把鱼目混珠的大众纠集在一起,只能依靠物质主义的大旗……整个社会堕落到凄凉的穴居时代,我认为这预示着自罗马皇帝卡拉卡拉\footnote{Caracalla,186—217年,他为增加税收和兵源,给所有自由人以罗马公民权。}建立起的文明将会遭受某种破坏。希特勒鼓吹的无阶级社会是个没有四肢的动物。但我相信大自然在最开始就是有定式的,大自然最厌恶没有定式。

我是在柏林的一家旅馆里进行写作的,这家旅馆安静得就跟一门榴弹炮一样。此刻,楼下房间里住着一个妇女,名字可能是道林斯基,她是前面我说过的那类人,正在对着电话倾诉她的离婚细节。窗户大开,所有粗俗的细节就像被直接灌输到周围炽热、静止的空气中一样。最后,无论我是否愿意,我听到了为什么道林斯基要出乎意料地摆脱夫人的怀抱。我听到“游行”这个词,我记起来了,昨天我看到德国女青年联合会在城里游行,游行队伍里尽是罗圈腿和大屁股,奇丑无比,与丑陋的城市遥相呼应,丝毫没有一点喜庆的气氛,等于是在向“舒适和愉快的生活”宣战。

就在我思考着19世纪的巨变时,回忆着女人游行,我又想到了另外一件事。就在70年前,德国因富裕而变得愚蠢,竟然同意让普鲁士做组织者和拉皮条者,不仅狗受到牵连,更恶劣的是像道林斯基这样的局面也受到了牵连。普鲁士是靠一片片小土地拼凑而成的。普鲁士永远不想成为一个国家。为了维系这个庞然大物,普鲁士人把全部精力都放在了战争机器上……结果是德国没有中产阶级,没有贵族,没有真正的知识阶层。因此,在冯塔纳所描述的汪达尔人和卡舒布人混合而成的寡头政治集团消失后,立即就浮现出了完全非德国、完全殖民性质的元素,比如,当神圣罗马帝国在建造大教堂的时候,德国人仍然在肚脐附近刺上绿色的蜥蜴文身。易北河在德国历史上是一条具有重要意义的河流,有些鸟类和植物种就是不从左岸越过这条河到右岸去是很有道理的。在易北河与维斯瓦河之间,就是我们前面说到过的那些罗圈腿妇人的家乡;在这里繁衍的种族永远呼喊着要更多的东西,集聚着大量受压抑的攻击欲望,是所有撕毁条约、掠夺行径的策源地,希特勒在过去五年里将这些行径乔装改扮成为国家行为——当他把这些行径作为政治家的本性的时候,德国竟然没有一个人敢去质疑他。

这里出生的人,档次差,偏好短期行为。这种偏好导致造假成风,普鲁士国王的巴洛克式宫殿,正面是在泥巴上镀金建成的,这样的要求被认为是实在的、权威的、完全合理的,因为国王有枪做支持。这里永远都有人在喊,“我要更多!我要更多!”。这种偏好导致造假成风,普鲁士国王的巴洛克式宫殿,正面是在泥巴上镀金建成的,这样的要求被认为是合理合法的,因为国王有枪做支持。兴登堡,被穷鬼奉若神明,社会上有一股巨大的丑陋的崇拜欲,他的雕像竟然建造得比国王广场的树木还要高。在这里,不仅嫉妒他人的才华,还嫉妒他人的财产,突然就能爆发盗窃行为,于是普鲁士人对穷山恶水出刁民的崇拜,竟然成为了德国的国教,而且还准备推广到全世界去。如果没人拿着枪反对他们,他们准会去做的。

我想起了那个下士给我讲的故事。他在教堂的门口总是让士兵别坐在那里“浪费时间”,赶紧走个过场出来(从祭台,经风琴,走过牧师,然后走出大门!),这样可以把时间用在其他有用的地方。只要极为唯利是图的国王仍然“以我家族的荣耀”使用国家的军队,这位下士的举止就是能被容忍的。为了发展经济,拿着法本公司制造的武器,穿着人造纤维长袜子和假外套去打仗,在我们这个富裕的世界里,这简直就是令人憎恨的灾害。德国变丑了,变恶毒了,变成了一个灾难中心,每25年来一次,起点就是俾斯麦建立的德国,这个国家是普鲁士的殖民地。

在这里,我触及到影响今日欧洲政治的核心问题。在凡尔赛宫,普鲁士的寡头们虽然知道自己对奥地利的前途负有责任,但仍然犯了一个令人难以置信的愚蠢错误,他们肢解了奥地利。可是,如今这些人已经消失在历史中了。此后要做的是普鲁士人带着雇佣兵去执法——这将会是一场灾难,我们知道这场灾难肯定会降临到我们头上的。

\section{1937年9月9日\ 理性与自由的终结}

很可能是因为有人告发,盖世太保突然出现在神学家特奥多尔·哈克(Theodor Hacker)的家里,哈克保存着一份日记手稿,盖世太保正是为搜查这份手稿而来。一个盖世太保拿起了那份手稿,但恰巧另一个盖世太保提出一个问题,拿起手稿的人的注意力被分散了,没有看手稿便放下了。可怜的哈克——他原本就不是一个意志坚强的人——很可能分分秒秒都在为自己的脑袋能否保全而忧心忡忡。

我的几个朋友借机警告我要小心写作。我的写作全出自我的内心需要,不能停止,所以我只能漠视警告,继续写日记,我希望我的日记对记录纳粹时代的历史会有帮助。一夜又一夜,我把这些记录藏在我家树林的深处……我保持着警惕,以防有人监视。我还不断转移藏匿日记的地方。

这就是我们目前的生存状态。我有几个朋友,他们大约四年前离开了德国。我的这几位已经离开的朋友啊,你们是否知道,我们现在是完全没有法律保护,随时都有被疯子们指控的危险。

想起你们就觉得奇怪,因为我只能穿越太空和横渡大海才能听到你们的声音,你们的声音来自那个至今已经被禁止接触的世界!很奇怪,我又来到几年前我们谈话的地方。我想念你们,即使你们中的大部分是我在政治上的敌人,我也想念你们——哦,请相信我,这里不许表达任何不同意见,必须绝对一致,真无法容忍这里的生活。

不过,如果你们回来,当我们再次相会时,你们将不再认识我们。不知道你们是否能理解,逃到文明世界比较容易,难的是留在这个危险的边区村落对野蛮人进行非法的观察。你能理解我们的处境吗?在这漫长的几年中,我们的内心充满了仇恨,躺在床上恨,站起来还恨,在漫长的噩梦中恨——虽然我们眉梢上挂着非法的仇恨,但我们没有法定权利,没有任何妥协的余地,开会时每个人都必须喊“希特勒万岁”。等你们回来后,我们还能有共同语言吗?这些年来,你们一直生活在文明的环境里,你们能理解我们过的这种像死一般的孤独生活吗?你们能理解我们由于生活在这充满了悲哀气氛的地下墓穴而目光深邃吗?或许,等你们回来时,会不会在远处就被我们眼睛中喷发出的目光吓坏?

按照1789年的理念建立的世界是怎么一个样子?——那个世界包围着你,仍然是你生活和思想的基础,这是不言自明的道路,就如同螃蟹有保护壳一样,对不对?请理解我:我们知道那段历史——大百科全书运动是个历史过程,这个运动的起点是文艺复兴,文艺复兴让人民不信神而去信别的东西——信神曾经是不可或缺的生活方式。请不要冤枉我,我不是想厚古薄今,或患有身陷大灾变的幻觉。然而,难道我们现在不是正在经历1789年的最后一个阶段吗?法国资产阶级在1790年就一边呼喊着“国家万岁”,一边暗中夺取国王留下的权力遗产,难道他们不是最不稳定的现象吗?巴尔扎克曾经说过,“总有一天资产阶级会听到自己《费加罗的婚礼》”,难道他不是预见到了俄罗斯和德国的悲剧了吗?圣茹斯特\footnote{St. Just,法国大革命中涌现出的著名政治家,热月政变中试图为罗伯斯庇尔辩护,被投入监狱,1794年7月28日被处死。}不是在很久之前就预言未来会出现极权主义国家吗?难道不能说吉伦特主义\footnote{Girondism,法国大革命时期吉伦特党的主张,代表工商业大资产阶级利益。}在克虏伯、福格勒、劳士领身上得到了最终的体现吗?他们抛弃了所有规矩,霸占了德国的核心,成为德国社会的焦点——他们是军事化了的吉伦特主义,丝毫不要道德规范,甘愿做信仰的敌人,虽然在战场遭遇惨败,但在意识形态取得了滑铁卢战役一样的胜利。

就纳粹而言,我肯定大家都同意一个观点,纳粹是国家的最大破坏者,这个破坏者总是倾向于采取神秘手段。你们这些我的老朋友,也许会反驳我,但我仍然要说,在近1500年以来,德国一直都没有国家主义,但今天不同了,德国有了国家主义,却失去了国家,因为我们一看到每条裤子上的扣子都印着“德国制造”字样时,我们就会露出喜色。我们都同意,由于全民都处于沮丧之中,所以希特勒被蒂森\footnote{Thyssen,19世纪末经营钢铁业发家,被称为“鲁尔之王”。}先生及其朋友扶上台,形成了一个由财阀控制的政府,绝望地想把19世纪延长……

哦,等我们见面时,在否定德国的现状方面,我们肯定不缺少共识!然而,当我们谈论未来时,我们还能有那么多共识吗?由于你来自一个有着现实基础的文明社会,你可能不会理解或者会断然拒绝我们现在看到的现实:

希特勒搞的那套东西,仅是一个症状,代表对世界的一次滑稽的深刻干扰;德国持续了五个世纪的理性主义和自由思想到此结束了;在人类占据的土地上,出现了一种非理性的新因素。

在我们之前殉难的人能排成长长的一队,我们如何能漠视世界危机的征兆……漠视写在墙上被你认为无法抹杀的人类理性?严密科学的基础发生了动摇,这难道仅仅是个偶然吗?这就是为什么万有引力仅是在“广义物理”中是正确的原因?依靠最新光速测量,天体物理学家怎么能把那个昨天还仅是一个小天体的地球,突然变成了宇宙的中心?当哲学家们发现五个世纪后的崩溃即将发生,竟然卑鄙地开始讨论如果真崩溃了,什么东西会在这次毁灭中遗留下来?

我认为人类的精神正在经历着伟大的演化,这会影响地球上的物质生活,如果地球上的居民的道德情操继续恶化,地球就将会被破坏得无法生活,最后将会在某种宇宙的大灾难中破碎。我所看到的不能算是宇宙的大事,而是历史大事:难以计数的人们以及他们的思想将会不可避免遭受一场灭顶之灾,这场大灾难正在酝酿之中,我已经在地平线上看到了大灾难的恐怖和希望。

这场彻底的大崩溃给人的感受呈病毒性扩散,其意义是不言而喻的。这种感受就好像一场巨大的风暴到来前,人内心产生的神秘、恐惧和震颤感。对这群毫无精神内涵的人来说,还能意味着什么呢?我们生活在巨大的精神真空之中。任何时候,如果人们意识到这种精神真空和可怕的混乱局面,都是会引发大灾难的。因为有精神需要和物质需要,大众只能自甘堕落,挖洞穴躲起来。这对人来说是必须的,就如同猪必须有泥巴一样。但如果明天他们脆弱的蚕茧被风刮跑了怎么办?

我毫不怀疑当代的“罗马皇帝们”生活在类似与罗马帝国相似的精神衰落中。我非常清楚,这些罗马帝国的可怜后裔根本不了解他们所陷入的衰落是无以复加的。在纳粹的炸弹像雨点般落在苦难的西班牙之后,我再次阅读了里尔克\footnote{Rilke,1875—1926,著名德语诗人。}和斯特凡·格奥尔格\footnote{Stefan George,1868—1933,著名德语诗人。}的诗作……当我把他俩的诗作放下时,我知道我已厌倦他们的作品了,因为他们的作品在我们这几年呼吸的空气中霉变了。我认为,虽然里尔克是个诚实人,能深深地打动人,但他的诗令人疲倦,在形式上走到了尽头;格奥尔格是一场世界性大火中的红色火苗,但显得有点装腔作势。

现在,如果有艺术家说他能写弦乐四重奏作品,或者说能建造出一座不亵渎上帝的大教堂,难道他不是个骗子吗?作为艺术家,我们难道不是站在墙前等着一只看不见的手来结束我们的性命吗?这就是陀思妥耶夫斯基所谓的“世界末日,”他是在70年前的日记中写下的。如今,一群预示着世界末日的骑手闪电般地冲向我们,这难道不是预示我们会怎样彻底失败吗?

不,我不相信地球会在千禧年毁灭。我热情地相信生命有自我恢复能力,我所预见的灾难仅是这个星球所经历的许多次灾难中微不足道的一次。然而,我有个信念,自文艺复兴时代起,生活就对人的身体状态产生影响,这种影响在过去几年来完成了,致使人的肉体和精神越来越不平衡——没有这种平衡,人在地球上就无法生活。

有几个在去年那场神圣的奥林匹克运动会上护理运动员的医生告诉我,在极度热衷于运动的这一代年轻人中,女孩子会有月经紊乱问题,而男孩子虽看上去勇猛无比但都会有性功能失调问题(不仅是冠军,而是普通运动员也一样)。没有什么能比这个例子能更好地说明精神状况对身体健康的影响。汽油是摩登时代幸福生活的基础,但它却比酒精对人类的腐蚀更甚。

所谓的“大众”由具体的人构成,他可能是穿军装的将军,也可能是大学教授,更有可能是为数众多的车床操作员。大众的数目在爆炸性的增长中,他们的增长破坏自然增长的规律,这种增长从生物学上看是不稳定的,以至只有癌细胞的增长能与之媲美——这样的增长在地球上曾经上演过寿终正寝的一幕。在大约两百年的时间里,繁荣的罗马萎缩成了一个乡下小镇,宏伟的罗马广场几乎淹没在麦田里。

生活方式发生巨变令古人痛苦不堪,但对现今的大众影响不大,因为他们基本上依靠技术和机械化生存。然而,地球上速记员如今坐得满地都是……大量人浮于事的官僚用分发完全没有用途的调查表的方式去瘫痪那些仍然有产出的生产领域——他们绝对不会住手的,除非原来的市场国家自己也建立起“本国的工业”,使得欧洲无法继续输出多余的产品;到那时他们像兔子一样的繁衍方式就无法维持了。

奥特加·伊·加塞特\footnote{Ortegay Gasset,1883—1955,西班牙著名作家。}仔细观察了平淡的现实,他发现今日的年轻人把无线电和电动机看作是天经地义的事,这表明这些年轻人已经脱离了现实。他巧妙地引用了科学家魏尔(Weyl)的名言来说明这点。魏尔曾经说过,“只需要一代人的冷漠,就能破坏技术赖以生存的学术气氛。”

种种迹象标明,一个伟大的文明正在走向末路,其最有活力的时期早就结束了,原因是像蚂蚁一样活着的大众根本没有理性,这威胁到了技术存在的基础。难道我们还能继续漠视这个事实吗?

大众购买技术产品时,根本对技术不抱有什么敬意,觉得自己是局外人,对技术中包含的理性劳动丝毫不感兴趣——他们是罗马帝国卡拉卡拉\footnote{拉丁语:Caracalla,公元211年—217年任罗马皇帝。颁布安托尼努斯敕令,让罗马公民权赋予全体罗马人民。——编者注}皇帝时代的近亲,他们知道罗马时代是舒适的,但懒惰得不愿去遏止这个时代的分崩离析。

我不相信大众知道自己对技术产品的依赖程度。我认为,当世界走向灭亡的过程开始的时候,他们真想知道的其实是政府安排的德国和瑞典之间的下一次星期日足球比赛是否能按时举行。在我看来,他们的命运已经确定,不可避免。马上就要来的第二次世界大战就是结束一切的开始:理性主义占主导地位的时代结束了,这个时代的遗产是那个这个星球能自我更新的假说。接下来是X时代,这个时代有一种新生活模式,但就是没有理性。

这样一来,大众群体知道他们的命数已定,于是他们便会去消灭不随波逐流的人。在德国,希特勒的政权就是想维持这种没有不同意见的大众群体,有待被消灭的目标是胆敢坚持原则说“不”的人,这些人给这个政权带来的伤害远较张伯伦软弱的无休止的绥靖政策所带来的要大。我们这一小群人会牺牲,但我坚信我们的牺牲能换取精神的复活。我们在地球上的余生将会是残破的,受虐待的,但我们除了死,什么也不希望有。

我写下这样的文字,心里绝非没有恐惧。我知道说大话的人总有一天要偿还……

但我们无法回到我们昨天共享的生活中去了。当你回来时,你会把令人诱惑的生活展示在我们面前。我们受了太多的苦,不能不相信所有通向美好生活的道路都充满了艰辛。地狱之门在我们眼前打开,但并非没有目的,无论谁见过这地狱之门,都再也回不到现实中美好的交际酒会中去了。前面,我描述了一个希特勒青年团员把十字架丢到大街上,并大喊大叫,“滚吧,你这肮脏的犹太人!”我还讲了有关希特勒的事,描绘他如何在贝希特斯加登(Berchtesgaden)向聚集在一起的暴民展示自己的情况。此后,一些神魂颠倒的女人吞食希特勒曾经践踏过的沙土……哦,这实在是太耻辱了,这些人根本不是传说中的有精神闪光点、动作优雅的反对基督者,而是畜生的粪便,在所有方面都类似于中产阶级的反对基督者……

哦,人可不能沉沦到这般地步。这群暴民,我与他们的联系仅仅是有相同的国籍,他们不仅根本不知道自己已经堕落,还准备随时要求其他人跟他们一样吼叫、一道吞食沙土、一起退化。

回到家里,我打开陀思妥耶夫斯基的书,他的书在德国的受禁程度无人可比。我再次阅读《群魔》中的人物——将军妻子的儿子彼得·斯捷潘诺维奇所说的话:

\begin{quote}
所有人都成了奴隶,(他们)受奴役的程度相差无几。

每个人属于众人,众人属于每个人,这样就实现了伟大的平等。

为此,首先要降低教育、科学、智力的水平。

只有大知识分子才能有较高的教育水平和科学水平,所以他们没有用。

大知识分子总想攫取权力,然后当暴君。

他们一心只想着当暴君。他们破坏多,建设少。

应该流放他们,处死他们。

西塞罗的舌头要被割掉,哥白尼的眼睛要被挖掉,用石头把莎士比亚砸死。

打倒文化。我们已经有足够多的学者了。纪律是第一位的。

通向知识之路是贵族之路,所以我们必须破坏它;

我们要酗酒、诽谤、监视;我们要把天才掐死在摇篮里。

我们都要降低身份去做分母!

彻底的平等,绝对的服从,绝对不许有个人风格,教皇在上,我们围着他,我们下面是地狱!

……

一或两代人的道德堕落是必不可少的;这场堕落是巨大的、卑鄙的,所有人都会变成恶心的、残酷的、自私的爬虫。

世界将会遭受一次从来没有见到过的颠覆。

俄罗斯将被笼罩在黑暗中,世界则为自己昔日的圣灵哭泣……
\end{quote}

陀思妥耶夫斯基是正确的,世界的末日就要到来了。这个令人悲愤、充满了诅咒的旧世界就要结束了。

\section{1937年9月9日\ 纳粹“革命分子”}

作为巴伐利亚王室的客人,我在霍亨施旺高堡(Hohenschwangau Castle)住了几天。夜深了,在与主人进行了一番长谈后,我试着走回客人住的那一侧。走廊非常曲折,就跟迷宫一样,我又不熟悉环境,连走廊灯的开关都找不到,于是只好在楼梯上坐下来,一边与寒冷聊天,一边等着天亮。

主人给我讲了各种各样的故事,这些故事对我来说似乎非常遥远:有一个可以放三支哈瓦那雪茄的烟斗,是俾斯麦曾经用过的——因为他想抽一次有三倍的烟量;在关键的1888年里,他曾与老德皇一起吃过早餐,那时皇帝的胃口极佳。最后,他告诉我,在第一次世界大战期间,形势极为艰难,令人沮丧,他当时是集团军司令官,他的集团军的后备部队只剩下半个连的兵力,可供使用的汽油只有1200升。

最后,主人让我看《柏林画报》刊登的一幅戈林的照片。照片是在戈林的书房里拍的,看上去他是一个拥有幸福家庭的男人,妻子是前索尼曼选美冠军。这一家人站在一块哥白林挂毯前面,这块挂毯曾经是维特尔斯巴赫(Wittelsbach)公爵的收藏品,但被随手偷走了,可能一起偷走的还有其他物品:照片中房子男主人手指上戴着的巨大钻戒、女主人戴着项链和耳环。我们谈起了这位品味高雅的艺术爱好者的家谱:他出生在罗森海姆(Rosenheim),母亲是女仆。他想进巴伐利亚的军校,但没有合格——他的申请书被送到皇储手中——于是被送往普鲁士。如今,在穿上一身奇怪的军装之后,戈林先生的家谱变了,他成为中世纪威斯特伐利亚一位将军的后代。他显然精神错乱了,觉得自己就是普鲁士的国王。我有个熟人,他最近经常见到克里娜赫(Karinhall),在这位前索尼曼选美冠军的潜泳伙伴们住的房间的门上,都挂着瓷制的名牌,上面写着“未来第一夫人”、“未来第二夫人”等等。

但这就是他们处世的方式。他们现在是统治阶级,个个都拥有奇怪的家谱,都从德国北方的贵族中挑选“副官”,这些副官与古代围着博克尔松的那些人是同一类人。

戈林先生正式要求妻子按照贵妇的标准穿着打扮。戈培尔也有供他表示敬意的祖先,他的这些祖先是德国中部一些龌龊的王侯。希姆莱这个人,总是努力地维持一种简单的生活方式,可在他的随行人员中也包括一名皇室跟班。最糟糕的是女人们,这些过去的女招待,大多数都跟过好几个男人,虽然戴着大量从皇族偷来的珠宝,但仍然去不掉身上的厨房气味。她们的行为既像电影明星,又像妓女,玩弄起了宫廷阴谋:“戈培尔夫人,你丈夫按规定只能有两辆公车,但怎么能动用三辆呢?”

但他们的为人就是这样的。他们这些革命分子都是肮脏的小资产阶级,他们矫揉造作,无法摆脱昨天带着狗链子的感觉,如今他们坐在被赶走的主人的餐桌上,在快燃烧尽的蜡烛旁,吃着餐桌上的残羹剩饭。

在回家的路上,我又听到了一条新的丑闻。纳粹上台的第一年就宣称,作为1789年哲学的自然延展,决斗是男人的天赋权利,并高调宣布国家批准各阶层都能以决斗来消除不同意见。如果主人和仆人因皮鞋没有擦好而产生不同意见,也能用左轮枪来了结。但在这份新的特许状下,第一个倒下的是纳粹自己的人——根据古代的法律看,这不是最糟糕的。罗兰·施特龙克先生是个有才华的记者,就我所知,他是个好人,有品位。

有一天,施特龙克发现一名愚蠢的年轻党员,按照纳粹“纵欲”的指示,竟然与自己的女儿睡觉。他把那家伙叫出来杀死了。这件事之后,决斗的规定被废除了。如今,仅是提出决斗挑战,都要受到惩罚,被判很重的徒刑。司机擦不干净车,不能再决斗了,只能换人。

对这项政策,我与弗兰肯斯坦之间有不同意见,他到过很多地方,他是个怀疑论者。我刚读完了《诽谤》这本英文小说,内容是两个英国骑兵军官因争执而寻找报仇机会的故事,他俩先是用纸牌赌博,然后是拳击,最后是上法庭;玛丽·博登(Mary Borden)在没有进行宣判的情况,就隐约地告诉了她的非英语读者这两个人的性格是怎样的。

我如今虽并不提倡那个著名的给学生放血的愚蠢建议,但也不愿否认一个事实,自1918年官方禁止决斗后,所有的荣誉标准都被充公了,荣誉变成了一项无规的博弈。诽谤者可以无所顾忌,因为不会像过去那样能遇到严重后果——这种情况并非起始于纳粹。请不要拿古代的理由出来说服我,因为那基本上是谎话。让诽谤者上法庭实在是太小的惩罚——此外,当诽谤案涉及家庭名誉和私生活,决斗绝对是更加人性的解决方式,因为不会把家庭的内情透露给报纸,不会在街上惹来风言风语。近30年来,我们看到欧洲人衰颓得很厉害。我们必须阻止这个趋势。

我回家要路过慕尼黑,自从普鲁士人占领了这座城市后,我就尽量躲开——这座城市曾经是何等的欢乐和壮丽!不久之前,这里还是像处女地一样的牧场,到处都是充满田园风味的宁静气象,这在全德国都是独一无二的,但如今却被毁了,到处是沙土堆,森林被砍伐,火车在城市中间疾驰而过,市内建有巨大工厂。这些都是德军总参谋部干的,他们简直就是典型的野蛮人,不理解有些东西失去了就不会再有。我现在已经认不出过去的那个欢乐的、高雅的城市,过去生活在其中的人都那么的年轻和幸福;它从来不是一座大城市,而是一块供农夫和农场主生活的土地。如今,到处能看见大屁股女人、普鲁士官僚的妻子,她们推着婴儿车,从路德维希大街上的佛罗伦萨宫前通过。在旅馆的走廊里,到处能看到刚被提升为军官的军士,穿着令人厌恶的的长统靴站在房间的门前。那宫廷剧院,在弗兰肯斯坦离开后三年,已经堕落为一个四级的旅游公司,里面挤满了一群群金发女人(德国少女联盟)。各家旅馆里都挤满了从德国北部来的企业经理的夫人们,这些娘们儿是搬运工痛苦的根源,她们总是把集合点选在有纳粹匪徒的地方。不,我不忍再直视这座被普鲁士的野蛮破坏了的城市,等这座城市终见天日时,我才会回来。

在那个过去很隐蔽的王宫花园里,希特勒异想天开,他要建造一座“最大的歌剧厅”,并把拱廊里的罗特曼的壁画拿掉了,换上了画家齐格勒\footnote{Adolf Ziegler,1933年被纳粹授予慕尼黑研究院的艺术教授,后来被任命为帝国艺术委员会的主席。}的作品。希特勒让齐格勒去清除“德国艺术中的颓废”。他的地位类似于德国画家的首领:他是个不想给自己退路的人。他拥有“女性阴毛大师”的称号,这个称号是他的同事送给他的,因为他对那类事物的表现情有独钟。

在慕尼黑,普鲁士化正在有条不紊地进行,人们对此的感情基本上是高昂的,这在30年前的摄政时期是无法想象的。豪森区(Haidhausen)吉辛区(Giesing)这两个慕尼黑外围的城区,与英国伦敦的怀特查佩尔区类似,布满了欢乐的小巷,此时变得不安全起来,一群青少年举着“红锚”旗帜,向穿着纳粹制服的人发动恐怖袭击。由于持德国北方口音的人不会被人骂,他可以穿着皮衣和高帽子走过吉辛区,而不会受到“红锚”分子的骚扰——他们只对穿纳粹制服的人发动攻击,特别是党卫军成员。据说“红锚”涉嫌几起谋杀案,所以当作是无害的粗暴行为被取缔了。警察发现了几具党卫军成员的尸体,认为跟讲义气的兄弟会有关,但实际情况很有趣,这一伙人都是反纳粹的年轻人,他们被迫加入希特勒青年团,如今扮演明暗两种角色。这件事的背景几乎近于不可理解,很像芝加哥的黑帮,组织者被认为是慕尼黑的一名律师——这样的事竟然发生在我们这个快活的、相当幽默的城市里,在20年或30年前,这座城市还是由年长的元老统治着!如今的德国,一个魔鬼摆脱了束缚——唉,我们之中没有一个人知道如何才能再次把这只魔鬼拴住。

\section{1938年3月20日\ 奥地利,绥靖与冷漠}

现在轮到奥地利了。

我们几周前就预见到要出事。很自然,我们知道这意味着什么,它意味着危险和有预谋的暴乱……这场卑鄙的表演就是为了干预找借口。如今,每条街道上都有坦克车队和炮兵车队在行进,指挥官都是激动不已的年轻党卫军军官。希特勒青年团里的那些半大的孩子,争着要当英雄,纷纷参加军队,可敌人是个欧洲大国,有700万人,不算是个小国家。

这是一场大规模的野蛮入侵,有人看到奥地利领导人性命不保而流露出肮脏的满足,看到侮辱和掠夺而极度兴奋——我认为这件事很不光彩,我深深地感到羞愧……

可怜的奥地利被愚弄了,强大的普鲁士发动攻击最终能找到的唯一罪名就是:奥地利能使人回忆起古老且高贵的神圣罗马帝国。

我在萨尔茨堡已经待了几天了,如今街上到处都能看到从柏林来的长着土豆脸的女胖子。感谢他们的好汇率,他们唱着歌买光了我们所有的东西,包括那些在柏林无法买到的东西,我们的货架空空如也。他们就好像是一群等主人出远门后寻欢作乐的仆人,在找到了酒窖的钥匙后,跟自己的女人狂欢起来……

\begin{verse}
主人离开了家

让我们把剩下的酒喝光

请给我一个吻

再吻我一个

生活啊生活

这才叫生活
\end{verse}
 

她们唱的就是这类歌曲。这时,正好来了一大群德国少女联盟的女成员,她们正心醉神迷地向古老的大街上驶过的坦克车队挥手致意。下一期的《柏林画报》一定会登出一幅照片,并配上文字说,“当地居民正热烈欢迎德国解放者……”我们都知道戈培尔的编剧才能,这个瘸子从前是个卖男装的销售员。

我听说许士尼格\footnote{Kurt Schuschnigg,奥地利总理。}被关在一间肮脏不堪的监牢里,并受到不公正待遇。那些胆敢坚守阵地的人下场都很悲惨,纳粹对此很是满意。

很难想象作家不来参加这场宏大的酒神节,纳粹的作家肯定会来。布鲁诺·布雷姆(Bruno Brehm)先生来了,还带着秘密印刷的小册子,借以庆祝这件大事。在这本小册子里,这个叛徒竟然写了一首赞美希特勒的诗,赞扬希特勒圆了德意志帝国的梦想。

但德国北部的报纸有更加拙劣的表现,他们说“奥地利回归德意志帝国”——普鲁士人有权做霍亨索伦王朝和哈布斯堡王朝的合法继承人……这就如同一个发了家的养猪人,娶了一个倒了霉的大户人家的女儿,于是他就可以宣称自己是直系继承人,有权继承这家人的盾牌。

我的表兄L先生是少将,参加了上述政治盗窃行动,当我跟他谈起这件事时,他无法理解我为什么没有狂喜得眼睛发光。我问他,如果像老毛奇那样有教养的人接到这样的进攻命令,他是否会立即辞职。这些普鲁士军官,身为历史上伟大将领的继承者,实在太让人害怕,令人不敢相信,他们丝毫没有意识到他们所扮演的卑贱角色。就是因为看到了他们的荣誉感已经被抹杀了,道德有缺陷,否则正确与错误之间存在神圣的界限,我这才断定德国精神堕入了可耻的深渊之中。

我听说了几个能够深深打动人的故事。有奥地利军官自杀了。镇守布雷根茨(Bregenz)的部队在明显不利的情况下仍然与入侵者展开斗争。镇守萨尔茨堡的赖纳团,具有悠久的历史,看到祖国受到侮辱,士兵们从城堡的窗户中纵身跃下。为什么许士尼格面对这样的好机会没有下令开枪?这时下令开枪,能让世界从令人费解的死气沉沉中苏醒过来。这样卑鄙地蹂躏一个小国,邻近的国家竟然冷眼旁观,耸一耸肩。没有人在最后时刻站出来阻止这样的暴行。似乎人们故意等着看眼镜蛇出洞。

然而,我敢预言这些国家最终会为自己的懦弱被动而后悔。他们将付出的代价是不可估量的;他们总有一天要付出代价。这是对和平的巨大破坏,但罪犯竟然逍遥法外,似乎变得比从前更加强大,而我们这些在德国内部的反抗者,因此而变得更加弱小无力。

我们和跟我们有同样思考的人,会不会死在纳粹的机关枪下?而且其中可能还包括奥地利人的机关枪。如果真是这样,我们就要感谢各国政府的冷漠了。虽然第二次世界大战已经爆发,但我仍然要再次问那个问题。五年前,在纳粹夺权的时候,欧洲国家能采取行动,哪怕只是警察采取行动,就能把纳粹这帮坏蛋揪着领子投入监狱。

但各国政府实际上是怎样做的呢?他们袖手旁观,这使得德国内部无法进行任何抵抗。他们现在干什么?他们仍然在等闲旁观,忙着思考如何不惹希特勒先生生气——这就使得进行抵抗的希望更加渺茫。未来,他们还可以做几件事:惩罚那些用肮脏的交易使得可耻的1933年1月的那一天变得不可避免的那些人;惩罚那些躲在幕后的军人和工业资本家。但有一件事却已经无法做了:德国人已经无力阻拦那个在他们的绥靖政策下变得越来越强大的纳粹政体。由于他们在政治上的冷漠,德国内部的抵抗力量遭到了破坏。他们是在让德国内部手无寸铁的民众去做那些拥有世界上最强大海军的政府不敢做的事。

总有一天你们会受到谴责和指控的。

就在我写下这些话的时候,头顶上嗡嗡地飞过大批轰炸机,飞机的轰鸣声整整持续了一个小时,就好像这些飞机在与一个世界大国打仗一样。我是个德国人,我围绕着这片我生活并热爱的土地奔走着。我就是死,也不会离开这片土地。每一棵树倒下,每一片森林消失,我会战栗;每天寂静的峡谷被破坏,我也会战栗;每一条溪流受到这些强盗的威胁,我还会战栗……

我知道这片土地是有生命的,是世界跳动着的心脏。我崇拜这片土地的心跳,不论这片土地是否被鲜血和污渍所覆盖,但我同样知道那轰鸣声是对正义、真理的否定;生活中一切有价值的东西都被那轰鸣声所否定。我相信德国这幅漫画是被一只摆脱了束缚的恶猴子涂抹出来的。

你们躲在那里不动,我恨你们:我恨你们假装睡觉。我在死的时候仍然要恨你们、诅咒你们。我要在我的坟墓中诅咒你们,那诅咒将会如幽灵般笼罩着你们的子孙后代。我除了诅咒,没有其他武器可用。我知道诅咒会使我凋谢,我不知道我是否能活着看到你们的衰败。

但我知道,如果一个人真心地爱德国,就必须去恨眼前的这个德国。为了看到一个崭新的德国,我宁愿死十次。

写到这里,我的内心犹豫了。马上就要过复活节了,收音机里传来《马太受难曲》的旋律,这仿佛是在嘲笑我。

德国,我的德国……是的,就是这个旋律,这是为我们在唱。

现在怎么样了?

现在,在我们的头顶上,那些白色的野兽驾驶着低能的机器,向着残忍和罪恶飞去,驱赶走了春天的宁静。我哭了,但不是悲哀的哭,而是气愤和耻辱的哭。

\section{1938年7月\ 财产危机}

施梅林\footnote{Schmeling,德国拳击家,德国历史上第一位世界重量级拳王。}先生在纽约被打了。

根据那个暴民之王的命令,我应该相信这次失败是全体德国人的失败,因为一个拿高薪的年轻屠夫在纽约被恰巧与我同族的另一个拿高薪的年轻屠夫打倒了。我们四个人在这个闷热天的黎明静静等待这场搏斗的结果。当我们听说整场表演只用了两分钟的时间时,我们都大笑起来。同胞们,不要信电影中的荡妇、拳击场上的拳击手,还是去信神灵吧,我希望你们能在死之前意识到这点。

有一天早晨,我注意到有三个男人在窥探我的牧场。我不认识他们,周围的环境很宁静,他们的出现与环境极不和谐。他们带着各种测量仪器在我私人的牧场上做勘测。当他们看见我的时候,甚至都没有打个招呼。于是我问他们是谁,他们说是柏林西门子的员工。戈林手下最大的企业西门子似乎要在这里建一座厂房……

他们做这件事,没有询问,没有安排,没有通知,没有一丁点儿的合法性。我问他们,如果我不通知他们,便在西门子拥有的土地上钻井,不知他们有何感受。我们双方展开了活泼的对话,因为我想认识一下这些从柏林来的“绅士”,让他们说说什么是财产权和礼貌。我把我的工人叫来,没收了他的仪器,并锁了起来。

这引发了他们的强烈抗议和威胁。第二天,来了一名助理委员。他温和地谴责我使用武力,并通知我两天后一个委员会要来找我。这个委员会按时到了:包括五个巴伐利亚政府官员和一名奥地利工程师,这名工程师的翻领戴着纳粹的十字记号。我这才知道了这个项目,它需要破坏这条美丽的河谷,拆除我的那栋早期哥特式风格的房子,大水要淹没400公顷的土地。

所有这一切就是为了获得4000马力的能量,那等同于一颗炸弹的威力。可这样的事,竟然是个宣称偏爱农民的政府做出来的,他们有一句口号是:“德国如果不为农民建国,那就什么都不是。”他们一张嘴,我就听出这个项目根本不是为了4000马力,而是德国北方的工业资本在向南方转移。他们嗅到战争正在迫近,通货膨胀随之要发生,这些工业资本家正在把纸币转化为固定资产——这些资产要从农民手里偷,无论多少代价,无论是否害死人。这些工业家名义上说是为“大家好”,借以掩盖他们野蛮抢劫的行径,他们是德国皇帝和旧贵族的后继者。

我想到了这条无与伦比的宁静河谷,18个世纪以来,人类在这里繁衍生息不断。我无法隐瞒我的愤慨……

巴伐利亚的官员虽然默默不语,但咧着嘴傻笑,这说明他们站在我这一边。但那个奥地利工程师信奉希特勒主义近乎疯狂,慷慨激昂地表达意见。他谈“什么是对社会有利的东西”,我却问他这样的“社会”有何价值。当他说可以没收我的财产,我宣布我可以离开这里,但他必须躺在担架上先离开这里。

在德国,这样的话如今是很难听到的,他气得哑口无言。他以为我的左轮枪就在衣袋里,所以坐在椅子上既疑虑又害怕。那几个巴伐利亚人盯着我,仿佛看到了一个奇迹,而那个工程师则改口说可能需要几年的时间才能解决这个问题。此后,委员会离开了。

后来,我在慕尼黑知道了这件事的背景。水利建设部的一位内部人士告诉我,很久之前有一个方案,既能获得同样多的能量,还能保护整条河道,但这个方案被议员阿诺·费舍尔(Arno Fischer)否决了。他不仅是部长,还是地下内燃机的发明者。他想让这个项目使用他的技术。毫无疑问,他可以因此中饱私囊。所以,这才出现破坏整个河谷的计划。还有隐蔽得更深的,巴伐利亚有个巨大的化工厂,生产炸药,其后台老板是权势熏天的戈林。几年前,戈林还经常出现在债权人大会上,如今他成为这一带居民命运的主宰。

这就是这件事背后的原因。根据我的密友提供的消息,一旦遇到这类事,就意味着在德国境内的财产已经没有安全性了。诸位先生,我们会知道一切的。我宁愿我的财产和全德国的财产被炸成碎片,也不愿让它们落入这帮人手里……

\section{1938年9月\ 近乎穴居的大众生活}

在从柏林回家的路上,我感到很紧张,有点压抑,因为捷克的危机越来越深重。在上巴拉丁(Upper Palatinate),我从卧铺车厢窗口向外望去,只见一眼望不到头的火车装载着大炮和弹药向边境方向驶去。德国,我的意思是说当前这一代人的德国,非常认真地掌握了拦路抢劫的技能,它正处于一种极度不同寻常的心态下。那位所谓的元首的意志成为了宇宙的真理,所有反对派,包括帝国边境之外的反对派,都变成了罪犯。是的,当然有外国牵扯进来,破坏协定不算什么大事,元首想怎样就应该怎样……

如果大家鼓起勇气,齐声对那个被一系列成功的政治偷窃冲昏了头脑的元首说“不”,结果又会怎样?这会让他猛地发现自己不再是习以为常的世界中心,有这样的结果就足够了。只要出现这样的情况,他肯定立即就会从舞台上消失。

然而,所有的征兆都表明,这一次,欧洲仍然会像对待德国进攻奥地利那次一样不会有任何作为。这样一来,希特勒的地位就会更加稳固。对于我们这些不太坏的德国人来说,我们心中尚存一息期盼:期盼着爆发一场战争,把我们从这次“蝗虫灾难”中解救出来。

我与P先生进行过一次长谈,但他无法理解为什么我会有这样的感觉。当然,他是个商人,而我一直有个看法,国家主义的本质其实根源于商业利益。不过,这也取决于他是否把这个善于诡辩、勒索、诈骗的政府视为合法。自从1933年1月30日之后,我一直无法不把他们看作是罪犯,虽然这个国家形式上现代,但实际上弄虚作假。如果一伙匪徒闯入我的家里,攻击我,虐待我,当警察破灭而入来救我的时候,我难道会抱怨警察吗?

我现在可以证明,用公民投票给予希特勒霸占奥地利的合法性,是拙劣得无法再拙劣的伪造。我家里有四个成人,都投了反对票。此外,我知道镇子上还有至少20个值得信赖的人也投了反对票。然而,根据官方的投票结果,没有一票反对“希特勒的行动”。

到处都能听得到阴谋和刺杀的传言,传言的中心是党卫军的禁卫队和所谓的条顿骑士团(主要由助理药剂师和邮局职工组成)。最近,我在慕尼黑也碰到了……

像往常一样,我住在火车站附近的那家小旅馆里,当我在三楼的窗前刮胡子时,什么东西从窗前掉了下去,接着传来轰的一声。当我走出旅馆时,发现街上躺着一个人,两条腿大张着,脑袋枕在人行道上,下面有一摊血迹,脑壳爆裂了。他穿着黑色的短裤、白条的睡衣。

路人围了过来,呆呆地看着,一个骑自行车的人激动地诉说如何差一点被这具从四层楼落下来的人体砸中。一个妇女大声叫喊着,她当时看到这个人爬到了窗户上,然后跳楼了。旅馆的门房拿来一些纸张,把尸体盖上了。两个清洁工清除了血迹和脑浆。后来,一辆环卫开车过来,尸体被抬到一旁,用水管连接到最近的消防龙头上,清洗了路面,那死尸的脚从盖着的纸张底下伸了出来。

那具尸体被隔离在街边,当风吹起那堆纸时,就能看见尸体伸出的脚。我向门房打听情况,他告诉我,此人在早晨六点钟时来到旅馆,穿着党卫军的制服,看上去有点醉意,在旅馆的最高层要了一个空房间。后来,他又要了一升啤酒和一瓶白兰地酒——在那间顶层的小房间里,如今空荡荡,我们在乱哄哄的床上找到了那只酒瓶,喝空了四分之三。地板上放着那人的黑外套,床上洒满了弄皱了的明信片,是游客在里斯本的贸易广场和埃及的塞得港买的。

几个小时之后,警察公布初步调查结果。此人在旅馆登记的名字自然是假的,他在巴德托茨(Bad Tolz)的党卫军的学校受过训练。他参与了一起针对希特勒和纳粹的阴谋活动,正在躲避追捕。

整件事都令人不愉快,而尸体就更加令人不愉快了。这件事使我想起汉斯·冯·比洛(Hans von Bulow)几年前告诉我的一件事,他是伟大的比洛将军侄子。1918年,一名前普鲁士军官在芬兰战役中被俘,成了一群布尔什维克的厨师。此人在俄罗斯人的战俘营里蹲了好几年。在有了这样的经历后,又在俄罗斯革命中生活了一段时间,他变成了一个彻头彻尾的强盗。大战开始前,自1912年他就获得了提升,拥有了军官的特权。现如今,西伯利亚四年的战俘生活让他变得更加野蛮,蜕化成为一个蓄着大胡子的冷血杀手。作为匪首,由于犯了数不清的罪行,而被判处了死刑。但令人难以置信的是他面对行刑队黑洞洞枪口时的表现:他要了一根烟抽。他点燃了烟,抽了几口。就在开枪命令下达前,他扯下裤子,转身背对着死亡,拉出一大堆屎,他一边拉屎,一边接受子弹的“圣餐”。

我又和比洛谈起这个老故事。表面看很类似于肖邦的情况,当死亡来临的时候,他说道:“呸!”这一代人蔑视死亡,这诱使人去崇拜他们……

但蔑视死亡是错误的。此刻在死亡面前的勇气仅是人群中的一种冷漠而已。愤世嫉俗所要表达的仅是对人群生存条件的反映:既不好,也不坏,仅是某种满足,没有什么实质内容。我实在无法给予我沉闷的同时代的人更富有精神特点的描述了。

今天,有传言说维也纳要发生暴动。我不相信这是真的。那很可能是农贸市场里的妇女们的闲扯。大众的生活就像机器人一样,吃完了饭马上就与用过氧化氢漂白了头发的女人睡觉,不断地生育后代,就像白蚁一样成堆地生活在一起。他们重复着至高神灵发出的咒语,谴责他人或被人谴责,死掉或致他人死掉,就像植物一样过着呆板单调的生活。即使面对父辈的传奇、高贵的历史、本民族文化的大量精品,都丝毫不会产生激动的心理。

即使在这种近乎穴居的情况下,也并非是不可容忍的。真正不可容忍的是他们迫使几个不愿过穴居生活的真正人类返回山洞中,如果胆敢拒绝,就威胁着要加以灭绝。

我们在赫拉克利特(纪元前五世纪的希腊哲学家)的作品中能读到:

\begin{quote}
他们实际上不知道,多数人在作恶,只有少数人行善。

希腊的厄斐索斯人甚至命令老年人去上吊,把城市留给年轻人。

他们赶走了杰出的赫尔谟德尔(Hermodor),并大叫道,

“我们不许有道德出众的人——这样的人必须离开这里,去别的地方。”
\end{quote}

\section{1938年12月\ 被忘却的伤害}

戈培尔鼓动迫害犹太人,我冥思苦想,希望能发现其中奥妙。就在这个政权仍然需要和平的时期,此举无疑会招致全世界的敌意,从而不可避免地引发战争。我无法理解他们的动机,即使我设想自己是个纳粹,也无法找到理由说服自己。

我知道独裁者每五个月就要进行一次烟火表演,为的是能留住“贱民”的忠诚……这就是拿破仑三世为什么在塞瓦斯托波尔之后,还要去远征中国、马坚塔、索尔费里诺、墨西哥,最后还去了苏丹。

毫无疑问,所有这一切足以解释11月9日发生的一系列事件,如果不是希特勒想要战争的话——如果他不是想自掘坟墓,战争肯定能避免。

我与L先生谈论起这些想法,他在外交部工作,为人十分勤奋。他笑我的分析太复杂。他解释说希特勒总是喜怒无常;他的样子就像波斯王阿塔泽克西兹(Artaxerxes),如果他不能马上得到想要的,便会倒在地上啃地毯。

如果L先生说的是正确的话,这就是所有可悲的、无穷羞辱的原因。但我倒想说两件我亲眼看到的事。第一件事,演员桑尼斯的侄女,被迫从一个地方逃亡到另一个地方,她对此感到厌倦,在一个寒冷秋天的晚上,跑近了大山。我们找了她几天,最后终于找到了她:但她已经死了。

第二个故事更加令人心碎。为了保护隐私,我不准备提及当事人的真实姓名。这个故事是流芳百世的利奥·冯·聪布施(Leo von Zumbusch)的遗孀告诉我的。

老姑娘X独身住在马克西米兰街的一套两居室中。一名在纳粹统治下红极一时的演员想要这套两居室。他发现这两居室实际上是由一名老犹太妇女继承,于是公开指责这个老女人占有这套房。在我们这个伟大的时代,这等于是要把她投入集中营,让她在饥饿中慢慢死去。X女士很清楚这意味着什么,她觉得自己太老、身体太弱,无法承受这样的折磨。于是她向自己学生的母亲要来了速效毒药。

这位朋友不仅有坚强的性格,还有卓绝的意志。首先,她用各种可能的办法保护那位疲倦的老妇人。当这个办法并不奏效的时候,她又去慕尼黑找一位药理学家,他是她丈夫的同学。她向他索取毒药……

这位绅士是希特勒的追随者,这引来了麻烦,因为他拒绝提供毒药。最后被逼无奈,他给了她一些箭毒素和氰化钾的混合物。她拿着毒药回到奄奄一息的X女士身旁。

X女士看到毒药,痛哭流涕,但又提出最后一个要求:在她离开人世之前,问这位朋友能不能为她唱勃拉姆斯的《最严肃的歌》。这位朋友是歌唱家,他同意了。老女人终于离开了人世。今天中午吃饭的时候,我们听说X女士死在那套房子里了。那位迫害她的演员P,当时站在她的门外听歌,听得都不耐烦了。

这些都是我亲眼见到的事。这里我没有交代当事人的真名。在第二个故事中,当事人是一名79岁舞蹈家,这位希特勒主义者住在维也纳……不久之后,她就要躺在病床上接受审判了。

另一个人呢?

我在这个世界上已经活了50多年了,经历过许多黯淡的时光,有一个经验总结:所有伤害给人的痛苦都是随后才来的,有时是数十年之后。或早或晚,有时甚至都到了被遗忘的程度,我推测那位P先生,当他偶尔在那套用伎俩获得的房子里品尝鸡尾酒的时候,会不会尝到箭毒素和氰化钾的混合物的滋味?

在他听收音机里进行曲的时候,会不会有时听到《最严肃的歌》?


\section{1939年4月\ 希特勒生日前夕}

漫长的冬天终于过去了,它简直称得上是一个“北欧”的冬天,在辞别了冬天后,我又来到了柏林。希特勒的生日要到了,为了这个国家的节日,大家都忙碌起来了,最突出的表现是旅馆里涌进来可供德国调遣的各种突击队员:有风暴突击队员,有超级风暴突击队员,有龙卷风突击队员,还有飓风突击队员。旅馆内每个房间的门口都是他们丑陋的大靴子。

首先,我与汉斯·阿伯斯\footnote{Hans Albers,德国著名演员。}见了一面。他在蒂尔加滕河畔(Tiergarten)有一套非常昂贵的房子,我们在那套房子里喝茶。他的房子里装满了令人生疑的“古董”,但他却坚信是真的。

但他是个好人,只是对衰老怕得要命。阿伯斯算是我们这里的名人,私下里是一个很朴实、有魅力的汉堡人。不过,他与德皇威廉二世有相同的问题。德皇与人促膝谈心时,是个很随和、快活的人,但他在公开场合时却令人难以容忍。阿伯斯和我独处时,还表现出了一定的伤感,他流着泪告诉我,他母亲在临终之际还唱着著名的《大海包围着的石勒苏益格–荷尔斯泰因》,而她就出生在荷尔斯泰因。

柏林散发着战争的气息。在我眼里,柏林永远不该像个暴发户一样:卑鄙、病态、荒谬。菜单上没有什么东西,红酒比平时更加不对味儿。餐巾布似乎也不干净。咖啡非常难喝,出租车没有汽油烧。旅馆的修理工被叫去修筑工事,旅馆里一团糟。过去被豪华外表掩盖住的混乱,如今完全暴露在外,这才是真正的普鲁士人的风格。

有一天晚上,在一个吹着口哨、挤眉弄眼的推销员的引导下,我走进城市西部一间坐落在地下室的破旧夜总会。按照柏林惯常的传统,夜总会营业时间要延续到早晨,但服务员太困了,趴在空荡的桌子上睡着了。在我待在夜总会这段时间里,夜总会里挤满了穿着党卫军制服的年轻乡绅。他们来柏林显然是为了参加即将举行的“皇帝”(指希特勒——编者注)生日庆典。夜总会到处是他们身上难闻的劣质香烟的味道,但他们糟糕的举止让人感觉更差。

他们很会玩,从香槟酒冷却桶拿出冰块,放进女人袒胸露肩的衣服里,然后大笑着把手伸到女人衣服里把冰块掏出来,而且手伸得特别深——深到令人感到难堪的地步。不知何故,有个蓄着长长的白胡子的老头,摇摇晃晃地走进这间地下夜总会,周围的人都嘲笑他。这老头用只有火星上的人才能听懂的语言,与周围的每个人大声交谈,谈的都是些第一次世界大战和自由军时代的皮条客说的话——这些话在最近20年来流行起来。

我坐在桌前,仔细看着他们的脸。他们个个都看上去很愚蠢,死板得令人同情,坚硬得令人感到绝望,但他们祖辈都是些沾染着鲜血的古老名字,他们的父辈曾经是欢乐的一代,让当时的民众、各国大使、副官大为惊奇,因为拥有日耳曼人巨大的胃口,并能在足球场上飞奔。

看着这些人,就仿佛是在看一条横亘在当下的我们和昔日的历史之间无法逾越的渊薮。确实,啤酒肚和眼袋没有了,脸变得消瘦了。猛一看,他们就像是把大褂留在了衣帽间的屠龙者或长天使。看着看着,他们说的那些妓院里的行话,露出粗俗的表情,又让我产生了新的联想。

首先,他们看上去空虚得吓人。继续观察他们,还能发现他们的眼睛里闪着亮光,总是突然闪起来,这不是因为年轻的缘故,而是这一代人特有的,立即就能让人想到一种绝对疯狂的野蛮本质。

第二天,在总理府前,我目睹了这场庆祝活动,那里有大量群众聚集,阅兵部队用鼓、钹、风琴演奏出震耳欲聋的声音。我听到了喧闹声,看到了女人疯狂的面孔,还看到了让女人疯狂的“那个东西”。

他站在那里,辉煌得无与伦比,姿势与往常一样,双手交叉放在肚子上,穿着银饰的制服,帽檐低低地压在前额上,像一个有轨电车的乘务员。我用望远镜仔细观察他的脸。他满脸的肥肉在晃悠;而且是悬挂着,没有骨骼的松弛——像熔渣,像凝胶,像生病了一样。上帝赐予人脸的光泽,在他的脸上丝毫寻不到踪迹。相反,他的那张脸因缺乏男性特征而变得神秘,且带着敌意,那敌意源自他的性无能,于是他才把怒火转化为对他人的残酷。

人群山呼“万岁”,那呼喊声中透露出愚蠢和低能……疯狂的妇女,精神恍惚的青年,整个民族的精神状态跟嚎叫着的疯癫道士一样。

我和弗兰肯斯坦一起回到旅馆,我俩今天是偶然相遇的。我们聊起昨天的见闻,他提醒我,德国有许多贵族听命于这个罪犯,冯·阿尼姆家族、里德泽尔家族、冯·凯特家族、冯·克莱斯特家族、比洛家族都有家族成员担任“集团军群司令官”或相当的职务……他们在接受这些荣誉地位时,没有考虑这等于是在玷污自己的名声和祖宗的声誉。这时我再次想到了头脑迟钝的群众那笨牛般的嚎叫;我看着被这群人山呼万岁拥戴着的摩洛神\footnote{在旧约中亚们人和腓尼基人所信奉的神灵,需要用儿童向他献祭。},思考着他的生理缺陷;我觉得我们全都陷入了这片耻辱的海洋中了。

不,即使是遭人恶骂的威廉皇帝年代的人,也从来没有达到过如此唯我独尊的地步。这就是常说的,过去的人再坏,也不如当今的人坏。不,眼前这一幕简直是污秽!这些庆祝活动不值得看,不值得去领会。撒旦已经松绑了,一大群魔鬼已经骑在我们身上了……

这个民族疯了,并会为自己的疯狂付出高昂的代价。这个夏季充满了不祥的预兆,如果医生无法治愈这疯病,那就只能用火和铁来修复了。

在回慕尼黑的火车上,D先生告诉我,他在一战时期是希特勒的连长。他说希特勒是个经常处于茫然中的人。作为信使,希特勒经常勇敢地走入“死亡的血盆大口”之中。不过,一旦危险过去了,全连的人都认为他是个笨蛋。

有关希特勒的铁十字勋章,也有一段传闻,一直都无法驱散。我现在仅是转述一下,因为我无法证实。一位熟悉那个时期授勋仪式的军官提及的一件事引起了我的注意,第一级铁十字勋章是批量发给普通士兵的。如果这是真的话,希特勒的勋章是自己挂在胸前的。

我不喜欢近几年形成的背后诽谤人的恶习。我不会这样做,所以我仅是记录我听到的,既不支持,也不反对。显然,希特勒除了在政治上,在其他方面也在撒谎。他不断撒谎,就是为了提高自己的声望。例如,1923年11月9日,希特勒从统帅部逃跑,还编造出一个奇妙的故事,他在枪林弹雨中救出一个正在哭泣的孩子。没有人看到过那孩子。毫无疑问,编造这个故事的目的是为了用一个催人泪下的电影去掩盖他可耻的逃跑行径。

D先生还告诉另外一些能说明希特勒性格的事。在希特勒夺取权力之前,他俩见面,希特勒总是尊敬地开口说,“上校先生,”而D总是用“你”称呼这位前信使。

这就是双方的用语,一直沿用到1932年。如今,D先生是慕尼黑的检察官,而另一方是戴着装饰着银线的有轨电车乘务员帽子的大人物……如今,这位德国的摩洛神是生与死的主宰——他过去住在巴厄街的一间连家具一同出租的房间里,如今他大胆地把他在那片充斥盗匪的地区用过的装饰品送给一个主权国家(被西班牙独裁者佛朗哥拒绝了)。

与普鲁士人打交道就是如此,特别是跟那些冒牌的普鲁士人:虽然他们总是试图掩盖,但无法掩盖刚入伍新兵的心态;即使反复无常的命运把他们吹到权力殿堂中,他们依然无法摆脱这种心态。当D在说话的时候,一队年轻人排队在春天闷热的天气里从楼下走过。他们背的不是舒服的帆布背包,而是筒状行李,这些背包能带的东西不多,不实用,但能让人回想起兵营和阅兵场。普鲁士人就是这样。每个行李卷,都是整整齐齐的,随时可以打开使用,这是军士梦想中的有秩序的生活。这种生活毁了德国,如今他们又希望在全世界推广。很快,德国就要面对结局:要么从普鲁士人的霸权中获得解放,要么毁灭。没有第三种可能性。

柏林举行庆祝“皇帝”生日的活动,给布鲁诺·布雷姆先生一次成为桂冠诗人的机会:布雷姆先生在向我做自我介绍时说自己是最纯粹的君主制主义者,还在两年前写了那本粗俗的有关德皇在大战中情况的书……到了1930年,布雷姆先生还在谋求获得维也纳犹太人文学圈的青睐;他把自己的书献给那些犹太文人的犹太妻子,还说什么是“忠诚的回忆”;可几年后,到了现在,他却写煽动性的反犹太人文章。

哦,一旦风声有变,我相信他立即就能翻出旧时代大公的爵位供他做政治借口。无论怎样,我知道他比班诺·冯·梅肖(Benno von Mechow)更善于见风使舵,此人在1933年变成了一个天主教徒,就好像天主教就要赢一样。然而,事实并非如此,他又转投纳粹。不幸的是,这也没有对纳粹产生他希望看到的结果。

但他们这类人不是都这样吗?看看他们的共同点,都是大战前刚从军校毕业,大战中一直没有升官,但依靠把自己在大战中的特殊经历写成书而成名,此后,不得不面临一个现实情况,由于缺少讲故事的基本能力,仅凭幻想无力推出第二本书,难道他们不都是这个模式吗?所以,他们是一杯茶,喝完了就加开水,三次,四次,五次,显然茶味是一杯比一杯淡。这几年来,他们所做的就是把旧书再改写一下,仰仗高强的写作技巧,填漏补缺。

这几位永恒的军校学生,老得都能指挥一个师了,却模仿汉姆生,模仿阿达贝尔特·施蒂夫特(Alalbert Stifter);对血液、土壤、土味、下水沟的臭味有专门知识;前拉斐尔派,具有提尔泰奥斯(Tyrtaeus,公元前7世纪的希腊哀歌体诗人)的敏锐观察力;仍然保持着大天使和屠龙者的童贞,而且瓦塞尔曼\footnote{Wassermann,德国细菌学家,在1906年发明了梅毒血检。}反应呈阴性。

人群中没有一个是完整的。就像冯·塔内写的斯泰恩的故事一样,所有人都不完美。世界上根本没有什么人可以做“朋友”。

我曾经说过:对我们这些幸存下来的人,最难忍耐的是孤独。那些与我们有相同思想的同志们一个接着一个消失了。

今天,当我下火车的时候,我听说马克斯·莫尔(Max Mohr)死了。

他是1934年的移民,死的时候在上海做医生——大约是一年前的事;可我现在才知道。大战中,他是一名既无私又勇敢的军官,喜欢滑雪和登山,他不仅是个医生,还是个农夫。

他是戴维·赫伯特·劳伦斯(David Herbert Lawrence)的朋友,劳伦斯写了那本令人无法忘怀的《六月的即兴创作》,这本书用浪漫主义的手法讲述了那次革命。他还是两本小说的作者,这两本小说写得更加有力,简直令人难以置信。在走路的人中,没有人比他更加疑惑。站在雪山顶上的人,没有人比他更加坚决。

劳伦斯和莫尔,看到这个受股票市场控制的世界,他俩感到厌恶,于是奋起反抗,难道他俩不就是这样的战士吗?他俩属于同一个乐队,一个还没有旗帜的乐队,这样的乐队在各地都是分散活动的——然而,只要太阳还照耀在地球上,他俩就不会向绝望低头,而是坚持宣传自己的主张,他俩是不是这样的一对人?

《拉德斯的友谊》写道:

\begin{quote}
从前的人没有肚脐。那个时候根本就没有这个东西。

后来,普鲁士人发明了肚脐……这可能是因为他们有一种天生的秩序感,这样人在出生的时候就有了一种可以随身携带的纪念章。
\end{quote}

没有人曾经以这种无法抵御的方式嘲笑自己祖国,也没有人曾经怀有对祖国有这种像暴风雨一样的爱。从来没有人曾经以如此迷人的方式去踢那些我们痛恨的人和事的屁股:威廉皇帝纪念堂大街和法本公司、鲁尔工业区、以及同性恋俱乐部里的年轻人;老式桥牌和柏林郊外的慈善性卖淫活动。

哦,巴伐利亚,我的巴伐利亚,这片世界上最美丽的土地,如今正在上演一出由远在天边的神仙演的戏剧,剧情发展到了第四幕和第五幕之间:村子里道路空荡荡的,本该到了挤奶的时间了,南面的卡文德尔山脉,就像是满身白灰的阿瑟王的妹妹变成的妖精。往远看,一队穿黑衣服的人走过田野:一个老农夫被带出房子,然后被杀死了。整个过程很慢,不慌不忙。他们把他带出了他出生时的房子,他的生命就此结束了,邻居把他抬到一间小教堂里,教堂里有几座洋葱般的塔楼。在更远的地方,另外一些邻居正在搬干草。大型的种马拖着涂成了绿色的马车厢,这是巴伐利亚最后一批种马,因为今后就要有卡车了,卡车简直就是钢铁猛龙。

落日的余晖洒落在眼前的景象上——你内心中隐藏的对家乡的恐惧是什么?那落日的余晖伴随着你的青年时代,也伴随着我的青年时代,当你给我《拉德斯的友谊》这本书的时候,正好是有不祥预兆的1931年7月。

两年后,他移民到了上海。在远离卡文德尔山脉出生地的地方,一场心脏病打倒了他。在一艘驶向北海的船的后甲板上,他的骨灰被撒向海风中,撒骨灰的地方离他的家乡很近了,就在赫尔戈兰岛的海面上……

与商人小说家的作品的格调不同,他的小说里包含了勇敢的战争故事,那些绝妙的军校学生过着不受拘束的的生活,并“幸运地突然死去,这是我们大家都盼望的结局。”

现在轮到你了。一盏灯熄灭了,接着又是另一盏。最后,戏院变成漆黑一片。那舞台上刚才还有光亮和动静,现在却变得一片空旷。冷风不时从背后的包厢里吹出来,吹到舞台上。

现在,活着需要勇气。勇气,就是每天都要树立的个人意志。这几年,继续生活就是继续仇恨。勇气是需要的,信念需要努力奋斗才能变成现实。

\section{1939年8月\ 在大战的门槛边}

我去过詹宁斯\footnote{Emil Jannings,1884—1950,获得奥斯卡金像奖的德国演员。}的家,它坐落在沃夫冈湖区(Wolfgangsee)。他家的房子曾经让人深刻印象,但如今阴郁的气氛笼罩着这栋房子——房子的主人害怕战争。他担忧自己的艺术收藏品,担忧股票和证券,担忧房子中央供暖用煤问题,担忧来年是否有足够多的腊肠种类供晚餐之用。他是个演员,不多不少,真正的一级演员;作为一个男人,他是个富裕的中产阶级,害怕即将到来的世界风暴,其中最害怕他钓鱼时的午睡时间和吸雪茄的时候受到严重干扰。

雅宁斯告诉戈培尔在柏林弄出了各种大丑闻。戈培尔在与演员弗罗利希(Frohlich)的妻子私通时,被弗罗利希发现,弗罗利希狠狠揍了戈培尔一顿。不过,真相并非如此。雅宁斯说他是目击者,我敢肯定他说的版本是真的。

真实情况似乎是弗罗利希和詹宁斯离开晚会现场,准备开车回家,发现妻子在车里与戈培尔部长在一起……促膝谈心。弗罗利希没有打戈培尔,但后来抽了妻子几耳光。他的贴身男仆把他妻子通奸的丑闻传了去来。

这是真事。弗罗利希没有勇气采取公开的行动。但流传的版本故意增加一个令人高兴的结尾,让部长先生挨打了。突然之间,一首名叫《我要做幸福的弗罗利希》的歌曲流行起来。

8月末,我与几年前是内阁成员的K先生一起在基姆湖(Chiemsee)度假。他年轻的时候认识俾斯麦。我们谈起他在大战中的经历。他谈了25年前大战爆发后最初几天在东普鲁士边境时的情况。在宣战的前几天,晚上都是满月,巡逻队在田边走来走去,大家排着队,一个跟着一个,避免践踏田里的庄稼。开战后,很难让农村里来的新兵克服心理障碍,骑马进入等待收获的麦田……

不久之前发生的事很快就成为昔日的传奇:一名普鲁士骑兵用长矛把一个俄罗斯兵从战马上刺下来,由于刺得太深,长矛拔不出来了。看着那俄罗斯人,他突然痛苦得哭了起来。那俄罗斯人摇着手,要求他看在基督的份上别再伤心了。

另有一件事:一个犹太小孩被军事法庭判处死刑,罪名是通敌。他被带到刑场上,但他不理解。有人把判决书给他看,但他看不懂,问道:“请告诉我这张小纸条上写的是什么?”

还有一件事:一名穿着俄军军装的老俄罗斯农夫,在被俘后告诉我们,他和战友们在双方相距1000米时准备开枪(这么远什么都打不到)。双方相距500米时,他们开枪了(看不清是否击中了目标)。在双方相距100米时,所有俄罗斯人丢掉手中的枪,不再开枪——“先生,这么近的距离开枪是犯罪!”

我们接着谈起了俾斯麦,这时希特勒重装部队正好在楼下通过,向萨尔茨堡进发。K先生当时做外交工作,要在首相面前做一次讲演。俾斯麦的书桌上摆着一盘子腊肠,当这位年轻人讲话时,那个老吃货就不时切下一块跟大拇指一样厚的腊肠,盖上一块相同厚度的黄油,大口吃起来,连块面包都不要……

我丝毫不怀疑一点,国家领袖的身体状况对这个国家的外交政策有深刻的影响。拿破仑帝国的兴衰说明了这点。如果如今掌握德国命运的人能像俾斯麦一样吃点心,我很想知道德国将会有怎样的命运。

如果我想否定俾斯麦的地位和他的深厚的爱国情怀,那肯定精神有问题。但我们正在收获他种下的庄稼,普鲁士的过度工业化就是他埋下的种子。我现在越来越相信,伟人悲剧性的误判必将产生严重后果。我们要感谢这届政府中那些整天想着工业化的笨蛋,就是他们才造成德国人口泛滥,人像兔子一样繁殖,大量的人找不到工作,于是更加贪婪地去夺权。

我们正处在第二次世界大战的门槛上,周边国家都反对德国,这都源自俾斯麦所创立的国家。我确信这场普鲁士人挑起的战争在没有放第一枪之前德国就败了。除非我们知道了即将到来的这场灾难的梗概,否则就不会谈论未来的好日子。我只有站在这场战争遗留下的一堆不太大的废墟上,才能看清未来——这是一场相当短的战争,不可避免,很快就要到来。

今天的天特别清澈,有秋天的气息,可今天又是最后一个和平的日子。按照天上星星的指向,一对公牛拖着犁耕地,一会儿爬上山坡,一会儿又爬下山坡,这种耕地的方式从基督诞辰之日就没有改变过。

然后,一场大浩劫正要袭来。这里的人们感觉到了危机,内心很焦虑。在德国,只有德国的农民,特别是巴伐利亚的农民,仍然在听从自己内心的呼唤,他们是对现实保持清醒头脑的阶层,还没有被全国性的疯狂和几次成功的政治盗窃冲昏头脑。在乡下,只有希特勒青年团的小流氓们是积极分子,他们在拉道(Radau)受过培训,以为战争就跟散步一般进入奥地利和捷克那样容易。

第二天早晨,在回家的路上,铁匠出来告诉我一个新闻:那个肩负着德国命运的侏儒又迈出了一大步,所有的高音喇叭里都是他的那权迷心窍、精神分裂的大叫大吼。

我同那人握了握手。已经七年了,他和我一样忍耐着,愤恨着。我丝毫不怀疑未来是一场巨大的磨难,是无法避免的。但有一件事我十分肯定,那个大魔头今天等于在自己的死亡令上签了字,为此我已经忍耐了六年,它让我熬过了生命里最黑暗的时刻。

我恨你,每个小时都恨,只要能让你死,我会高兴地放弃我的生命。如果我能在死的那天看到你也要死,我宁愿去死。我要跟你一起堕入深渊。当我让这仇恨自由发挥的时候,我整个人几乎都被它笼罩,我无力改变这一切,因为我不知道如何应对。我不许任何人轻视我的感情,傻子不知道仇恨的力量。仇恨改变现实。仇恨是行动之父。我们的房子已经被染污了,被亵渎了,要想逃离这栋房子,只能下决心去恨撒旦。只有这样,我们才能赢得在黑暗中寻找爱的权利。

\section{1939年9月20日\ 纳粹正在征服世界}

所以,纳粹(此时此刻我不想称呼他们是“德国人”)……纳粹正在征服世界,可他们除了这样还能做什么呢?这个夏季总是在下雨,真是令人感到绝望,但夏季过后,迎来了一个清爽的秋天,空气中飘浮着秋天的雾状芳香,大地坚实,仿佛受命让坦克开过来碾平一切障碍一样……波兰骑兵,波兰军队……如果波兰是我们的了,那么明天全世界就是我们的了。

是的,纳粹正在征服其他国家,但对内的征服力度也许比在战场上更大。报社的编辑们以残忍的方式把我们的森林转化成新闻纸。他们为我们这个伟大的时代发明了一套崭新的语言体系,宣布1899年的目标实现了,宣告妇女在战争中的承诺,宣传德国妇女支持战争,讨论德国古老的土地波兹南(Posen)。当有人提醒他们,波兹南在腓特烈大帝时代就归波兰,而在第一次坦南堡战役中从但泽来的士兵就为波兰打仗时,他们马上就变得卑鄙起来,威胁要向盖世太保告密……

纳粹真的是在征服世界,而他们的“战时评论员”真的是在给德语增色,因为他们“用一阵猛烈的火花,让敌人为和平而烧成灰烬”,此外他们还“让敌人嚼着东西躺倒在地”。当有人告诉他们,他们的德语是公共厕所墙壁上的德语,是男妓的德语时,他们马上变得非常卑鄙,咆哮道他们是战士,战士就这么说话。如果你不信,就把你送到集中营里去。

哦,是的,纳粹就是要征服世界,他们就像威廉皇帝在1914年那样不断地去征服世界,啤酒馆的常客再次征服了全世界。最近,在我住的地方的一家咖啡厅里,一名老军医用时髦的用语说波兰人和英国人都是“猪”——可这老头一辈子都没有见过一头“活着”的猪。

当我起身对这类完全不适宜的语言加以反对的时候,他看着我,样子就像一头被阉割过的雄鹿,感觉大地在他脚下崩塌了一样,他咕哝着说他如何一直以为我是个爱国的人。

\section{1939年9月22日\ 希特勒青年团团员的来信}

亲爱的莱克:

我刚从波兰战场回来,即将奔赴西线战场,此刻我正坐在家里给你写信。我现在是空军上校,刚从波兰战役前线返回,已经执行了11次任务——其中有几次任务相当特别,比如俯冲轰炸部队和军用列车。有一次,纯粹是出于对飞机的热爱,我捞到一次轰炸华沙的机会——当然,行动中还有许多飞机,我们要进行垂直俯冲,从高度16000英尺俯冲到2200英尺。我在执行任务的过程中没有负伤,只是机身上有些擦伤。下一个目标是英格兰。我相信肯定不止一次。

你听我说:我不撒谎,我们就要去做听起来似乎不可能的事了。我们是神的孩子,神是最怜悯人的。但我们总有试探命运的办法:我们搂住飞机的脖子说,“祝福我,否则我不松手!”

亲爱的莱克,我有好多年没有给你写信了。我们有很多空闲时间,因为我们是人,我们不必太匆忙。我要求飞往东普鲁士,不知是否太唐突。如今我飞遍了东普鲁士的上空,从北端出发,飞越全境,然后去攻击敌人,每次带弹飞行1600英里。每次执行任务几个小时后,我就能从波兰的废墟飞回来,给马祖里湖增添荣耀。有时飞机上有死人,也有飞机失踪的情况,还有的时候飞机受伤严重,只能在一堆搅在一起的金属条上降落。当你在一片沼泽的南边追上一支波兰陆军的运输队,25分钟之后,你离开那个坟场,你不会问你的代价是什么,你会觉得你自己根本不会死。莱克,我不知道你是否了解波兰人,不知道你认为我们攻击波兰人会持续多长时间。我只是想告诉你我所知道的:即便是欧洲变成了废墟,甚至英格兰也变成了废墟,我们还要继续打下去。

我是希特勒青年团的团员。我不是老兵,但被给予一个有权的岗位。很自然,我是国家社会主义的忠实信徒。莱克,我知道我们犯了一些可怕的错误。有些地方腐败到了极点。但我知道这些错误并不是致命的——更准确地说,命运是伴随着精神的,这是我的伟大的朋友莱克曾经说过的,我有信心相信我的第三帝国的精神。你和我,显然正沿着截然相反的方向前进。奥地利、苏台德区、波希米亚、默默尔这几个地方是给我的圣诞礼物。在仲夏,在战争的中期,一想到此时维也纳已经与第三帝国合为一体,我就感到幸福。对波兰的战争,是我想打的,我期待已久,盼望了很长时间了,所以我充满激情地参加战斗。我很高兴参战。有11次,我带着东普鲁士的荣耀,飞过散布着上万名挥手道别姑娘的田野;这11次,我都回来了,深感幸福,那是一种从来没有过的幸福。如今必须把战火烧向英格兰,这样东普鲁士才能打破孤立状态,突破跟牢笼一样的边境线,这事必须坚决地去做。这是个困难的任务。我相信英格兰既不是信条,也不是偶像,而很可能是块难以裁剪的皮料。但为了赢得战争,在1918年已经付出了难以置信的代价,却又在1933年,或肯定地说在1935年让敌人再次变得强大起来——这显然是政治上的浅薄之举,极为荒谬!现在又要与德国开战?不,莱克,这绝对不合理,十分荒谬,这是拿议会做掩护的政治,浪费每一次有利的战机,只等到觉得自己感到受辱时,再去挑战敌人。

我一点都不知道未来将何去何从。不管那么多了。我坐在轰炸机的机舱里,只管向我看见的人射击。但我逐渐认识到,这是一场英国和德国之间你死我活的斗争。莱克,我不管有些人提出的警告,说什么天就要黑了,英格兰会获胜。我们并非不熟悉夜间作战。请注意,我们在波兰获得了极有价值的作战经验,学会了如何对付与我们为敌的民众。波兰人打仗很勇敢。但我们仍然把他们打成碎片。我并不恨波兰人——现在更加不恨,他们现在已经是毫无斗志的原始人了。不过,又冒出了新问题,德国农民被人从背后射杀,这就迫使我们用新的日耳曼冷酷去解决这个问题,我们不在乎把多少波兰知识分子推到墙壁前射杀。一些最重要的波兰知识分子已经被我们枪毙了,一旦出现紧急情况,可供调遣的德国农民数量,肯定比波兰知识分子多。我不知道这样的方法是否适合于英格兰。但有一点我能肯定,我们的行动方针逐渐变成:“如果你不做我的兄弟,你的脑袋就要落地。”我已经庄严地下定决心,无论哪个国家的人,只要他敢违背我们在东方建立起的新秩序,或试图破坏国家社会主义,我们就要把他打倒。我将毫不留情地做这件事,这就是我目前的心事。我不认为敌人会很温和地攻击德国,这场战争是生与死的决战。英国人傲慢地宣布用饥饿战对付妇女和儿童,这明确地显示了他们的同情心和人性究竟是什么货色。

莱克,我说的这些让你害怕了吗?但我没有逼迫捷克和波兰做我们的死敌,倒是英国挑选这个关键时刻向我们宣战。打仗不能战栗,不能束手无策,这就是我想说的。这场新的世界大战,肯定会牵涉许多人的利益,但我有个信念,一定会有数万像我这样的人,他们会强迫那些意志不坚决的人做他们该做的事。

我希望在英格兰作战时要使用冷静得跟数学一样的理智……不能再像威廉皇帝打仗那样的模糊、漫不经心。

这个国家在第一次世界大战后,虽然精疲力尽,但在国家领袖的领导下,神奇般地又有了令人生畏的新战争能力。我认为大多数英国人聚集在城市里生活,能力被削弱了,无法做出英雄的举动,他们除了古老的贵族政治外,简直一无是处。当然,德国与英国几乎没有差别,也是继承了贵族政治的残余,但德国有一个新梦想。无论如何,未来就是一场大屠杀。如果我将来发现自己变成一枚火箭从天而降,我仍然要在最后一个时刻真诚地坦白说:我们曾经有过快乐。

我说完了。万事如意。

你的X

\ 

这是暴徒写的信吗?或者是逃犯写的信?不,写信的是一位有着可爱的蓝眼睛、笑声无法抗拒的男孩子……具有莱茵兰中产阶级血统,有传统操守,有文化修养。他写这封信,完全是因为有这几次丝毫没有痛苦的胜利和“国家社会主义者的虔诚”。他的机身可能被轻微“刮伤”,但一个男人绝不在上帝鼻子底下飞过时犹犹豫豫说:“我不放弃;祝福我。如果你不祝福我们,我们就会用完全新式的德国冷酷对付你们,把一群天使推到墙边去。”

这些胜利的意思就在于此。如今,每个大喇叭里放出来的都是这种腔调和无耻的语言,从报纸作者的嘴里流淌到穿着军装的暴徒嘴里。没有人敢顶嘴,因为盖世太保会来抓你。孩子谴责父母,哥哥故意不理睬妹妹。总之,对德国有用的,就一定是正确的……

这场全面战争的后果,就是让地球上住满了新一代的德国人,仿佛新一代的德国人还不够多……为什么?因为德国体制的效率很高,有条件生产那么多的德国人。

一对年轻的夫妇发现视觉有问题——周期性的失明症,似乎是男方的家族遗传病。于是那名年轻男人立即做了绝育手术。然而,作为优秀的德国人,他俩必须生小孩。丈夫坚决地送妻子去参加“青春之源”俱乐部,这是党卫军下属机构,办公地点在伦巴赫广场(Lenbachplatz)附近的残破的犹太教会堂里。

这家机构的办公室里摆着一本相册,里面是纯正北欧金发血统的党卫军男人的照片。顾客根据自己的偏好从中挑选出一个,然后向“青春之源”俱乐部指出所选的种马。不久之后,顾客怀孕了,接着做了母亲,生出一个日耳曼杂交小品种,取名海因茨–迪特尔,或埃克。这个小家伙长大后,就会具有彻底的德国式的冷酷,去发动闪电攻击,消灭那些胆敢违背新德国秩序或国家社会主义信仰的人。

“青春之源”俱乐部负担相关的一起事项,地址没在伦巴赫广场13号,电话是X,结果保证是德国血统。妓院里也能生产德国血统!

这就是我们将要过的生活,这就是赢得了一个又一个胜利的民族的生活。

坦白地说,我不认为这一切背后有什么思想性。我反对“青春之源”俱乐部,反对生产日耳曼小杂种。我不相信有什么屠龙者,不认为光看面孔就能判定天真无邪。我不喜欢德国少女联盟里的那些女孩子们披在肩上的金发辫子(看,我们多么健康,像乡村姑娘一样!)。我还不喜欢希特勒青年团的鼓点。我不认同德国新秩序。我喜欢神话德国——要知道德国60\%是斯拉夫人……德国的神话可以来自莱比锡郊区一位对条顿人感兴趣的剑术师的儿子,他的埃达是萨克森州的施科伊迪茨的一位高中老师。

经过多年的观察,我认为这些都是自欺欺人的大骗局,其背后则是受压迫的大众的欲望:贪婪、不满、松懈、发情期、性放荡、自闭,不仅拒绝上帝,还拒绝众神。在罗马帝国衰败的时期,各个城市内的暴民表现出许多相同的特征,比如:“年轻的民族”的那种冲动、好斗的喧嚣、不断挑战其他国家;无论提出什么要求,其他国家都必须答应,因为只能这样才能跟年轻国家打交道!

实际上,我们面对的是无可救药的大众,他们没有组织,没有固定的道义,痛恨纪律、形式、规定。出现这种情况的责任,可能要归咎于大大小小的商人和工业家,本世纪\footnote{20世纪——编者注}初,第一次世界大战之后,他们匆忙地想给大量无家可归的人一个可以相互拥抱着取暖的蚁窝。对当时的领袖而言,最令他们满意的大众是粗糙简单的人。为了使大众分散注意力,不去关心真正的社会问题和经济问题,他们设立了这些最令人舒服的假意识形态和虚拟符号——发电厂里的天神邪教、高音喇叭中的午后茶会、性病诊所中的“青春之源”俱乐部、猴腺医生。

无论怎样,德国自从陷入了越来越不现实的地步……就必须靠撒谎过日子。要想治德国的病,必须要用毒方,要用历史从来没有见到的剧毒药方。

任何人现在都必须恨德国,真心地恨,恶狠狠地恨,只是为了德国曾经有过的荣耀的过去,因为这样才是真正地爱德国——就像父母恨自己不幸走错了道路的孩子一样。

\section{1939年11月\ 拜访弗兰肯斯坦}

我在慕尼黑写这篇文字的时候,这座城市仍然处在贝格勃劳凯勒啤酒馆刺杀案件引发的骚乱之中。报纸流着鳄鱼眼泪哭诉“卑鄙的谋杀团伙针对最伟大的德国人发动了最大胆的攻击”。我认为在慕尼黑至少有一千人对刺杀失败感到沮丧。记者们嘲笑他们自己写的文章。官方对刺杀给出了一个说法,奥托·斯特拉瑟(Otto Strasser)是在英国情报部门的指使下放置炸弹的。这个说法引发众笑。没有人怀疑这场表演是纳粹释放的烟幕弹,为的是鼓动群众仇恨英国人,给希特勒先生戴上殉道者的光环。

我与奥托·斯特拉瑟是通过写信相识的。虽然他是个巴伐利亚人,但从混乱的1932年开始自称是“普鲁士的雅各宾党人”。在那年的夏季,他缠着我叙说一些下流的政治主张。他的兄弟格奥尔格在罗姆政变中被杀,那是个诚实的家伙,但说起话来没完没了。格奥尔格在1932年的晚秋来看过我几次,当时他正好是一颗冉冉升起的新星,他感谢我告诉他1932年末和1933年初的政治内幕消息。我永远不会忘记他在11月里说的一句话,当时纳粹虽然取得了许多胜利,但在大选中失利了。

“他说他要自杀,借此吓唬他的追随者,”斯特拉瑟说。“他是如此地歇斯底里,他们真的不必认真对待他。他是不会去自杀的。他的胜败在此一举。如果我还算是了解他,我认为他会采取绝望的最后一击去夺权的。如果最后这一下失败了,他就完蛋了。他就会像雾霭一样散去。”

格奥尔格在罗姆政变中因反对希特勒而丢了性命。我听说他腐烂的残肢被发现在一片庄稼地里。当他的孩子被告知父亲的死讯时,他们的反应是极富德国精神的:“他(希特勒)枪杀了爸爸,但他仍然是我们的元首。”斯特拉瑟有一位朋友叫格拉塞(他在罗姆政变中被杀死在安马利街自己的住宅里),格拉塞的妻子在听到丈夫的死讯时也说了极为相似的话。

我在皮尔森湖(Pilsensee)上的恒晨多夫(Henchendorf)待了一周的时间,拜访了我的朋友克莱门斯·冯·弗兰肯斯坦。就在开战前两周,克莱门斯还在伦敦指挥了一场音乐会,是丘吉尔的座上宾。与朋友待在一起的这几天,我渐渐恢复了元气,当时正值深秋,湖面上显得很忧郁。我谈论起刚出版的斯泰芬·乔治\footnote{Stephan George,1868—1933,德国诗人。}的信件,这些信件是写给雨果·冯·霍夫曼斯塔尔\footnote{Hugo von Hofmannsthal,1874—1929,奥地利作家。}的,这些信件显示出令人难以置信的傲慢。我告诉克莱门斯我有一次与乔治见面的详细情况,当时这位作家坐在一张高大的扶手椅子上,旁边有两个银质烛台,他询问我对亚里士多德的看法。两个小时之后,这位海德堡的国王斯泰芬,以极大的热情撕咬火着车站普通候车室提供的牛排和泡菜,满嘴流油。

之后我们谈起了汉斯·普菲茨纳\footnote{Hans Pfitzner,1869—1949,德国作曲家。}写的一封奇怪的、令人难以置信的信件,他在这封信中向德国戏院的总督抱怨——他作为一位德国大师,被埋没了,威尔第(Verdi),就是那位其作品“浸透着野蛮和鲜血”的作曲家,他的作品却被不断演奏……

有趣的比较:普菲茨纳,喜欢沉思,业余的娱乐音乐作曲家;威尔第……他敢把自己的名字与音乐巨匠的名字相提并论,而音乐巨匠的音乐流畅得就跟他的气息一样。

我们就普菲茨纳的事谈了很长一段时间,演出他的《爱情花园的玫瑰》时,玫瑰色的纸花像下雨一样落在舞台上。在《帕莱斯特里纳》第二幕中毒药的剧情。我坐在宫廷剧院看后一出戏没完没了的排练。保罗·格拉纳(Paul Graner)注意到一件事,普菲茨纳看着台下坐着的配角、歌手、临时演员,满脸令人讨厌的不高兴劲儿,然后偷偷离开了剧院。“有些人在笑,他把这些人的名字记录下来了,”格拉纳说。

普菲茨纳还有一个随便改变乐器演奏家的音乐的习惯。有一次,他看到双簧管乐谱的上方写着“垃圾”二字。他立即跑到指挥那里,要求解雇那个双簧管演奏家。最后,那个双簧管演奏家只被罚了5马克了事,理由是“为了养老基金的缘故”。

柏林大戏院的一名小提琴手告诉我,普菲茨纳最近在歌剧院指挥演奏威尔第的咏叹调,有人鼓掌,他打断掌声说,“别笑。这不过是一曲街头音乐而已。”普菲茨纳在音乐界仅是个辛苦的钟表匠,他以小人之心度君子之腹,仇恨才华横溢的威尔第是应该的,是完全符合逻辑的。

我与克莱门斯相识足有30年了……那时他光彩照人,刚被老董事们提名做皇家剧院的总监。1943年,纳粹解除了他的职位,解除他职位的方式很有教育意义。有一天,克里斯蒂安·韦伯(Christian Weber)在慕尼黑市议会宣布,不能再把慕尼黑歌剧院视为一个在文化上很重要的机构,必须做出调整。为了显示德国人民的精神状态,我要简略地总结一下他和被他批评的人……

\begin{quote}
克莱门斯·冯·弗兰肯斯坦先生:

职业:作曲家,许多大剧院都把他的作品当作是保留剧目,世界闻名的指挥家。被解雇了。

住址:慕尼黑市韦斯特普耳区一条配备着普通家具的小别墅里。

克里斯蒂安·韦伯先生:

职业:在做出慕尼黑大剧院的判断之前,他是蓝野猪酒馆的保镖;几次因为打架斗殴被判。作为希特勒的亲信,他如今是慕尼黑赛车协会的主席,还是慕尼黑市逊纳菲尔德大街一家妓院的老板。

住址:慕尼黑市的主教宫,富丽堂皇,1782年教皇皮亚斯六世曾在此居住。

\end{quote}

我还要记录另外一件事,第三帝国的元首发布了下面的禁止命令:

\begin{quote}
不许谈论纳粹高官过去的和现在的私生活;

在刑事法庭上不许出示任何有损这些新培养出小神仙的证据。
\end{quote}

《公开诽谤》杂志上刊登了一首诗:

\begin{verse}
谷子变成了种子,

国家已经变质,

我们蜕变了,

兴奋了坏孩子。

过去发生过的

如今再次出现:

好人没有了,

坏人随处见。

一旦这卑鄙

像冰一样破碎,

人们肯定说这是一场黑死病。

到那时荒野上的男孩子

会用稻草编织出一个人形,

把痛苦变成兴奋,

那古老的恐怖变成闪电。
\end{verse}

这首诗是高特费里特·凯勒\footnote{Gottfried Keller,1819—1890,瑞士诗人。}写的,几乎以离奇的准确性预见了未来。凯勒的这首诗,如今在德国非常流行。几乎家喻户晓,大家都在朗读——实际上,我是在施瓦宾(Schwabing)的施泰尼克酒馆听到的。老施泰尼克亲自朗读了这首诗,在场的顾客都吓坏了。盖世太保很气愤,但无法把一首诗送进集中营,也没有阻止在人群中传播。我们的状况还没有坏到不许听凯勒的诗歌的地步。

\section{1940年1月\ 纳粹德国的女人}

我曾经提到过的尤妮蒂·米特福德自杀了。她在慕尼黑的一家饭店里向自己开枪,但只是负了伤,没有死。后来,她被带回了伦敦。在伦敦,她用毒药自杀,这次真的就死了。这是一件大事,不仅是因为她视自己是未来的德国女王,而且还因为她还爱着我们德国的小白脸。严肃地说,在表达了对死者的尊敬之后,我仍然要说历史上的男疯子破坏力相当大。然而,能达到男疯子那个程度的女疯子造成的破坏更加恶劣。最坏的是那些假装出来的救世主。这样的人物,我们有不少,街上的人称她们是“纳粹母”。英格兰也有自己的这类人物,这类女人系着甘地式的白腰带。英国人应该心怀感激,因为如今英国又少了一个这样的女人。

在此期间,又有一桩丑闻在慕尼黑流传开来。这桩丑闻与费舍尔先生有关,他是希特勒在花匠广场(Gartnerplatz)的剧院的总监。他还是瓦格纳省长的亲信。不过,慕尼黑警察局长艾伯斯坦非常恨他的政敌瓦格纳省长……最近,费舍尔先生与一名极为年轻的女士在女王酒店吃饭,并且狡猾地定了一个双人间过夜。大约午夜时分,他俩上楼了……稍后,一阵刺耳的尖叫声传遍了各楼层。酒店里所有人都跑来出事现场看热闹。住在旁边房间的两个年轻人冲进了发出叫声的房间。他们看到那个年轻女人穿着的睡衣被撕扯开来,而费舍尔先生身上除了项链,一丝不挂。那女孩哭诉她还“不到15岁”,显然费舍尔先生想强奸她。然后,她用慕尼黑郊区吉辛的方言当面咒骂费舍尔先生。

那两个前来救人的男人亮出了身份,原来盖世太保。由于那女孩年幼,而且哭得厉害,因此费舍尔被捕了。酒店的老板讲述了幕后情况:那女孩和盖世太保都是艾伯斯坦安排好了的,这样至少可以消除政敌的一个追随者;费舍尔先生是个笨蛋,自己落入了圈套。他现在应该在检察官的办公室里,然后会被投入监狱。但我怀疑此事是否会发生。可以肯定,他一定会像软木塞一样从这片污水池里浮出来,再次大声哭喊起来。不久之后的某一天,他又会恢复活力,洗刷干净罪名。整个过程需要花点时间,这就如同要给纳粹装甲运输部队的头目欧登伯格(Oldenbourg)恢复名誉一样,他因贩卖白兰地酒而入狱。

朱利叶斯·施特莱彻(Julius Streicher)是第三帝国反犹太人运动的男高音,但被一组省长组成的陪审团宣判有罪,因为他接受了纽伦堡一个富裕犹太人的贿赂。有传言说他被枪毙了。但我从一开始就相信他毫发无损。就像我预测的那样,施特莱彻这个在纳粹夺取政权前几年就曾造过伪证的人,安全地度过了危险期。如今,他住在抢来的豪宅里,我认为他就住在那里。

最新的消息说德国的“大救星”\footnote{指希特勒——编者注}也有情人,名字叫爱娃·布劳恩(Eva Braun)。当然,我们大家都知道这是怎么回事,这位女士可以被称为“正式情妇”。她的情人把她安排在上萨尔茨堡(Obersalzber)的一栋豪华别墅里,两人距离很近,随时可以见面。她在这里扮演第三帝国的第一夫人的角色,甚至可以说是德国女王,她向人们极为公平地展示着自己的魅力和优雅,如果她不这样做,就会被追求者投入集中营。一名喜欢恶作剧的官员偷听了两人之间的一段长途电话,他说他听到希特勒基本上是爬在自己金发碧眼女朋友的胸脯上哭诉他被注射了大量荷尔蒙和维生素。注意,上萨尔茨堡有一大群年轻姑娘做妾群,侍候着“皇帝”,就如同她们的前辈侍候博克尔松一样。当年,以色列第一个国王索尔沮丧的时候,年轻的大卫就为国王弹七弦琴。与此类似,当这位曾经住在慕尼黑贫民窟里的纳粹皇帝不高兴的时候,这些年轻姑娘要为这位纳粹皇帝跳舞。

我是个保守主义者,但我要郑重地说,即将到来的革命会给德国最后一次恢复秩序的机会。如果让这次机会流逝,现状就永远不会改变了,中产阶级就会一直沉陷在污水坑中。我要说这是所有普鲁士贵族的共同命运,几乎没有例外。

我们德国的伯里克利还涉嫌与另一桩小事人牵连。1930年,他的侄女自杀身亡。从来没有人解释过,为什么这个姑娘会在那年圣诞节刚过之时,就在摄政王大街的寓所里结束了自己的性命。有人说这姑娘与一个犹太人相爱,因为有负罪感和害怕而开枪自杀了……但有迹象表明另有原因。当时似乎有许多细节被掩盖了,因为在魏玛共和国的警察局里和政府中,有许多人愿意为“德国的下一个主人”帮点小忙。

\section{1940年10月\ 病入膏肓的国家主义}

现在,我来到了菲拉赫(Villach)疗养,每天都可以去伐克尔湖岸边泡温泉。阿尔卑斯山就在不远处连绵起伏。这地方有点斯拉夫人的味道,弥漫着秋天的萧瑟之气,这让我想起了地处边疆的马祖里湖,那是德国南部一个四面环山的悲凉地区:女孩手帕的化学颜色刺人眼;污秽的餐厅提供的沙拉里有一股机油味,这都是战争的缘故;这里就像电影一样到处是可悲的贫困现象。富裕的东提洛尔(East Tyrol)就在附近,两者之间形成强烈对比。

天气不好的时候,我只好躲在旅馆的房间里,这里有一股巴尔干人的味道。街上穿得整洁点的人只有交通警察。

温泉区里有很多疗养的人。许多人的前额飘落下一绺头发,这是维也纳房屋管理员的标准发型……与我们德国的那位流浪贵族的发型一样。我在旁边的更衣室听他们说话,他们的声调颇有巴尔干人味:他们谈猪肉的价格、谷物买卖、女人等等。有时还会讲几句有关希特勒的笑话。不过,这种情况极为少见。这里的人不太关注他——这里是边境地区。

如今这个富有启发性夏季的回忆像洪水一样涌上我的心头。我记得在初夏的那些日子里,蓄着长胡子的长辈们簇拥在公告亭前看胜利的消息,他们的眼睛里闪耀着贪婪和幸福的光芒。他们从来不曾想过,希特勒的胜利,将会把他们原有的富有节制的商业道德变得完全无法辨识。我再次看到,大家就像被一系列成功的政治抢劫冲昏了头脑一样,看到新闻影片里有人被烧死,报以雷鸣般的掌声:坦克爆炸了,坦克中的人燃烧得像火炬一样跳下坦克,看到这样的情景,残忍的暴民狂喜地大叫大喊。我看到:喝得酩酊大醉的纸牌玩家用啤酒杯瓜分世界各大洲;如果没有听见“希特勒万岁”的问候语,邮局职工立即就大眼瞪小眼;女速记员穿着男朋友从法国偷回来的丝袜;休假回来的英雄,故事讲得天花乱坠,说什么用香槟酒的泡沫刮胡子……

1914年的热情完全不能与此相比——在第一次世界大战时,牧师的妻子们在运兵列车的梯子口派送薄薄的三明治,她们的脸色流露着极度害怕的表情。人们觉得每扇门、每扇窗户都有危险钻进来。为了麻痹自己的恐惧,人们在看到开赴前线的军队和军事动员机器在平稳运作时,便高兴地大声呼喊。

如今发生的情况却截然不同:带着恶意、狡诈、匪气。在1914年的时候,德国中产阶级对将军们和工业投机家们轻率地开动的吞噬人命的轮盘赌毫不知情。人们还仍然保持着过去中产阶级所拥有的传统的稳重和正直……或者说类似于灵魂的东西。今天,灵魂都被淹没在垃圾、污水、鲜血中了。然而,我仍然相信有灵魂,而且每天都祈祷能复现。

如今发生的情况截然不同,最严酷的特征是毫无特征的肉搏。人们只关心他们从巨大掠夺中获得的战利品——在1870年,至少在梅茨骑兵战役中浮现传奇故事。虽然战争招贴画企图掩盖事实,但色当对大家来说就如同一出伟大的戏剧。

今天没有骑兵部队向前冲锋的星光闪耀。大体上看,战役就是长得一模一样的机器在前后运动。这场战争之所以变得如此彻底的机械化,大部分原因是指挥战争的人是彻头彻尾的白痴的缘故。你打开收音机,听到的是车轮上的军队滚滚向前的巨大声音。你能知道的可能就是你这一方死人的故事,剩下的就是诸如西斯尔从法国的图尔昆(Tourcoing)给他的特丽萨送回丝袜的故事,或某些军队出纳员从法国解救出一批白兰地,以及如今所有酒吧都在用咖啡杯喝这解放酒的故事。

威灵顿在滑铁卢说的话,让普鲁士的英名流传百年。在色当战役中,不走运的皇帝在战场上企图用匕首自杀。可如今,人们对色当突破的印象是什么呢?是法国人的悲剧吗?或者是攻占索姆河一线吗?

什么都没有……我敢肯定,三周后,跟我一起在电影院看新闻电影的那800人谁也记不起他们所看到的战役和地点。我一直认为,汽油给人类的损害要比酒精大。我敢肯定美国人和英国人与我们德国人的感受是一样的。但如果一国之国民都像非洲霍屯督部落的人那样尚武好斗,结果将会是很糟糕的。普通德国人都要看星期日足球赛,对足球赛结果总是高兴得大喊大叫,但第二天就忘记了。人们已经习惯于听到胜利的消息,而且必须有一个接着一个的胜利,这对个人来说是很有吸引力的——但人会变得越来越野蛮,贪婪的水平会日甚一日。我能听见可怕的暴风雨之声正在不远的地方低沉地吼叫。

不错,我是德国人,但我仍然要说:每个民族都在自己的地宫、拱门、潜意识里隐藏着魔鬼、虚幻、无法实现的欲望;德国人把整个次序倒了过来,让这些隐藏的东西释放出来,就像把潘多拉的盒子打开了一样。一场暴风雨正横扫苦难深重的地球。德国醉心于夺取胜利,已经病入膏肓。听听战争评论员的话,听听咖啡馆里的话,听听德国军人的话,已经退化为冷血的街头流氓语言。报纸对已经被放逐的德皇大肆攻击,因为他在1916年阻止了那个派遣一支齐柏林飞艇去摧毁伦敦的计划。小传达员口出冷血狂言,那些身上还留存着美好年代风韵的老年妇女,却用粗话骂敌人的政治家,让餐厅的男招待目瞪口呆。

人们在背后私下做着“交易”:倒卖偷来的绘画、雕塑、酒窖,这些东西可能存在,也可能不存在……买卖证券、丝袜、没有主人的法国工厂——里面包含偷来的机器、勺子、肥皂、橡胶制品。在柏林,绝对是每个人都在做“交易”——我最近去过柏林,亲眼目睹了一切。不仅普鲁士贵族的妇女忙于做交易,就连侍女、药店老板、职员、高中生等等也在做。他们嘲笑我,说我过时了,他们觉得我太不合情理,以为应该留在基姆高的峡谷里,在思考现在和将来的幸福生活是什么的时候,让机会白白流逝掉。

这就是今天之德国。确实,德国南部一直怀疑普鲁士人胜利后的喧嚣,早就准备好了消音器。大部分工人和几乎全部知识分子都极为反对纳粹的统治。农夫仍然抱着古老的、顽固的思维习惯和生活习惯,对纳粹的胜利仅耸一耸肩,不愿“参与”其中。

但这又能有什么用呢?工业在幕后操纵;自鲁登道夫时代,总参谋部就受工业的控制。权力用恐怖作工具,工业家们紧抓着这把工具。他们控制着所有影响舆论的渠道,让大众处于麻木的、愚蠢的状态——大众就是那些拿工资的人、办公室职员、政府中的低级职员。此外,商人和堕入现实的贵族也加入到大众队伍中,构成中产阶级的一部分,而且这阶层有新官僚出现,成员的流动性很大。这些人比俄罗斯的布尔什维克更加唯物,虽然每天都在活动,但一点都不理解生活这出戏是如何演的。

自第一次世界大战之后,我头脑里就一直萦绕着一句名言……这句名言为我提供一丝痛苦的希望,不过我从没有无产阶级志向。这句名言来自巴尔扎克的《塞沙·皮罗多兴衰记》:“这是资产阶级在庆祝自己的费加罗的婚礼。”

应该说巴尔扎克的观点是很保守的,这点与我一样。此外,他的观点与民族主义分子的观点有天壤之别。要想做保守派,就必须相信地球上古老的规律不可改变:当清扫这些脏东西的日子到来的时候,地球就会颤抖。

这就是撕裂我内心的地方——同时也是撕裂许多德国人内心的地方,这些德国人跟我一样,认为德国不等于是德意志银行和德国钢铁协会。德国知识界的残余可怜得几乎就要变成菜农,既温顺,又没有组织。为了“国家好”,我必须“调整”自己。具体讲,我被要求去神话第三帝国,那个从简陋的宿舍里走出了的绅士摇身一变成为领袖。我需要假装唱赞歌,为刽子手歌唱,为破坏条约的行为歌唱。我需要加入他们的尖声喊叫之中。当看到敌人像火炬一样从真正发生爆炸的飞机上坠落时,我需要大喊大叫。

是的,这简直就是厚颜无耻的要求,足以让人喘不过气来,因为它要求一个人放弃他在国外旅行和交谈中所了解到的一切,转而听信宣传部的对其他国家的评语——在宣传部里,销售员变成了外交家,教师变成了驻外记者!为了弥合与上帝的差异,我需要采纳卑鄙的漠视上帝存在的观点,因为这个观点对德国有用!我相信自己懂得历史规律和地缘政治的理论,如今却不得不放低身段,与这个国家的暴民和流氓同流合污。我认为这个政权的宪法就是破坏国际协定,而其生存基本上就是依靠宣传!

最近柏林的电影院上映了一部新闻影片,影片中希特勒站在贡比涅森林的那辆历史性的列车车厢前面,在听到法国投降的消息之后,竟然像意大利人那样用一只脚跳舞;看上去简直就像一头装嫩的肮脏老猪,比那个仍然在赎罪的德皇更加不值得尊敬。

在德皇所犯的罪行中,我认为有这样一个,他在与奥匈帝国皇帝弗兰茨·约瑟夫一起看《波茨坦保镖》这场管弦乐的时候,拍了一下穿着蓝色军装的保加利亚国王费迪南的后背,这位国王当时正在弯腰看地图。

然而,我仍然记得从前那个三月的早晨,我们村的一个农夫从镇上回来,带回了老德皇驾崩的消息。君主对社会秩序很重要,他们像外衣一样给民众尊严。那个农夫因为是君主的仆人而显得高贵。我就是在这样的讲究义务和服从的传统中培养出来的。可是就在那家电影院里,周围的乌合之众看到跳舞的元首后竟然狂喜地喧嚣起来,我从来没有像现在这样为自己的同胞感到耻辱。我站起来,离开了。我这样做是可以理解的,但左右的人用肮脏的语言评论我;他们希望我为那个蹦跳着的烂货鼓掌。如果我胆敢说出真实想法,他们可能就会私自惩罚我。

哦,我还记得7月里那个炽热的下午,罗森海姆广场的高音喇叭里传来希特勒的胜利讲演,他在“给伦敦最后一次和平的机会”——这一天,我是永远不会忘记的。空气里充满了贪婪和胜利后人们的狂暴紊乱的欲望,压得人透不过气来。一些老顽固分子威胁说,“我们要用吸尘器去占领英格兰。”有个在后方指挥部工作的勇士,喜欢说大话,毫无例外地挽着一个办公室浪漫女子,装扮成战略专家的样子宣布道,“拿下英格兰最多只需14天。”

站在这群疯子中间,我知道形势在那天晚上就会发生改变。我知道英格兰的回答肯定是“不”,这就让我感到孤独得就像在一个人站在地球的北极一样,尽管我周围站着上千人。

我边写边思考未来的结局,我能想象出德国被英军占领后第一天的情况,一名英国军士想不出干什么更好,便一枪毙了我:由于犯了政治错误,其他国家也纷纷夺取了胜利,这是我能想到的。我远不会错以为这里的人都是恶人,其他地方的人都是好人。然而,我不会漠视一个事实,一个欧洲疯子在德国跳的死亡之舞就要结束了。此人是个国家主义疯子,欧洲必须做出决定,要么消灭他,要么被他消灭。

我为什么要尊敬一个从沾满灰尘的历史画卷中“再次拿了”出来的国家主义这个概念呢?这个概念,在建造德国宏伟教堂的时候没有人听说过,它在1789年前确实是不存在的,可如今纳粹却要传承它。

我必须把爱和恨这些朴素的人类情感看作是一种哲学,这种哲学把重商主义包裹上英雄主义的外衣,代表着资本主义企图夺权的欲望,如今就跟卢梭一样泛着平庸的腐臭味。国家主义跟吉伦特主义(Girondism)的旗帜一样,都是肮脏的破烂货,被伟大的卡莱尔\footnote{Carlyle,1795—1881,苏格兰历史学家,主要著作有《法国革命》、《论英雄、英雄崇拜和历史上的英雄事迹》、《普鲁士腓特烈大帝史》。}称为历史上最恶劣的。只有在彻底的无政府、无目的状况下,这种哲学才能成为一种恐怖力量。当然,法本公司喜欢希特勒——因为希特勒给他的毒气工厂披上了哲学的外衣!

雇用这个阴险的强盗,鲁尔的商人知道这意味着什么。为了保证重商主义的意识形态不面临土崩瓦解的危险,难道我应该假装与德国马车夫的距离要比我与法国历史学家的更近吗?我与这些历史学家维持通信关系已经有数十年了。民族主义本应该保护我们国家的传统,如今却成为了野蛮人彻底蔑视他人的工具,在这种情况下,难道还不许我抗议吗?

为建造一家纤维素工厂而破坏一片森林,其代价是什么?如果一座德国大教堂阻碍了一条高速公路的建设怎么办?当德国被系统地转化为洞窟,精神中心被破坏,民众失去组织形式,混乱无序是唯一形式的时候,残余的德国灵魂的价值有多高?

我们必须清楚地说明一点:如果国家主义真的像其辩护者说的那样,是人类基本的约束力,那么为什么到了相当近代的法国大革命时才被发现呢?为什么这种基本的力量在“尼伯龙根之歌”\footnote{一部用中古高地德语写的英雄史诗。大约作于1200年,作者为某不知名的奥地利骑士。是中世纪德语文学中流传最广、影响最大的作品。——编者注}那个年代没有呢?德国事实上存在1400年了,但没有国家主义。只是到了现在,国家主义才盛行起来,甚至戈培尔也曾开玩笑道,这个由拿工资的职员、疯狂的军人、未被玷污的女打字员组成的混合体是否能被称为一个国家。这个问题如何解释呢?如果国家主义真是有朝气民族的特征,那么这样的国家怎么会在其保护下道德沦丧,古老的传统丢失——人们被赶出家园,有信仰的人受到嘲讽,有思想的人受到侮辱,河流被污染,森林被破坏?

如果说德国的国家实力正处于高位,那为什么我们的话语却庸俗到前所未有的程度?为什么所有社会形态都变得恶劣了?我们怎么会变得如此地背弃协定、如此地不守信用?如果不仅德国官员说下流的语言,就连德军总参谋部和“前线评论员”也说,德国怎么会变得如此下流?

你假装活在我们德国“最伟大的时代”里,试着去建造大教堂,你肯定会被说成用石头亵渎。在收音机里听女声朗读德国神话,你仿佛就进入了一家妓院。说一串曾经塑造这个国家的人名:德文格、施特古魏特、托雷克、施佩尔、赫尔姆斯·尼尔,然后在说“德国”这个词,你会听到一大堆让你窒息的谎言。置身于一群唱海顿歌曲的人群中,你会觉得自己在一个喧嚣场所,像往常一样,到处是噪音,还能闻到男人的尿骚味……难道这就是国家主义?当年,就在大谋略家腓特烈大帝在危难中拔剑自救转危为安的时候,他是否知道国家主义这玩意儿?

现在是1940年,不是1848年。当被问及与“德国”有关联的东西时,我们不会想起圣保罗大教堂;我们会立即想起的是“德意志银行”和“德国钢铁联合会”。所以,让我们把这个简单问题提给国家主义者:

你们这些时髦的人,对技术统治论了如指掌,肯定会同意一个观点,一个国家的地缘政治重要性可以用人们穿越其国境的次数来衡量。但运输业不断使用新技术,随着时间的推移,运输越来越不需要穿越国境了。从前,从默默尔到林道,需要24小时,如今只需要2个小时就够了。所以,技术发展降低了德国的地缘重要性,它已经下降到类似于萨克森魏玛大公国的地位。然而,就像当年的大选帝侯跟着吱吱叫的牛群后面去法兰克福一样,我以相同的敬畏之心看待这块具有相当的地缘重要性的弹丸之地,因为冷酷的现实主义和唯物主义在此地毫无用途。技术统治论,本来应该是极富逻辑性的,如今也自相矛盾起来了!

相互依存和政治独裁是否相容?难道技术不能融合不同的人,并使他们的需求和口味标准化?试想你开着每小时跑200公里的汽车,跑一小时就来到了边境线,一个蓄着大胡子的条顿人摆着手不让你同行,声称这不符合“国家利益”,那么又何必制造跑这样快的汽车呢?

我很乐意看到技术被扔到地狱里去,有许多被人捧上天的概念最后都是这个下场。我预计会有一天科学被摆在次要位置上,因为人类有了截然不同的核心利益。由于科学现在很重要,有权的人以为科学的普遍性能永远保持下去。例如,一个地区有富余的柠檬,另一个地区缺少柠檬。强大的运输工具不断在两地之间穿梭,但仍然无法把柠檬从一地运到另一地。如果这类事不断发生,那么要技术有何用途?技术除了给我们的生活带来臭气、噪音、肮脏、喧闹的人群之外,什么也没有提供。

国家主义,无论多么大声地辩护,几乎马上就要完蛋了,致命一击来自这场暴民的战争。明天,一切都将成为过去,变成一个丑陋的美梦。我不是总想着实现一个统一的欧洲,但我知道如果再不着手实现欧洲统一,代价将是极其巨大的。欧洲必须设法阻止战争,否则欧洲统一的愿望就会在摇篮里看着大教堂被摧毁,美妙风景变成废墟。

今天,我从菲拉赫回家。我站在码头上,湖面在秋天的太阳下闪着微光。一条猪鼻蛇想睡午觉,爬进岩石的阴影中。我看了那蛇半天,那亮晶晶的爬虫也有意识地回头看着我,眼睛里透露出奇怪且悲哀的光芒。家乡有个传说,这种蛇是由陈旧的泥土变的,能吸收人类残忍行径和罪恶产生的有毒酸液。

我站在那里看了很长时间。然后,我走回湖边。太阳在山背后落下去了,我感到一丝秋天的凉意;还感到一丝悲哀,因为又是一年要结束了。我们被欺骗了,因为克虏伯要更多的钱,将军总想自夸。

即便是这条宁静的峡谷,也没有能逃脱他们的噪音的眷顾:在峡谷里的大街上,一个连的士兵正在行进,指挥官是个中尉,骑着一头母马,像是一头在篱笆上跑着的狗。士兵们唱着一首纳粹新编的进行曲。过去的“老进行曲”别认为太“软绵绵的”——这首新进行曲很像是酒吧里的音乐。

昨天,一名尽责的地方纳粹干部在贴满夸张的口号的布告板上又写下一行大字“上帝惩罚英格兰。”到了今天,又有新的内容了。那新内容,简直太不可思议了,非常无法无天,我恐怕是第一个看到的。有人在布告板上把“英格兰”三个字抹掉了,换上了神圣罗马帝国的一个最边远的地区的名字——普鲁士。上帝的愤怒将会降临到普鲁士身上。

\section{1940年11月9日\ 慕尼黑的怪异}

施特劳斯博士是德国陆军系统的心理医生,军官晋升前需要他做精神状态诊断。他告诉我一件事,他问一个年轻人读完《浮士德》的感受,那位年轻人回答说:“唉,那浮士德还是个孩子。但你知道,医生,跟格雷琴做那种事,他不应该。”

伟大的歌德留给后人的遗产就剩下这么点了。

第24次1918年慕尼黑起义周年祭刚结束,我们又谈起了这次起义的事。我是个保皇党,我反对这次起义。过去的情景涌上我的心头:从莱茵兰调来炮兵部队,镇压起义军,而起义军在皇帝宣布退位的晚上之前,一直坚守阵地,他们的阵地就位于我在帕兴(Pasing)的住处的窗户前面。很快,起义军被缴械了……瘦弱的老马,破碎的马轭,饥饿的炮手……机关枪不断地扫射,死人躺在到处是脚印的地上,平躺着的尸体看上去很小,好像已经变成大地的一部分……

此后,新成立的共和国就开始了凯旋阅兵式!古老的慕尼黑中产阶级是坦率的,他们只能如此!旗手没有旗子,赶紧申请,在游行的前几天才拿到采购令;在旗手的后面是老兵方阵,后面跟着一个弯腰驼背的穿燕尾服的老头……哦,我记得人群中有抗议者,他们戴着高帽子,在人群中晃动,就好像革命群众中竖起的矮烟囱。

当时有一辆四轮马车,上面站着几个疯狂的守旧派,他们用榔头猛烈地打击王宫的金属遮护板。还有一幕令人难忘:在伦巴赫喷泉的石头动物上,坐着一个老人,他的样子就跟亚述人有翅膀的神牛一样,但胡子是暗褐色的,一群情绪激动的人围着他,他给他们做一场长篇演说。

这就是慕尼黑的革命。

在本世纪\footnote{20世纪——编者注}初的慕尼黑,负责颁发驾驶执照的官员都戴着高帽子。我记得革命的第二天发生了一件事,卡尔广场有大批民众在游行,宣称要推翻一切皇权,突然人群中喧闹起来:不知何故,有传言说国王路德维希要回来了——这位国王30年前就在施坦贝尔格湖淹死了,但他的臣民都不相信他死了。他为理查德·瓦格纳(Richard Wagner)建造了城堡,但毁了自己。当时大雪封山,瓦格纳乘坐八匹马拉的雪橇,赶马车的是涂着白粉、戴着假发的男仆。

但这就是慕尼黑:极为不讲政治,具有一种怪异的赌徒心理,这点普鲁士人根本理解不了——所以说慕尼黑天生是柏林的死对头。

纳粹这帮人,只想着技术救国,而绝对无法改变巴伐利亚的形象,即使他们再掌权10年也不行。如果他们赢了战争,最终仍然逃脱不了失败的命运,原因有两个:(a)他们没有灵魂;(b)他们不幽默。他们是人类笑声的敌人,他们对幽默的害怕,要超过对一场新战争的害怕。

回慕尼黑:我最近去了一趟慕尼黑,旅馆里没有暖气,服务质量很差,餐布显得不干净。餐馆仅开几个小时,门刚打开,就有一群饥肠辘辘的疯狂暴民涌进餐馆。这群人你推我搡,挤进餐厅,飞奔到最近的餐桌上,蹲坐在桌前,睁着充血的眼睛,露出牙齿,等着上饭。过了一会儿饭来了,饭碗里只有一片纸一样薄的怪肉,漂浮在更加怪的汤里边。整个过程就像动物园的工人给猩猩开饭一样。

但这不是我最后这次去慕尼黑这座被普鲁士人破坏的优美城市的主要印象。在火车站附近,一长队人在逊纳菲尔德大街排队等候,这是个可怕的、令人无比沮丧的大街。我问他们在干什么,有人告诉我,这些人排队等着进妓院——大白天等着进妓院,队伍排到了火车站广场,妨碍了交通——他们是休假的士兵、战争劳工,甚至还有几个妇女。(这几个妇女显然排错了队,她们不知道排此队的目的,遭到其他排队的男人的嘲讽。)我听说每天都是如此。如果有运兵车到站,还需要请警察来维持秩序。遇到这类情况,队伍长度被限制在100人以内,于是在本地区家庭里工作的女仆就会应招来“帮忙”。

现在的慕尼黑就是这个样子。这个明智的体系都归克里斯蒂安·韦伯管理了,他在成为希特勒的红人前,只不过是一家名叫“蓝猪”咖啡厅跑堂的,他如今居然可以批评像克莱门斯·冯·弗兰肯斯坦这样的艺术家,住在皇宫里的教皇殿。最近,当局任命了一大堆陆军元帅和小神仙,这让慕尼黑人的幽默感得以发挥。为了满足戈林对职称的渴望,他获得了“世界大元帅”的称号;考虑到戈培尔的欢乐性格,他获得了半个“世界大元帅”的称号;但克里斯蒂安·韦伯仅获得了逊纳菲尔德元帅的称号。

当然,正如我说的,这个称号与逊纳菲尔德大街的有序繁荣有关,同时也使这位大人物晚上吃饭时有大批漂亮人儿的陪伴。

\section{1941年6月\ 俄罗斯人与信心的限度}

我与德国驻莫斯科大使有点小联系,所以我知道发生了什么,以及即将发生什么。

有件事很可怕,这些人内心异常黑暗,已经辨不清方向,意识不到发生了什么。他们不断做白日梦:“元首正在与俄罗斯谈判。”他们把眼睛睁得大大的,足以囊括半个世界,闪耀着比平时更加贪婪的目光。“俄罗斯会允许我们穿越他们的国土去攻占印度!德军已经到了高加索!”此时,东方的上空正乌云密布!

“明天我们就要踏上印度的土地。”党卫军一级突击队大队长塞梅耳维贝(Semmelbei)认为自己能取代某个傲慢的英国贵族,成为印度总督。德国人对英国人的痛恨,起源于1899年的布尔战争。这是一种暴民情怀,但如今仍然主导着大众的感情。这是政治寡头的一种痛恨——那些操纵公众舆论的小学教师们,感到生活在一个有阶级的世界里就是一种侮辱。例如,艾琳·塞利戈(Irene Seligo)女士是一名工资待遇不高的《法兰克福报》驻外记者,她不能原谅英格兰没有给予她大使级的待遇。她以为,自己一到英国,就应该立即被接到白金汉宫,接受国王和王后的款待。

在这些表面现象之下,是大量隐藏着的欲望。有人想移民到英格兰去,盼望获得一片咖啡种植园。还有人想把英国的衣服和烟草运回国。我国的那位被称为标准的德国妇女的“有经验的秘书”,正盼望她的党卫军未婚夫能给她运回齐本德尔式家具,以便布置在她梦想中的四室住宅中。

柯思亚·露西滕贝尔格(Kostja Leuchtenberg)告诉我,纳粹已经完成了他们的“经济计划”,海外职位都有了人选,尼日利亚、肯尼亚、西南非的所有职位空缺全都给了德国工程师。

最近,我看到了詹宁斯拍摄的电影《克鲁格总统》。看到妇女俘虏营的场面,观众没有激动。对布尔妇女施加的残忍暴行,也没有让观众激动。然而,当观众看到一位英国贵族,戴着嘉德勋章去拜会维多利亚女王时,他们这才激动地大喊大叫起来……这就叫白蚁的仇恨,白蚁不把好东西啃成碎末绝不善罢甘休。德国的工业化,造就出一大群白蚁。这就是克虏伯、蒂森、劳士领、霍希等诸位先生的社会理想。

哦,仍然有几个人还没有被吸入这个白蚁堆中。知识分子还没有,他们曾经是德国社会最好的成员,如今仅占总人口的3\%。农民也没有,他们在任何时代都是社会的基石,他们没有被宣传所愚弄,他们知道劳士领先生所倡导的“扩展工业平原”是个威胁。慕尼黑也没有,这个被认为是“运动的摇篮”的地方,跟纳粹划清了界线,正扮演这旺底(Vendee)在法国大革命期间的作用。

3月底,德国坦克部队沿着维纳公路向西南方向去“惩罚”塞尔维亚那个小国家。我看到路边站着一个老农民,每当有一辆坦克轰隆着驶过时,他就会猛地吐出一口唾沫。当赫斯(Hess)乘坐飞机去英格兰后,大多数农民表示高兴,他们说,“王储走了,”他知道事情的结局,所以提早引退了。

显然,当年拯救老德国的是知识分子、农民、巴伐利亚人。社会大众,就是那堆白蚁,正想着让德国和俄罗斯人谈一笔交易,要知道当时东线战场还没有开火。后方没有人知道真实情况。从来就没有一个民族会如此绝望地、愚蠢地跌入大灾难之中!

此外,从来没有一个民族的领袖是如此地恶劣、如此地不负责任!舒伦堡(Schulenburg)是个有风度的老派人物,在莫斯科深受尊敬。1940年冬天,在莫洛托夫来柏林之后,他警告德国可能会受到攻击,但希特勒甚至没有接见他。考斯纯(Kostring)是驻莫斯科的武官,被希特勒大骂是亲俄分子,因为他呼吁重新评估红军的实力。这只晴雨表,就是不说“天气晴朗”,于是他们把他打碎了。

在我还是个男孩子的时候,德军总参谋部的军官来我们家,我听到了他们的谈话。他们都是老毛奇训练出来的,以极为谨慎的态度对待俄罗斯人!可如今的参谋官都是鲁登道夫训练出来的,他们仍然按照第一次世界大战的情况做计划。虽然他们的教官已经死了,但他们仍然打算派步兵方阵发动进攻。

德国的工业家正计划用廉价的无线电和消费品淹没俄罗斯!腐蚀俄罗斯人的办法就是向他们提供大规模生产的电子消费产品。那些生活在伏尔加大草原上的俄罗斯人是难以解开谜团,那个德国西部人永远理解不了俄罗斯人——俄罗斯人想要自己的体制,不想变成西方人——可如今,聪明的德国生意人却把俄罗斯人视为德国人。

那种认为俄罗斯人可以用小装饰品、废弃的大礼帽拉拢腐蚀的观点,是极其愚蠢和傲慢的。这是第一个低级错误。第二个低级错误是低估了作战距离。我在10年前去过一次俄罗斯,看了看乌拉尔山脉北面的村庄和伯朝拉河\footnote{位于俄罗斯欧洲部份的东北部——编者注}流域,那时十月革命已经有14年了,当地人还不知道沙皇已经下台,甚至不知道爆发过世界大战。

但最糟糕的是低估了斯拉夫人的灵魂。如今知道了,但噩梦已经成真了。我永远不会忘记1912年我去圣彼得堡时听说的一句话。当时有个俄罗斯农夫平生第一次看飞机起飞,他说:“他很可能每月拿30卢布,最多35卢布。每月35卢布,他就敢违抗上帝的意志!”

德国的技术官僚由于不理解这句话,所以觉得他们很低级。但他们会在俄罗斯寒冷的巨大领土上遇到这些农夫,这些农夫的表现绝对不会附和他们梦想中的哲学;在这个由这些低级的农夫构成的恶鬼世界里,宣传是不管用的,他们可以放弃一切,但绝对不会放弃他们信奉的上帝。

上一个复活节,我与柯思亚·露西滕贝尔格谈起这件事,他两年前刚从兰特金矿(Rand mines)回来。他是个俄罗斯人,既了解西方,也了解东方。他认同一个观点,这场战争在历史上第一次让斯拉夫直接与西方对垒。在这里,虽然希特勒不断谈论德国的“命运”、“历史上最强大的国家”,但德国像其他西方国家一样对俄罗斯既轻蔑又不信任。俄罗斯在24年前遭遇不幸,如今饥寒交迫,就像我曾经提及过的亵渎者,按照陀思妥耶夫斯基说法,俄罗斯人距离上帝比距离那些怀疑上帝的人要近。

昨天,黎明时分天气就是炽热的,我拧开收音机的旋钮,吃惊地听到戈培尔宣布对过去的盟友开战。我把收音机关上,心情十分忧虑。这场战争很可能会吞并我,吞没我的财产,吞没我的生活,吞没我的孩子。我很有可能会被希特勒这新花招拖入漩涡。

然而,我的第一反应是大笑。我从来没有放弃对这个民族的内核的信任,虽然目前被掩盖起来,无法分辨,但仍然存在着。这个国家必须经历一次伟大的自我毁灭,才能消除丑恶现象,这个学习过程是痛苦的,代价是巨大的,最后才能放弃对邪恶的克虏伯、霍希、廉价收音机三位一体的信仰,转而去信仰真正的神灵。

撒旦由于过度自信,超越了自己能力的限度,落入法网,从此再也无法自由活动。这就是事实,我内心因此而感到无比愉快。我恨你。我醒着时候恨你,在梦里同样恨你;我恨你破坏人的灵魂,破坏他们的生活;我恨你,因为你是人们笑声的天敌……哦,你看你也是上帝的天敌,我恨你。

你每次讲演都在嘲弄被你压制得无声无息的精神,但你忘记了思想产生于悲愤和孤独,思想的力量比你所有的折磨人的手段更加致命。你威胁所有反对你的人,但不要忘记了:我们的仇恨是致命的毒药。那毒药会慢慢地流入你血液,当我们的仇恨达到你血液里足够的深度的时候,你会大喊大叫地死去。

让我这一生就去完成这项任务,让我的死成全这项任务!这个承诺来自那些真正遭受你打击的那些人,现在我把它记录下来,因为你可以这样做,我也可以这样做。

如果你把上帝驱逐出地球,你就会在地球的底下遇到上帝。我们这些潜伏在地下的人,会高兴地给上帝唱歌……

\section{1941年9月\ 德皇的不完美}

最近,在上巴伐利亚的小火车站加尔兴(Garching),我第一次看到了一整列车的俄罗斯战俘。

我应该说不是看到的,而是嗅到的。一条支线上停着一列闷罐车,夏日的微风带来一股股人类粪便和汗液的恶臭。我走近一看,发现粪便地板缝渗漏到了铁轨上。“他们挤在那里就跟牲口一样。”说这话的军人似乎不赞同用这种办法对付这些毫无防卫能力的人——实际上,他似乎很不高兴。“他们在战俘营里饿得没有办法,只能将地上的草拔起来吞掉。”

我们这个地方发生了一件事。有一个农民家里很穷,儿子最近去了趟美国,带回来惊喜。父母虽然穷得跟要饭的一样,但名声很好。我欢迎儿子回国,他俩拥抱了儿子,并做了一顿大餐。他俩的浪子大吃大喝完,上床睡觉了。那天晚上,他曾向父母炫耀自己的几百美元的钞票。儿子睡下后,父母为儿子的事争论了很长时间。最后,父母达成了一致意见,母亲从厨房拿来一把大菜刀,切断了儿子的脖子,就因为儿子的钱来路不正:这里的人民是诚实的,换种说法,他们是正直的……

长时间以来,我一直信奉着一个理论:正派的人一定会采取从来未见到过的方式去反抗当前的种种恐怖现象——在这背后隐藏着一个极广阔的进程,将引发巨大的精神错乱,释放出大量的丑恶。当我把我的这个理论说出来的时候,遭到了别人的嘲笑。这些人说我在捏造噩梦,他们对我说,人在战争期间心理会变得粗糙。随他们说,我的理论需要数十年才能得到证实。

现在看来要扩展这个理论。几个好人的死,必须算是这场社会病的征兆,仿佛他们的死是计划中的事,就像符合某种可怕的逻辑一样。克莱门斯·冯·弗兰肯斯坦去年冬天病了,就在他计划来看我的前几天。症状似乎是流行性重感冒,并按此进行了治疗。但他的病情没有好转,被送进了医院。

最近,我去看了他,他瘦得让人害怕。我的一个医生朋友今天送给我一份慕尼黑医学周报,上面罗列了几个肺癌的病例,是按照字母顺序排列,这个不慎重的办法让我很痛苦,因为第一个病人就是克莱门斯:他是洁身自好的人。在我看来,克莱门斯是德国最高贵的人!

就在同一天,克莱门斯的表兄把他严重的病情告诉了我,仿佛命运就是要带走我所有的好朋友,而独孤是我们苦难的一部分一样。他的表兄是厄文·舍恩波恩(Erwein Schonborn)伯爵,在维森特海德(Wiesentheid)有一处恢宏的地产,是前德意志帝国首相霍恩洛厄(Hohenlohe)的侄子。他有谦和的脾气,故意不去做外交官和法官,却宁愿终生做医生。在经历长时间训练之后,他成为一名外科大夫。他的早餐在一家沙龙里吃,这家沙龙的墙上挂着拉斐尔编织的挂毯。然后,这位极为富裕的贵族,离开沙龙里的客人,骑着自行车去看病人——他给人看病从来不收费。如今这位极有文化素养、拥有众多朋友的好人,在多年的操劳之后患病了。

我与弗兰肯斯坦做朋友,不仅是因为有共同的运动爱好,共同的人生经历,还因为对生活有共同的态度,相信未来有好日子过,这点实际上是我俩之间最重要的共同点。当我想到我就要失去这位极为看重能与我一起改变这个国家现状的人时,我开始战栗起来。

戏院里,灯光闪烁,然后熄灭了。舞台空空荡荡,一股冷风从后台吹来。观众椅子上只有小虫子留下来了。在一片孤寂中,面对着散落的观众,最后一幕必须演完。

当然,柏林可没有这么忧郁!柏林现在是信心十足,在胜利的浪潮中乘风破浪,好像回到了威廉皇帝最惬意的时期。有功的人,都获得了希特勒先生丢给他们的大好处……有几个专门为这个政权的小神仙们准备的餐厅,他们在这些餐厅里吃“工作早餐”时,以极高的效率做交易,就好像每天都在过生日一样。我上一次去柏林,我去了几年前去过的一家餐厅,那里普鲁士贵族的后代的愚蠢行为曾吸引了我全部的注意力。这次与我跳舞的是K夫人,我初看到她,感觉她像一个餐厅餐具柜。这个女人的乳房巨大,这种富态的样子是她那个阶层的女人40岁后常见的。

那位曾经的身材苗条的女人,打开手提包,露出一对漂亮的黄铜蜡烛台。根据她展示出的证据,这对蜡烛台曾经用来给拿破仑在圣克卢(Saint-Cloud)的桌子照亮,那地方很久以前被大火烧毁了……

显然,被偷走了——但如今是在战争中的世界,难道不是吗?我不想要这被偷来的东西,于是借口没钱加以拒绝。可那女人竟然给我上起了经济学课。她对我说,银行太喜欢发钱,纸币自然就会贬值,我作为两个孩子的父亲,要抓住眼前转瞬即逝的机会。除了蜡烛台,她还拿出法国白兰地、巴黎的内衣,最后还说能提供一对纯种锡利哈姆犬。这对犬是K夫人的一个在法国雷恩的朋友抓住的。显然,手提包放不下犬,无法现场展示。

我说都不要,那女人的热情急速减退,转身走了。我估计她认为我是一个傻子。她宽阔的后背晃动着,传送出她深深的蔑视。

保罗·维格勒(Paul Wiegler)是乌尔施泰因出版社坚持工作到最后的人,他也为乌尔施泰因兄弟工作,这家出版社坐落在在科赫大街上。他告诉了我出版社老门卫的故事。不知道何故,这个老员工仍然与此时身在纽约的前雇主保持着联系。根据他获得的信息,原先是百万富翁的乌尔施泰因大哥,如今年事已高,生活在饥寒交迫之中。虽然我不认识这一对不近人情的乌尔施泰因兄弟俩,但我过去有机会了解他们的像蚂蚁一般小的业务和清教徒的行为规范。如今他俩一贫如洗了。他俩也许有兴趣知道如今有了一个官方机构,叫帝国商业道德办公室,配备很全,不仅有电话、卡片目录,还有秘书。

我利用这个机会去拜访了一下弗里德里希·利奥波德(Friedrich Leopold)亲王,他是我父母的亲密朋友。亲王后是已故皇后的姊妹,还是德皇的表姐。德皇和皇后在公众的要求下,离开了祖国。她还是弗里德里希·卡尔(Friedrich Karl)亲王的儿媳妇,这位亲王是马尔拉图战役的指挥官。她如今80岁了,仍然很敏捷,根本不去想他的皇家姊妹。她既无偏见,身体又好,她总是骑自行车穿越整座巨大的城市,从哥林尼克(Glienicke)来施特劳斯贝格(Strausberg)看我的姻亲。当她谈及自己的皇家表哥和他做皇帝的举止时,态度一点都不随和。

当然,她岳父辉煌时期在哥林尼克留下的遗产几乎没有剩下什么了。大部分城堡被卖掉了,她手中的资本以悲剧的方式缩减到零。她有三个儿子,一个儿子在上一次大战爆发的第一天就倒下了,第二个儿子在赛马比赛时死于事故,最后一个儿子由于养成了不良的生活习惯,让她极为悲哀,这个结果给她造成双倍的伤害。纳粹对这类事的嗅觉很灵敏,上台后不久便就听到了风声,从此一直勒索这位母亲。他们定期把亲王投入监狱,然后索要赎金。亲王被释放后几周,马上又会被抓入监狱,游戏重新开始。后来,亲王去找戈林先生谋求帮助,被安排在接待室等待,接待室坐满了打字员和党卫军的笨蛋。在等了2个小时后,前皇家普鲁士步兵上尉(已经退伍了)和当今世界大元帅的榜样出现了,他双手插在衣兜里,嘴里含着永远离不开的香烟,向这位前梅斯战役获胜将军的媳妇打招呼:
\begin{quote}
“你想要什么?”
\end{quote}
我们谈到死去的德皇。当大皇子的死讯传来时,他的反应令人难以忘却。威廉二世本应该重视儿子的死讯。为了安慰悲伤的父母,他发去一封电报:“地位高责任重。”这封电报就这么几个字。

我必须承认一点,随着时间的推移,我对那个已故的、被人遗忘的德皇有一些更加具有善意的想法。我似乎觉得把他流放到荷兰的多尔恩(Doorn)等于是赎罪了。我只见到过他一次,当时他还“在位”,有些军事细节或什么其他东西让他生气了,他便大声叫喊,猛烈地摇晃着短粗的手,这很不符合他作为皇帝的身份。后来,当他知道了我不为普鲁士王室效劳,而是我对维特尔斯巴赫王室\footnote{1180—1918年间统治巴伐利亚的德国家族。}效劳的时候,德皇在多尔恩亲手雕刻了一个木制纪念品送给我,这让我很高兴。这份霍亨索伦家族的装饰品让我开始重新认识了德国的君主。

如果说我比普通德国人对那位已经过世的君主有更多的了解,那是因为我的圈子与宫廷一直都存在联系。宫廷的老人,比如各种代理人和负责人,他们很乐于在马祖里湖区的狩猎宴会上透露宫廷里的内部细节。我们这个圈子里的人比媒体早五年就知道克虏伯和奥伊伦贝格的丑闻。我知道一件真正的哈姆雷特式的故事,此事发生在1896—1897年间,当时威廉皇帝还在位……

我的舅舅马塞尔是德国驻圣彼得堡大使馆的副官,经常往来于圣彼得堡和柏林之间,中途喜欢在我父母的家中休息。这使得我们能以最快的速度知道柏林发生的所有情况以及有关沙皇的传闻。我记得有一个七月的早晨,吃完早餐,我在父亲的书房里看报,我爸和舅舅这两个男人在餐桌前并排坐着。

我应该解释一下那段时间报纸说徳皇在“霍亨索伦”号差点负重伤的事。当时他在甲板上散步,冒着烟的船帆从桅杆上掉下来,砸到了他的眼睛。报纸说,德皇本人虽然仅负了点轻伤,但很痛苦。在这起小灾难发生后的几天,报纸又登出一则布告,对这场事故负有看管不利的值班军官冯·汉克海军上尉,在一次郊游中死了,他的尸体和自行车被人从挪威的一处瀑布下面拖了出来。

如今,舅舅告诉了我报纸上没有说明的后台事件。冯·汉克海军上尉是个狂热的自行车爱好者——那时这项运动颇为流行。他在“霍亨索伦”号的甲板上骑过几次自行车,而当时德皇也在场——德皇恨这项运动。威廉对他产生了敌意,仅让他在一个小范围内骑。当冒着烟的船帆掉下来的时候,正好汉克在值班,这真是最大的不幸。

这时,一件不可思议的事发生了,它足以把这名20岁的男孩子吓呆。德皇下令所有值班人员到甲板上来,怒火中烧的德皇伸手打了汉克一巴掌——在同伴面前挨打,汉克忘了自己是谁,竟然回手打了德皇,这其实是我们都具备的本性。

死一般的寂静。在场的人都目瞪口呆。汉克转身走下甲板。一天后,船到了挪威,这名海军上尉要求上岸,这一要求很快被批准了。那天晚上,这名军官的尸体和自行车被从瀑布下拉出来了。很自然,他自杀了——绝对不是谋杀。他因为侮辱了德皇而自杀。他的亲戚在22年后向我证实了这点。

仅凭这一件失去自我控制的事,就去评判德皇,显然有失公允。他在私下里是一个好心人,但缺乏安全感。然而,一旦他被迫出现在公开场合,他就会变得狂暴,为了克服这种不安全感,并且显示出他是个“能照顾好自己”的人,他才采取一种特殊的姿态,目的是显示出他是个严厉、不妥协的军人。

我的一个朋友曾经亲身经历过德皇变得越来越严厉的这种转变。在一次军事演习中,他负责招待德皇。德皇在他的庄园里散步,愉快地说着话,态度自然,很有魅力。然而,军演开始后,副官们出现了,他突然变得暴躁起来,像是一个世人皆知的那种大叫说话的古罗马皇帝;非常容易被激怒,处处让人感到痛苦。

与此同时,他从来就没有穿过合适的衣服,这使得这位必须做事精确的军事首领具有一层戏剧色彩,让人感到既悲又喜。他的腰带、剑饰、军装总是有让人诟病的地方。一名英国海军军官告诉我,德皇有一次去英国海军部访问,突然提出要去观摩英国地中海舰队编队射击表演,他们对皇家来访完全没有准备。那位英国海军军官形容了德皇爬上英国旗舰舷梯的情形,德皇穿着闪亮的海军上将服,脚下却穿了一双极为不合时宜的夏季专用白鞋,这让大家全乐了。

刚被流放的时候,德皇住在阿莫隆根(Amerongen),一个英国女士曾看见他参加一些荷兰贵族的婚礼。德皇站在圣餐台前,穿着令人羡慕的将军服,带着黑鹰纽带——但脚上穿着丑陋的皮绑腿,我当兵的时候管这叫“裹小脚”。最近,我看到德皇拍摄的最后几张照片,其中一张是他与妻子的叔叔在一起,此人在多尔恩做了10年的贵族长。照片中,德皇安宁地坐在公园的凳子上,穿着一身柔软的装束,手抓着拐杖,双脚舒服地交叉着——看上去就是一个有风度的老绅士。唯一的问题就是他穿那双温暖的鞋,鞋的扣子系得不对——确实有没有照看好的细节。

我说这些不是无礼——我是觉得有趣,感觉德皇是个矛盾的人物:仿佛有一只无形的手,一方面允许他在这些小事上斤斤计较,另一方面又温和地提醒他人类活动中的悲喜剧——“你看,德皇,你应该在这些事情上很精确,但你离完美差得很远”——或者类似的说法。我不相信这些是偶然的事。我相信我们所看到的是仁慈上帝的手。这就好比在情绪激昂的诗朗诵中或政治讲演中,打字员仅暗中修改关键单词中一个字母,整个意思竟然荒谬般地变反了。

人们常说,德皇威廉本性奢侈,唯有这样才能发挥皇室的作用,这个说法流传很广,不过我不相信。我认为最残酷的笑话就是让这个怕羞、缺少个人能力、缺乏安全感的人去统治德意志帝国,而这个帝国自成立之日起,就浸泡在麻烦之中:政教之争、阿尼姆丑闻、社会危机、两代德皇相继驾崩。

我不认为他应该为解除俾斯麦的职务负责任,这件事他不能单独做主,而且他并非主谋。严肃的历史学者不会在这件事上谴责威廉。有没有可能让一个德国保守贵族一夜之间变成工业家?俾斯麦和法本能相容吗?

我认为德国人为了使自己良心不受谴责才把责任推到一个人身上。德国在一夜之间切断了与过去的一切联系,摒弃了自己的理想和偶像,这才在三月的那一天解雇了俾斯麦。我认为德皇仅是代表人民赶走了俾斯麦——他仅是在为时代作最后的表白,因为每个德国人都是一个微缩了的威廉二世:一样的看到发展就高兴,一样的大喊大叫,一样的漫无目的,一起放弃了昔日的港湾。德国人都跟威廉二世一样地喜欢惹事,一样地缺乏机敏,一样地不顾一切地自恋——然而,虽然非常没有安全感,却基本上没有害人的心。

1905年,我去托尔博莱(Torbole)看加尔达湖(Garda),当时那里正在召开药材商的大会。那天,代表们白天开会,晚上带着妻子乘坐汽轮在湖上游览。他们唱着《平静的大海》这首歌,歌声在湖面上回荡——他们非常自信,以为大家都像他们一样兴高采烈。他们悬浮在梦想和现实之间,最终将感到后悔——德国这个符号是无害的,但完全迷失了方向。

我不是霍亨索伦王朝的追随者,或要为其辩护,我只不过觉得这是很显然的事。我不是皇室的内臣,也不是个倾向独裁的人。不过,我想说,这些唱歌药剂师的儿子们拒绝承认徳皇在1918年是他们的真实代表,这才是令人感到耻辱的事。

1914年,七月的最后几天里,在柏林街头,我看到有望不到边的人群站在徳皇的城堡前面,对着徳皇的窗户有节奏地喊道……

\begin{verse}
“我们要见徳皇!”

“我们要见亲爱的德皇!”
\end{verse}

这就是他们叫喊的。他们按着拍子叫喊,人群既看不到边,也听不到边缘——就好像受过训练一样,在一声号令下,一起爆发出激情,大声叫喊起来。

那是1914年7月底的事——可220周过去了,或者说1540天之后,各种极度不恰当的言论和冷嘲热讽都抛向了他。这是在他执政26年之后发生的事,这么长的时间,足以改变领导层了,他的缺点大家都知道了。这位灰白头发的老人在1540天里做的令人诅咒、令人耻辱的错事,难道比他在执政的26年里还要多吗?

我知道君主制不可避免地要被推翻,但我认为方式不对,因为在这件事中整个德国都有责任。我不认为德国人高人一等而有权嘲讽别人,戈培尔在报纸上写的那篇有关徳皇的文章就是一例。相反,我认为每个德国人都要反思自己的罪恶和缺点——特别需要反思的人是德军总参谋部的人员、德国北方的财阀、普鲁士的贵族。

在君主最需要帮助的时候,鲁登道夫在哪里?鲁登道夫手下的那帮将军们,在伙同工业寡头把能力不强的君主拖入那场血腥的赌博之后,他们又去了哪里?就在那个时候,那个灰白头发的普鲁士统帅又在何处?兴登堡是知道他的君主有弱点的,也知道君主那时正需要帮助。他显然可以除了挥一挥毫无用途的手之外再做点有用的。事实上,他提出的建议对将军们最合适:让威廉自愿去国外。

在纸上写下“忠诚是荣誉的标志”不难。难的是一生就宣誓忠诚一次;而且一旦宣誓,就再也不能像还钱要借条那样把宣誓收回来;难的是致死都能坚守誓言。

这才是瑞士卫队要做的事。1792年8月10日,虽然国王逃跑了,但他们仍然用生命保卫空荡荡的王宫:他们忠诚于自己的誓言。

在俄罗斯可能还会取得更多的胜利,甚至有一天历史会欢呼这些胜利是伟大且重要的(我不想相信这点)。然而,这些昨天还在信誓旦旦表达忠诚之心的将军们,不过是见什么杀什么的政治罪犯,他们绝对不会获得那些瑞士农夫的荣誉的。没有人会在他们的坟墓上方放置一个大理石的狮子像。

在过去的这八年中,我们痛苦,我们悲愤,我们忍受着无法忍受的耻辱,这样的遭遇使我们对生活有了新的看法。我们获得了新的生活机会。这次新机会,有可能是最后一次了,我们必须利用这次机会严肃地审视我们自己,自问是否能防止1918年那样的懦弱行径。

“这不好,也绝对变不好,”正如哈姆雷特所言。这话简直就是我们说的。从这些乌云一般的胜利中,根本冒不出来什么好东西,因为乌云里充斥散发着臭气的罪恶。一个把宣传和谎言当作立国基础的国家,是不会有什么好结果的。一个既虚伪又自大的民族,会把自己的罪恶推卸给古老先贤,会向罪犯宣誓效忠,会随时准备追随撒旦,并宣誓效忠,这样的民族是不会有什么好结果的。如今,魔鬼把赌注提得越来越高!

就在人们盲目地痛饮胜利的欢喜之酒时,一场巨大的风暴正向他们头顶袭来。可在今天的德国中,清醒的人却成了孤独者。他因为有见解而孤独,他能看到当结局来临的时候,他就要履行自己所有的承诺,无论是说过的,或是写下的。在过去对生活的众多期盼中,只有一个能留存下来:我们这个时代会要求那些不能融入主流的人去殉难,当殉难的时刻来临时,走上殉难之路的人可以从信念中获得力量。

人类的愿望,如果足够大,真的都能变成现实吗?

\section{1941年9月\ 超越历史的日常生活}

这就是我们今天在德国的生活……

星期一,宣布了一次巨大的胜利。到了星期二,就没有一个人还有记忆了。报道说抓了大量战俘;谁都不信能有那么大的战俘数。每一天收音机里都会宣布新的胜利——吹牛刚开始,我们便关掉收音机,感觉好像是受到了侮辱。

我不理解为什么人们不喜欢质疑。凭什么说这些钳形攻势是“历史上最宏大的”?为什么被包围的敌人比色当还多?或许是口蹄疫的原因,或许是融冰稍早了一点的原因。有时我在想这可能是我们的将军有白蚁堆的思维,他们看不到白蚁堆之外的情形。我又想了想,随后否定了这个想法。可能另有原因,更加复杂的原因,非常怪异的原因,无法用语言表达的原因。

我不知道那些原因到底是什么,但我有预感,就跟其他人有同样的预感一样。这个原因就在我们中间,但看不见,摸不着。

如果事情出乎我的预料,果然是真的,即实际结果与宣传相符,我的感觉仍然如同当今:历史被超越了。

我们的生活里另一件超越历史的事:布鲁诺·布雷姆先生,这位从前犹太人文学圈的招待员,从利沃夫发回大量充满血淋淋真相的信件,报告在那里发现了大量死尸,并归罪于犹太人。所以,我们既没有荣誉,也没有真理或正义,仅过呆板单调的生活。社会下层,就是除了戴纳粹标志的那些人之外的社会成员,都在忍饥挨饿。官员——这些人是从前的裁缝、银行学徒、神学校学生——忽悠大家过前线般的艰苦生活,但他们却自己拿着“外交官的粮票”,价值比普通人的粮票高出三倍。

最近,瓦格纳省长来到我们这个小镇子上,镇上的鸡基本上全被杀光了,供他的随从大吃大喝。我们的素食大王希特勒先生在慕尼黑附近的索恩(Solln)有自己的蔬菜农场,温室周围拦着电网,还有党卫队巡逻。

与此同时,德国下层社会成员能感到德国的食品工业正在疯狂地向化学化工转变。食糖用冷杉树的果肉制成,腊肠用毛榉树的果肉制成,啤酒用臭乳清制成。酵母是化学制品。橘子果酱上了颜色,以便让人觉得是真的。奶油也一样,但其着色物质更加低劣,不能被消化,会伤害人的肝脏,这就是为什么今天有那么多肝胆病人的原因。每个人的眼睛都是黄色的,我有几个医生朋友,他们说癌症的病例在过去四年里翻了一倍。

真正的普鲁士人,都是捡破烂的高手,他们能把本来够人们消费的德国天然产出换成人造物质。罐头蔬菜也是人工染色了的。除了供年轻军官狂饮的酒或者是黑市上军需官们卖的酒之外,其余都是勾兑出来的。肥皂一股臭味,就跟“新德国”的腐败一样难闻。去年冬天我用几个月的粮票换了一双滑冰鞋,只走了半个小时就坏了,原来是纸板做的。有一个故事,一个男子穿了一件用合成木制成的外套,他的朋友在他的肩上拍了一下,他心不在焉地以为谁在敲门,于是说:“请进。”

如此造假的不良后果已经开始显现出来了。树浆发酵产生的气体和粘土面包产生的气味,使餐厅里的空气变得对人体有害。人们已经不再注意屏住呼吸了。如此系统的血液中毒,人们身上出现疮和脓肿,汗液变臭。每天人们都必须寻找食物,看到邻居有家境好的便会产生嫉妒心理,于是大家的行为都变得污秽、懒散,这种情况在不久之前是难以想象的。

附近湖边有一家别致的航海学校,学员都是工业家的女儿。对外,这所学校显得很有势利,对内则是个小妓院。在这个小妓院里,苗条的小姑娘与那些虚张声势、粗俗的、野蛮的教师睡觉。在湖边村庄的小咖啡厅里,我听戈林私人医生的肥胖妻子讲了其中的详细情况,而且非常全面讲解了戈林夫人想实现的人工播种计划是如何实现的。

由于缺少男人,导致怪事频出。虽然不许私藏法国战俘,但法国战俘被认为能给人带来快乐,有德国北部的农妇把法国人藏在独轮手推车上的土豆堆里运回家。

在附近的一个村庄里,一名30岁的丈夫去俄罗斯打仗,她等于守活寡,在她65岁的公公的协助下,在沼泽地里把自己的两个孩子勒死了。在我住的那个道德水准很高的小镇子上,也有了大城市的生活,因为纳粹通过“母亲和孩子”机构送来了德国北方的女性;最后一些本地居民也感染了性病。在战俘营的卫兵的帮助下,这些女人在小镇上建立了一个“快乐岛”。

最近,我在去镇子的路上,听到有人呼救:这群女人中的一个,有个三岁大的孩子,由于她正跟她的情侣在一起玩耍,没有照顾好孩子,孩子掉到河里溺水了。我花了一个小时的时间,想把她救活,但没有成功。孩子死了。那女人最终来了,虽然她因失去亲人而悲痛,但那天晚上我看到她与情人在摆放她死去孩子的房间的窗前散步。

这太过分了。那天晚上,村子里的人为那个女人安排了一场寻找伴侣的音乐会,有大口水罐,有火焰号角,有灭火设备。演出仍然沿用老规矩,有50年的历史了,目的就是要以最简单、最有效的方式维持村庄的繁荣。然而,无知的牧师出面干预,说是违法的。

现在看到一些已经消失的东西又死灰复燃了:这可能是好,也可能是坏,表现出神鬼的贪婪和兽性。我不知道世界是否就要终结了,反正陀思妥耶夫斯基说末日到了。但有一点我知道,人类活动的改变要花费很多年的时间,改变后的就改不回来了。一个傲慢文明的独裁统治就要结束了。

\section{1942年1月\ 愚钝的理想}

这个冬天像暴徒一样扑向我们。令人讨厌的北欧人,最近几年总是在祈求,于是才有了这一系列的严冬。一连八周的时间,荒凉的景象使人的灵魂感到沮丧,大地被白色覆盖了。我在雪中挖了一条通道,足有一人高,从我住的房子通往仓库。站在挪威式的冰雪块山的时候,我发现已经跟二层楼一样高了。

就这样,我被与世隔绝了两个月的时间。去取一磅肉,需要滑冰两个小时。去最近的银行或牙科诊所,简直就等于是去北极,需要24个小时。去慕尼黑,需要在脏得令人恶心的火车上旅行两天的时间,火车上挤满了包裹和浑身肮脏、味道难闻的旅客——平时乘坐汽车,只需90分钟。这样浪费时间,都是因为这个政体拿走了所有的东西,却不给任何回馈。面对这种局面,即使你绞尽脑汁,也无济于事。由于缺少修理工,你必须自己去修电器、补屋顶、疏通管道,这样才能让家庭正常运转。

最近,我在冰封的树林子里散步时发现一只被狗咬伤的小鹿。我把它带回家,由于受了致命伤,它死在我的手臂上——它死前眼睛还流泪了,显得无限悲伤。这是对造物者的控告,因为他竟然让自己创造的一种生物遭受如此大难。有一次,我在南大西洋目睹了一艘捕鲸船捕杀一头母鲸鱼和她的孩子的过程。鱼叉手是个红胡子的爱尔兰人,他不断地把鱼叉插入那鲸鱼的体内。那巨大动物的身体被撕裂了,肠子流出体外,但她依然在被血染红的海水里不断前后游动。虽然她的身体在抖动,但仍试图用自己的身体保护小鲸鱼。看着那脸上长满了雀斑、狞笑着的鱼叉手,再看着那舍命保护孩子的可怜的动物,我意识到世上有撒旦,但上帝也是存在的。

冬天也改变了战争。俄罗斯的雪水中浮现出了一个幽灵,一个复仇的幽灵。我诚实的同胞害怕极了,为了壮胆,他们相信奇迹能改变一切。他们希望用毒气在10秒钟里杀死一个大国的全体国民,幻想“原子弹”能帮忙,只需三枚就能摧毁英伦三岛——对,还有海底隧道,据说正在加莱挖一条跨越英吉利海峡的海底隧道,在一个良辰吉日,德军通过这条隧道,把勃兰登堡\footnote{Brandeburg,普鲁士王国的前身。}的敌人打倒在地。

我在去萨尔茨堡的火车上还听说了另一件奇怪的故事,这个故事与英吉利海峡海底隧道的谣言相关,故事的内容是有关哈巴格航运公司的西奥多·科赫(Theodor Koch)船长。科赫是一位很有绅士风度的船长,外表俊秀,举止完美,他因此有极为出彩的职业生涯,很受公司信任,他受命掌管一艘大船跑北大西洋的纽约航线。我敢说许多英国人都记得这位姿态优雅的船长。由于科赫曾经在帝国海军中指挥一艘轻巡洋舰,在战场他受命负责英吉利海峡中的岛屿,这个岛屿是刚从英国人手中夺来的。但去年秋季,有一名盖世太保的高官,来到了岛上。他俩进行了一次长谈,说话声音越来越高,最后科赫拔出左轮手枪自杀了。

这个故事有一个奇怪的细节,前厅的值班员隐约地听到了部分谈话内容,他有好几次清晰地听到谈话中涉及隧道问题。我很想从与我同行的那个汉堡人那里得到更多的消息,如果他愿意,他显然能告诉我更多,但他拒绝了这样做。

从我这方面讲,我认为这群人在我知道有隧道这回事之前就一直在争论这件事。事实上,这帮纳粹亡命之徒害怕丧命。他们想尽办法逃跑,甚至想跑到月亮上去。

这就是我在这个寂寞冬日的生活。还有其他事吗?最近,在女王饭店住我隔壁的前德国国家银行的总裁亚尔马尔·沙赫特(Hjalma Schacht)先生,如今失去了昔日的威风,他说话非常恶毒,而且声音特别大,足以让我也能听见。他说如果通货膨胀能避免,他会很高兴地相信世上有永恒运动。次日,我在黑尔比希咖啡厅与几个老顾客一起喝咖啡,其中有几个是天主教神学家,他们讨论的重点是应该采取何种惩罚措施。剥光宣传部长的衣服,让他裸体站在海拉布伦(Hellabrunn)动物公园的猴子笼里,观摩的价格要提高,但节假日可以低价。那个大人物,必须把他放在笼子里,带去做环球巡演,“手拿着吊环,脚趾上挂着铃铛”。这样的惩罚,作为再洗礼教徒的国王博克尔松就受过,中世纪的政府把他像金丝雀一样装在笼子里,带到各地展示,让各地的人在喝咖啡时捉弄他,让他在被处死前体验一下绞架下的幽默和警句。

这样的谈话让我很愉快。一想到希特勒光着身子,在百老汇的舞台上演唱《霍斯特·威塞尔之歌》(纳粹党歌),我心里就感到无比满足,这个主意深深地打动了我,无论对公众或个人都极有意义。未来德国应该出现能够产生欢笑的革命,因为德国已经有10年没有欢笑过了。安全阀打开了,一些受禁锢的愤怒可以发泄了。如果像1919年那样,以“法律和秩序”为理由关上安全阀,那就会有人把炸药包投向那些释放政治烟花的人。在一场彻底的大革命到来的时候,让希特勒那帮人殉道,有可能拯救我们,即使从最保守的角度看,我们需要让大众摆脱愤怒,让他们大声吼叫,直到疲惫下来。

我与M先生有过一场长时间的讨论,他是教诸如大众心理学和大众病理学这类社会科学的,但纳粹禁止他教课了。我清楚地记得他也谈到了相同的想法——欧洲人口在过去140年里增加了2.5倍;他还谈及斯宾格勒好几次。斯宾格勒的思维显然很单一,他只关心让婚外生育合法化,呼吁允许曾经当过兵或牧师的农民儿子多次结婚。但我要指出这样有可能产生令人震惊的意外后果。

M先生的谈话主题转向了技术给现代人的日常生活带来的变化。我无法同意他的观点,即“工人只是大众的一部分”。实际上,大部分工人阶级不愿跟着大众宣传跑,而仅是其中一小部分生活条件不拮据的中产阶级。此外,我注意到从前的两个实际例子:罗马帝国和哥伦布登陆前的印加帝国——当民众突然变得对大众宣传很敏感的时候,社会就不太健康了。这是衰败的征兆,当时正好是罗马皇帝卡拉卡拉在位。在这个时期,胎盘成了神秘的迷信物,社会出现明显的腐败迹象,并出现令人恐惧的暴力破坏活动。

这就是大众这个群体的特点:他们傲慢地走到了历史的舞台上,可已经病入膏肓了;突然间成为了社会关注的目标,但无力长时间维持自己的形象;命中注定在空荡荡得令人感到可怕的舞台上,这种现象在公元400年的时候被一名希腊的记日志者描绘下来了——罗马曾经有数百万人居住,如今仅是个有数千居民的小镇子。古罗马广场变成了玉米田,方形石柱在玉米穗中随处可见。

如今还有谁认为人口的增长仅是执行古代“人口繁育”法令的结果?这个法令与现代的人口繁荣景象是极为不同的。人口的增长,并非源自我们如今看到的大城市拥挤的生活条件。事实上,人口爆炸是一场全球的灾难。即使是在我们这个宁静的小村庄,我从教堂的记录中看到,人口在近百年增加了三倍。这是科学发现的后果吗?我比较倾向于认为当代科学是先有了大众群体的后果,因为当代科学领域宽、能力强,倾向于用合成出来的化学毒素取代大自然的产物,而不是相反的因果关系。我还认为,当前科学正努力开发便宜、标准的商品,比如收音机、大众轿车、尼龙丝袜,并想用这些新商品去取代老商品,这进一步说明了这个因果关系。

我们如何看待如今的“健康水平的提升”?确实,如今传染病被根除了,人口平均寿命增加了,束腰和啤酒肚没有了,德国人有了新的形象。但我希望回到过去,我宁愿生活在过去的不好之中!上帝保佑,至少让我还能不时在劣等人群中看到一张真正的人脸,就是过去熟悉的那种没有被美容的德国人脸。但愿历史存在意外,能够让我们摆脱这既疯狂又空虚的面容——希特勒那帮人全是这种面容。

再说说人口寿命:这主要是依靠甄别出身心健康抑或不适合生存的婴儿。我曾经提到过,给男女运动员看病的医生发现,他们很容易出现性功能紊乱的问题。这是大自然对人类的警告,大自然不能允许人类只关注肉体的发展。

就在这类心灵空虚的人繁殖得越来越多的时候,出现了一种对哲学令人震惊的感情,人们感到自己不再与时间有关。世袭阶级没有了,既没有了牧师或帝王,也没有了具有神圣权力的立法者和法官。如今没有人关注哲学,这造成人类的各种活动无法定型。于是哲学理论名存实亡了,因为大学里的哲学家所做的让他们看起来私很像一群很受人尊敬的夜班员,坐在一起玩一场永不休止的塔罗克纸牌戏,而且所用的都是老掉牙的玩法。

形式艺术的宝库已经被攻破,内容遭人篡改,被亵渎了。德国的建筑设计风格变为了“新现实主义”,但结果是建筑结构越来越不现实,就跟一年没有剃的大胡子一样。建教堂被人说成用石头亵渎,写出来的弦乐四重奏曲是如此的虚张声势和无聊,虽然很激昂,听上去像莫扎特曲子,但其实是噪音。

自然界痛恨没有结构的丑陋,即使是珊瑚也有自己的生长结构,但人类却慢慢地变得越来越没有结构,越来越痛恨结构。现在的理想变得很愚钝,级别和职业技能的差异都被认为是荒谬的:教授像个运动员,侍者像个贵族,贵族像侍者的领班。商人像纯种马,骑兵军官在兰特矿业股市里投机。只有流浪者或是强盗还保留着真正的专业特征。

过去只有英雄和大思想家才能有的名誉,如今被流汗的脏手给亵渎了。在今天的德国,只要你是一个啤酒厅里的吵架能手,你就能获得同在色当夺取大胜利的毛奇相同的陆军大元帅的头衔。如果你的样子既像一个马贩子又拥有纳粹党籍,你就可以被称为是一个合格的“政治家”。如果歌德知道像赫里伯特·门泽尔(Herybert Menzel)、约瑟夫·马格纳斯·韦娜(Joseph Magnus Wehner)这样的人也被称为是作家,他肯定会把自己的作品藏起来。如果腓特烈大帝不在库勒斯道夫\footnote{Kunersdorf,1759年,普鲁士在此地惨败于奥俄法联军。}自杀,也会在得知自己与一个流浪汉相提并论后自杀——这是一种令人耻辱的相提并论,简直是德国的耻辱!在德国的辉煌时期,产生了令人难忘的圣母玛利亚像和屠龙骑士像。今天我们被赐予希特勒青年团版的圣乔治、德国少女联盟版的玛利亚,这两种形象离理想之间的距离,在相貌上就如同戈培尔之于道林·格雷\footnote{Dorian Gray,王尔德笔下美少年。},或在品格上就如同奥托·格布勒\footnote{Otto Gebuhr,演过许多部腓特烈大帝的电影。}之于腓特烈大帝。

虽然我说的可能给你一种印象,好像纳粹和像白蚁一般的德国大众是全部肮脏和懦弱之事的渊源,但其实我不是这个意思。我看到,无头脑的大众舔食文化孤岛的现象在各国都有。比如,这种现象在英格兰特别突出,英格兰几乎要投降了。虽然这种失败很可能是命中注定的,但我们只有实事求是,并且勇于献身,失败是可以避免的!与1789年的法国大革命一样,这并非是不可避免的唯一结局。我们目前面临的情况是劣等人想夺权,我们是可以与之抗争的,办法就是依靠自己的力量,坚信真理在我们手中。如果有必要,可以牺牲换取胜利。

我永远不会改变一个观点,没有头脑的大众绝对不是无产阶级。大众可以出现在大公司的董事会里,可以出现在富裕工业家的子女们中间,但出现在工人中的机会比较少。我们要对付的是一场瘟疫,这是事实。这场瘟疫源自社会的上层,其病因是一种不知名的生理学溶解。

有鉴于大众所患病症及其病情发展趋势,我强烈呼吁抵制他们,反对他们,我相信我的呼吁是合理的,因为我考虑到他们在生理上是无能的,肯定在地球上活不长。我已经谈论过,在人口增长和肉欲横流之间存在着神秘的联系,在现代的白蚁堆和上帝引发的某种恶性肿瘤增长之间也存在神秘的联系。癌细胞和大众群体:都是有缺陷的生物结构,都趋向于提早衰亡,都具有繁殖的爆发性,都处于破了现有形式的无组织状态。这种疾病就像黑死病潮流一样向我们扑来。显然,仅从这点看,就能看出其具有深层次的原因,这会不会威胁人类的全部文化?

我很乐观,相信这片在上个世纪压在我们头顶的黑云,在某一天就会突然消失,即使我们还需要再忍耐几年末日的恐怖。从经济上看,大众人群是不可能的,等到不久之后,工业品覆盖了整个地球的时候就能得以验证。到那时,过多的人口就会显示出其荒谬的本性:由于人口太多,不仅养不活,而且养那么多人口也毫无意义,就好像在18世纪末,受封建制度分封的人多得既无法维持,也毫无意义一样。大众人群缺乏活力,如今有许多非理性因素威胁其生存,这种情况就如同地平线上正在聚集的暴风雨云一样。在那暴风雨到来之前,完全可以想象得出来斯宾格勒预见得情景,他看到了世界上最后一把小提琴被打碎在地,莫扎特的最后一份四重奏曲被大火烧成灰烬。但有一件事非常肯定,基于理性的生物,在非理性或反理性的潮流中根本无法生存。精神空虚在我们这个时代是无穷无尽的,这使得非理性势力可以大举入侵。

生活不能没有思维,如果肉体和思维之间的平衡被打破,结果只能是死亡——死亡是永恒的,无论这座以繁殖人口为目标的妓院内部有何种改变都一样。这场大众革命也许会毁灭哥特大教堂,让巴赫的舞曲永远沉默,但一群退化了的足球运动员绝对无法从这场被他们引发的大火中生存下来。

\section{1942年2月\ 国家主义者的历史观}

国家主义者的历史观:德国的历史书是披着金发的谎言。纳粹的国家主义:就是一种思想状态,如果你不仇恨什么人,你就不够爱国。

德国人撒谎说掌握了解释最困难问题的完美工具——任何解决问题的方案都是可信的,不变的,能被任何人理解的。《黑色军团》杂志告诉我们,人生活中没有悲剧,至少党卫军没有。这份杂志还说,“悲剧是教皇发明出来控制人类的。”

这就是这些人激起人们愤怒的原因:他们把这种野蛮强加给我们,并试图让我们屈服,强迫我们跟他们一起去压制所有不同意见,跟他们一样不分黑白。我们就要灭亡了,因为我们不知道他们的“舒适的科技生活”实际上就是一个大骗局,是虚假的生活方式。我们错以为他们提供的便利就是真正的好产品,这就等于我们把油漆当作了彩虹。

他们真正想的是让我们无法辨别黑白,把世界变得死气沉沉,就像荷兰画家勃鲁盖尔(Brueghelian)笔下的古人拿着一罐肉罐头一样;把汽车旅行奖励给徒步者;把昔日那种昂贵的真丝袜子奖励给今天办公室里穿尼龙袜的女孩——全部目的就是让观赏者变得恶心起来。

我们应该相信过去日本木版画用的颜色与油漆的颜色是一样的吗?我们都知道祖父如果去意大利会怎样旅行,他要花费两年的时间,要让这次旅行终生难忘。现代旅行,坐着高速火车横跨意大利,从北部的维罗纳到南部的塔兰托仅需四周的时间,在如此的观光后,马上去疗养院恢复体力。这两种旅行是一样的吗?在乡下干草堆上突然进行的野合,与纳粹的妇女元首萧尔茨·克林克(Scholtz Klinck)用动作或不用动作的讲演是一样的吗?

以前,我在谈到职业差别或阶级差别时,我说过流浪汉是一种有尊严的职业,是最后一种具有独具特点的职业。虽然当时仅想鼓励一下流浪汉,但我现在急切地想收回这个鼓励性的说法。如今,似乎罪行也要在社会发展的祭台上作点奉献。最近动态是纳粹要组建帝国妓女委员会,议会已经完成了辩论,在一个评判委员会的监督下做了试验,并科学地设计了职业发展计划。这个委员会受到德国最高权力机关的保护,由最能胜任的德国宣传部承担责任。

事实上,已经不缺少组建现代德国工会的条件了——也许只缺一个小条件:工业家缺席。工业家们,只有你们缺席。

\section{1942年3月11日\ 德国人的雷鸣}

我正在阅读叔本华的书:为了能更好地理解德国人的理性品质,我做了几点笔记,以备未来之用。

费希特\footnote{Fichte,1762—1814,是康德的学生。}的第一本书已经出版了40年,但他仍然被视为与他的老师康德处在同一级别上——就好像他俩有什么共同点一样。

利希滕贝格\footnote{Lichtenberg,歌德把利希腾贝格的作品比作能指明问题所在的神奇魔杖。}的作品,不仅没有再版,而且如今在面世32年后竟然被免费赠送。然而,克鲁格、黑格尔等人的作品已经再版了好几次了。

不知何故,我突然想到,当爱国主义进入科学的领地时,就像对待小脏孩一样被人抓住领子丢出大门。

现在有一种说法,德国人发明了炸药,我感觉很难相信这点。

看完海涅有关德国的论断(从康德到黑格尔):基督教在某种程度上锻炼了德国人在战争中为获得胜利而显示出的野蛮性,基本上无法改变这种情绪,如果有一天,那个稳健的十字符号失去了魔力,那种北欧诗人所吟唱的没有理性的、狂暴的怒火就会爆发出来。然后,那些石头雕刻的神仙就会抹掉眼睛中留存千年的灰尘,从废墟中站立起来。然后,雷神也会跳出来,举起他的那把巨大的锤子,把哥特式的大教堂砸得粉碎……

我要警告康德的信徒、费希特、自然哲学家——你们不要笑我。我预计像精神世界一样,物质世界也要爆发一场相同的革命。理念是行动之父,就如同闪电是雷鸣的序幕一样。当然,这是德国人的雷鸣,所以非常沉重,而且缓慢。但它终将会来的。当它发出隆隆声——这种声音在人类历史上从来没有听到过——你马上就会知道:它终于来了。然后,会出现一声怒号,天上的鹰都会掉下来,吓得最遥远的非洲狮子赶紧钻近神圣的洞穴之中,尾巴夹在屁股下面。这样的事要发生在德国,相比之下,法国大革命就跟一曲柔和的牧歌一样。

\section{1942年5月\ 布满危险的城市}

吕贝克(Lubeck)和罗斯托克(Rostock)这两座哥特时代杰作一般的城市遭遇了灭顶之灾,每个人都在悲叹,但谁也不会比我感到更难过。

它们为什么要挨炸?仅在30年前,罗斯托克还是地处富饶的农田包围之中的一座城镇,宁静安详,经济自给自足。后来,有人提出在吕贝克和罗斯托克建立军工厂的想法。这些军工厂本可以建在一些了无生趣、建筑物没有什么有价值的小镇上。但工程师们不愿生活在乡下,而这两座城市的市长又想给自己的城市带来发展机遇。结果就出现了与慕尼黑所遇到的类似情况。慕尼黑的福气要感谢克虏伯,他在一战的时候就在慕尼黑兴建了工业区。

如今人们为永远失去了两座大教堂而哭泣,但忘记了我们的种种不幸的根源是工业偏执症。人们哭泣,但不扪心自问。失去了保存在昂贵的巴洛克风格橱柜里的家居工具、螺丝起子、手锯怎么办?那些独一无二用于吸引狩猎客人的水晶玻璃器皿怎么办?战后,那些工程师、战争生产委员会的领导、市长、社区领导都能够为他们拿着国家的钱所做的轻率的无耻赌博行为负责吗?

他们不会出来负责任的。多年以来,这些跟害虫一样的北方工业集团吞食了我生活的宁静河谷,赶走了农户,为未来埋下了不安定、贫穷和不满的种子。就是因为这些人心存巨大的歹毒,这才使这个地区有了“大发展”,他们就跟吕贝克和罗斯托克的市长一样!

过去几年来,这里建造了一座巨大的地下仓库,用于存放炸弹和毒气。农民的土地被征用,却没有给任何补偿。这里埋藏着大量肮脏的化学物质,其破坏力大得难以想象,把一个原本宁静的地区变得危机四伏。每天晚上,我都能听到火车运走大量毒气弹或其他武器的声响——只需一次空袭就能把这个美丽地方变成烟火弥漫的地狱。

下面所述之事很具有代表性。负责保管毒气库存的技术员,是一位来自普鲁士的绅士,他参军前是受人尊敬的消防队员,如今已经是上校军阶了。有一天晚上,他喝醉了,开始向仓库的警卫射击。警卫反抗他,他就用拳头揍警卫。第二天早晨,他又企图贿赂昨晚警卫值班的下士。这些警卫都是巴伐利亚人,他们嘲笑了他。当晚,他就消失了,消失在纳粹军事机器的齿轮中了。除了这次倒霉事,他在所有方面都很完美,很可能会去其他地方任职:也许他会去波兰的某个仓库做主管,在那里他会掌控当地居民的生死,肆无忌惮地向活人射击。

他们竟然任命这样的人来掌控化学品“地狱”的主管,只要他犯一点小错误,就能让整个地区毁灭——人畜草木全都无法逃脱。

在罗斯托克遭遇轰炸的时候,我有个亲戚,他是个妇科学家,他的诊所在第一天晚上的空袭中就被炸毁了。他的住所在第二天晚上的空袭中被摧毁了,包括全部家当。这位62岁的老头,穿着睡衣,从窗户里逃脱了,他一生勤奋劳动的果实化为乌有,最后只剩下一条老命。

我要说说恩斯特·涅基希(Ernst Niekisch)的事,他在四年前被纳粹判处终生监禁,这一判决在国际上受到很大的关注。我听说他在监狱里被谋杀了。涅基希原来是巴伐利亚的中学校长,但他是我遇到过的最聪明、最非凡的人。1919年冬天,在慕尼黑起义期间,我自愿去拜里舍霍夫酒店做俘虏,另外还有50位绅士,我们都是巴伐利亚老亲王利奥波德的支持者。当时,涅基希在政治上是一位名人,担任士兵委员会的主席,他尽全力保证我们俘虏受到正常的待遇,我也许要说我们受到了绅士般的待遇。涅基希对上届政府有明显的偏爱——这点与资本家不同——其结果是我们仍然能用我们自己的轮盘赌具,并且赌光我们剩余的钱,都是能发出铃铛一样响声的1870年前的老款银币。

我第二次见到涅基希是在1930年。那时,他变成了一个狂妄的小团体的领袖,他的这个小团体与鲁登道夫的坦南堡同盟很相似。德军总参谋部里亲俄派办的一份相当不错的报纸,向这帮昔日的军官、自由军的支持者、饿着肚子的学生提供支持。当然,他们是希特勒派的死对头,因为希特勒派极为仇视俄罗斯人。

我两次受邀参加了涅基希组织的“活动”,一次是在大门紧闭的洛伊希滕贝格城堡里(Leuchtenberg),另一次在图林根的森林搭起的帐篷里。活动包括:军队式的节俭餐饮、早锻炼、晚间进行话题敏感的谈话。参与谈话的人来自三教九流,我从来没有见到过人事这么复杂的团体,其中有各党的秘密代表,左翼党有,右翼党也有;有饿着肚子的高中和大学的学生,他们是在经过长途跋涉之后才来到这里搭起帐篷的;罗斯巴哈(Rossbach)的右翼组织残部;受人诬蔑的军队牧师;退休的将军;魏玛共和国的军队的间谍;政治流氓,甚至还有党卫军中的反对派,他们两年前在罗姆叛乱中遭到镇压。

涅基希本人长得就跟一个球一样,具有希波克拉底般的洞察力,不过眼界狭窄、悲观,但无论怎样,他是个“大叛徒”。当他开始尖刻嘲讽和恶毒攻击纳粹和希特勒本人时,他的命运就注定了。德军总参谋部缺德,也注定他命中倒霉。1918年之后,德军总参谋部变得很实用主义,喜欢政治投机,总是不守承诺。很显然,一旦希特勒得势,成了掌握德国命运的军事独裁者,涅基希就会立即被他的支持者抛弃。

涅基希之所以被认为是严重的叛国,毫无疑问,主要是因为他的出版物。不知为何,他的出版物竟然到了1935年才引起纳粹的邪恶才子罗森伯格(Rosenberg)厌恶。涅基希利用一切机会取笑这位纳粹的大神仙。但这并非无中生有,罗森伯格把自己比作是大西庇阿,甚至是克伦威尔。

具有讽刺意味的是独裁者能置人于死地,却无法不独裁。我有个疑问:四年前那几个把涅基希投入监狱并实施政治谋杀的法官,要不要为此负责任?他们能心安理得吗?

\section{1942年6月\ 异常陌生的旧世界}

我去斯图加特会见我的出版商时,遇到一位老妇人,她是30年前沉没的“泰坦尼克”号上的幸存者,至于那场海难的原因,至今仍然模糊不清。这位夫人告诉我,那只船下沉的过程中,发生了一件令人匪夷所思的事:海水都已淹到甲板上,救生艇已经被放入海里,可仍然有乘务员端着装满了三明治的盘子,低声地说,“请用三明治!”——这样的忠于职守,带给我们永恒的幽默感。几乎没有人会想到这个故事其实代表了英国人的灵魂,值得让约瑟夫·康拉德\footnote{Joseph Conrad,1857—1924,英国作家,善于写反映海员生活的小说。}去写。

这几天,我一直陪着病危的克莱门斯·冯·弗兰肯斯坦,他的癌症已经进入晚期。我陪他去看了我的大夫,因为他认为这样可能有好处:他曾经是个很自信的人,不久前还夸耀自己的体力,如今没有我的帮助他钻不进轿车;他在拥挤得令人窒息的候诊室等着,这里挤满了肥头大耳的资本家和歇斯底里的女演员,而此时的天气也热得要命。

一般说来,会诊的效果并不好;他知道自己希望渺茫;这仅是再演一幕滑稽戏,完全是为了展示妻子的慈爱和体贴,从而能略微缓解一下她的孤寂。

我们坐在那里,扮演着死亡这出戏的角色。我们与瓦尔特施皮尔(Walterspiel)一起吃早餐,他竟然没有认出克莱门斯,克莱门斯的面貌变化实在是太大了。我们知道30年的友谊就要结束了,这次面对面恐怕是最后一次了。

我再也无法欣赏到你的分析,再也不会为你的与众不同而惊奇:第一眼就被你的仪容所感染——以及那外表之后的慈善之心,总是极力在想帮助别人。

我们一起去看克莱门斯的表兄厄温·舍恩博恩(Erwein Schonborn),他目前在钮威特尔巴赫诊所治病。他来信说他的病不重。但当我见到他时,我发现他的样子变得吓人,瘦得像皮包骨,跟克莱门斯一样一副要死的样子。

克莱门斯从前是我聊天的伙伴,他平时说话喜欢说反话,甚至有点嘲讽人,此时他谈话的腔调是我从来没有听到过的。这两个男人之间的谈话用的是一种温和语调,就像兄弟之间的互相体贴,这是两个有血缘联系的男人,在即将永别时正在向对方敞开细腻和孤寂的心扉。

我就要失去他俩了。他俩是我的伙伴和朋友,像他俩这样的好人在德国几乎没有了。他们这种人,有远见,有宏大的世界观;心胸宽广,具有伟大的人类精神;他们是我志同道合的同志,我们能一起完成建立新德国的任务,那是一个崭新的德国。

室外是毫无同情心般生机盎然的夏日,这座我们深爱的城市发出刺耳的尖叫声,这声音让我们感到异常陌生。室内有正要死去的人,他们因痛苦和绝望而发出柔和的呻吟;对往昔的回忆是悲伤且沉重的,我想起了我们一起去滑雪冒险、讨论问题、参加节日庆典的共同经历。

我回到家里,深感孤独,精疲力尽。整个世界都好像失去了光彩一样。生活就好像潮水一样退去了,眼前是一片荒凉的海滩沙漠——这时你心里往往有一种感觉,那潮水永远不会再来了,这一生都不会再有了。太阳也仿佛变小了,天上的明星一个接着一个地熄灭了。

上帝很善于对待他的子民,能理解这点的人是令人敬佩的:约瑟夫·康拉德的《归宿》讲述了盲目的维利船长殉难的故事——有一件事,我根本不想提及,戈培尔先生禁止人们阅读约瑟夫·康拉德的作品。

在这个炎热的夏日夜晚,我仿佛迷失了自我。遥远的世界与我之间被冰雪隔离开来。天堂离我更远,我的朋友不久之后就会在那里读《圣经》。我孤独地活着,而且感到越来越孤独——我们生活在被撒旦征服了的人群之中,只有意识到痛苦,未来才能有所改变。

与世隔绝是人生活中最后的时刻:在这个时刻,人用死亡去证明真理。

然而,我想问问仍然生活在有声有色、舒舒服服的旧世界中的你:你知道我们生活在黑暗中吗?你知道迈向死亡的道路要迈过痛苦的门槛吗?你知道只有激情和痛苦才能为明天埋下种子吗?

\section{1942年10月30日\ 艰难的蜕变}

我是在阿特奥町(Altoting)旅馆的窗户里看到慕尼黑挨炸的,当时我正在那里检查物资;那可怕的红色闪光改变了挂着满月的秋夜。我听到远处传来低沉的轰隆声,估计爆炸点离我有80公里远,爆炸声音持续了三分钟——在这三分钟里,遇难者一定是喘不出气窒息而亡的。最后,西面的天空出现一片巨大的火光。

在接下来的几天里,人们谈论着巨大的损失,大部分人是因窒息而死的。即使过了五天,仍然有人被从废墟中挖出来,他们被废墟掩埋了,无法动弹。有人挖出来时,已经死了,脸上仍然挂着死前的狰狞样。

英国人知道很多纳粹的高官都住在索林区的豪华私宅中,那个区不幸地连续遭到三次空袭。住在此地的沃纳·贝根格林\footnote{Werner Bergengruen,1892—1964,德国小说家和诗人。},丢失了所有手稿、收藏品,因为房子里所有的东西都跟着房子一起消失了。第二天,有人看到他的精神处于震惊和绝望之中,蹲在房子的废墟上,并从废墟中随手捡一些东西给过路人:拉丁初级读本、一尊小铜像、一对中国文物。旁边放着一个手写的广告,上面说一位德国作家要大甩卖剩余财产。警察想赶走他,但他拼命抵抗,站在旁边围观的人群很同情他,警察只好撤退了。

空袭那天晚上,希特勒先生碰巧在慕尼黑,在给可怜的平民拉响的警报响起来之前,他已经躲进了私人防空洞,地板上铺着地毯,有浴缸,据说还有电影室。所以,当数以百计的人被埋在瓦砾中挣扎着呼吸的时候,希特勒也许正在看电影……

很自然,希特勒会宣布日后一切都会重建,会比从前更好。很有可能就是几个年轻的加拿大人把圣母大教堂炸成了废墟,这时希特勒就会派纳粹的建筑师施佩尔(Speer)来安慰我们,不要为失去几座大教堂而悲伤。我认为希特勒在看到这些哥特式的杰作损失了后会暗中高兴的,因为他自己就很想变成一座不朽的建筑。为了建设他的宏大剧院,难道他不曾威胁过要摧毁特埃蒂娜教堂(Theatinerkirche)、王宫花园(Hofgarten)、洛伊希滕贝格宫(Leuchtenberg)吗?在基姆,据说要建立一所领袖学校,就是那种为未来培养总理的种马农场,地点在这座宁静湖泊的东岸,绵延一公里半。这样原先宁静的湖岸会变成一块大石头,大石头中矗立起一座130米高的塔。人们觉得他要建房子就像是下命令,不能有反对意见。

他是个恶毒的人,心胸狭窄,所以他痛恨一切耿直的、健康的东西,因为这些东西跟他的本性背道而驰。他憎恨一切合法的东西,所以我们传统中珍贵的东西都被他视同仇俦,因为这些东西不谄媚他的虚荣。看着这只大猩猩,我们是不是真的成为这个逃脱了束缚的野蛮原始人的囚犯?这样说是不是太过分了?

所以,我们继续过着羞耻的、不光彩的、充满谎言的生活。我们这些软弱资产阶级的抗议,除了能重复说着有关这个政体的老笑话之外,在剩下的时间里只能用来吞食政治宣传的果实。

戈培尔在报纸上连续发表了几篇文章,一名佃户的妻子来找我,看上去她很害怕,浑身发抖。上帝保佑,她如何保护她的孩子呢?根据报纸上说,她的孩子会被拖到收容英国人、美国人、俄罗斯人的收容所里!请注意,这名妇女在美国住过几年,工作是洗衣女工;她只能说一点英语,对波士顿有良好的印象——可她相信有关报纸上说的外国魔鬼的故事。这是真的,这些人昨天还很明智,有辨别是非的能力,今天就像患了思维疾病一样。只要报纸的口气坚定,他们就都会相信。

最新一条消息是戈培尔捏造的,他说我们的“领袖”在不预先通知的情况下访问了一个镇子。然而,镇子上人好像感到了领袖发出的光芒,自发地排队等待领袖的到来!如果这样的故事出自德意志帝国的官员或魏玛共和国的官员,众人的嘲笑声大得足以迫使他辞职,而且嘲笑声会一直伴随着他的余生。但戈培尔的故事通过广播就会播送了,还被相信了,还被消化了,还没有一个人敢笑出声。

实际上,只要是以政府的名义在广播里或报纸上说的,说什么都有人信。如果戈林先生在每次说话必备的号声下,赞扬他的那只名叫巴伐利亚国王的猎狗,我认为那些昨天还觉得自己与北方人截然不同的、骄傲维护自己特点的人,将会欢呼着表示顺从。

空气中有一种怪诞的东西正在向我们迫近,永无休止的谎言压垮了我们的生活节奏。希特勒那帮人来了之后,连续九年时间了,夏季只出现在日历上,大雨从天而降,就像史前的大洪水。年年葡萄欠收。植物学家说,一些秋季开花的的植物提早到了春季。我听到东线的一位动物学家说,在北高加索,原来生活在印度热带地区的蛇已经能在伏尔加河畔找到,这里是欧洲的边缘。所以,全都乱了,正常的秩序荡然无存。如今折磨德国的灾害,是不是与这些现象在本质上是一样的?

8月,克莱门斯死了。那个住在英格兰的他最爱的兄弟,听到他的死讯后回来了,他的死让他的兄弟痛苦不堪。就在八周之前的一天,他在皮尔森湖的那间小房子的上空乌云笼罩,他弹奏起他的歌曲中那首我很喜欢的曲子《莉塔派》,这是一首充满忧愁的歌。我坐在他身旁,悲痛地看着他的手指,如今瘦得如同火柴棍。在他演奏的过程中,我们中间闪过一道蓝光,灯熄灭了,保险丝碎了。看来那时老天爷就把把我们分开了。

昨天,我与H先生就人性残忍的形式正在发生改变这个话题进行了一番讨论,特别谈论了东线的恐怖战况。这使我想起了大约40年前发生的事,其恐怖的情景至今仍然历历在目。我那时是个军官学校学生,正在哥尼斯堡(Konigsberg)休一次短假,我的一个老同学邀请我去参加一次解剖学课。

大部分学生都外出度假去了,解剖室放置着一张看上去很油腻的解剖桌。当我走进解剖室时,看到一名胡子拉碴、肮脏、长着灰白胡子的老助理员正在从一具刚运到的尸体上把头颅取下来。那颗头颅被左轮枪的子弹完全打烂了。

我逃跑了,可那老家伙挥舞着匕首,像一只蝙蝠一样追着我。在走廊里,他告诉了我这具尸体的故事。一个同性恋的药剂师,开枪杀死了自己的恋人,因为这个恋人想勒索他,然后他自己也开枪自杀了。由于没有人收走这具难看的尸体,于是这位前药剂师就被搬进了解剖室。

两年后,命运再次捉弄我。此时,我变成医学院的学生,走进同一间解剖室,发现面前的解剖桌上躺着那个药剂师。我立即辨识出这具本不该出现在这里的遗体。尸体上任何可疑的东西,都被那老助理给拿掉了。这个可怕的老头,就好像一个移动的讣告牌一样,守候在桌子上这具可怜的、变了形的人体旁边。

我永远不会忘记第一次摸死人硬邦邦的肌肉的感觉,更忘不了第一次切死人肉的感觉。我周围还有其他三个年轻人在观摩。他们遇到了与我一样的问题,他们那种小孩子才有的诚实表情透露出跟我一样的要与恐惧和厌恶做斗争的愿望。房间里站满了这样的男孩子,站在装饰得显得可怕的桌子旁边。他们本来是柏拉图的学生,朗读着《伊利亚特》的诗篇,如今放弃自己的人文学科,不得不来站到散发着尸体腐烂臭气中,进行尸体解剖分析……

我们成功地完成了这样的跨越,这点可以从老师和助理脸上那冷嘲热讽的表情上看出来,研究生的表情更反映了这点。我记得只有一个初学者丢掉了刀具,跑走后再也没有回来。剩下的人继续工作,去完成艰难的蜕变,让自己成为蜕变的奴隶。即使今天回忆这段经历,也让人感到可耻和难过。我相信他们都是中产阶级家庭的子弟,社会地位稳固;我至今还跟他们中的几位保持着联络,我知道他们在业余时间阅读波德莱尔\footnote{Baudelaire,1821—1867,法国诗人。}作品,偶尔还演奏弦乐四重奏消遣。对如今弥漫世界的野蛮的纵欲,我知道他们的感受与我的类似。

然而,在厌恶之余除了冷嘲热讽,我们还能做什么呢?从切割开始,整个房间里就充满了猥亵的笑话,人们交头接耳说着流行语,并找理由大笑,但声音中带着忧虑和痉挛。这种情况持续了几周时间,即使到了半个世纪后的今天,我仍然感到羞愧。我们每天都在做着这恐怖的行当,每天的玩笑变得越来越猥亵。利用死尸演的闹剧越来越多——死尸变成可怜的玩偶——那些猥亵姿态暗示这些死尸变成了活人。在我的生活中,只有这段时间被迫演出一出低劣的游戏,驱使我们演戏的力量是如此的野蛮和巨大,就如同压路机一样;这是一出闹剧,名字是《在小便池里拉大便》,该剧的结局很令人沮丧,有《伍采克》的悲剧特色。

后来,我意识到这只不过是我们抵抗恐怖的手段。但眼前的恐怖如何抵抗呢?它从坟墓里钻出来,像鬼怪一样排队穿越晚秋的夜空。

从法国巴黎传来一个消息,拉雪兹神父公墓正在挖地寻找海涅的遗骨,由于没有发现遗骨,所以必须做点什么事,整个墓穴被挖开后大敞着。告诉我消息的那个人说,冯·卡尔先生在慕尼黑的马里昂巴德旅店的院子里被党卫军踩死了:一群20岁的笨蛋踩死了一个70岁的老人。

H先生,我和他今天从哲学角度讨论过非人待遇的问题,是不是?他从东线回来了,他看到有三万犹太人被屠杀了。

这次屠杀仅用了一天,实际上是在一个小时内完成的,机关枪子弹打完了后,就用火焰喷射器。全城的人都跑到现场观看,有下岗的士兵,有乳臭未干的小青年。躺在婴儿床上的婴儿,高兴地看着天空上那些有耀眼颜色的光环,而他们的父母也仅有19岁或20岁。哦,人都退化了,生活变得没有荣誉感。哦,子弹把我们和那些被撒旦烧死的灵魂分隔开来!

你觉得我们有缺点,而我们在孤独中忍受着恐惧。你批评我们不反抗。在肮脏的掩体里,我们的抵抗力默默地消失了。牺牲者的鲜血流尽了,但毫无意义。有人像夏绿蒂·科黛(Charlotte Corday)一样进行刺杀,但却无人知晓。魔鬼出山了,是上帝放那魔鬼出来的。“上帝给了他巨大的权力。”我们只能推测上帝为什么会这样做。为什么上帝要挑选他走上历史的舞台呢?未来等待我们的是什么?上帝在幕后要做什么呢?

我们头顶的天空仍然是漆黑的。我们难过极了,我们深受折磨,而你却不受折磨。死人不受折磨。

小心啊,有人会把我们从困境中解救出来!

\section{1943年2月\ 纳粹被推到了墙上}

英美在非洲登陆的消息,以令我吃惊的速度传播开来。虽然有禁令在先——不许听盟国的无线电广播,但这一消息这是不胫而走,在一个小时内就变得众所周知了。当我看到这条新闻产生的效果的时候,更是感到惊奇。每个人都对这一决定性的战局变化感到高兴,这意味自己的国家要战败。巴伐利亚人甚至还想让战火烧到阿尔卑斯山。

不仅如此,整个镇子——甚至可以说是整个地区——都兴奋起来了,仿佛每个人都喝了香槟酒一样。突然,人们的谈话变得坦率了,脸上似乎散发着光芒。漫长的冬日带给人们的艰难就要过去了,温暖的春风在寒冷的冰雪上吹拂着。每个人都感觉到一只奇怪的手已经把纳粹推到了墙上。这则消息无论好坏,反正都是有益的。有个中学校长,平时最喜欢说纳粹用语,如今已经不再说“希特勒万岁”了,而是换成了“您好”。本地区的纳粹主管召集人们开会,请求人们看在上帝的份上不要烧毁他家的住宅,因为他仅是执行党的命令。

这就是那条消息在各村庄产生的效应。希特勒咆哮起来,但实际上他非常恐惧。人们不再难为情,把他视为救星的时期过去了;过去,他的画像在普鲁士新教教堂中,就如道林·格雷的画像一样,放在他的著作的旁边,他的著作是女仆们的圣经。神像头上的光环已经退去。

这段时间,每个人的脸上都显露出某种恶毒的满足感,平时虚伪的表情一夜之间全都烟消云散了。衣服领上的纳粹标志摘下来了,对纳粹党不敬的行为也不再被记入个人档案了。在我们附近,有个农民,他的部分土地因战争而被征用,有个仲裁委员会要求他去作证。这个人怒不可遏,骂仲裁员是恶棍,政府官员都是窃贼,而且是世上最恶劣的窃贼,是“疯狂的麻烦制造者”。他离开仲裁委员会时,重重地把门撞上了。几年以来,在场的官员们从来没有听到如此激烈的语言,都目瞪口呆了,而且此人至今仍然没有被抓起来。

与此同时,情况越来越妙不可言。德军医院氯仿麻醉剂和吗啡开始短缺。看到伤员躺在冰冷货车厢里臭气熏天的稻草堆里,医生进行了抗议。在柏林,由于缺少胰岛素,一个单位里所有的糖尿病患者都死了。

我曾经读过德国皇储回忆录,其中有一部分写的是1870—1871年之间的事,我发现组建德意志帝国时发生的事令人震惊。

有几段有关确定德意志国徽的谈话,其所用语言极其轻佻——就好像讨论护发素的商标一样。当时谈到了俾斯麦选定的颜色,皇储谈了看法,他说国徽可以采用“像舞厅一样的绿色和金色”,上面写着“今晚有舞会”。

德意志帝国就是按照这种方式建立起来的吗?国家的复兴就是按照这种方式进行的吗?这是建立一家咖啡出口公司的方式,是公司未来股东按规定商讨组建股票公司的方式——但这个方式却用在了组建德意志帝国上!

在皇储的日记中,随处可见欠考虑的地方,充斥着皇家的傲慢——这预示着他儿子威廉二世未来的样子。在字里行间,任何人都能感到缺少德国人原有的稳健品格,却多了匪徒的贪婪,并且用嘲讽的口吻去否定德国伟大的精神传统。德国没有健康国家应该有的神秘内核,所有健康国家都值得铭记的“超乎现实世界之上”的东西。

不,德国是个混合物,混合了一点儿决斗俱乐部里的浪漫情怀和“体操之父”路德维格·扬恩的体操术,混合一点黑格尔的属性和弗里德里希·李斯特\footnote{Friedrich List,他对农业在国民经济中的地位和作用有着非常深刻的论述。}的一部分健康肌肉——这是一碗充满了欲望的肉汤,包含了整整一代人想尽可能快但又毫无痛苦地富裕起来的欲望。当巴伐利亚的路德维希拒绝皇帝的头衔的时候,当我祖父辈中的某些人拒绝那个反动的疯子组建的公司的时候,这些人的直觉难道不是很准确吗?普鲁士帝国渴望权力,想通过用殖民的办法统治祖国去赢得骄傲。有这样的欲望,其结果是不会好的。

德意志帝国自诞生那日起,各种麻烦就纷至沓来:1873年的经济失败;控制教育权输给了教会;有人刺杀威廉一世;企业用法律手段镇压社会主义分子的企图遭遇可耻的失败;威廉一世和弗里德里希三世相继逝世。这个时期过去18年之后,就接近这位日记作者的死期了,此时又发生了几件悲剧事件:德国皇帝差点没有死在街上,最后被一名路过的马车夫救了,那车夫扯下导致皇帝窒息不通畅的插管;新德皇一上台就逮捕了自己的母亲;最后是威廉一世的葬礼,举着旗子站在皇帝的棺材前那名副官,简直可以用酩酊大醉这个词来形容,走路的时候还需要别人扶着,而走在游行队伍中的俾斯麦,由于害怕受风感冒,竟然戴着金色的假发,那时虽然是六月的天,但着实有些冷。这就是上帝报复的方式。

哈姆雷特说过:“这不好,也绝对变不好。”

我在普林(Prien)的时候,拜访了一位老者,他在路德维希二世时是步兵,那时那位君王还没有遭受不幸。很奇怪,这个老人和我遇到过的许多曾经与国王有接触的人一样,绝对不相信路德维希二世精神有问题,却认为这是柏林的阴谋。在柏格堡(Schloss Berg),他看见两个医生激烈争论,他俩相互指责对方捏造心理诊断结果。无论如何,我不想介入这段争论,但我许多年以来一直怀疑当时处理这个病案的古登医生,他对国王表现出撇不清楚的敌意——在路德维希二世被囚禁前,他让同事恶意攻击国王的至高无上的地位,而之后又难掩喜色:仿佛他是个有权驱逐国王的心理医生。

进一步看,虽然德国北方人很喜欢说路德维希国王和奥托国王是由于基因问题而产生心理疾病,但他俩的基因问题不是哈布斯堡家族和维特尔斯巴赫家族近亲通婚的结果。相反,有问题的基因源自他俩的祖父普鲁士大王储威廉,他通过他的女儿巴伐利亚的玛丽女王把胚细胞缺陷传给了他的两个外孙。

所以,基因问题,最有可能不是个巴伐利亚问题,而是从柏林输入进来的。此外,无论是否精神有毛病,欧洲没有一个国王能像路德维希国王一样,死后仍然有影响力。在1918年起义期间,人们相信他回来了。如今,又有传说他回来了,而且不仅是农民在说——有人在雪夜中看见他乘坐着一辆奇怪的雪橇在飞奔。

最近几年,我在朋友圈里太常见到死神,所以有几句话需要补充一下。我要说的跟我住的房子有关,这是一座很孤单的建筑物,房龄足有600年了,一直被认为有鬼魂,这个鬼怪是个修道士,他从河面上飞来,在餐室的窗户前显灵。

当然,我没有看到过那鬼魂。我在家里曾经听到过奇怪的声音,有时像保龄球的声音,有时午夜时分电灯会突然熄灭,有时我卧室的门会无缘无故地突然打开。按照习俗,我认为这些是野猫所为,或电力中断,或门锁坏了,从来不把这些看作是鬼魂显灵。

但我家里最近出了些新情况非同一般。去年秋天以来,我周围的人死讯不断,我们家的人都注意到了阁楼里有点特别,那里一直是鬼魂故事的起点。我们都闻到那里飘出东西腐败的味道,而且不是只有一个房间能闻到味道,整栋房子都飘着那股味道,阁楼里的味道没有了之后,会突然出现在第一层,过一会又能在第二层闻到。这肯定不会是某种暗示,因为我又几个客人从慕尼黑来,他们根本不知道情况,等他们进入自己的房间后,又都出来说有股腐烂的味道到处弥漫。

很自然,我们最初是想可能床垫中有死老鼠,或是在地板格子里。我们赶紧仔细查看,但什么都没有发现。这时最不可思议的事发生了:那气味开始愚弄我们,在一个房间里跟我们玩藏猫猫。不仅在是房间,而是从一个家具转移到另一个家具,一会儿在椅子上,过了一会儿又转移到容易被人忽视的灯泡上,在消失后会突然出现在我的大提琴的弓上。我们找不到原因,于是只能不予理睬。此后,就像我在过去一年中失去好多朋友一样,那气味从10几个人面前完全消失了。我说的这个故事,很可能会遭受戴着牛角框眼镜的各类知识分子的嘲笑。

还有一件事,我有个朋友叫戈瓦丁斯基(Gwosdinski),他家的房子在空袭中被彻底地摧毁了。他前几天给我写信说,灾难来临前,他家的小萌猫彻底地变成了另外一个动物,无缘无故便垂头丧气,哀鸣不已。空袭那天晚上,小猫被救了出来,但仿佛在大火中看到了什么东西,又跑回房子中。房子的废墟中没有找到小猫的遗体。这里,我仅是提及事实,既不支持,也不否定。最近我遇到一件事,只能用超自然来解释。

我年轻时就很关注大保守分子冯·海德布兰特(Heydebrandt),他是我父亲在帝国议会里的同事。1918年秋天,他退出了政治舞台,我再也没有听到过他的消息。1924年10月的一个夜晚,我梦见他死了。几天后,我真的听到消息说他死了。

我又想到了希特勒先生,他躺在德国艺术馆的界牌处。他手拿着锤子,猛敲了三下。当他举起手臂的时候,锤子的头脱离锤子的手柄,落入远处的混乱的人群中找不到了。我能看到那个迷信到歇斯底里程度的家伙被这个恶兆吓坏了。

他的敌人把这看作是个好兆头,我们当时就希望这个政权马上就崩溃。我们已经等了10年了。我们因为忧伤所以头发变得灰白,我们强令自己吞下不共戴天的仇恨毒药,宁可去死,也要消灭我们的敌人。我们坚守着权利,绝不投降,宁愿牺牲我们生命中最好的时光。仇恨使我们像蜜蜂一样以生命为代价去使用蜂刺。

如果想在希特勒的统治下获得和平和繁荣,就意味着不仅要付出泪水,还要去抢劫和谋杀。我们中绝对没有人会这样做。至少我不知道谁会去!我只知道我的朋友会不妥协地去战斗——我只知道有人宁愿去死10次,也不愿去支持魔王获得胜利。我只知道人们宁愿与上帝一起哭泣,也不愿和魔王一起欢笑!

欢笑该结束了,魔王知道这点。结束的时间到了——不是英雄地结束,而是丑恶地结束,是在耻辱中结束,是在蜕变中结束,是在全世界的嘲笑中结束!有时历史会允许自称是伟人的人动一动历史这部巨大的机器。这会持续一段时间,傻瓜不会马上被消灭。但那机器突然加快了步伐,越来越快,他被甩入那机器,被碾压得粉碎。斯大林格勒:他会陷入法国人称的“光屁股”的状态,另外还有一个更具有破坏性的说法在人们中传播,比任何宣传词传播得都快,这是“哥法兹”,意思是“历史上最伟大的陆军元帅”。“哥法兹”可以扮演一会儿亚历山大大帝。但不久之后,历史会露出真面目,让他泪流满面。

“哥法兹”……

\section{1943年3月\ 断头台来帮忙}

魔王\footnote{指希特勒——编者注}获得了来自斯大林格勒的消息。很自然,在他走向灭亡前,总是先有好消息降临。我们再次感到了恐惧。

希姆莱:1934年的新年庆祝会上,我见到过希姆莱一次。当时我和克莱门斯在混乱中被挤到他所在的那群可疑的人群中。此时,这个样子很像是个地主的管家的中产阶级后代,硬是把我拖到角落,问我谁是亚诺·雷希贝格(Arno Rechberg)。

雷希贝格先生很富有,是共济会高级会员,在推翻冯·泽克特(Seeckt)的行动中发挥了重要作用。在他的努力下,这才有了洛迦诺会议(Conference of Locarno,1925年秋天,德国和法国这两个死敌在这次会议上郑重地达成了和解)。此外,他不仅是绅士俱乐部的背后老板之一,还是冯·巴本内阁的成员。由于我与雷希贝格先生仅是萍水相逢,不想深谈,于是反问了希姆莱一个问题:为什么第三帝国的元首要问像我这样地位低下的人有关大人物的问题?

他用近视眼吃惊地看着我。我说“元首”这个词有点外国口语,他没有听懂。为了不显得无知,他走开了。他身上有一种令人感到恐惧的力量,虽然他仅是个副官,还带着无法抹杀的小资产阶级气息,但他有想杀谁就杀谁的绝对权力。他肯定很像富基耶·坦维尔(Fouquier Tinville,法国大革命时期的恐怖检察官):古板的官僚,主管地狱的大臣。

慢慢地,这个人踩着别人的肩膀爬上了高位。现在我们就生活在这样的环境里,就好像法国人必须熬过热月,推翻罗伯斯庇尔一样——简直就是地狱,什么都是非法的,随时可以指控别人,随时可以用斧头砍人头。纳粹就是嗜血的虐待狂雅各宾党人,他们成立即决法庭,在五分钟内就能判人死刑。判决书上盖上“清算充公”的字样——这意味着执行死刑并剥夺财产。受害者被推出后门,刽子手已经在那里等候了。只需要15分钟,所有的法律手续就宣告完成。接下来是高大的断头台,大学解剖室堆满了掉了头的尸体,由于尸体太多,大学的管理层拒绝再接受这类哑巴客人。

人头落地小事一桩:有点脑子的人早就知道他们必败,可谁敢怀疑他们,他们就砍谁的头。谁敢使用英镑,他们也要砍谁的头。那些胆敢诽谤大将军的人,掉脑袋掉得特别快——因为他们诽谤了上萨尔茨堡的凯撒,此人前不久还自称是大西庇阿,如果谁敢质疑他不如上帝,他定会勃然大怒!他是最大的刽子手,他修改法律,将“奚落元首罪”的处罚从原先的六个月监禁变成砍头。特劳恩施泰因县的检察官告诉我,德国目前有11座断头台。最近有一次慕尼黑的断头台坏了,最后只能借斯图加特的断头台来帮忙。

但断头台的工作效率从来没有如此之高。一个巴拉丁人,因为不愿让儿子去前线送死而被砍头。一名斯图加特银行总监,因为在火车上被人听到说战局不妙而被砍头……哦,克里斯蒂安·韦伯先生的妓院里有个妓女的头颅也被残忍地砍了下来,要知道韦伯先生不仅是两座妓院的老板,还是那位历史上最伟大的将军的密友。这个妓女的罪状是按照老板的吩咐在提供肉体娱乐后要求外币付钱,并私自留下了一点小费。

估计仅柏林每周就要砍掉16个人的头。在维也纳,普鲁士人的仇恨处于白热状态,那个数字是每周20人。刽子手每周有两个固定休息日,休息完之后要大规模地生产。他每处死一个人,得一笔佣金,政府还给他一笔固定收入,所以他从这项事业中有利可图。我能想象,在报纸个人广告栏目中,一条符合新式德国人在柔情和残忍方面达到标准平衡的广告如下:

\begin{quote}
政府官员

军人,工资丰厚,有养老金,身材高大,金发,面貌良好,热爱自然,人生观积极,寻觅一位有相同精神状态的女人,也是金发,有意结婚。身高1.6米以上,年龄25岁以下。
\end{quote}

绅士们喜欢金发。那些金发的名叫英格丽、维布克、阿斯特丽德、古德龙、伊索尔德的夫人们是不会理会丈夫的职业的。丈夫毕竟是在为政府做重要工作。你认为我夸大了刽子手的社会接受程度吗?且让我讲一个最近发生在维也纳的故事。

M夫人是著名的女演员,在维也纳城堡剧院(Burgtheater)处于全盛期时,她是演悲剧的主角,后来成了一家造酒长的所有者。她常跟一位官员一起吃饭,因为他与黑市有关系,这对卖酒有帮助。有一天,这位官员带来一个“熟人”一起吃饭,那人极少讲话,似乎不愿跟任何人来往,绝对不看人的眼睛,当被问到是否住在维也纳时,她用德国北方的方言回答说她仅是临时来维也纳小住,因为在此地有业务。后来,那人走了,这位夫人听说刚才晚餐同桌的客人是刽子手……

继续这个话题,慕尼黑最近发生一件事,我当时在一所即决法庭旁听,看到一名65岁的老医生被判8年徒刑,差一点就上了断头台。法庭的屋顶很低,光线很暗,有霉味,墙上有一道道的烟痕,由于疏忽,墙上还挂着一幅前摄政王的照片,看上去就跟到了另一个世界一样;被告是个老人,颤颤巍巍,结结巴巴;控方和检察官是个圆滑的金发荡妇——不巧是个瑞士人——她是老者的女佣人;助理法官是两个公务员;法官,干瘪瘪的脸,满脸怨恨,一个畜生,一块脏东西,一个从下巴伐利亚的深沟里钻出来的纳粹恐怖分子……

不,这个人的名字不叫福克斯。几天前,福克斯刚把肖勒兄妹(Scholl)俩送上断头台。这个人叫罗斯多佛(Rossdorfer),昨天还是普拉特林镇的法律顾问,如今这个下贱东西找到了一个机会,要把数十年积累起来的对“知识分子”的仇恨发泄出来。那瑞士荡妇指控这位老医生,是受人指使才做的。审讯简直不能再快了。当她再次说出她的指控时,那金发的家伙几乎要喊出“纳粹万岁”的口号。老人结结巴巴地想为自己辩护,还没有说出三个词,就被法官给呵斥住了。那两个助理法官认出了我,感到了害羞,躲避着我嘲讽他们的眼神,假装做出合乎法庭规矩的样子,问了几个客观的问题。那老人感到法庭上情况有变,鼓起勇气开始讲话,立即就被一阵可怕的声响给打断了……

原来我们既明智又公正的法官采取行动了,这位过去在为持刀行凶的农民和逃避税收的小贩写法律文书时,文书语法都不通顺的写字员,此刻咆哮道,“胡扯”,并重重地把文件砸在桌子上。他怒火冲天,疯狂地对那老头大吼大叫。接着,他做了一件法庭里从来没有见到过的事。他从椅子上跳起来,跑到老人的前面,在老人的面前挥舞着拳头,咆哮道:“你听着!如果你继续说,我就打你这张老脸!”

就这样,证据有了,可以审判了。那老人被判了8年徒刑,对他这个年龄来说,这很可能就是死刑。

我走出了法庭,思绪万千:英国议会法庭,出于对查尔斯·斯图亚特(Charles Stuart)曾经戴过的那顶显赫的王冠的敬畏,并考虑到他已经是一个深受折磨临死之人,于是派一支仪仗队把王冠送到他住的公寓里;那个深怀恶意的法国法庭,虽然判夏绿蒂·科黛死刑,但仍然接受了她的请求,在绞死她之外,允许一名艺术家给她画像,因刺杀马拉的女刺客希望给子孙后代留下“一个榜样和一个警告”。这些事就发生在150年前,几乎是相同的日子,世界在此间不仅一蹶不振,而且还多了几分罪恶:比如,这位苛刻的绅士,用拳头收集证据,如果被告胆敢不承认,他就揍被告的脸。

我心情沉重地徘徊在慕尼黑,看着空袭留下的破坏——这座城市曾经是那么令人快乐,那么的美丽。就是这个时候,我听到肖勒兄妹殉难了。

我从来没有见到过这两个人。农村很闭塞,我仅听说过他俩的事,但我无法相信其意义竟然如此重大。肖勒兄妹是最先有勇气去面对真理的德国人。他俩虽然死了,但其事业将会继续下去,因为殉难者的事业总是后继有人。他俩埋下的种子,日后肯定会开花结果。这兄妹俩大胆追求真理,几乎达到了勇敢挑战死亡的境地。出卖他俩的是大学的学监,由于他害怕挨揍或遭受惩罚,后来不得不受到监护。

他俩被处死了,这是罗斯多佛这类法官做出的第二例。他俩虽然死了,但仍然散发着勇气和牺牲精神,所以达到了生命的巅峰。

我从几个与他俩有交往的年轻人那里知道了他俩的一些情况。这些人鹤立鸡群,是斯瓦比亚人中的好人,他们生活在宁静中,就如同生活在与世隔绝的修道院中,但他们像英年早逝的圣者一样带着光环。他俩在法庭上的举止——特别是那个姑娘的举止——具有鼓舞人的力量。他俩当着法官的面,嘲笑法庭、纳粹、疯狂的希特勒,为我们这些还活着的人带走了永恒的冰冷气息。他俩就像圣堂骑士团被判刑的骑士一样复述了对法官的警告,“一年后将会被上帝审判”。一年后,骑士的诅咒在某种程度上实现,因为教皇克莱芒五世和法国国王菲利普四世都死了。我们等着看未来会发生什么……

肖勒兄妹,带着年轻人的令人叹为观止的尊严死去了,死得很安详,为后人铭记。他俩的墓碑上要刻下这样一句话:“他知道如何去死,所以永不会做奴隶。”这句铭文应该能让这个在过去10年里严重退化的民族感到脸红。

总有一天,我要去给他俩上坟。站在他俩的墓碑前,我们会感到耻辱的。

这就是我们这个民族的两个孩子的故事:他俩是上帝最后的孩子,但同时又是精神最先获得重生的德国人。

但希特勒先生现在正忙着其他事;正当我们的大教堂和国家博物馆被炸成废墟的时候,希特勒先生却努力为纳粹的音乐谱曲,他要让这首曲子不仅旋律搭调,还听上去和谐。最近有人看到戈林去参加一次为手下下举办的聚会时,穿着长到脚脖子的皮大衣,系着镶着偷来的钻石的摩洛哥皮带,踏着红色的摩洛哥皮鞋。我敢肯定他看上去伟大极了——这位陆军元帅从来没有指挥过一次战役。但历史上有先例:那个不幸的罗马人卡力古拉\footnote{Caligua,公元12—42,罗马帝国第三代皇帝,有名的暴君,后来被自己的卫队长刺死。},在变成一个大疯子前,就曾经踏着红色的摩洛哥皮鞋,出现在目瞪口呆的下属面前。

\section{1943年8月\ “坏德国人”与新德国}

我发现我的医生科里布莱特(Kleeblatt)在断头台旁为继子的死而悲恸。肖勒散发的小册子是他继子写的,所以跟肖勒兄妹俩一道上了断头台:科里布莱特医生以巨大的努力,才没有让继子的尸体被分解后送入解剖室放在装有来沙尔消毒液的瓶子中。

但死人的灵魂已经开始工作了,工作的效果已经使纳粹统治阶层出现士气低落的现象。几周以来,处于下层的人士,比如地方官员、镇长、体制的捍卫者,都故意做出姿态,表示对纳粹失去了信心。他们普遍表现出厌恶的样子,这样让大家看出他们对现状不满意、不愉快。如今在邮局,职员很可能把官方的通知轻蔑地丢到一旁,咕哝说他已经“受够了欺骗”。

这些人态度的转变背后有秘密吗?这些绅士们最近几天收到一封发自某个“革命执行委员会”的信,告诫他们必须对自己在官场上的行动负责;从前的指控和罪行已经被记录下来,如果继续这样做,将会产生比较恶劣的后果。很有幸,我拿到了一封这样的信:

\begin{quote}
我们有文件证据,可以证明你们自1933年以来的活动,当希特勒主义崩溃后,你们要为自己的活动负责。因此,本执行委员会通知你们正受到最严密的监视。如果你们胆敢再支持现政权,或有报道说你们伤害反对派,将会宣布对你们实施死刑,并且包括你们的全家。现政权垮台的那天,就是绞死你们之日。
\end{quote}

这封信见效了。这封信的邮寄方式难以置信,是从几个地方用挂号信寄出的,寄到巴伐利亚的信发自因斯特堡(Insterburg),寄到东普鲁士的信发自巴登(Baden)或符腾堡(Wurttemberg)。

这封信很有效。小官僚们马上就停止工作了,中学老师又去教堂了,妇女领袖沉默了,纳粹的地方组织成员再也不去开会了。我开车带着一名特罗斯特贝格(Trostberg)医生的妻子去慕尼黑,她告诉我,她的丈夫,当时是地方医疗机构的领导,过去不给犹太人治病——包括交通事故。现在,这个妇女说,她丈夫精神出了问题,不断抱怨生活没有意义,纳粹的宣传不真实,甚至想自杀。他为什么这样?因为那个神秘的革命执行委员会已经给他判了刑,把他开除出医疗队伍,理由是他“不人道的举止不适合做医生”,现政权被推翻之日,就是判决执行之时,可能还有进一步的惩罚。

顺便说一句,送信的人要携带着信件跑遍半个德国,每次送信都有可能丧命。

前天,M先生,一个由几个学生、艺术家、知识分子形成的组织的领导,来到了我的家里:他是个命途坎坷的人,儿子当兵死了;虽然63岁了,但有30多岁人的耐性,他的脸色憔悴得跟死人一样。他发展我参加了一支把生命系在刀锋上的队伍。

在仲夏的夜晚里,我们边走边谈,一直谈到天明——我们谈未来的宣传计划,谈英语无线电广播(很不幸,常错误地理解德国的公众意见),畅谈未来的行动计划。

举几个例子:

\begin{quote}
立即并有效地不让柏林和普鲁士成为政治影响力的中心。

立即建立一个组织,在纳粹被推翻后,能够负责清理德国南部地区。

立即赶走所有在1920年后来这里的普鲁士人。立即破坏1933年之后在巴伐利亚建立的军事工业。

此外,为了保证能建立起一个新德国:充公所有重工业;立即国有化工厂;指控所有瓜分希特勒政权的土地的高层叛国者,最先审判的应该是冯·巴本的内阁成员、迈斯纳、纽赖特、兴登堡、施罗特等等。

立即指控继续打仗的将军。
\end{quote}

这个头开得不错;虽然有点夸张,但无论如何算是去纠正过去80年错误的雏形。

就因为我们策划这些东西,我们就是坏德国人吗?这取决于你指的是哪个德国。如果你指的是那个正要灭亡的俾斯麦的异教大德国,那我们就是坏德国人。我们不恨此前的那个德国,那个德国孕育了且集中精力抚育了许多伟大的思想,而且已经持续了上千年。然而,从波茨坦有新人来了,他们用了几十年的功夫就毁了德国,把文化遗产拖入了废墟,用假货取而代之,让早期德国的千年历史成为了记忆。

不,我们不坏。我们是在为德国未来考虑,未来不能建立在果冻上。如果仅是像1919年那样仅做修补工作,我们的后代照样会死于普鲁士人发动的大屠杀中。猛一听很矛盾,但事实上,我们只有把德国从普鲁士的霸权下解救出来,我们才算胜利了。我们要抛弃他们重商主义的上层建筑,不让毫无意义的、受政府资助的过度工业化来束缚我们。只有这样,我们才能使生命变得有意义,才能移走阻滞人类发展的累赘,这才是上之上策。我必须这样做,尽管这可能需要数十年的时间。只有这样,我们才能保证,即使我们不成功,至少要能让法本公司战败。

\section{1943年8月20日\ 技术制造的灵魂真空}

有一份英国人散发的传单,指责党卫军杂志《黑色军团》把战争的基础不放在人类的行动上,而是放在魔鬼的诡计上。英国人还攻击这份杂志把灾难归咎于非理性原因。我现在不是《黑色军团》的支持者,更不支持甘特·德阿奎恩(Gunter d’Alquen)——此人的真实姓名可能是甘特·舒尔策(Gunter Schulze)。我认为这场战争的罪魁祸首是有血有肉的真人——那些贪婪地想要更大利润的公司董事、那些想更多军功章的将军、那些已经晋升为政客且为美化自己想要更多香脂的流浪汉——这些人在战前从来没有如今这样的机会如此表现自己的欲望。

说了这么多,我们解决问题了吗?我们能解释当前的呆滞吗?要知道,就在1928年5月,当希特勒宣布他们那帮人基本上是和平主义者的时候,德国人以几乎疯狂的态度接受了他们。那时的场面与现在的差别实在是天壤之别。我曾经提到过,有些女人吞食希特勒途经之路上的泥土,我们怎样解释这种现象?那个希特勒青年团的男孩子把十字架丢到窗外,并大叫道,“滚吧,你这个犹太猪!”这件事是不是更好解释一些呢?因为更能表现出一代年轻人的道德沦丧、野蛮无礼、想谋杀他人的倾向。难道英国人就不怕这群人一夜之间变得疯狂起来第二天便去攻打其他民族吗?

我必须承认这些问题让我感到沮丧。这表明在欧洲大陆和大西洋的岛国之间在思想上有条鸿沟。他们还试图用19世纪的老方法去描述眼前的这头历史怪物。

我们肯定要审判那些已知的幕后操纵者,我希望看到希特勒、戈林、戈培尔、巴本等人被绞死。我们这些德国人肯定要承担起祖先对我们的期待,背着铁十字勋章,穿越悲哀之谷,直到我们的最终目的实现。

然而,现在是不是有一个国家,由于缺少看问题的视角,否认在其国内会出现大规模的精神错乱现象?人们会不会指责手无寸铁的德国知识分子无所作为?有可能,但即使是英国内阁,虽然他们有各种武器,却懒惰得在希特勒掌权的前两年也没有用烟把鼠洞里的褐色老鼠熏出来。

我不想玩相互推卸责任的游戏。让我感到心烦的是这类思想方法,这种方法使我们看不到真正的问题,在我们这个时代的巨大危机面前盲目自满。如果一个国家听不见那预示未来变故的惊雷般的马蹄声,这个国家将会遭遇不幸!居住在撒旦的太阳下,又不学会相信上帝的民族,将会遭遇不幸,因为那太阳正在以可怕的方式升起!那些无法接受下述事实的人将会遭受不幸:

理性主义已经衰落了。这种异端邪说统治世界已经有400年了,如今已经不值得信赖了。伟大的非理性神秘主义又在敲门了。今天,我看到美国飞机第一次实施了空袭,在大白天轰炸了雷根斯堡(Regensburg)。这是我第一次近距离接触战争。那些飞机悄悄地飞过峡谷,看上去就跟白色的小鸟一样……我看到其中一架被防空炮击中了,冒出黑红的烟,然后燃烧着大火掉了下来。我看见一只降落伞脱离了大火。我看见降落伞的绳索着火了,人体坠落在地上。我驱车去泽布鲁克(Seebruck)看残骸。飞机残骸砸出一个14英尺深的大坑,汽油在大坑里燃烧,并冒着泡。发动机钻到地下太深,没有人企图把它挖出来。在大坑的周围,散落着人体的碎片——有一只脚、一个手指头、一条胳膊。这些人体碎片被装入一个土豆麻袋。

在W附近,有几个美国人比较幸运,安全着陆了。后来,他们被带走了,有两个难民对他们吐痰,押送美国人的士兵说不许,并挥舞着枪不让这些没有防御能力的人受辱。你只有在非资本家的普通人中间,才能发现人类的正派天性和天生的不愿做暴徒的本性。

从汉堡传来的消息超乎想象——街上的沥青融化了,活人被沥青煮死了。好好的城市都变成了废墟,到处是死人,高矮不同的活人围着死人看,远看就好像殉教者的参差不齐的石头王冠一样。据说有20万人死了。

我一般不信道听途说。我相信亲眼所见。在这件事上,我亲眼看到太多的事实。

我听说了大量关于汉堡人的故事,当这座城市着大火时,出现了大量极为野蛮的、混乱的举止,有人患了健忘症,有人穿着睡衣逃出家门在街上乱跑,他们眼色惶恐,手拿着空荡荡的鸟笼子,既忘记了过去,又不知道明天将如何度过。

这就是我在上巴伐利亚一个小火车站上看到的情况,当时正好是八月初酷热的一天,有40或50个这样可悲的人,打碎了一扇火车车厢的窗户,正忙着钻进火车厢,虽然火车站的站长气愤的大喊大叫,又推又搡,但这些人仍然我行我素,争着钻进车厢抢座位。

接着就发生了难以逃脱的宿命。一个人的手提箱是纸板做的,边缘脱落了,手提箱里的东西洒落在站台上,有一堆衣服,一套修指甲的工具,一个玩具。手提箱里还有一具烧焦了小孩的尸体,就跟一个木乃伊一样,一个几乎疯狂的妇女拖着这具尸体。几天前,这具尸体还是那名妇女的家人。到处是沮丧的哭喊,厌恶的喊叫,歇斯底里般的怒吼,有一只狗在狂吠着。最后,一名官员同情起这群人,安顿了他们。

我还听到另一个传言。燃起的大火吞噬了大量氧气,使离火场很远的人窒息。炽热的磷光体落下来,像下雨一样,把成年男人和女人的尸体烤成小木乃伊,所以有无数的妇女携带着自己家人的遗体,在家园的废墟上徘徊。

面对眼前的景象,你仍然不认为这场战争还不致于终止了一个时代吗?你还要避免承认技术革命已经走到尽头了吗?事实上,这场技术革命留下了一层可怕的灵魂真空——也许能填补这真空的只能是一群新崛起的信奉非理性和非机械主义的魔鬼。我们已经无法回到从前了,有人会怀疑这点吗?来拯救我们的真的是那骑着黑马的天国骑士吗?

\section{1944年7月2日\ 另一种愚蠢}

今天,我从施泰因(Stein)骑自行车回来,路上遇到一群从德国北方来的年轻的德国女工,她们被“分配”到那家坐落在阿泽峡谷的化工厂里工作——“分配”这个词很可爱,目前在德国商界很流行——她们挤在一起就像一堆海蚌一样。与其他德国人一样,她们像军人一样排队前进。她们很难看,看了简直让人生气,完全没有了女人味,就跟德国少女联盟一样。她们小跑着,就跟一群梳着金发小辫子的奶牛一样。我必须要解释一下她们为什么对德国很重要。

因为她们唱的歌很重要。这首烂歌是从布尔什维克那里借来的,旋律断断续续,简直就是垃圾。但副歌部分的歌词值得关注:

\begin{verse}
着火了

原来是歌剧院

那是我的家乡

那是我的家
\end{verse}

现在你应该承认,对那些挨炸的人们来说,这是一首很精彩的歌!一问才知道,这群面色迟钝的“母牛”来自汉诺威,那里的剧院确实在大火中烧毁了。我不敢说面前看到的是某种反抗,或是某种自嘲,或是纳粹已经把德国妇女改造成了“母牛”。或许这仅是另一种愚蠢而已。

我必须多告诉你点事,我在特劳恩施泰因(Traunstein)火车站,听到两名柏林爱乐乐团成员的对话。他俩显然思维仍然敏捷,气愤地告诉我一件在慕尼黑流传的小故事:最新一期的军方简报说:军方简报还没有来得及公布。

很好!确实很好!我希望看到对现政权的反抗,但不是这种戏谑笑话,而是党派政治。这最多反映了被纳粹阉割的德国人的孤独、软弱、呆滞。

为了公平,我应该补充说一点,我听说穆尔瑙(Murnau)有一个巴伐利亚人组建的党派在活动,在平茨高河谷(Pinzgau)圣约翰(St. Johann)附近有个奥地利人的组织在活动。参与这些党派的成员,很自然都是军队里开小差的,或丢了工作的工人——但如果我们不早进行这类活动,世界会多么的可悲啊!

我又想起那个困扰了我11年的难题。我就是不能理解为什么德国人全都变得疯狂起来,军事将领们竟然让希特勒先生操纵(一个65岁的老人除了尊严外还有什么是重要的?)——他们应该把那个慕尼黑乞丐丢到大门外去。我不理解为什么德国妇女会疯狂地支持希特勒。我又想起那个11岁大的男孩子格里哥·斯特拉瑟所引发的问题,他在1934年夏天看着希特勒谋杀了他的父亲,仅四周之后,他却解释说:“那是元首做的,元首做的是对的。”

哈,疯狂,大规模的疯狂,像醉汉一样的疯狂,规模大到无法衡量,这将会是一场最恐怖的酒醉,世界上还没有人见识过!看看你们对无线电广播的操纵——看看那些麻木的大众,你们把人类社会转化成了白蚁堆!结果是真正的知识分子被禁言,这个因素不容忽视,因为结果就是暴民社会。我在美国和苏联都看到了这个现象。我本不该在这里提及苏联的例子,但苏联可以说是今日世界上最阴暗的糟粕。

我需要补充一点,我谈论到的人不是无产阶级,而都是中产阶级——小官僚、小学老师、邮局工作人员。他们属于是桑巴特\footnote{Sombart,1863—1941年,德国社会学家,主要著作有《19世纪的社会主义和社会运动》和《现代资本主义》。}所谓的“进步的桎梏”的阴暗阶层。这是我在读奥特加·伊·加塞特\footnote{Ortegay Gasset,1883—1955年,西班牙著名作家。}的《大众的反叛》看到作者的一条引用的词汇。加塞特极不赞成今日的德国,因为他是个具有世界眼光的保守派。

我认为我们中那些正在收集材料日后写《第三帝国史》的人,最后很可能都不得不把书名取成《邮政员和小学老师的革命》。

\section{1944年7月18日\ 烧焦的大地}

我在基姆高的家中,见证了慕尼黑遭受的最猛烈的空袭。连续三个小时,飞机的嗡嗡声一直没有断过,爆炸声根本没有停息过,就连大地都在颤抖。即使在我们这里,离慕尼黑还有90公里远,窗户玻璃也被气压冲碎了。不久,有飞机从头顶飞过。在很近的地方,我听到两声爆炸声,可能是防空炮开火了。我看到一架银色的金属鸟——我看不出是德国的或是英国的——旋转着落在大地上,就像被秋风吹落的疲惫树叶一样。落地点离我们这里有10公里远。

谁能保证炸弹不会从我的房顶上坠落下来?我的财产虽然很少,也是拼命挣来的,而且还熬过了通货膨胀,但就凭一颗炸弹就能将之炸毁。一家英国广播电台说霍庖丁(Horpolding)是军需品供应站。这个地方离我们这里仅有8公里远,也在轰炸范围里。我眼前的这条河的下游水本来很清澈,谁也不妨碍,但如今被工业废料给污染了,那些导致德国毁灭的将领们对此要负责。

我看着我房子里的东西,我珍爱的图书室、中世纪的小雕像、中世纪的枝状大烛台和绘画,此时这些东西给我一种奇怪的感觉。哈,你有没有见到过一个临死之人床边摆放的财产?你一定知道他拥有这些财产已经毫无意义。

公路上逃离慕尼黑的人流看不到尽头。在慕尼黑,有数万被炸弹驱赶出家的人,蹲坐在马克西米里安广场(Maximiliansplatz)旁边的街道上,度过阴雨连绵的夜晚。家庭破裂的老妇女,肩上扛着木棍,木棍上系着包袱,里面是裹着在这世界上的所有家当。这些可怜的无家可归之人,穿着被烧焦的衣服,眼睛里仍然流露出对大火灾的恐惧;那能震碎一切的爆炸,废墟下的死尸;在地下室里,不仅有可怕的死人,还有被污水和排泄物窒息的人群。

可希特勒先生不会为此担心。为什么要担心呢?我们都听说过,他在深埋地下的防空洞里白天看小说,到晚上就看电影——对杀人犯和情绪不稳的魔鬼来说,夜晚是痛苦的。那个笨蛋施佩尔为什么要担心?他简直就是这病态的、机械的、幼稚的一代人的典范,只会用他的那简单的语句表白自己。我必须再说说施佩尔:他跟巴本一样,就把屠夫的良知和荣誉感与愚蠢相结合,搞破坏不找理由,简直是在犯罪,就是他的脸是我看到的纳粹集团的候补队员中最令人恶心的,仅次于德国的那些诸如克虏伯那类假保守派、假贵族——施佩尔这家伙竟然以为自己是达·芬奇的化身。

\section{1944年7月20日\ 指挥大师的嗜好}

玛丽亚·欧采芙丝卡(Maria Olczewska)来看我。我们谈到了指挥大师富特文格勒\footnote{Furtwangler,1886—1954,德国指挥家。}——这个话题我极不想谈及。显然,他有一种利用“金发美女”的指挥方式。这是一种奇怪的嗜好,无论是真假,是一个人选择后的结果。

我无力去改变。

\section{1944年7月21日\ 背叛希特勒}

有人想刺杀希特勒。行刺的是冯·施陶芬贝格(von Stauffenberg)伯爵。我认识他的父亲,那是一位具有完美人格的人,是德国贵族的典范。这次刺杀的意义——说明将军们要起义了,这是我们期待已久的。

哈,现在才动手,先生们,有点晚了。你们制造出了那个魔鬼,只要进展顺利,他要什么,你们给什么。你们把德国交给了这个十恶不赦的大罪犯,无论他在你们面前摆放下多么难以置信的誓词,你们都表示效忠——还记得你们是谁吗?你们全是德意志帝国的军官。你们让自己成为了他的奴隶,他身负着10万条被他谋杀的人的性命,他是全世界都痛恨的目标。

现如今你们背叛他了,就如同你们昨天背叛魏玛共和国一样,就如同前天你们背叛君主一样。哦,我丝毫不怀疑,如果这次政变能成功,你们能拯救我们,我们还算是这个国家的残留物。但我很遗憾,整个国家也很遗憾,你们失手了。

然而,想一想你们自己的身份,你们是普鲁士邪教的实施者,你们传播罪恶,人类对你们嗤之以鼻——你们能做未来德国的领袖吗?不行。

我是个保守派。在德国,保守派几乎销声匿迹了。我的思维源自君主制度,因为我是受君主制主义教育长大的。只有君主在世,我的身心才能获得健康。就因为这点,我恨你们。你们是卖弄风情的女人,总是见风使舵!你是历史的叛徒!你们是那些工业寡头的可恨走狗,他们那些人如果上台,肯定会破坏我们国家的社会结构和政府组织的!你们甘心情愿地为克虏伯公司制订掠夺俄罗斯的计划,可是如今计划失败了,这才反映出你们在政治上的浅薄,在地缘政治上的无知!你们不懂什么是适度,什么是秩序。你们在所有可能的场合都提倡无神论,否认灵魂的存在——除了你们无聊的普鲁士实用主义之外,你们恨一切美好的东西。

鲁普雷希特亲王(Rupprecht)曾经告诉我,在许多年前的第一次世界大战中,他那时是个集团军司令官,他请求鲁登道夫去救库西城堡,那是一处珍贵的建筑物,位置处于两军阵地之间。“那座建筑根本没有军事价值,对我们没有,对敌人也没有。双方都没有想到把它用于军事目的。可我要求救那座城堡,不能使之受损,因为害怕有损我们的声望。但声望是相当没有用的——这反而引起了鲁登道夫的注意。鲁登道夫下令炸毁了那座城堡,这就如同无故打我一顿。”

“但他那座城堡并非是因为我想保护它。他恨库西城堡,是因为他恨一切不属于军营的东西——精神、品味、优雅,换句话说,他恨一切使生活变得不同寻常的东西。”

哈,这位伟大的老毛奇的渺小侄子,却住在妙不可言的城堡中。许多年以来,他们的每一宗叛国行为,每一次的杀人劫掠行径,都被掩盖过去了,因为希特勒让他们再次掌控这支被普鲁士化的品质恶劣的军队。作为回报,他们保护希特勒,他所犯了任何一桩罪行,他们不仅为他辩解,而且采取行动保护他。他们高兴地看着空袭的受害者、集中营里的囚徒、受宗教迫害的人士。他们几乎不提及“德国”和“德国精神”,因为只要改朝换代,他们就会失去权力。

如今,这家公司要倒闭了,他们马上就背叛了,借此证明自己在政治上无罪——就如同他们背叛那些在他们追求权力的道路上变成无用的人一样。

整个国家都哀叹那颗炸弹被放置在错误地点,爆炸时间也不合适。我内心中深深的遗憾,与大家是一样的,无以言表。但我要对那些将军们说:一旦德国被从普鲁士的邪教中解放出来,他们就要被枪毙,同样要被枪毙的人还有那些发动这场战争的工业家、为这些工业家唱赞歌的记者。我们还不能忘记梅斯纳、兴登堡的儿子,以及那些在1933年1月30日的那场巨大错误变故中负有责任的人,绞死这帮家伙时,一定要把绞架再提高20英尺。那些不判绞刑的,也要治他们的罪,让他们终生卖火柴和废纸,借此提醒我们是谁偷走了权力,把无尽的悲惨的生活倒在我们的头上。

我帮不了他们。

\section{1944年8月16日\ 呼吸着死亡的气息}

空气里弥漫着死亡的气息。我不是指我所在的这个地方,国外的广播说——枪毙了5000名军官;纳粹在滥杀无辜,根本不是看是不是与刺杀有关联,只要是不喜欢的人就杀。是的,有的时候是把一家人一起枪毙才算完成任务。

不妙啊,我预感到有什么可怕的东西正向我们扑来,弥漫在夏日的空气中,这让阳光看上去都变得可怕起来,仿佛我们就生活在葬礼的火炬中一样。我们满脑子里全是即将到来的灾难,恐怖和死亡气息包围着我们。我们正在变成一个粗俗的民族,我们的年轻人都被灌输了政治无赖合理、消灭种族合法的理念,我们的军事领袖变得各个具有不撞南墙不回头的性格。

我们呼吸着死亡的气息。虽然奥宾(Obing)是个祥和的小村子,但这个村的妇女运动主席却不那么祥和,她最近告诉我,她赞美元首,因为他好,如果这场战争以失败告终,她就准备放毒气熏死所有德国人。哦,我不是在写小说。在我的脑海里,她简直就不是人。我观察着她:皮肤是日晒黄褐色,40多岁,眼睛里放射出疯狂的光芒——我要提醒你们注意,除了小学教师之外,这些母狗就是崇拜希特勒的疯子中最疯狂的群体。

听了她的话,人们的反应如何?这些巴伐利亚农民准备反抗吗?要知道这些农民的父辈可是具有独立精神的。他们难道不会把这个浑身燃烧着火焰准备去赴死的女人丢到奥宾湖里去吗?

这些农民从来没有想到这点。他们跑回家去了,困惑地摇着头,低声相互嘀咕着没有办法的不幸。

另一方面,慕尼黑电厂的工人们说,到了最后清算的时候,他们要用电烙铁给每个纳粹党徒的脑门上烙上一个纳粹十字记号。这个想法很好,但可以补充一点细节使之完美:再让他们穿着褐色衣服了却余生如何?

\section{1944年10月9日\ 德国正在变成一座垃圾堆}

吉斯勒(Giesler)先生想出一套新的监视技术。如今,每座镇子和每座村庄都有“居委会”,无论白天或黑夜,居委会的检查人员都能破门而入,有权占用人们的生活空间。由于这些人具有“分配劳动力”的权力,他们可以凭借自己的喜好,迫使妇女提供“自愿的劳动”。

这就是发生在我们住宅里的情况:如同天上的闪电,丝毫没有预先通知,连大门都没有敲一下,一周前被派到我们的这个宁静村子里的小独裁者们就会出现在我的客厅里。自他来到村里,他就费力不讨好地想改变农民打招呼的传统用语。在我家,他在征用两个房间的基础上,又征用了我的书房,并且和气地承诺让每个房间住一个女人和三个孩子。为了让这帮人住下,我家的哥特式的墙壁和地板被钻了许多窟窿。他们命令我把书房里的图书(包括绝版和手抄本)转移到地下室去,这些书说不定在那里会被老鼠给啃掉。“别发火,好多书都进了下水道。为什么你的书不能下地下室?”

他说这话时眼睛里闪着卑鄙的光芒:他过去是一名税务员,此时意识到自己手中的权力,觉得自己的形象已经膨胀得跟拿破仑一样大了。在第一次世界大战中,我记得一群守卫防空洞的人用从列文(Lievin)的米索森城堡里拿来的绝版书和无价的手稿烧炉子。但这些士兵处于饥寒交迫中,自然是找到什么就烧什么,他们内心并无恶意……

但现在发生的情况却大为不同。眼前这个地位卑贱的小官僚,他是在对受过教育的人表示不满,他对一切比他好的东西都不满——他觉得等待已久的报仇时间到了,他要向社会地位比较高的人报仇。

此外,这个贱货心里还揣着另一种仇恨:他仇恨任何有精神的东西。就是这种仇恨,使得德国的中产阶级在19世纪中叶抛弃了自己最好的传统,变得异常尖酸刻薄,这在历史上是从来没有过的。

此后,又发生了许多事。在特劳恩施泰因,一名军官警告我:毕希纳(Buchner)先生把我放在了他的那份问候语有问题者的清单中了。除此之外,他还说,许多人都知道我涉嫌7月20日的刺杀行动,这样我就有必要立即找到与我有相同观点并“被炸得无家可归”的人,警惕有可能偷听我们也在听到广播电台的人,知道那些不可能告发我的人。我们找到了几个从慕尼黑来的给家具装软垫的工匠,我认为这几个人是可靠的,他们是一个“被炸得无家可归”的艺术家推荐的。我们还抓了一个美国人,此人在慕尼黑的生活至今没有受到影响,后来的事证明他是个极好的人。做这些事花费了大量时间,以为谁也不能肯定能留在自家里,需要乘坐拥挤的、肮脏的火车来往于慕尼黑,在纳粹政府部门的接待室等待,必须听周围办公室里受到良好照顾的女人们的哈哈大笑,吃很多质量有问题的冰激凌和饼干。

所以,虽然纳粹的统治基础在动摇,但我也陷入了困境,不知道何时我的房子就会不属于我了。但有一点很让人发笑,我因此而获得了对这个体制更深入的了解。

我走了很长的路,来到指挥部,指挥部里的情况很符合人们的想象,坐满了“指挥官”,这些人过去全是办公室管理员,如今各个都在扮演成吉思汗——空气里弥漫着神秘的恐惧感,但他们都装出粗鲁和野蛮的样子。

屋里还有一名盖世太保,我拿着一封介绍信找到了他,他的行为举止出乎预料之外:安静、有尊严、有礼貌、负责任,做评议员工作;他有礼貌地请我等一下,允许他先把香烟抽完,这是有教养的人的标准行为举止。盖世太保的问候语是“赞美上帝吧!”,而地方办公室里是震耳欲聋的“希特勒万岁。”

我描述我的状况,指出省长手下的人威胁我随时可能来搜查我的房子,这位盖世太保官员实际上指出这类威胁是有原因的,因为纳粹可能在两三周之内就会垮台。

目前的气氛很怪异,弥漫着恐惧和观望的情绪,混在其中的还有怒吼时发出的疯狂。德国正在变成一座垃圾堆,这都是借了那位大神仙的光——这都是因为世界末日的细菌导致的奇怪病症!

想一想,这些白蚂蚁,无论白天晚上,时刻攀附着组织,就好像葡萄攀附着葡萄树架一样——再想象一下,如果知识分子把他们的脑袋切下来,他们照样能为组织拼命工作!气氛紧张极了,明天,任何时候都有可能,天空里就会出现闪电。除了纳粹的死党,人们都知道要发生什么,他们有怨气;他们的怨气随时都有可能发生,在邮局的窗口前,在有轨电车上,在傻乎乎地等着买报纸上说有的东西而排出来的队里。

人们的神经紧张极了,随时可能争吵起来,接着就会动手打架。我看见一个正在等着上有轨电车的16岁的女孩,抽了一位瘦弱老人的嘴巴子,因为她嫌弃那老人下车太慢了。我上前回敬了她两巴掌——打在她脸上,就是为了回敬这贱货的抱怨。我的举动让那个年轻的女孩大为惊诧。

我从前没有见到过这样的事。在慕尼黑革命期间,所有人都很有礼貌,这与希特勒的时代完全不同。慕尼黑被污染了,被扭曲了,就如同德国其他城市一样,这都是普鲁士蝗虫的罪过,现在慕尼黑给我的感觉,就如同芝加哥给我的感觉一样。

哦,如果一座城市昨天还像是幽默的母亲,今天却成了废墟,在这废墟城市中漫步,那可是极为令人讨厌的。我走过一条街,一栋房子垮塌了,顿时升起一阵巨大的尘土,一堆5米高的垃圾覆盖住了我们刚刚走过的小路。现在我在写字,仍然能闻到死尸的腐臭味,因为在那堆垃圾下埋着17具银行职员的死尸。这时一场虔诚的纪念会开始了,这几具死尸被从破裂的下水道里流出的人粪尿给淹没了,垃圾堆的顶上放着幸存者留下的十字架,而依靠吃尸体长得肥肥胖胖的老鼠,在垃圾堆和十字架上毫无顾忌地飞快地奔跑着。

现在仍然打不通电话,也没有人在服务窗口外面排队等候;商店里没有东西可卖,或者是有货但没有人来买。尽管如此,生活在城市废墟中的人,仍然要继续活下去,他们像没有头脑的动物,无论白天或黑夜,冲到饭馆去抢敞开供应的食物,就好像动物园里的猴子等着吃午饭一般。他们喝光化工啤酒,盲目地相信宣传说的一切,就因为他们,我们才被一个疯子统治了12年的时间。那些被剥夺了昔日好时光的人,被这群恶毒的猴子统治了12年的时间,难道他们不应该期待自己的国家被打败吗?这不是悲剧,这是耻辱。

\section{1944年10月\ 将命运交付刽子手}

有人被逮捕了,又有更多人被逮捕了。简直就是疯狂地逮捕人,但这无法掩饰他们内心的极度恐惧。

托尼·阿尔科\footnote{Toni Arco,此人在1919年刺杀了巴伐利亚的内阁大臣艾斯纳。}被逮捕了。我敢肯定他痛苦地后悔在25年前刺杀了艾斯纳(Eisner)。沙赫特(Schacht)被逮捕了,老胡根贝格(Hugenberg)也被逮捕了。沙尔纳格尔(Scharnagl)市长也被逮捕了,同时被逮捕的还有几个与皇室有关联的女士和几个年轻的新教徒。

人们不留任何痕迹地就消失了。几周后,或几个月后,竟然还是杳无音信。家庭就这样被拆散了,蒙在鼓里。A被逮埔了,FR据说也进了监狱,他的兄弟是伯爵,也在去维也纳的路上失踪了。只知道他在奥地利一座火车站的站头上露过一面,当时带着手铐,左右有两个看守。就在两年前,他的两个儿子被这场战争吞没了。

有些消息是关于巴伐利亚国王的,既奇怪又严酷。M先生收到从意大利北方传来的消息:“不要为上校担忧,他在多洛迈特脉(Dolomites)很安全。”在当时的情况下,“上校”无疑指的就是75岁的巴伐利亚国王——这位君主沉浸在年轻时的回忆中,给我讲述了他与奥地利老国王弗朗茨·约瑟夫以及与俾斯麦见面时的有趣故事。俾斯麦曾向他描绘92岁的威廉一世在早餐桌上令人羡慕的胃口。很显然,威廉一世现在肯定被迫正在外国土地上后撤,从一个山头撤到另一个山头。

M先生是在10月初收到这封信的。现在是月底,有谣言说国王被杀死了。纳粹实在是再低劣不过了。然而,我对目前局势有个看法,打死敌人比让敌人活着更危险。

10月13日这一天,既美丽,又炎热,我被逮捕了。

早晨6点钟——秘密警察最喜欢这个时间——我听到门铃大作,看见我们泽布鲁克的警官站在楼下,他是个好人,道歉地解释他来是要执行一项令人不快的任务,要把我带到坐落在特劳恩施泰因的军队监狱去。

我承认我不是很担忧。四天前,我没有理睬让我参加国民自卫队的“邀请”,理由是我心绞痛犯了。然而,我像其他好公民一样赶紧去地方政府做解释,我说刚接到俄罗斯来的消息,儿子在俄罗斯失踪了,这才引发了疾病。

我犯了一个错误。欺骗,在炎热秋日的明媚阳光下的欺骗;欺骗,老练的欺骗,几乎算是无耻的欺骗,那警官的欺骗。我们渡过了河,去乘坐火车。我家里的女人忧郁地向我挥手,这让我思绪万千。几个小时后,我确实发现那不是一个无足轻重的警告。

军营的大门在我身后重重地关上了。在我和明媚的秋天之间,出现了一道围墙和一个警卫。我站在一个有警卫的军营里,这里充满了皮革的味道、汗的味道、猪油的味道。这里的首长是一位年轻的军士,是个斯瓦比亚人——这个人身上有一种日耳曼人特有的暴躁、好动、严格。这些特点,我一直不觉得有什么好的,反倒给世界带来很大的麻烦。

我给负责这片地区的少校打电话。电话里传来的辩护声是冰凉的,这说明我不能问,只能听。就在这时,我看到一个年轻军官正骑自行车穿越军营,我认识他。我叫住了他,他不愿跟我握手。我向他解释,我是被捕进来的,在老德皇的东线军队里,我这样的人被称为“令人恶心的人”。听到这里,他笑了,并与我握手,还亲自打电话。电话筒里传来噼啪声,他脸色变得惨白。他把电话挂了,然后用多了几分的严肃劲儿说,我的罪状是“破坏军队的士气”。他低头离开了。

犯了“破坏军队的士气”罪的人,要被处以极刑。最近,我听说,有一个被判犯这种罪的人,被用亮度达到数千烛光度的光线照瞎了,后来就变成了解剖课的尸体。

夜幕降临了,军营一片漆黑,我被锁了起来。

囚禁室宽2步,长6英尺,像个混凝土棺材,有一张木床,还有一个恶臭的痰盂,一个有铁栅栏的窗户高高地挂在墙上。爬到床上,我能看到一片极小的天空、兵营、军官宿舍、地处兵营之后的小树林:一片我们巴伐利亚高原上的可爱的松树林,这片树林与疯狂的普鲁士的军国主义毫无共同之处,普鲁士人是破坏巴伐利亚的害虫。

窗户里就能看到这么多了。在墙上,猥亵的言语和时间记号仍然清晰可见——多少周了,多少天了,多少小时了,甚至多少分钟。墙上还有一大堆苏联的红星,这给人一种印象——似乎整个苏联军队曾被囚禁在这里。最后,我看到几个刻在墙上的字,可能是用钥匙刻的,其意思很适合我:“上帝啊,你为什么要遗弃我?”我读着这句话,心里难过极了。写这句话的人像我一样马上就要赴死了。

确实没有谁说过要处死我。不过,我忽视他们的歹意,他们就是想找到点理由对付我,他们只需给刽子手随便找个借口就行了。

你会相信吗?一位60多岁的老人,一生都过着受人尊敬的生活,而他刚接到消息说他儿子在俄罗斯成了战俘,他因为伤心而去“破坏军队的士气”。你去问问普林镇的老医生,他们会给你真正的答案。

然而,发生在我身上的与绞刑告示无关。

我一夜都喘着粗气,室外是野蛮的军营里的噪音。我们被关押在这里,不许过宁静的夜晚。他们关门的时候,总是用尽全力把门撞上。当有人要求去茅坑,走廊里就会传来看守恶毒的咒骂声。凌晨三点,下岗的士兵在矮护墙上跺脚,仿佛大象踏步。五点半,虽然我们起床对任何人都没有好处,而我们睡着了却不会损害任何人,但门仍然会裂开一条缝,有人怒吼道“起床!”——我们因为忧虑而一夜失眠,此时刚好睡着。

我思考着谁是害我的人,他为什么要把我交给刽子手。我认为可能是地方长官,我曾把他告上法庭,因为他骂我养看家狗是懦夫——他在这桩案子中失败了,所以他可能会报复我。我想起咖啡厅里的那个老宣传员,他夸夸其谈后,我没有鼓掌。我想起那个居民代表,他发现我的打招呼用语违背规定,而且我两次试图把他赶出我的房子,我才不管他是否是在“为政府做事”。我想起了如今在这个腐烂的国家里满地爬的小东西,他们依靠指责他人养活自己——他们有的大规模地屠杀,有的小规模的谋杀,但如今都受法律保护,他们就不想一想,有一天他们也会落入刽子手的手中。

然而,我并没有因为联想到他们未来的遭遇而感到高兴,这点让我思绪万千。奇怪,我进化了。在10年前,我曾经制定可怕的计划要报复他们。今天?今天我不再想报仇的事,《圣经》强调不报仇的段落反映了古老的高尚想法。报仇?几年前,我将一个处于绝望边缘的老朋友带进我的家里。虽然我善待他,给他钱,但他竟然企图破坏我的婚姻。我狠狠地揍了他,狠得不能再狠了。但我心里的安慰感,也就持续了三天。

如今怎样?我早就认识到了这点小事在永恒的世界里显得微不足道。如果我冒险进入上帝创造出的人性中,并杀死了他,我就帮助他像一个英雄一样死去,而不会延长他不名誉的生命。我自己曾经让许多人流泪;难道我没有为这些泪水付出过努力吗?或许我的报偿在未来才能偿付。难道我不知道死亡正在逼近我吗?死亡不仅把我和我爱的人分隔开来,也把我和那些想诋毁我的肮脏的人分隔开来——虽然我不想看到这些人,但这样的付出是必须的。

不是基督徒也能理解这些道理。但只有基督徒才能公开地接受这些道理,像个英雄一样去死。1912年,在一艘英国人的沿海汽轮上,我那时还是一个跟威廉一世的儿子一样无知的年轻人,我跟汽船上除我之外唯一的旅客在甲板上做晚间散步,他是老式中国知识分子。我对他讲解了基督世界的情况,我们陷入一场空前的巨大痛苦之中。那个庄严了老人是个道家学者,青岛大学的教授,他愉快地看着我,并平静地说,可见基督教仍然面临着一个大任务要去完成。他说话的方式给我留下了深刻的印象。

三十年后的今天,我所犯下的几宗大罪压弯了我的腰,我的生活几起几落,所以我知道生活不易。很显然,基督世界面临着一个大任务要去完成。可撒旦正在统治世界,所以我们要准备第二次把撒旦埋人地下墓穴,再来一次尼禄火烧罗马,从而使罗马精神像个胜利者一样得以重生。

\section{1944年10月14日\ 绝望地阅读}

我以为就跟去一家旅馆住一夜一样,所以只随身携带了一个小提箱。他们来搜查武器:这不是个好预兆。我要求请律师,但被粗暴地拒绝了。

很快我进了监狱。站在床板上(这违反规定),我看到了完美的秋天景象。我已经被剥夺了欣赏美好景色的权利,就如同第一次世界大战偷走了我们的权利一样,就如同1920年代的通货膨胀一样,就如同希特勒时期一样——四分之一个世纪就没有了,这是人最美好的一段时光——被这群战争疯子给抢夺走了。

走过军官宿舍内的院子,钻过虽说便宜但如今仍然被视为优良的门帘,从一个房间走进另一个房间,你看见一群新涌现出的金发军官,他们过去可能就是电梯服务员,你可能还在他们的手里放过2马克做小费呢(就是这双手可能要清理杂物和垃圾)。他们崛起了,但我们却在过去的12年中倒霉了;显然,帮助他们崛起的钱可是我们出的。他们的那位患了精神分裂症的领袖,一分钱都没有,但从1918年起,他们崛起了,成了现在的这个样子。德国简直就是希腊传说中国王三十年未清洗过的马厩,留给我们去清扫!

现在,他们在阅兵了,我从早晨听到晚。领头羊厉声叫喊,后面的250只组成的方阵齐声呐喊。真是祸害了,听听他们那愚蠢的歌声,看看他们的脸,就知道他们的精神早就被宣传所阉割。他们向前行驶着,发出轰鸣声——看这里,每5个人跟着一台机器,再看那里,一个巨大的怪物笨重地前进着,喷射出恶臭的气体,上面坐着10个人,接着又是一个新式的机械魔鬼上坐着5个人。这些铁皮做的魔鬼跟士兵有什么关系?应该把团徽从他们的军装上撕掉,换上用金线绣出来的螺丝起子和油壶。

我想澄清一点:我是个老兵。我17岁时,骑马走在定音鼓的后面,这才是我感到自己该做的——士兵该做的。但随后而来的机关枪和四缸引擎引发了一个问题,士兵这种职业是否还存在,如果士兵不存在了,难道说政治家、国王、诗人、知识分子还能存在吗?或者说传统职业都被职业娼妓替代了。(由于规定太多,公开的娼妓已经销声匿迹了。在戏前和高潮时,女方必须两次高喊“希特勒万岁!”)以我为例,我变成了一个和平主义者……这不是因为我很喜欢天生脆弱的文艺作品;不是的,我是因为想在那个该死的谎言葬礼上做司祭——这个谎言就是“士兵”的概念可以随意滥用!

下午,我被带去参加一个听证会。听证会由一个地方领导人主持,他带着军士的徽章,看上去很像一个有身份的巴伐利亚小中产阶级(他可能是邮局柜台后的职员,也可能在一家繁忙的律师事务所工作)。我坚持说我是因为受到愚人的指控才来这里的,那家伙肯定是一个低劣的阴谋家,那位主持人的脸色全变了,对我发出刺耳的尖叫,就如同一个大喇叭一样。我等着他消耗掉肺里的全部力气,然后真诚地盯着他的眼睛,以一个没有防守能力的人的勇气站在他的面前——这就是我当时想要强调的。

随后大量的指控就像洪水一样落在我的头上:

——我曾经错误地报告了我的军阶(我回答说,我流过太多的血,不再重视军阶)。

——我早年犯过错误,轻视民兵组织。我证明实际情况与此相反。

——我曾经组织妇女游行,反对把十字架从建筑物中转移走,在游行时没有按照要求喊“希特勒万岁”。此外,还贬低德国货币的价值。

我是这样回复质疑的:我们这里是军事法庭还是党的法庭?对德国货币的指控,我希望提供更多的细节。

我的努力是不会有结果的。接下来,我遭受了大量恶言漫骂,猛烈得就跟火山的熔岩一样,不让我有一点反驳的机会。我沉默了。他们把我带走了。

但我不是一个很容易就会放弃的人。他们把少校叫来了。当我见到他时,我知道:只能依靠高官才能救我了。他是个幽灵,一个男宠,一个被枪炮声吓怕了的战战巍巍的假木偶。他什么事也办不成,没有什么能办成——这个人是可怕的机器。他的眼睛里流露出病态的残忍……

我知道这类人。我在自由军时期见到过他们。这些不信神的鬼魂,内心充满了虐待狂般的怒火,穿越时间又来到了今天,他们是纳粹体制中的军官,各个都是涉嫌严重犯罪的残废人。

我又孤独了。远离了那个我称之为家的地方,那里还有落日的最后一片红云;在这里,一边吃饭,一边听着皮靴的笨重声音。在监狱里有一个很奇怪的现象,人们很容易就能做任何最卑微的工作,以便让生活更加容易一点。你能一点都不恶心地清扫发出恶臭的角落。你能毫不迟疑地躺在爬满虫子的草垫子上。你那身由连在位王子出门购物都要来打个招呼的伦敦裁缝制作的西装,都可以毫不在乎地在破木板子床上来回蹭……

只要你做这些小事,你就能很快使监狱的生活变得容易一些。有些好意的小伙子,把门锁打开,突然你就有自由了,可以在小隔间外面靠着门走一走了。你不必真的去走;想一想也足够了。

次日,我进一步融入狱友们的世界。我打开门闩,第一次面对面地与我的狱友邻居见面了,我俩可算是住在混凝土的小隔间中的难兄难弟。此前,他们仅是通过打信号相识,方法是敲击墙壁,我很快就学会了这方法。在这支操着多国语言的队伍中,能看见平淡无奇的销售员的脸、惨白的脸、基层小官僚的愚蠢的脸、穿着军装的职员的脸。在他们之中,波兰人和捷克人(甚至丹麦人和瑞典人)就像是杯子里的骰子一样被投掷到这里来了,但他们能给我们大家以安慰——这就如同我在外国听到了德语一样。

——有个小家伙,穿着摩洛哥警察衣服,但几乎盖不住身体,就像一头熊一样昏睡不已。他已经多休了五天的假期,有个本地姑娘勾引他,而他也喜欢在她家的池塘钓鲤鱼吃。

——L,是个诚实的迪纳拉人,脸长得跟马一样,但表情很严肃。由于痛恨本族人,在边境一带流浪了五个月。在一个军事安检站,他被抓住了,发现身上带着一把左轮枪——这是一件很严重的事。抓他的那个士兵,突然良心发现,对着他的耳朵低声说抓他很后悔。这个案子很严重;他可能会因此而丧命。

——T,克罗地亚人,被指控与俄罗斯人在边境地区做买卖——有笨蛋把指控信投递到了军队,而这事跟军队一点关系没有,就如同我跟火星人一点关系没有一样——其实,他是个有品位的人,友好、可爱,按照这个瘟疫区的标准,他算是有教养的。在小隔间的阴暗角落里,坐在滋生蚊虫的稻草垫子上,我们谈论了一会儿他的祖国。他描绘了塞尔维亚人是如何把他们这些和平的酿造葡萄酒的人从多瑙河边的村庄赶走,因为塞尔维亚人想在那里定居。

“相信我,收成很好;谷仓里堆满了麦子,大桶装满了黑麦,工棚的地板上堆满了一捆捆的玉米和烟草。实际上,那个春天就有传言,我们要被赶出这块土地,忧郁的老年人都相信传言是真的,但我们年轻人嘲笑老年人胆小。塞尔维亚的官员为让我们放心,向我们保证没有此事……然而,就在传言变成现实的前两天,他们还严厉惩罚敢于传这类谣言的人。”

“所以,当事情发生的时候,你可以想象我们的震惊程度。只给我们12天时间离开,离开我们的村庄、酒窖和工棚。为了换取在波斯尼亚相对等的条件,以及完好的农场及其建筑物和富饶的农作物,我们被告知要把产品、财产、农具都留下。”

“老年人知道将会发生什么。那天晚上,许多人割喉自杀了,还有人上吊,甚至跳入多瑙河淹死。我们剩下的人遭遇悲惨,被投入一个伤寒病流行的集中营。后来,我们被用封闭的货车运走了。在此后的14天的时间里,我们生活在死人的恶臭排泄物中。”

“到站后,我们中的一部分人被锁在一座大宅院的冰冷房间里;另一部分人被丢进一所托儿所的破烂不堪的花房里;第三部分人被放进一座充满了虱子的营房里,那里曾经住过伤寒病人。先生,这就是向我们承诺给予的‘同等繁荣’的村庄。”

“如果是过去的朝代,”我回答说,“我说的是奥匈帝国,他们绝对不会如此的残忍。你都难以想象,他们只需要你向维也纳帝王的双鹰国徽宣布效忠就行了。”

“知道了。先生,但人们还是想过自己的生活。”

他指的是民族自决。民族自决的思想极其愚蠢,自1789年就开始四处传播,它引发的烈火烧毁了整个欧洲——其火焰之所以是如此地具有破坏性,就是因为欧洲知识分子的火焰太温和了,致使寻找上帝的火焰在地球上熄灭了。

我悲伤地躺下来。我生到这个星球上的时间太早了。我活不过这愚蠢的思潮。

……每天过得都很悲伤。寒风从墙缝里吹进来,秋天的太阳消失了,秋天走得太快了,在这石头棺材里看黄昏,就跟世界末日一样。

只要还有光线,我就阅读,而且是绝望地。我读极度愚蠢的回忆录,读充满了党派傲慢与偏见的日记,读呼吁让已经腐烂的拿破仑理念复兴的疯狂文章,这些理念正在荼毒着我们的生活……

“从前,人与人之间的差距很大——如今,完全没有差距。从前,有命运的说法——如今,有按日付的工资。名声——那是什么?称一公斤名声给我——多少钱?我们买可移动的假牙,在肚子里养殖肠菌类。我们把生活变成不连续的碎片,生活气息越来越少,每一代人都留给后一代人一个更加混乱的世界。公主们呢?她向那些为她父王工作的工人医院方向骑自行车,这些工人在她走过的时候,几乎一动不动,甚至都不会跟她打个招呼。”

这几段文字是一个男人在1915年写的,他在家里的女人希望他成功的驱使下,变成了毫无个性的大众中一员。

在天刚变冷的那几天,我又被叫去听证。这次听证会让我大吃一惊,因为幕后有变化。前天还是北风凛冽,如今是暖风袭人;上次还对我大吼大叫的首长,如今对我极为体贴。我真怕他在这次晚间听证会结束时亲吻我道晚安。

谜团解开了。在少校办公室的外面,挂着一件皮大衣,上面有党卫军的标志,这件大衣是德特尔将军的,就是他带给了我奇迹。他是比我年轻的一代人,他温和地斥责我。这次谈话对那位前下士如今成为首长的人是有影响的,但我不知道我是否受到了重视,或对方仅是喜欢这样谈话。

“将军先生想乘坐汽车还是步行?”这是普鲁士军营里的行话。这句话的意思是说那个马屁精可能马上就会自己瘫倒在地或自己跑远得无影无踪。

奇迹发生了。一个小时前,我可能要被埋葬在监狱里,如今出现了我不敢想象的一幕:我今晚就会被释放。

我被带了出来,然后再次被锁起来,经历着每一个可能会被释放的囚犯都有的经历:忐忑不安地等待每一个小时,唯恐最后一分钟发生变故。你寻找着每一丝使紧张精神得以缓释的线索:在监狱里最后的时刻与刚进监狱的时刻同样的恶劣。

我很幸运,一次空袭缩短了时间:我们就像在动物园里被打字员、洗衣女工、厨房助工们盯着看的动物一样,被带到一间狭窄的、低矮的地下室内,里面有各种管道——自来水或厕所下水。可以假设,我们这些人最好是淹死在这些污水管道里,然后尸体在室外被炸弹炸成碎片……

从地下室的窗户里,我能看到一小片天空和一片比较大的兵营的院子。哈,那单调得一眼望不到头的一排窗户,裸露在外的工棚,向每一个方向看都是世界末日般的丑恶——恐怖似乎是军国主义的根本特征……他们恨一切有精神和美丽的东西。他们迷恋的,或许是一个特大号的奇形怪状的骰子筒。他们拿自己对丑陋的热爱做原料,构建出一门要全世界的人都必须崇拜的宗教。

不,必须把他们连根拔掉,义无反顾地追杀他们,要用所有想得到的和想不到的手段去羞辱他们,把他们消灭到最低限度,当有关他们的记忆全部被清除了之后,世界才会有和平。

两个小时后,我离开了兵营。我感到自己像个被埋葬在了一个大墓穴中——不仅污秽,还充满了被阉割后的记忆。

古代有个迷信:获得自由的人不能回头看,以免被叫回来。所以,我一直向前走,绝不回头看,但我的那位下士朋友跑出来,用刷子一边给我的外套掸土,一边还说:“让这件事快点结束吧!”

你,我们的年轻朋友:让我们结束这件事,以我们共同的仇恨的名义;以痛苦受难的人类的名义;以世界的名义……

回到家里,我才听说究竟是什么原因使我能出来。要不是那个德特尔出面干预,事情就会是另一番模样。

\section{后记(Richard J. Evans)}

\subsection{}

弗里德里希·莱克(Friedrich Reck)的《绝望者日记》,是来自第三帝国内部的一份反纳粹的文件,极为有力,极为动人,但不好归类。这份文件最初由斯图加特的一家不知名的小出版社于1947年在德国出版发行。不久之后,这家出版社便停止了商业运作。当时没有引起人们的注意。到了1964年,有一家报社,刊登了一系列名为“被遗忘的书籍”的文章,这才引起人们的注意;而且是相当大的注意。1966年出版了纸皮书,还有了几种语言版本。该书的英语版本在1970年出版,印了2000册,最后一章中原先被删节的部分又恢复了。这本书不是传统意义上的日记;作者没有按时间次序写事件;而是包括了作者的思想、反省、故事、回忆。记录开始于1936年哲学家奥斯瓦尔德·斯宾格勒的死亡,终止于莱克死前一个月。

在希特勒的独裁统治下写这样的文字是有风险的。1937年9月9日,莱克记录了他的朋友警告他有危险,因为盖世太保袭击了一个熟人的家。“出于我内心的需要,”他写道,“我必须漠视警告,继续写我的日记,目的是写出一部纳粹时期的文化史……”后来,他用一个铁皮盒子把手稿藏在自家的田里,并把最后几页藏在干草棚中。到了战争最后几周,他的日记手稿才被从埋藏的地方挖出来,传给后人。但那几页藏在干草棚中的手稿被老鼠啃了。他的妻子把日记给了他的朋友科特·特辛(Curt Thesing),他是一位大众文学作家,曾翻译了亨利·福特的自传,并为一家名叫维特的出版社工作过几年。她请他为她的丈夫出版这本日记。特辛尽全力转录(大部分日记是打字,文稿有缺页,这意味着最后一章不全)。书名取自莱克跟特辛谈话中提及的名字:《绝望者日记》——不是莱克的绝望,而是德国的绝望。

这本日记之所以如此的非凡,是因为它猛烈地、公开地表达了对希特勒和纳粹的深刻仇恨。像其他保守派一样,作者视那位纳粹领袖是个庸俗的煽动家,他的统治,根源于1789年法国大革命引发的一场欧洲历史上的大灾难(1937年9月9日的日记)。莱克从一开始就不妥协地在几乎所有方面反对希特勒的独裁统治,这点与其他反对希特勒的人不一样,因为那些人或多或少都受希特勒的影响,甚至支持希特勒对犹太人的仇视,比如那些策划在1944年刺杀希特勒且不幸失败的人。虽然莱克是个保守派,但不反对犹太人。他与许多犹太人保持友谊,比如他在1930年与犹太数学家利奥·佩鲁茨(Leo Perutz)关系很好,直到佩鲁茨1938年移民为止。在纳粹统治下,与犹太人交朋友是危险的。他的秘书伊拉玛·格拉泽(Irama Glaser)就是一个犹太人,在1933年4月死去(可能是自杀),他深情地描述了自己对他的秘书能力的依赖。

莱克像斯宾格勒那样持有文化悲观主义,认为魏玛共和国毫无希望,因为太唯物、太僵化,破坏了德国良好的道德标准,腐蚀了社会结构和权威,对来自美国的低俗流行文化不加以拒绝。1925年,他写了一本讽刺美国商业文化的小说《美国式的结算》。他称自己在1933年出版的另一本小说《禽兽案例:机器的命运》为“德国第一次向美国主义宣战”。不过,在1933年,希特勒高调发动“德国革命”,其目的就是为了赶走美国主义,这正好应验了斯宾格勒提出的历史循环的观点。如果莱克能想到这点,他会有所醒悟的。纳粹刚夺取权力才几个月,莱克就意识到将会发生什么:确实,第三帝国比魏玛共和国还要坏。莱克看到,纳粹是非理性的大规模爆发,他知道德国因此会遭受灾难。另一方面,莱克不像大部分保守分子那样怀念1914年前俾斯麦和德皇统治下的和平、繁荣、秩序、稳定,而是能清楚地看到,纳粹主义的根基是德意志帝国领袖的傲慢、贪婪、不负责任,这点可以从他在1943年2月的日记中看出来。然而,只有极少数保守分子会把前德皇威廉二世描绘成为一个自命不凡的小丑,这点可以从莱克1941年9月的日记中看出来。日记对“普鲁士在德国南部和奥地利的行径”进行了旁敲侧击;1937年5月,正当许多反希特勒的德国保守分子希望普鲁士主义复辟的时候,莱克出语惊人,谴责这等于是在“推动普鲁士主义”。在他死后才出版发行的小册子《白蚁的灭亡》中,他把纳粹德国比喻为没有思想、完全丧失责任感的白蚁堆,他甚至批评说“普鲁士病毒将会与被害死的病体一道下地狱。”

\subsection{}

这本日记并非是莱克唯一向纳粹倾吐蔑视的作品。在1930年代,他出版了两部主要作品,它们很有特点,体现了作者的“内心转移”,他不公开反抗,也不离开德国去比较安全的国家,但借历史对比批判希特勒和纳粹。这两部作品都是莱克在与好朋友厄温·冯·阿雷廷男爵(Erwein von Aretin)谈话后萌发了写作动机,此人是巴伐利亚的主要保皇派,战后做听证的男爵。莱克问他“如何与希特勒的独裁做斗争”,随后产生了写一本有关1534—1535年间明斯特的再洗礼教徒起义的书。在叛徒博克尔松的带领下,一些极端激进的新教徒占领了明斯特这座城市,建立起一个神权政治的乌托邦,但很快就蜕化为一场大祸害;博克尔松变成一个独裁者,他引入一夫多妻制,处死持不同政见者,建立起一种恐怖的统治,一直维持到这座城市被攻破。随后,再洗礼教的领袖被抓,并被处死。这本书的副标题是“历史的弥天幻觉”。莱克用文学的形式写出了一本历史书,该书于1937年初出版发行,让细心的读者在字里行间看出希特勒和纳粹的影子;莱克在1936年8月11日的日记亲自做了今昔对比说明。

与此类似,他根据历史材料,写了一部历史小说:《夏洛特·科尔黛:刺杀的故事》。这本书于1937年末出版发行。书的开篇就是对雅各宾派檄文执笔者让·保尔·马拉的长篇分析。许多人相信法国大革命变得越来越激烈,马拉要负主要责任,比如巴黎监狱的九月大屠杀和实行恐怖统治。莱克把他描绘为一个魔鬼般谋杀人的精神病患者,准确无误地让读者看出他对年轻的贵族妇女夏洛特·科尔黛的迷恋之情。这才使科尔黛有机会在1793年7月23日进入马拉的寝室,并刺杀了浴缸里的马拉。她之所以要刺杀马拉,是因为她相信这样做能拯救法国,使法国不陷入恐怖和内战之中。科尔黛因此而上了断头台,死后成为一个浪漫人物。莱克一直崇拜她;1929年,他出版了一本有关马拉的小说,此后还经常写这个话题。在这里,对比是明显的,就如同博克尔松一样,读者都能看出来。写类似书的不乏其人,字里行间透露出批判希特勒的意思,但很少有人像莱克这样大胆:写博克尔松和科尔黛是极富有公民勇气的举动。不过,纳粹的审查机构似乎不觉得这些书是针对纳粹政体的;在他们看来,明斯特的再洗礼教徒的统治仅是德国历史的一部分,而马拉遇刺是法国历史。莱克的这两本书都没有被禁,结果官方也没有找他的麻烦。他巧妙地把博克尔松隐藏在神秘的学术研究中,加了拉丁文后缀、脚注等学术手段,似乎帮助他逃过了审查。直到有读者抱怨这两本书的政治倾向,纳粹官方才将他的书下架。

或许他的名声也帮了他。在社会上,他是个知名作家,不写政治性的作品,而只写娱乐性的大众小说;德国公众和学术界不把他视为一个文学巨匠。正如他在《白蚁的灭亡》中所写的那样,他用第三人称的口吻写自己道:“他写小说,写异国情调的小说——即使他写了如此重大的话题,他能有多严肃呢?”接着,他又补充道:“这些小说都有一个共同的话题,就是大众和他们不可避免的命运。”这其实是个事后说明。他卖得比较好的作品有:《海外贵妇》《纽约来的贵妇》《高悬在蒙特卡罗上空的炸弹》,其中第三本是一部幽默滑稽的短篇小说,卖得极好,莱克获得大量的版税;后来被改编成为一部很成功的音乐电影,参加演出的有几个德国影视大明星,比如有Peter Lorre、Hans Albers,到了1960年又重新拍摄了一遍,Eddie Constantine也出演了。这是个虚拟的故事,说的是一位退了休的英国船长,指挥着一艘巴尔干小国的炮舰,来到了蒙特卡罗,在一个狂欢节化妆舞会上爱上了一位蒙面贵妇,但在赌博中把炮舰上的钱都输掉了,这使得他没钱给爱人买一串珍珠。于是他威胁炮轰蒙特卡罗,除非统治蒙特卡罗的亲王把赌资退给他。实际上,那位狂欢节上的蒙面贵妇就是这位蒙特卡罗的亲王假扮的。这位亲王解决了一切纠纷,故事圆满结束。纳粹的审查人员怎么可能想到写如此虚幻作品的作者会用严肃的文字反对政府呢?

\subsection{}

除了写文章、短篇小说、长篇小说之外,莱克在一战前还赢得了戏剧评论家和游记作家的名声。他在《南德日报》(与现在的《南德日报》无关)的文化副刊做过几个月的编辑。到了30岁时,他成为了一名职业作家。1884年8月11日,他出生在东普鲁士的马祖里,在非常舒适的环境下长大。除了他的姐姐嫁给一个很成功的律师之外,他与两个兄弟都不很成功,没有达到父亲的期待:他的大哥放弃了军人生涯,做了艺术家;另一个哥哥陷入沉重的债务,在23岁时自杀了。根据父亲的计划,莱克最初做了军人。1904年,他停止服役,转而去学医(这会使他的服役期变短),后来全力投入文学创作;10年后,当第一次世界大战爆发时,他被判定不适合从军。此后,他靠写作过着不稳定的生活。他经常借债,并在1923年的通货膨胀中损失了大量钱财。他写书没有赚到多少钱,反倒是秘书死后留给他一笔相当大的遗产。

1907年,莱克娶了比他年长4岁的女学生Anna Buttner,他俩在慕尼黑市郊与4个孩子一起生活。不过,他俩的婚姻生活并不顺利,在1920年分居,在1930年离婚。到了1933年,莱克在上巴伐利亚的群山之上的波茵(Poing)购买了11英亩的地产,他在此地长久居住下来。1935年之后,他是和新妻子(Irmgard von Borcke)以及他俩生的三个女儿住在这里。对莱克来说,这片土地代表了他与自然在没有现代工业和现代化干扰下的融合;只有一条蜿蜒的小路通往美丽的基姆湖边,这片土地上长着柏树,有一栋中世纪的石头房子,在上巴伐利亚的旅行手册上是这样描写的,“那里有一栋古老的石头建筑,位置异常孤寂”。接下来的几年里,他在这里写作,并把日记藏在这里。他不时去邻村拜访,或去更远的慕尼黑进行社交活动,在咖啡馆里、文艺圈、俱乐部里与朋友和熟人交往。

莱克在自我封闭的状态下写下的文字,包含了许多有关纳粹德国日常生活的珍贵细节。他评论了选举造假的事(1938年9月);一些德国利用反犹太人为自己谋私利(1938年12月);开战前柏林战时节俭的悲惨境界(1939年4月);战争中普通德国人越来越艰苦和匮乏的生活(1940年11月9日、1941年9月、1942年1月);东线巴巴罗萨行动后虐待苏联战俘(1941年9月);在盟军的空袭下德国的基础设施开始崩溃,电话系统遭到破坏(1944年10月9日)……

比这些细节更加引人注目的是莱克对公众心理变化和不同阶层观点的记录。当1940年夏季德国攻占法国后,他记录下了德国统治者的极度兴奋,他的记录与我们从其他渠道获得的信息是吻合的。然而,他在1940年10月就感觉到英国军队最终会占领德国,这个预言无疑会被那些他称之为“被胜利冲昏了头脑”的德国人视为荒谬之极,但该预言在四年半后实现。1941年6月,他听说了入侵苏联的消息,他的态度是阴郁的,他有一种不祥的预感,但这次许多德国人跟他的感觉是一样的。纳粹一直对天主教抱有敌意,1941年发动了一次没有持续多长时间的贬低天主教的运动,莱克记录了1942年1月几个天主教徒嘲笑纳粹领袖的笑话,他所记录的轶事极有价值,说明了纳粹引发的负面反应(如果有人举报,说这样的笑话会被判死刑的,所以说笑话的人冒着很多的风险)。自1942年5月之后,莱克的日记变得越来越担心盟军的空袭,再次表明了公众的心态(他说汉堡有20万人死了,但空袭1943年8月20日刚开始,他的数字太夸张;公认是死了4万人,尽管这个数字也很坏,但离谣传的数字相差甚远)。1942年10月,莱克记录了东线有3万犹太人被杀,这表明仅有一部分大屠杀消息传回了德国,实际上大屠杀已经进行了好几个月了。

1943年2月初,德国在斯大林格勒战败,德国的公众情绪发生剧烈变化,这点被活生生地记录在日记中了,不过有点主观夸张;但莱克注意到,在1943年2月之后,对反对政府的言行的严酷镇压变得越来越强烈。在被处死的人中包括肖勒兄妹,当时还是学生,他俩在慕尼黑张贴谴责纳粹罪行的传单;莱克对他俩的死表达了极大的痛苦,描绘了他俩临死时的表现,这部分是这本日记中最感人的。在刺杀希特勒的军事政变爆发的第二天,即1944年7月21日,他记录了自己的态度,他的态度与其他德国人截然不同。大多数人谴责这起叛变行为(莱克说,“整个国家哀叹炸弹没有在预定地点爆炸”,这个说法太过乐观)。莱克谴责刺杀者行动太晚了,并视之为背叛而一笔勾销,因为他们背叛了他们为之服务的每一个政体,比如,他们在1933年背叛了魏玛共和国之后,便投奔在纳粹的大旗下。莱克预言,这些人不会是“未来德国的领袖”。他说的不错:他们代表了反动的势力(按照他的说法,他们是普鲁士异教徒、罪恶的传播者、人类鼻子中的恶臭),无论发生什么,他们注定要失败。

1944年8月16日,莱克捕捉到了一个有趣的故事,此时距离纳粹失败的日子越来越近了,空气中弥漫着越来越浓重的要自杀的情绪;就在战争晚期,大量纳粹领袖纷纷自杀。然而,他也临近被捕的状态。1944年10月9日,他注意到纳粹地方官员怀疑起了他。10月13日,他被捕了,原因是他拒绝参加“人民冲锋队”,这是一支临时军事组织,由适龄应征人员组成,大部分人都50岁或60岁了,充当抵御盟军进攻的最后一道防线。他面临一系列的指控,最严重的莫过于“破坏军队士气”这一项。他们被投入了一所军队监狱。正如他日记最后一章所描述,他最终被军队监狱释放(很可能在监狱里呆了一周时间),回到了在波茵的家中,但这不意味着麻烦就此结束了。像许多生活在纳粹德国的人一样,莱克成为被告密的对象,告密者很可能是他的出版社内部人员。盖世太保获得一封莱克写给出版社的信,他在信中花费很多篇幅抱怨了通货膨胀吃掉了他下一本书的版税。1944年12月31日,星期日,新年之夜,他被捕了,原因是他侮辱了德国货币。他被关押在慕尼黑的盖世太保监狱里。1945年1月3日,他被采了指纹。这所监狱在1月7—8日的空袭中严重受损。1月9日,他和大部分狱友一道被转移到达豪集中营。他的编号是137838。集中营的条件差得惊人。有数千其他集中营的人被转移进来,因为那几座集中营即将被苏联红军攻克,这致使本来就不富裕的集中营供给更加匮乏。疾病和营养不良四处蔓延,在1944年底的时候,呕吐、饥饿、虱子、斑疹伤寒折磨着囚犯:1月份有1903名囚犯死了;2月份有3991名;3月份有3534名;4月份有2168名。莱克病了,被送进病号隔间,但条件一样很差。年龄50岁以上的死亡率是80\%,莱克就在这个年龄组里。官方的死亡记录证明他在1945年2月16日死于斑疹伤寒。三个月后,战争结束了。他死后第二天,他的号码就被盖世太保放进“离开者”的名单中了。三月初,盖世太保写信给波茵所在的特劳恩施泰因地区的地方官,要求他们通知莱克的妻子他在2月16日死了。


\end{document}