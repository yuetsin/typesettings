%-*- coding: UTF-8 -*-

\documentclass[UTF8]{ctexart}
\usepackage{graphicx}
\usepackage{float}
\usepackage{CJKpunct}
\usepackage{amsmath}
\usepackage{geometry}
\geometry{a5paper,centering,scale=0.8}
\usepackage[format=hang,font=small,textfont=it]{caption}
\usepackage[nottoc]{tocbibind}
\setCJKmainfont{SourceHanSerifSC-Light} %设置中文字体
\punctstyle{quanjiao} %使用全角标点	

\title{儿戏}
\author{Alice Munro}
\date{}
\begin{document}
\maketitle

\newpage

我想,那之后,家里肯定是议论纷纷。

真伤心,真可怕。(我妈妈。)

应该有人看着。辅导员到哪里去了?(我爸爸。)

想想,这种事也可能发生在……(我妈妈。)

没有。别瞎想了。没有。(我爸爸。)

~\\

要是我们再一次经过黄色的房子,也许我妈妈会说:「你记得吗?记不记得你以前多害怕这房子?可怜的小东西。」

我妈妈有种习惯:对我在遥远的婴儿时代的种种毛病,她总是抓住不放,甚至可以说,如数家珍。

~\\

如果你还是个孩子,每一年,你都会变成一个不同的人。
通常都是在秋天,当你丢掉暑假的困惑和懒散回到学校,升了一级的时候。
这是你记录变化最为明显的时光。
在此之后,你就不会记得哪年哪月在变化了,但是变化仍在继续,完全一样。
在很长的一段时间里,过去会从你身边溜走,走得如此轻松,完全是自动流失。
场景常常还未消失,已然不再相干。然后突然一个急转弯,某样东西遍地开花、处处涌现,想要得到关注,甚至还想要你做点什么。
虽然显而易见,实在没什么可做的。

~\\

马琳和沙琳。别人都以为我们绝对是双胞胎。
那几年流行给双胞胎起压韵的名字,邦妮和康妮,罗纳德和唐纳德。
另外,沙琳的帽子和我的很配。
这种叫苦力帽的帽子是宽大的低顶圆锥形草帽,用一个结或者松紧带系在下巴上。
这个世纪后期,这种帽子在电视上越战的镜头里就很常见了。
西贡街头骑自行车的男人戴着它们,走在路上的女人也戴着它们,她们的身后是被轰炸的村庄。

可能在那个时代——我是说我和沙琳露营的时候——大家说到亚洲苦力,不会觉得这种说法有什么不妥,说黑鬼,或者说像个讨价还价的犹太人,也不会觉得有问题。
我那时十几岁,还不会联想这些词汇的背景文化。

我们有这样的名字,戴这样的帽子,所以第一轮点名的时候,我们喜欢的辅导员——快活的梅维斯指着我们说:「嗨,双胞胎。」
我们还没来得及解释,她就已经在点别人的名字了。
我们喜欢梅维斯,但是,我们更喜欢漂亮的辅导员保利娜。

不过在点名之前,我们就注意到对方的帽子,互相有了好感。
否则的话,我们中至少一人,甚至可能两人都会把全新的帽子摘下来,准备塞到床底下,声称是妈妈逼我们戴的,我们自己不喜欢,诸如此类的话。

我挺欣赏沙琳,但我不知道怎么和她交朋友。
夏令营的女孩,除了少数几个稍微大一点以外,都是九到十岁的年龄。
在这个年龄,大家已经不再像六七岁的女孩那么容易交朋友、那么容易出双入对了。
我只是简单地跟在几个女孩子后面,她们和我是一个镇的人,但没有一个是有特殊之处的朋友。
我们进了一间还有空床的小木屋,我把自己的东西扔在棕色的毛毯上,听到身后一个声音在问我:「请问,能不能把我的双胞胎姐姐旁边的床让给我?」

是沙琳。不知道她在和谁说话。木屋大概能住下二十多个女孩子。
那个女孩回答,「当然行」,就搬走了。

沙琳用了一种奇特的腔调,逢迎、玩笑、自嘲,还有一种引人注意的兴奋,如同鸣钟的颤音。
很明显,和我相比,她不是一般地自信。她不是相信那个女孩一定会搬走,也不是强硬地说:「我先来的。」
(如果她是那种家里疏于教育的女孩—有些女孩就是这样的,她们来这里是由国际狮子会付的钱,或者是教堂支付的,反正不是她们的父母——她可能会这么说:「赶紧上厕所去吧,省得屎拉在裤子上。反正我是不会走的。」)
不是这样的自信。沙琳的自信是,她相信她想要别人做什么的时候,大家都希望这么做,而不仅仅是同意她的要求而已。
我本也有机会拒绝她,我可以说,「我不想当什么双胞胎」,然后不理她,收拾自己的东西。
但是,当然了,我没这样。
如她所料,我的感觉是受宠若惊。
我看着她兴高采烈地把包里的东西倒了出来,有些东西掉在了地板上。

我只找到一句话说:「你已经晒黑了。」
「晒黑太容易了。」她回答。

一开始是找我们的不同之处。
我们讨论后发现,她晒了变黑,而我晒了就会长雀斑。
我们的头发都是褐色的,不过她的颜色深一点。
她的头发是波浪形,而我的头发则茂密如树丛。
我比她高半英寸,而她的手腕和脚踝粗一些。
她的眼睛偏绿,而我的眼睛偏蓝。
我们一直热衷于观察彼此的不同之处,甚至把后背的痣和能看见的雀斑都列在了表格里,还有第二根脚趾的长度(我的第二根脚趾比大脚趾长,她的则短一些)。
我们详细地回忆了从小到大得过的病,种种意外事故,身体有没有什么部位修补或切除过。
我们两个的扁桃腺都摘除了——在那个年代,这是种普遍的预防手段——我们都得过麻疹、得过百日咳,都没得过流行性腮腺炎。
我拔过一颗上犬齿,因为它挡住了其他的牙。她的拇指被窗户夹过,所以半月痕不完整。

我们把身体的历史和特征都整理完了以后,就开始讲故事——家族的戏剧性事件、故事,以及区别所在。
她是家里最小的,也是唯一的女孩。而我是唯一的孩子。
我有个姑姑,高中的时候死于脊髓灰质炎。
她有个哥哥加入了海军。
那是战争年代,所以我们在营火晚会上会唱《永远的英格兰》、《橡树之心》、《统治吧,不列颠尼亚》,有时候会唱《永恒的枫叶》。
我们生活的背景,是空袭、战争、沉船,虽然距离我们很遥远,但是每天都在发生。
每隔一段时间,不远的地方也会有那么一回军事袭击,很吓人,又很庄严、很刺激。
要是我们镇或同一条街的某个男孩死了,他住的房子就算没有挂花圈,没有黑色的布帘,也会有一种特殊的沉重气氛。
使命完成,尘埃落定。
尽管房子里可能什么特别之处也没有,也许只是门口路边停了一辆陌生的车,表明有亲戚来,或者是牧师来这个失去亲人的家里坐一坐。

夏令营的一个辅导员在战争中失去了她的未婚夫。
她佩戴着他的表。我们认定是他的表,就别在她的外套上。
我们倒是愿意为她难过,关心她,不过她嗓门尖利,颐使气指,连名字都让人讨厌。阿尔瓦。

我们生活的另外一个背景是宗教。在夏令营,这个背景本来应该要强调的。
不过,因为是加拿大联合教会负责这里,所以不像浸信会或圣经基督教会那么喋喋不休,也不会像罗马天主教会甚或英国国教会,有那么多正式的认可仪式。
大部分女孩的父母都属于加拿大联合教会。
不过那些由教会付钱的女孩子,可能不属于任何教会。
联合教会用的是它最为亲切的民间方式,我们甚至都没有意识到,他们对我们的要求只有晚上祈祷,吃饭时唱圣歌,还有每天半小时的特殊谈话。
这种谈话就叫聊天,早餐以后的聊天。
不过,即便是聊天的时候,也很少提上帝或是耶稣,说得更多的是日常生活中的诚实品质、友好性情、纯洁思想,要我们承诺长大以后不抽烟、不喝酒。
没有谁对此提出过异议,也没有人中途退场,因为我们早就习惯了这种谈话,而且温暖的阳光下,坐在长椅上也蛮舒服的,一大早,我们还不想跳进水里,都嫌冷。

我和沙琳的这些事儿,即使是成年的女人也会做。也许她们不会互相数后背上的痣,不会去比较脚趾的长度,不过当她们相遇,感觉到彼此之间有一种不同寻常的惺惺相惜时,她们也会感觉到需要,需要了解重要的信息、重大的事件,不管那些事件是公开的还是私密的,以此填满两人之间的所有空白之处。
如果她们感觉有这样的温暖和渴望,那么她们几乎不可能厌倦。不管说的是什么琐事和傻事,她们都会一起笑,也会笑她们揭露的自私、欺骗、吝啬,以及纯粹的恶。

当然,一切都需要非凡的信任。不过,这种信任的建立,可以只在片刻之间。

我曾经观察过。围坐在营火边,搅拌树薯粥的时候,或者因为传说有野兽,剥夺男孩子们说话的机会,让他们去树林放哨的时候,这些漫长的时段就是开始的时间。
(我是个受过正规训练的人类学者,虽然比较差劲。)
我观察了,但从来没有参加过这种女性之间的秘密交换。
也并非全然如此。
有时候好像需要这么做,我也装作自己在参加。
但是对方永远能发现我的装腔作势,变得既茫然又警惕。

通常和男人交往,就不至于这么谨慎。
他们并不指望这样的交换,对这种交换绝少有真的兴趣。

我说的这种和女人的亲密关系,不是情欲的,也不是情欲的最初阶段。
情欲的关系,我在青春期前也经历过。
情欲的关系,也会有信任,也许还会有谎言,可能会走向游戏,总之不管有没有性玩笑,都会有一段热火朝天的短暂兴奋,紧随其后的则是不舒服、拒绝承认,以及厌恶。

沙琳告诉过我她哥哥的事儿,不过说的态度是一种真实的厌恶。
就是参加海军的那个哥哥。她去他的房间找她的猫,他正在对他的女朋友干这种事儿。
他们根本不知道她看见了。

她说他上上下下,啪啪啪啪。

你的意思是他们在床上互扇耳光?我问。

不是。她回答。是他的那东西进进出出的时候,啪啪啪啪。下流。恶心。

他光溜溜的白屁股上还有疙瘩。恶心。

我告诉她的是维尔娜。

~\\

回到我七岁的那年,我和爸爸妈妈住在一座当时我们叫作拼连住房的屋子里。
那时候,连栋式住宅这个词可能还没有。
总之,那座房子不是平均分割的,维尔娜的外婆租了后面的房间,我家租的是前面的房间。
房子很高,光秃秃的,很难看,刷的是黄色的漆。我们住的小镇太小了,人口加起来也没多少,不用划分居民区。
不过,实际上是有分区的,我觉得,我们的房子恰好位于体面地区和年久失修的地区的分界线上。
我说的是第二次世界大战以前,正好是经济萧条的最后时期。
当然了,我估计那时候没人知道经济萧条这个词。

我爸爸是个老师,他的工作稳定,但是钱很少。
我们另一头的街道渐渐消失,那条街的房子属于那些既没有稳定工作,也没有钱的人。
维尔娜的外婆显然有点钱,因为她说起领救济的人,用的是十分轻蔑的语气。
我妈妈肯定和她争论过,说那不是他们的错。不过没用。
这两个女人算不上亲密的朋友,不过,有关怎么安排使用晾衣绳,她们的态度真诚而友好。

这位外婆的名字是霍姆太太。有一个男人时不时来看望她。我妈妈说他是霍姆太太的朋友。

不过,实际上,每次他来的时候,妈妈都不让我出门玩,所以我没什么机会和他讲话。
他长什么样子我都不清楚,不过我认识他的车,车是深蓝色的福特V-8。
我对车特别有兴趣,可能是因为我家没有车。

然后,维尔娜就来了。

霍姆太太说维尔娜是她的外孙女,我们没有理由怀疑。
但是,从来没有任何迹象表明,她们中间隔着的那代人存在。
我不知道是霍姆太太去把她接回来的,还是她的朋友用V-8把维尔娜送来的,总之,那是夏天,还没有开学的时候,她出现了。
我不记得她告诉过我她的名字,一般情况下,她不是个健谈的人,我肯定也没有问过她。从最早的时候起,我就对她有一种强烈的反感。
在那时,我对其他人无论如何也没有过这样的感觉。我告诉妈妈我恨她。
妈妈问你为什么要恨她,她对你做什么了?

可怜的东西。

孩子用恨这个词来表达各种不同的感受,意思也许是他们吓坏了。
吓坏了的意思不是他们担心挨打。
拿我自己的感觉来举例吧,当你走在人行道上,一些大男孩喜欢骑自行车拦住你,冲你发出恐怖的怒吼。
你害怕的不是对身体的伤害。
好像我对维尔娜的恐惧,差不多是对诅咒或者阴暗企图的恐惧。这种感觉,小孩子都会有,也许是因为一座房子的样子,也许是因为一根树干,更多时候或许是因为霉烂的地下室、幽深的衣橱。

她比我高很多,不知道她比我大多少。也许是两岁,三岁?她瘦得皮包骨头,骨架子那么小,脑袋也那么小,让我想起蛇头来。
细密的黑发平滑地盖在这颗脑袋上,遮住了前额。
我觉得她脸上的皮肤很暗沉,很像我家旧帆布帐篷的盖布。
她颧骨突出来的样子,就像盖布被风吹得鼓起来的样子。她的眼睛永远都在斜视。

不过,我相信,大家看见她的时候,不会觉得她的模样有什么地方特别招人讨厌。
实际上,我妈妈说她楚楚动人,或者说几乎算得上楚楚动人。(比方这么说:「真可惜啊,她本来可以长得楚楚动人的。」)
但仅仅就我妈妈从她的一举一动中看到的而言,也没法反对这种说法。
她比她的实际年龄小得多。
这是一种拐弯抹角的说法,指的是维尔娜还没学过读书写字,也不会滑冰、打球,另外,她嗓音粗哑,而且不会压低自己的嗓门,她的措辞奇怪地断断续续,好像这些词儿结成了块儿,卡在她的喉咙里。

她干扰我、毁掉我自己玩的游戏的办法,不是一个小女孩,而是个大女孩的办法。
一个年龄不小,但是没有经验、没有权利,什么都没有的女孩,只有宁死不屈的决心,以及丝毫不明白自己并不受欢迎的无能。

孩子理所当然是一群保守得可怕的人,他们当机立断地反对一切边缘的、反常的、难以驾驭的东西。
我是家里唯一的孩子,备受溺爱,当然,也备受训斥。
我笨拙、早熟、羞怯,有自己私密的规矩和憎恶。
我甚至讨厌维尔娜头发上不停掉下来的赛璐珞发夹,还讨厌她老是想塞给我的红绿条纹的薄荷糖。
她会试图追赶我,硬把这些糖塞到我嘴里,以她独特的断断续续的发音方式,吃吃吃吃傻笑个不停。
直到现在,我还是不喜欢薄荷的味道。
还有,我不喜欢这个名字,维尔娜。这个希腊名字是春天的美女的意思。
我觉得它听起来和春天没什么关系,也不像青草地,不像花环,不像穿薄纱裙的姑娘,更像一抹顽固的薄荷污渍、绿色的黏液罢了。

我也不相信妈妈是真的喜欢维尔娜,因为她天性中有些许伪善,还因为她决心不让我快活,就装作同情维尔娜。
她要求我友善。开始,她说维尔娜不会待太久的,暑假一结束,她从哪里来的,就回哪里去。
后来,维尔娜根本没打算回哪里去,她又告诉我一个令人欣慰的消息,说我们自己很快就搬家,友善的日子不长了。
实际上,这时候距离我们真正搬家还有一年。最终,她耗尽了耐心,说我让她失望,她从来没有想到我的天性原来如此刻薄。

「她生来就是这样,你怎么能攻击她长的样子?这是她的错吗?」

这种话对我没有意义。
要是我有足够的辩论技巧的话,我会说我根本没有攻击维尔娜,我只是希望她离我远一点。
不过,我本来就是在攻击她,用不着质疑这到底是不是她的错。
不管我妈怎么说,我的这种态度,其实多多少少和我住的地方、我生活的年代里,周围没有说出口的看法是和谐的。
即便是大人们,他们的笑容里都藏有一种无法克制的满足感,以及理所当然的优越感,每当他们说某人「太简单」、「少根筋」的时候,我都能看见这种表情。
我相信我妈妈就是这种人。私下觉得。

我开学了。维尔娜也上学了。
她进了一个特别班,在学校操场边上一幢特别的楼里。这幢楼是镇上学校最早的教学楼。
不过在那个年代,没人有时间研究城市志,没几年这楼就被拆掉了。
楼里面有一块用墙隔出来的空间,让学生休息时间聚在一起玩。
那幢楼的学生,早晨上学比我们晚半个小时,下午放学比我们早半个小时,所以课间休息时,没人会去骚扰他们。
不过,因为他们都攀在墙上,看我们普通学校的操场到底有什么,挤得太厉害的时候就会出事儿,会用尖叫、挥舞棍棒之类的吓唬他们。
我从来没走近那地方,几乎没见过维尔娜。我是在家里被迫应付她。

刚开始的时候,她站在黄房子的角落里,看着我,我装作不知道她在那儿。
后来,她走进了前院,在我家的台阶上占个位置。
她在那儿,我要是想去厕所,或者我冷了,就非得经过她身边不可,而且还很近,很可能会碰到她,也许她会来碰我。

我从来没见过哪个人能在一个地方待那么长时间,而且眼睛只盯着一样东西。常常盯的是我。

我有一个秋千,秋千挂在枫树上,荡秋千的时候,我的脸要么冲着房子,要么冲着大街。
也就是说,要么我和她面面相觑,要么她的目光让我若芒刺在背,她甚至有可能过来推我一把。
她老是推我一个趔趄。
不过,这还不是最倒霉的。
最倒霉的是,她的手指戳在我背上,真像一支支冰凉的针管,直接穿透了我的外套和里面的衣服。
我还有一个游戏是堆叶子、盖房子。
就是我用耙子耙,或者用手抱,总之设法把枫树的落叶堆到一起,然后把树叶拼成房子的图样。
这里是起居室,那里是厨房,那一堆松松的是卧室里的床,等等等等。
这个游戏不是我发明的。
学校的看门人把落叶全部耙走烧掉以前,课间休息的时候,女孩子们都在操场上盖房子,她们堆出来的房子更豪华,有时候甚至还会有点装饰。

开始时,维尔娜只是看我在干什么,她永远斜着眼睛的表情,在我看来,是一种莫名其妙的优越感。
她凭什么觉得自己高人一等?
后来就到了她行动的时候了,她走过来,抱起一团树叶,大概是因为她自己也在犹豫,或者就是手笨,叶子漏得满地都是。
她抱起来的叶子,不是搁在一边备用的那堆,而是从我房子的墙上抱走的。
她把墙抱在怀里走了几步,扔进我整洁有序的房间里。

我冲她大喊大叫,叫她住手。
她弯下腰,想把她扔下来的树叶重新抱起来,但没办法聚拢,干脆又抛撒下来,等叶子全落到地上,她就开始傻乎乎地乱踢。
我冲她吼,一点用也没有,或许她把我的吼叫当成了鼓励。于是我只好低头朝她冲过去,正好顶在她肚子上。
我没有戴帽子,所以我的头发就碰在了她的羊毛衫或者是外套上。
我的感觉是,我的脑袋撞在了一个坚硬而又臃肿的肚子的刚毛上。
我喊叫着跑回家。我妈妈听完之后说的话更让我发疯:「她只是想玩,但不知道怎么玩。」

第二年秋天,我们搬进了另一座平房里,我再也不用经过那座黄房子了。
黄房子总是让我想起维尔娜,似乎它也学会了她斤斤计较的算计,恐吓似的斜视。
黄颜色似乎正是侮辱的色彩,而那扇并没有坐落在中间的前门,添了一分残疾的感觉。

我们住的平房和这座黄房子只隔三个街区,就在学校边上。我既然逃离了维尔娜,就忘记了小镇的大小和生活的复杂性。
有一天,我和学校一个同学在街上迎面碰到了她,我才明白这并不是真的,不全是真的。
这回应该是我们谁的妈妈派我们跑腿做什么事儿。我没有抬头看,不过我相信,擦肩而过的时候,我听到了吃吃的笑声,也许是高兴,也许是因为认出了我。

同学的话顿时让我觉得可怕。

她说:「我以前觉得这是你姐姐。」

「什么!」

「我知道你们住在一起,所以觉得你们肯定是亲戚。至少是堂姐妹。你们不是?不是堂姐妹?」

「不是。」

~\\

特殊班上课的老楼没通过安全检查,不能再用了,镇里就租用了圣经礼拜堂,学生们去那里上课。
圣经礼拜堂恰好就在我家的街角,过了马路就是。
维尔娜上学有两条路可以走,她选择了那条经过我家的路。
而且,我家的房子离人行道只有几英尺远,所以,实际上,她经过的时候,影子都会落在我家的台阶上。
要是她愿意,她可以把鹅卵石踢到我家的草坪上。
除非我家的百叶窗拉上了,否则她就可以偷窥我家的客厅和门厅。

特殊班的上课时间这时候和普通学校一样了。
至少早晨一样的,下午放学还是会早一点。
学校也许觉得,他们在礼拜堂上课,就不会和普通学生挤在一条路上了;
所以,这时候,我就可能在上学路上遇见维尔娜。
我随时随地都看着她可能出现的方向,只要一看见她,就躲回家里去,借口说忘带什么东西,或者说鞋子磨脚、我得找一块橡皮膏,再或者说我头上的发带松了之类的。
我不至于傻到说看见维尔娜了,我妈妈会说:「那又怎么了?你怕什么?她会吃了你?」

又怎么样呢?有污染?会感染?维尔娜看起来挺干净,挺健康。
而且,基本上她也不可能上来就骂我、揍我、拽我的头发。
不过,只有大人才会笨到以为她没什么力量。
力量,再说了,这种力量也是特别针对我的。
她的眼睛盯住的是我。至少我这么觉得。
仿佛我们之间有一种默契,这种感觉无法形容,也没法解决。
这种感觉固执得如同爱情一般,尽管在我看来,绝对更像仇恨。

我想,我讨厌她,正如有人讨厌蛇,有人讨厌毛毛虫,有人讨厌老鼠,有人讨厌鼻涕虫。
没有什么拿得出手的理由。她确实不会对我有什么实际的伤害,不过,她能扰乱你的五脏六腑,让你痛恨自己的生活。

~\\


告诉沙琳维尔娜的故事时,我们的聊天已经很深入了。
除非游泳或者睡着的时候,我们才停得下来。
维尔娜成了我的祭品,虽然并不是这么拿得出手,相比沙琳哥哥抽动的、疙疙瘩瘩的屁股,没有那么生动活泼、让人恶心。
我记得我和沙琳说,维尔娜的可怕之处,我没法描述清楚。
不过紧接着,我就开始描述,我感觉,我描述得应该还算不错,因为为期两周的夏令营就要结束的一个中午,沙琳冲进了食堂,脸上闪烁的是恐惧,以及一种奇怪的兴奋。

「她就在这里。她在这里。就是那个女孩。那个可怕的女孩。维尔娜。她就在这儿。」

午饭吃完了,按规矩我们要收拾东西,把我们的盘子、碟子搁在厨房的架子上,然后当天值日的姑娘会拿走去洗。
然后,我们会去糖果店排队,糖果店每天中午一点钟开门。
沙琳刚刚就是回宿舍拿钱去了。她是个富人,因为有个企业家爸爸,所以她就马马虎虎,把钱随随便便放在枕头里。
我除非是游泳的时候,钱都是贴身放的。
所有付得起钱的女孩子午饭后都要去糖果店买糖,把讨厌的甜点味道清理掉。
我们明明知道甜点难吃,但还都非要尝尝,想确认味道有没有我们想得那么恶心。
木薯布丁,烂糊糊的烤苹果,黏糊糊的奶油冻之类。
所以,我第一眼看见沙琳这种表情的时候,我以为是她的钱被偷了。
不过转念一想,这种倒霉事儿不至于让她的脸扭曲成这样,她眼里的震惊分明是种喜悦。

维尔娜?维尔娜怎么会在这里。认错人了吧。

这天应该是礼拜五。夏令营还有两天时间。
我们还有两天就要走了。
最后我们才明白,夏令营也有个特殊营。
在这里他们也还是特殊的。
他们过来和我们一起度过最后的周末。
人不算太多,估计总共也就二十个,也不全是从我们小镇来的,还有附近其他小镇的。
沙琳刚想清楚地告诉我怎么回事儿的时候,一声哨响,辅导员阿尔瓦跳到了椅子上,朝我们讲话了。

她说,她知道我们肯定会尽责尽力地欢迎观光客,也就是夏令营的新营员们。
新营员带来了自己的帐篷和自己的辅导员,不过,她们会和我们一起吃饭、游泳、游戏,也要参加早晨的聊天。
她说,她肯定我们会把这场相遇当成交新朋友的机会,语气里有我们已经熟悉的警告和叱责。

这些新来的人花了一会儿时间支帐篷,安排床位。
有些人明显毫无兴趣,跑开了,辅导员只好连吼带叫,把这些家伙抓回来。
这段时间我们正好自由活动,是休息时间,我们到糖果店买了巧克力、甘草糖、太妃软糖,躺在铺位上吃。

沙琳一直说:「想想,想想。她在这里啊。我简直没法相信。你觉得她是不是跟踪你?」

「可能吧。」我回答。

「你觉得我能不能把你藏起来呢?」

我们在糖果店排队的时候,我一直低着头,让沙琳挡在我前面,不让附近的特殊营员看见我。
我飞快地偷看了一眼,我看见了维尔娜的后背。
她垂下来的、蛇一般的头颅。

「我们应该想办法让你乔装打扮。」

根据我说出来的故事,沙琳大概以为维尔娜主动骚扰我。
我想这也是真的吧,除了她的骚扰没有这么显眼,比我能说出来的更加隐秘以外。
现在,我就让沙琳这样想吧,反正这么想会让她更兴奋。

因为我和沙琳一直处心积虑、躲躲藏藏,所以维尔娜并没有立刻发现我。
也有可能是因为她和其他特殊营的新营员一样昏头昏脑,不知所措,努力想搞明白自己到底来这里干嘛。
没一会儿,她们就被带到海滩的另一头去上游泳课了。

晚饭时,我们在餐桌前唱歌的时候,她们排着队进来了。

\begin{verse}
    在一起,在一起,
    
    我们在一起的时间越长,

    在一起的时间越长,

    我们的心情越飞扬。
\end{verse}

再后来,她们平静地各自走开,散落在我们中间。
她们全都戴着写了名字的牌子。
坐在我对面的那个叫玛丽·埃伦,类似这个名字吧,她不是从我们小镇来的。
不过,我还没来得及高兴,就看见维尔娜在隔壁桌,她比同桌的女孩都高。
不过感谢上帝,她和我坐的是同一个方向,所以吃饭的时候不会看见我。

她是那一桌最高的,不过也并没有我记忆里那么高,没有高得扎眼。
大概是因为过去的一年,我的个子长得飞快,她也许已经不长了。

吃完饭,我们站起来收拾盘子。
我一直低着头,一眼也没朝她的方向看。
不过,我还是知道什么时候她的目光落在了我身上,什么时候她认出了我,什么时候她笑了,她嘴角下垂露出微笑,她的喉咙发出那种古怪的吃吃声。

「她看到你了。」沙琳说,「你别看。别往那儿看。我站在你们中间,走吧,往前走就是了。」

「她也朝这里来了?」

「没有。她站在原地,一直盯着你看。」

「笑?」

「有点。」

「我不能看她,太恶心了。」

剩下的一天半,她会怎么迫害我?
沙琳和我一直在用这个词,实际上,维尔娜甚至根本没有靠近我们。
迫害。听起来像是大人的法律腔。
我们始终处于守望的状态,仿佛我们被跟踪了,或者只是我被跟踪了。
我们试图把握维尔娜的动向,沙琳和我汇报她的态度、她的表情,等等等等。
有两次,我自己也冒险去打量她,不过得沙琳先告诉我:「好啦,现在她不会注意你的。」

这两回,我看见的维尔娜都略微显得沮丧,或者说愠怒、困惑?
就像大部分特殊营的孩子一样,她很茫然,不清楚自己在哪里,她在这里干什么。
有几个孩子——当然,其中没有她——跑到海滩另一头的悬崖上,钻进了全是松树和杉树的树林里,引发了一阵骚乱。
还有几个沿着通往公路的沙石小径就打算出发了。
所以辅导员召集我们开了个会,要我们看好我们的新朋友,因为我们比我们的新朋友熟悉这个地方。
沙琳又捅了捅我的肋骨,当然,并不是因为她觉察到维尔娜有什么变化,信心没了、身材变矮了之类的消息,她只是持续汇报,汇报维尔娜狡猾的表情、邪恶的神态、讨嫌的外表。
也许沙琳说得对,维尔娜看见了沙琳,看见了我的新朋友、新保镖,一个陌生人的出现是某种信号,告诉她一切都已经改变,她不再可以确定了。
这种意识坏了她的心情,尽管我没看见她心情不好。

「你没告诉我她的手。」沙琳说。

「她的手怎么了?」

「我从来没见过这么长的手指。她要是双手掐住你的脖子,肯定能掐死你。肯定行。你说,晚上和她住在一个帐篷里,有多恐怖呀。」

我说确实挺恐怖的。

「不过,和她一个帐篷的全都是白痴,不会注意的。」

~\\

最后一个礼拜天,发生了一种变化,夏令营的感觉完全不同了。
并没有激烈的事件,餐厅的钟也在通常的开饭时间敲响,饭菜不比平时更好,也没有更糟。
休息时间到了。然后是游戏时间,游泳时间。
糖果店照常营业,我们也像往常一样,聚在一起聊天。
不过,你还是可以感觉到一种心神不宁、心不在焉的气息,连辅导员都是这样的。
她们没有像平时那样,一堆斥责或者鼓励的话就堆在舌尖准备着,而是大约会花上一秒钟时间盯着你看,似乎在努力回忆自己平时会怎么说。
其实,这种变化自打特殊营到来的时候就已经开始了。
她们的出现改变了营地的气氛。
以前这里是真正的夏令营,有自己的规则,还制定了奖罚制度,和学校或孩子们生活的任何环境一样。
而他们来了,每个角落都开始崩塌,充分暴露了它只不过是临时的。表演而已。

是否因为我们看见特殊营员的时候,就想她们是不是真正的营员,然后发现其实根本没有真正的营员?
部分如此吧。
不过,还有一部分是因为,夏令营就快结束了,夏令营的作息制度就要瓦解了,爸爸妈妈快来接我们回到日常生活里去了,辅导员也要回去做她们的普通人,也许她们甚至不是老师。
我们正处于一个即将被拆散的阶段,这期间的友谊、敌人、竞争,在过去的两个礼拜里活跃的一切,都要被拆散了。
谁能相信这所有的一切,仅仅是两个星期?

没人知道该怎么说,但是没精打采的气氛在我们中间蔓延,这是一种厌倦的坏脾气,就连天气表现的也是同样的感受。
过去的两星期应该不是每天都阳光灿烂、无比炎热,不过我们的印象大半是如此。
而现在,礼拜天的早晨,不一样了。
礼拜天的早晨我们不是聊天,是祈祷。
我们在外头祈祷的时候,云彩暗了。
温度没什么变化。要是非说有变化的话,应该是这一天的热量更多。
不过这一会儿,像是暴雨就要来了,而且,寂静无声。
辅导员们,还有礼拜天特意从附近小镇开车过来的牧师,都不时地抬头看,担心要变天。

确实掉了几滴雨,然后就没再下了。
雨水就此结束,没有下暴雨。
云朵稍微亮了一些,并不足以确保之后会阳光明媚,但足以保证我们的游泳不会取消。
之后就不供应午饭了,早餐结束厨房就关门。
糖果店的百叶窗也不会再拉开了。
午饭时间,我们的爸爸妈妈就会陆续赶来接我们回家。
会有一辆大汽车来接特殊营员。
我们多半已经收拾好了东西,床单抽掉了,粗糙的棕色毯子叠好了,搁在床脚。
这种毯子睡觉的时候总是潮乎乎的。

虽然我们都挤在木屋里,叽叽喳喳地说话、换游泳衣,可是,木屋内部还是展示了它的短暂,以及忧伤。

海滩上也是一样,沙滩上的沙子似乎比平时少,石头则比平时多。
沙子所在的那块地方颜色是灰的。
水看起来很冷,不过实际上海水挺暖和。
然而,我们游泳的热情已经消退,大部分女孩只是在水里漫无目的地跋涉而已。
游泳辅导员是保利娜和一个负责特殊营的中年妇女,她们两人只好冲我们拍手。

「快点。你们等什么呢?今天是夏天的最后一次机会哦。」

有几个女孩是游泳的一把好手,她们经常一下水马上就朝木排游过去。
还有一部分人游得不错,我和沙琳就是这类,我们会往木排的方向游,然后再转身游回来,证明我们至少能闷在水里游上两码的距离。
保利娜一般都是立刻游到木排那儿,待在深水里看着大家,以防有人出事儿,还要保证所有营员都确实在游泳。
不管怎么样,这个礼拜五,游泳的人还是比平时少,很多本应该游的人没有游。
保利娜也许是为了打气,也许是因为气恼,叫大家都下水去,不过她也只喊了这么一句,然后自己就在木排周围拍水,和几个铁打不动的游泳专家一起说笑去了。
大部分女孩子还是在浅水区里玩水,只游了几英尺或几码就坐在水里互相泼水,有些转了个弯就去找空瓶子做漂流物了,似乎谁都对游泳不感兴趣。
特殊营的女辅导员站在水刚刚到她的腰的地方。
不过,大部分特殊营员都在水没有淹没她们膝盖的地方。
这个辅导员裙装式游泳上衣上的花甚至没湿。
她弯下腰,用手溅起微弱的水花,一边笑一边冲她的学生们说,好玩吧。

我和沙琳待的地方,水至多到胸口,我们没有走到更远的地方。
我们和游得很差的女孩子们一起做做漂流瓶,懒洋洋地游游仰泳,或者蛙泳。
没有人训斥我们吊儿郎当。
我们试了试在水底睁着眼睛能游多久。
我们偷偷摸摸地游到对方身上,猛地趴到对方的背上。
我们身边也有一群这么玩的人在大笑大叫。

我们游泳的这段时间里,一些父母或者被派来接孩子的人已经到了,他们说自己没有时间可浪费,所以他们要接的女孩子在水里就被点名叫走,这又导致了一些额外的紧张和混乱。

「看,你看。」沙琳说。
她的声音咕咕噜噜的,因为我刚刚把她按到水底下,她这才抬起头来,湿淋淋的,还在吐水。

我看到了。
维尔娜正在朝我们走过来,她戴了淡蓝色的橡胶游泳帽,修长的手正在拍打水面,面带微笑,表情像是,突然之间,她收复了对我的权利。

~\\

我没跟上沙琳,我甚至都不记得我们是怎么道别的了,我们到底有没有道别都是个问题。
我的印象是,我们的爸爸妈妈差不多是一起到的,我们匆匆忙忙钻进各自的车里,回到了以往的生活之中——我们又能怎么办呢?沙琳爸爸妈妈的车当然不会像我爸爸妈妈的车这么破旧、嘈杂,性能还不可靠。
不过,即使并非如此,我们也绝对不会想到要让这两对父母互相熟悉一下。
大家,每个人,包括我们自己,都急于出发,远离喧哗——有人是因为丢了东西喧哗,有人是因为看见了孩子喧哗,有人是因为没看见孩子喧哗,有人是因为错过了大巴喧哗。
总之,一片喧嚣。

多年以后,一个意外的机会让我看到了沙琳婚礼的照片。
那个年代,婚礼的照片会在报纸上刊登,不光是小镇,城市的报纸也会登。
看见照片的时候,我正在多伦多布罗尔大街的一家咖啡馆里,一边翻看报纸,一边等一位朋友。

婚礼是在圭尔夫举行的。
新郎是多伦多人,毕业于奥斯古德法学院。
他个子很高,也可能是长大后的沙琳特别矮。
即使她的头发厚厚地盘在了头顶——就是那个年代流行的精致的钢盔头——也只刚有他的肩膀高。
这种发型让她的脸看起来平淡无奇,像是被压扁了似的。
不过我记得是克娄巴特拉妆,眼睛描得很浓,嘴唇淡淡的。
听起来很怪,不过那个年代,这种样子必然会被盛赞。
而这一切让我想起来的,只是她还是个孩子时下巴上长的滑稽的小肿块。

她,报纸上说的新娘,毕业于多伦多的圣希尔达学院。

那么,到圣希尔达学院上学,她肯定来过多伦多。
她在的时候,我也在同一个城市,我在大学学院上学。
某一个时刻,我们可能走在同一条街上,或者校园的小路上,却从来没能相遇。
我不觉得要是她看见我,会故意不和我说话。
我也不会躲开她。当然了,一旦我得知她在圣希尔达学院上学,一定会觉得自己才是个真正的学生。
我和朋友们都觉得圣希尔达是淑女学院。

现在,我是人类学硕士,我已经决心永远不要结婚,尽管不排除会有情人。
我披了一头长长的直发,我的朋友和我都喜欢嬉皮士的风格。
相比现在,那个时候童年记忆更为遥远,早已褪去了色彩,一点也不重要。

我本可以给沙琳写信,请她的父母转交,报纸上就有她父母在圭尔夫的地址。
不过我没写。我觉得祝贺女人结婚这种事儿,伪善到了极点。

~\\

不过,也许是十五年后,她给我写了信,请我的出版人转交。

「我的老朋友马琳,」她这样写道,「在《麦克林》杂志上看见你的名字,别提有多高兴了。
一想到你写了本书,我真是惊叹不已啊。我还没去买书,因为我们刚度假回来,不过我肯定会去的,而且一定会尽早看的。
我刚刚翻了翻外出期间积攒下来的期刊,在上面我看到你的照片了,很有个性,评论也很有趣。
我觉得必须要给你写封信祝贺。

「你是不是已经结婚了,但还在用婚前姓写作?你成家了吗?给我写信,告诉我你的生活吧。
我伤心的是没有孩子,不过我终日忙于志愿者工作、园艺,或者和我的丈夫基特一起出海。
永远都有那么多事情要做。我现在在图书馆工作,要是他们不订你的书,我就拧断他们的胳膊。

「我要再次祝贺你。我要说,我有点惊讶,但也不完全惊讶,因为我一直觉得,你可能会做一些特别的事情。」

这一次的来信,我也没有回。
似乎根本没有意义。一开始,我没有留心信结尾的「特别」这个词,但后来想起来的时候,它仿佛是给我迎头痛击。
然而我对自己说,她用这个词没有特别的意思,我如今依然相信这一点。

她提到的书,是我从一篇文章慢慢写成的一本书。
我一度失去了继续写这篇文章的兴趣,开始写另一篇,后来有空了,就当成业余消遣再回头继续写。
自那以后,我和别人合著了两本书,适时地满足对我的期望,不过,我自己写的那本,是唯一给我带来一些外界关注的,更不必说来自同僚的批评了。
现在,书已经不再印刷了。书名叫《偶像和白痴》,要是现在,这种名字绝对逃脱不了惩罚,甚至连我的出版人都很紧张,尽管他们也承认这个书名朗朗上口,容易记住。

我研究的是不同文化背景的人们的态度——我指的是那些不可能称之为原始的文化背景——研究这些人对精神或者身体异常的人的态度。
「有缺陷」、「残废」、「迟钝」这些词汇当然都进了垃圾箱,也许有好的理由——不仅是因为这些词表现出优越感,以及习惯性的不善态度,也因为这些词并非准确的描述。
这些词汇忽略了这类人群身上丰富、精彩,甚至卓越的(至少算是特有的)力量。
最有意思的是,除了迫害之外,其中还能发现一定程度的崇拜,有一系列的能力被归咎(这个词并非完全不准确)为神圣的、神奇的、危险的,或者说,有价值的。
我能做到的全都做了,我做了历史研究和当代研究,也涉及了诗歌、小说,以及宗教习俗的影响。
自然,因为我的专业背景,大家批评我过度文学化,批评我引用的信息全部来自现成的书籍,但那时候我不可能跑遍全世界。
我没有得到任何资助。

当然,我能看清楚其中的关系。
这种关系,我想可能沙琳也看出来了。
很奇怪,那些事儿感觉是那么地遥远,那么不值一提。
它只是一个起点。
那时候,对我而言,它和童年时代其他的事情差不多。
因为从那以后的旅行、长大后的成就。安全感。

「婚前姓。」沙琳这么写。
这种说法,我很久都没有听说过了。
这种说法和「老处女」基本相当,听起来贞洁又愁苦,明显不适合我。
早在我看见沙琳的婚礼照片前,我就不是处女了。
当然,我也没以为她是。
并不是因为我有一群情人,其中大部分甚至也谈不上是情人。
和许多与我同龄的、没有过一雌一雄单配婚姻生活的女性一样,我知道数字。
十六个。我敢肯定,很多比我年纪轻的女人,刚二十出头就已经达到了这个数字,甚至可能十岁出头。
当然,收到沙琳的信时,总数没这么多。我不能(不能是真的),也不愿意费心去算了。
其中有三个重要的人,按年代顺序排序,这三个人都排在前六位。
我说这三个人「重要」的意思——哦不,只有两个,这第三个人对我的重要性,远远大于我对他的重要性。
我说的重要是指,和这两个人在一起会有这样的时候:你想把自己撕开,远远不只是身体的屈服,想把你全部的生活安全地和他的生活搁进同一个篮子里。

我不让自己这么干,不过只是勉强能做到。

可能是这样的安全感不能完全说服我。

~\\

不久之前,我收到了另一封信。
这封信是学校转给我的。退休以前,我一直在大学教书。
我从巴塔哥尼亚旅游回来,发现这封信在等着我。
那时候,我变成了一个吃苦耐劳的旅游爱好者。
信已经来了有一个月之久。

信是打印的——对此,写信的人立刻表示了歉意。

「我的字写得很难看。」他写道,接下来,他介绍自己是「你的童年伙伴沙琳」的丈夫。
他说他非常抱歉,非常非常地抱歉,给我带来了一个坏消息。
沙琳住在多伦多的玛格丽特公主医院,癌症已经从她的肺部扩散到了肝脏。
可叹她终生都在吸烟。她活的日子不长了。
她并没有经常和他提起我,不过这些年以来,只要她提起来,都是为我杰出的成就而高兴。
他知道她的内心是多么地珍视我。现在,当她的人生走到尽头之时,她热切地想见我。
她拜托他来找我。他说,也许童年的记忆是最珍贵的。童年的感情。无可比拟的深厚友情。

哦,她现在可能已经死了。我觉得。

不过,要是她已经——这就是我想问题的方式——要是她已经走了,我到医院打听打听,也没什么风险。
接着,我的意识,或者不叫意识,随便叫它什么好了,变得清晰了。
我可以给他写一张便条,说可惜我出门去了,不过我会尽快赶去。

哦别,最好别写便条。
他也许借此便出现在我生活里,为了向我表示感谢。
「伙伴」这个词,让我很不舒服。「杰出成就」也是,另外一种不舒服罢了。

~\\

玛格丽特公主医院和我的公寓只隔几条街。
某个春光明媚的日子,我步行走了过去。
我不知道为什么我不先打个电话问问。
也许我想让自己觉得我已经尽了最大的努力。

在总台,我发现,沙琳还活着。
对方问我想不想见她,我根本没法说不想。

我在电梯间时依然在想,我现在还来得及转身走,现在还没到她楼层的护士站。
也许我可以走出电梯,搭下楼的电梯。
总台的接待员不会注意我的。
事实上,我一转身,她就在接待下一个人了,那时候就已经没有注意我了。
再说,就算她看见我走了,又有什么关系?

我估计,我会感到羞耻。
多半不是因为自己缺乏感情而羞耻,而是因为自己缺乏意志而羞耻。

我在护士站停下了脚步,护士给了我她的房间号。

那是一间私人病房。
非常小的房间,并没有明显的医疗设备、鲜花或者气球。
事实上,起先我连沙琳的人都没看到。
一个护士面对床,弯着腰,床上似乎是一堆被子,没有人在。
这堆东西仿佛放大了的肝脏,我觉得。
我希望自己赶紧跑掉。

护士站直了身体,转过身朝我微笑。
她身材丰满,是棕色人种,声音轻柔,有一种哄孩子的语气。
也许她来自西印度群岛。

「你是马琳。」她说。

这句话似乎让她颇为愉快。

「她一直盼着你来。你走近一点吧。」

我走近了。
我看见一个肿胀的身躯,一张严重毁坏的面孔,小鸡似的脖子。
对这个身躯来说,医院的睡袍实在是太肥了。
鬈曲的头发还是褐色的,从头皮上也就长出四分之一英寸的长度。哪里也不像沙琳。

我以前也见过临终的面容——我自己的父母,还有我害怕自己会爱上的男人。我不会为此感到震惊的。

「这会儿她睡了。」护士说,「她很希望你来。」

「她现在意识还算清醒?」

「清醒的。她只是睡着了。」

嗯。现在,我看出来了,是沙琳的模样。
哪里?也许是一丝抽动的表情,自信顽皮地将一边的嘴角隐藏起来的样子。

护士用她快活而又轻柔的腔调对我说:「我不知道她还能不能认出你来。不过她希望你来看她,还给你准备了东西。」

「她会醒来吗?」

耸耸肩。「我们经常给她注射止痛剂。」

她拉开了床头柜。

「东西在这儿。她说要是你来晚了,就由我把东西给你。她不想让她丈夫给你。你现在就来了,她会很高兴的。」

一只封好的信封,上面写了我的名字。
字母是大写的,字迹摇摇晃晃的。

「不是她丈夫。」护士眨了眨眼睛,咧嘴笑了。
难道她觉得有什么古怪的?一个女人的秘密,一个旧爱?

「你明天再来吧。」她说,「谁知道呢?要是她醒了,我告诉她。」

一到楼下的大厅,我就打开了信封。
沙琳的字写得很规矩,不像信封上的字那样张牙舞爪、潦草狂野。
当然,她有可能先写了便条,放进信封里,然后粘好放起来,以为有一天可以亲自交给我。
后来,她才觉得有必要在信封上写上我的名字。

马琳。我之所以写这张条子,是怕有一天我没法亲口和你说。
请你答应我的请求。
请去圭尔夫,去大教堂找霍夫斯德神父。永援圣母教堂。
教堂很大,用不着名字也能找到。
霍夫斯德神父。
他知道该怎么办。
这件事儿,我不能让C帮我,也永远不想让他知道。
霍神父知道,我告诉过他,他答应会帮助我。
马琳,求你帮帮我,谢谢你。和你没有关系。

C,指的必然是她丈夫。他不知道,当然了,他肯定不知道。

霍夫斯德神父。

和我没有关系。

我本可以一走到街上,就把纸团起来,扔掉。
我确实这么做了。
我把信封扔掉了,风把它吹进了大学林荫道边的排水沟。
然后,我才发现,便条并没有在信封里,还在口袋里。

我再也不会去医院了。我也不会去圭尔夫的。

她丈夫名叫基特。现在我想起来了。他们一起出海。克里斯托弗。
基特。克里斯托弗。C。

回到公寓楼,我发现自己搭电梯下了楼,到了车库,而没有上楼回家。
我上了车,套上外衣,出门上了大街,朝加德纳高速公路开过去。

加德纳高速公路,427号公路,401号公路。这会儿,正是交通高峰时间,不是出城的好时机。
我讨厌这种时候开车,我很少在这种时间出门,没有信心在这种情况下开车。
还剩下半箱油,另外,我还得上个厕所。
也许在米尔顿,我想。我可以停在公路边,加满油,上个厕所,再考虑考虑。
这会儿,我除了继续开车,没别的办法。朝北,然后再朝西。

我没有下车。先经过了米西索加出口,接着是米尔顿出口。
我看见公路的指示牌告诉我还有多少公里到达圭尔夫。
我像往常一样,脑子里大致换算了一下有多少英里,估计汽油够用。
我不给自己停车的理由是,太阳就要落下来了,越来越麻烦了。
现如今,即使在最好的天气,城市的上空也会笼罩一层雾霾。


我在圭尔夫转弯以后才下了车,迈着僵硬、颤抖的双腿去了卫生间。
随后,我加满了油,付账的时候顺便打听了大教堂的位置。
方向不清楚,但是对方告诉我在一座山上,到了镇中心,随处都能看见。

显然并非如此,尽管确实在哪里似乎都能看见它。
一个个精致的尖顶从四座塔楼里伸展出来,我以为它只是大,结果还挺漂亮。
当然确实很大,对于这么一座相对较小的城市来说,这肯定是最权威的主教座堂了。
不过,后来有人告诉我,实际上,它并不是主教座堂。

这就是沙琳结婚的地方吗?

不是,明显不是。
她当年参加的是联合教会的夏令营,那个夏令营没有天主教徒,倒是有不少新教徒。
那么,和C有关吧。不知道。

她也许悄悄地改了信仰。从那以后。

我及时地找到了到教堂停车场的路。
我坐在那里想我该怎么办。
我穿着休闲裤、夹克衫。
我觉得,到天主教堂——不是,是主教座堂——的要求是非常古典的,我不知道自己这一身是不是合适。
我试图回忆去欧洲大教堂参观的时候,是不是不许露胳膊?头巾?裙装?

上了山,是一片辉煌的、尊贵的寂静。
四月,树叶还没有开始发芽,不过,挂在上空的太阳毕竟已经很明亮了。
地上有一条低矮的雪堤,呈现出教堂前空地路面的灰色。

我身上的夹克衫,晚上穿太单薄了。或许是这里的夜晚太凉了,风比多伦多大。

这个时候,教堂也许已经锁门了。锁上了,空荡荡的。

高大的前门看起来确实如此。
我没有爬上台阶去试,我决定跟着两个老太太,她们和我一样老。
她们是沿着长长的阶梯从大路上过来的,看起来完全没打算走这些台阶,而是直接朝教堂一侧的便门走去。

里面的人更多,大概有二三十个,不过感觉他们不像来参加什么活动。
他们分布在教堂前排的座位上,有的跪着,有的在聊天。
走在我前头的两个老太太顺手在一只大理石圣水盆里沾了沾手,甚至没有抬起眼睛看看自己在干什么。
她们冲一个正在布置桌子上的篮子的男人打了个招呼,声音也不算低。

「这天气,看起来挺暖和,其实真够呛。」一个老太太说。男人说风快把鼻子刮下来了。

我看见了告解室。
如同一座座独立的避暑屋,像哥特式的玩具房,大量阴沉沉的木雕,深棕色的布帘。
而其他的地方都闪闪发亮,光彩夺目。
最高的弧形天花板是最神圣的蓝色,底下的弧线和直立墙连接,用绘制着圣像的漆金徽章来装饰。
彩色玻璃窗在这会儿的阳光照耀下,变成了一块块的珠宝。
我沿着侧廊小心地往前走,想看一眼圣坛,可是,教堂的高坛在西面的墙上,那儿的光线太亮,照得我睁不开眼睛。
纵然如此,我还是看见,窗户的上方画的是天使。一群天使,鲜艳、透明,纯净得如同光线。

这是最需要谨慎的地方。
不过,这里似乎没有人小心翼翼。
聊天的女人们确实是轻声聊天,但并没有轻到窃窃私语的地步。
其他人事务性地点点头、画画十字,就跪下忙自己的事情去了。

我也该忙自己的事儿了。
我四处张望,想找一个神父,不过视线范围内没有。
神父肯定也和其他人一样,工作了一天;他们现在肯定已经开车回家,进了起居室或办公室、书房,打开电视,松开衣领,拿了一杯喝的,心里想的是晚上不知道有没有好吃的。
他们来教堂的时候,是来供职的。他们穿上法衣,准备主持仪式。或者弥撒?

或者是来听告解的。问题是,没有人知道他们在不在。
他们的格子隔间,不是有他们自己进出的门吗?

我得找个人来问问。
那个在桌子上分篮子的男人,看起来不是光为了自己的事儿才出现在这里的。
他也显然不像引座员。
大家在这里都是自己决定坐在哪里、跪在哪里的,有时可能会因为宝石一样刺眼的光线干扰,站起来再换个地方。
我和他说话时压低了声音,以往我在教堂就是这个习惯。
他没听清,只好叫我再说一遍。
或许是因为困惑或尴尬,他犹疑不决地朝某间告解室的方向点了点头。
我应该非常明确,他才可能明白。

「哦,不,不,我是想找一个神父。有人叫我来找他。霍夫斯德神父。」

整理篮子的男人消失在侧廊稍远的一端,过了一会儿,他和一个神父一起回来了。
这个年轻的神父矮胖结实,脚步轻快,穿了一件普通的黑色袍子。

他叫我到一个房间去。之前我没注意到这个房间。
实际上,也不是房间。我们走过了一段拱道——不是门廊,到了教堂后面。

「这里方便谈话。」他说着,给我拉过来一把椅子。

「霍夫斯德神父……」

「哦,不是我。我得告诉你。我不是霍夫斯德神父,他不在这里,他在休假。」

有好一会儿,我都不知道怎么继续说下去。

「我会尽力帮助你的。」

「有一个女人,」我说,「她在多伦多的玛格丽特公主医院,快死了。」

「是啊,我明白,玛格丽特公主医院。」

「她请我来,我这里有一张她写的条子。她要我来找霍夫斯德神父。」

「她是这个教区的成员?」

「我不知道。我都不知道她是不是天主教徒。她原来住在这里,她是圭尔夫人。她是我多年未见的老朋友。」

「她什么时候和你说的?」

我只好解释,她没有和我说过,她睡着了,不过她留了张条子给我。

「那你不知道她是不是天主教徒?」

他的嘴角有一块开裂的溃疡,他讲话的时候一定很疼。

「我觉得她是,不过她丈夫不是。他也不知道她是吧。她不想让他知道。」

我这么说,想让事情更清楚一些。其实我也不知道是不是这样。
我有种感觉,神父可能很快就要烦了。「霍夫斯德神父肯定都知道。」我说。

「你没和她谈谈?」

我说她正在接受药物治疗。
不过,她不会一直在接受治疗,她一定有清醒的时候。我也强调了这一点,我觉得有必要强调一下。

「要是她希望做告解,你知道,玛格丽特公主医院就有神父。」

我想不出来还该说什么了。
我拿出字条,抹平了递给他。
我发现她的字迹没有我想象得那么正常。
只是因为有信封上的字做对比,才可能觉得这些笔迹容易认出来。

他露出了困惑的表情。

「谁是C?」

「她丈夫吧。」我担心神父问我他的名字,然后再和他联系。
不过,他只问了沙琳的名字。这个女人叫什么名字,他问。

「沙琳·沙利文。」真是奇迹,我记得她姓什么。
我又想了一下,确信我没有记错。
因为这个姓太像天主教徒的姓了。
这不是说明这位丈夫可能也是个天主教徒嘛。
不过,神父也许会推断这位丈夫堕落了,这样的话,沙琳的秘密就容易理解了,她的便条也会因此变得急迫了。

「她为什么只要霍夫斯德神父?」

「我想大概是有特别的事儿吧。」

「所有的告解都是特别的。」

他站了起来。我还是坐着不动。他又坐了下来。

「霍夫斯德神父正在休假。不过,他没出门。我可以给他打电话问问,要是你非这么坚持的话。」

「好的,谢谢你。」

「我不想打扰他。他最近情况并不好。」

我回答说,如果他的情况确实不太好,没法开车到多伦多,我可以开车送他去。

「要是有必要的话,他的交通问题,我们会解决的。」

他朝四周看看,没找到什么能帮他的,便取下别在口袋上的钢笔,打算在便条的空白处做点补充。

「我确定一下这个人的名字。夏洛特?」

「沙琳。」

~\\

在整个谈话过程中,我没有想过吗?一次也没有?
你可能觉得,当我瞥见广博的怜悯(即便有些小小的狡猾),我可能便崩溃了,崩溃是明智的。
但是我没有。不适合我。做过的事情都做过了。天使云集,却血泪斑斑。

~\\

我坐在车里,没想到要开发动机,尽管车里冻得我直发抖。
我不知道接下来该怎么办。
我知道我可以怎么做:找到公路入口,加入奔向多伦多的永恒、灿烂的车流;
或者,要是没力气开车的话,就找个地方过夜。
大多数地方都提供牙刷,至少也会有台机器卖牙刷。
我知道应该做什么、怎么做,但是我累了,太累了,没有力气做什么。

~\\

湖面上的摩托艇本来离岸边应该有些距离,特别是离我们夏令营的营区,以免摩托艇造成的水波妨碍我们游泳。
但是,最后一个早晨,那个礼拜天的早晨,有两艘摩托艇开始比赛,它们转着转着,就近了。
当然没有木排那么近,不过已经足以掀起波浪。
木排来回颠簸,保利娜提高了嗓门,她的叫声中满是斥责和惊恐。
摩托艇的噪声太大了,开摩托艇的人根本不可能听见她的声音。
他们掀起了一个巨大的浪头,浪花冲向岸边,我们这些在浅水里的人也站不稳了,有的随波浪起伏,有的失足跌进水里。

我和沙琳都没站住。
我们背对木排,因为当时我们在看维尔娜。
我们站的地方,水大概没到我们的腋窝。
我们被水抬起来,同时又被扔了出去,这时候听到了保利娜的尖叫。
我们可能也像其他人一样尖叫了,开始是害怕,后来是被浪头淹没再站稳的兴奋。
随后的浪头再也没这么大了,我们能控制好站着。

我们摔倒的时候,波浪把维尔娜卷了起来,朝我们的方向抛过来。
当我们重新浮上水面,脸上挂着水珠、胳膊胡乱扒拉时,她就在水面下,四肢张开。
四面八方都是尖叫、呼喊,浪头变小了,尖叫反而变多了。
错过了第一次袭来的浪头的,装作被第二个浪头打翻了。
维尔娜的脑袋没有钻出水面,她现在不再迟钝了,而是从容不迫,在水中轻盈得如同水母。
我和沙琳的手碰到了她。碰到了她的橡胶泳帽。

要是我们在力图保持平衡的时候,抓住手边一块不小的橡胶物件,根本没发现它到底是什么、我们到底干了些什么,这可能就是一场事故。
我想清楚了。
我觉得没人会指责我们。小孩子们都吓坏了。

是的,是的,不会知道我们干了什么。

这是真的吗?是真的。
从某种意义上来说,起初我们并没有做任何决定。
我们没有互相看,然后决定做这件明明是有意识做的事儿。
说有意识,是因为当维尔娜的脑袋想伸出水面的时候,我们的目光确实相遇了。
她想把头伸上来,如同锅里煮沸的团子。
她的身体在水下徒劳无益地虚弱挣扎,这种时候只有脑袋知道应该怎么办。

我们可能没抓住橡胶脑袋,橡胶帽子没有防滑设计,没有凸起的花纹。
我还能清楚地记得它的颜色,苍白乏味的蓝色,不过我没法描绘它的花纹——一条鱼,一个美人鱼,一朵花儿——花纹的脉络压进了我的手掌里。

沙琳和我的目光落在了对方的身上,我们都没有看我们的手在干什么。
她的眼睛睁得大大的,充满了喜悦的神采。
我猜想我的眼神也是如此。
我想我们并没有罪恶感,也没有为我们的邪恶得意洋洋。
感受得更多的是,我们仿佛正在做神奇地召唤我们去做的事儿,仿佛这是我们这辈子当中,让我们之所以成为自己的一个最高点,一个巅峰。

你可能会说,我们走得太远了,没法回来了。我们没有别的选择。
但我发誓,从来没有过选择,对我们来说,从来没有过。

整个事情可能没有超过两分钟。或者三分钟?一分钟三十秒?

要是说那时候,阴沉的云彩渐渐散了,可能太过分了。
但是,就在某个时刻——或许是摩托艇侵入的时候,或许是保利娜尖叫的时候,或许是第一个浪头打过来的时候,或许是我们手掌下面那块橡胶不再挣扎的时候——太阳突然出来了,海滩上来了更多的家长,辅导员叫我们别玩了,赶紧上岸。
游泳课结束了。
对那些住在远离湖边的地方,家乡也没有游泳池的孩子来说,夏天结束了。
私家游泳池只存在于电影杂志之中。

我已经说过了,和沙琳分手、钻进父母车里时的情景,我记不清楚了。因为无关紧要。
在那个年纪,事情说结束就结束。你会盼望事情结束。

我确信,我们没有说陈词滥调,没有侮辱,也没说过没必要的话。比如,不要告诉别人。

我可以想象,骚乱就此开始。不过,要是没有急剧的变化,不会迅速蔓延。
有一个孩子的凉鞋丢了。最小的孩子里有一个因为浪里的沙子卷进了眼睛尖叫不已。
一个孩子在吐,不知道是因为在水里兴奋过度了,还是因为家人来了太高兴,或者是偷吃糖的动作太快了。

很快,但不会是即刻,焦虑就开始弥漫。有人失踪了。

「谁?」

「一个特殊营营员。」

「该死。不知道出了什么事。」

负责特殊营的女人跑来跑去,穿着她那件花游泳衣,肥胖的胳膊和大腿上,奶油冻般的肥肉直晃荡,声音发了狂,眼泪都要掉下来了。

有人去树林里找,沿着林间小径往上走,叫着她的名字。

「叫什么名字?」

「维尔娜。」

「等等。」

「什么?」

「水上面有什么东西?」

\end{document}