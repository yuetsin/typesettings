%-*- coding: UTF-8 -*-

\documentclass[UTF8]{ctexart}
\usepackage{graphicx}
\usepackage{float}
\usepackage{CJKpunct}
\usepackage{amsmath}
\usepackage{geometry}
\geometry{a6paper,centering,scale=0.8}
\usepackage[format=hang,font=small,textfont=it]{caption}
\usepackage[nottoc]{tocbibind}
\setromanfont{STSongti-SC-Regular} %设置中文字体
\punctstyle{quanjiao} %使用全角标点	


\title{完全自杀手册}
\author{鶴見 \ 済}
\date{1993/7/7}


\begin{document}
\maketitle

\newpage
\tableofcontents
\newpage

\section{Preface 前言}

这本书,详详细细写尽自杀的方法。

这本书不是常自杀者的记录报告,也不是想说明有关自杀的理由。虽然可以将它当成是一本杂记书来阅读,不过整本书的走向是朝着「如何自杀」的方向进行。

啰啰嗦嗦的开场白,可能您已经厌烦了。

不知道从多久以前开始「为什么年轻人要走向死亡」这个话题就不断反复再反复地被谈论着。那时,比方说七十年代,所得到的结论是「虚无主义」、「不欢年代」等等。然而,像是「为什么不可以自杀」、「为什么一定要活着」这类的问题,却始终没有任何解答。 言归正传,目前需要的是一本能让「自杀」付诸实现的书。

有关这样的书,十年前出版的「自杀的方法」,几乎也只是写些拉拉杂杂的内容,令人烦腻之至。现在应该知道的是,纯粹的自杀方法。

在美国,只有一部可以进行安乐死的自杀装置,是由一位学者发明创造的。(案例 30)而 本书将是日本唯一的一本「以语言文字为工具的自杀装置」。

谈到这里,很想马上从吃药自杀的方法介绍起,不过为了让「现在为什么要自杀」这个问题更明白些,以及其它种种商业上的理由,不得不先写一些拉拉杂杂的东西。

\subsection{Chernobyl}

……我在等待时机,准备「大显身手」。二十年前发生学生暴动时,大家曾以为「厉害人物要出现了!」阿波罗号登陆月球、石油危机、苏联侵略某个国家、昭和年代即将结束,我想「这次的举动将会惊天动地」。然而却只是3级震度只不过倒下一面墙而已。学生彼此对视,笑着说「太棒了!」,活动即告结束。……\footnote{节录自知上寿『黎明』后记。}

八十年代即将结束时,曾经掀起一阵「世界末日潮」、「危险话题」,最受喜爱的乐团唱着 Chernobyl\footnote{切尔诺贝利核事故,是1986年4月26日于苏联乌克兰普里皮亚季市切尔诺贝利核电站发生的核反应堆破裂事故。该事故是历史上最严重的核电事故,也是首例被国际核事件分级表评为最高第七级事件的特大事故。} 的歌,小孩子的玩笑话全都充满死亡的味道,少女为了准备迎接世界大战而开始寻找同伴。我们则为「厉害人物要出现了!」、「也许明天会是世界末日!」而兴奋不已。

然而,世界末日并没有到来。原子弹始终没有爆发,全面核武战争的梦也早已消失的无影无踪。

八十年代的革命家,彻底的体验到挫折感。

最后大家终于明白,「大显身手」是不可能实现的,二十二世纪一定会来。

(当然,二十一世纪即将来临,因为不会有所谓的世界大战。)

世界绝对不会出现末日。只是稍稍接遇到「异界」及「外面」,并无法得到满足。如果希望有更大的刺激,如果真的希望世界走到末日,除非是做「那件事」。

\subsection{A Long Vacation}

说什么「枯燥乏味」并无济于事--因为我们运气不好,就出生在历史的这个舞台。
一直到二十二世纪,我们会每天早上七点钟起床,然后上学或上班,反复说一些毫无重点的话。

在学校,一次又一次不断的背英文单字、历史年号;在公司则一面说些「无聊透了」之类的,却又以一星期、一个月、一年的周期,反复好几个星期、好几个月、好几年的做实际上真正无聊的工作。

步调缓缓的最先进产品不断出现,步调缓缓的政治家继续贪污,电视内容步调缓缓的继续激动。

但是,当关掉电视环顾四周时,却又是一如往常的另一个每一天。

唤醒「关掉电视后 那种奇妙的黑暗」,正是这本书的另一个目标。)

三岛由纪夫在他的自传小说「假面的告白」中,提到:「日常生活」比战争还要恐怖。
我们 总是一忍再忍的过着这种「令人颤抖的恐怖日常生活」。
为的是能带来莫名其妙的「安定将来」。
一路上步步为营,小心翼翼的避免方向有所偏差。

没 有 像 电 视 连 续 剧 这 样 喜 剧 收 场 的 结 尾 。

只 是 ,奇 怪 的「喜 悦 」总 是 不 断 步 调 缓 缓 的 继 续 着 。

是的!关键词是「步调缓缓」和「反复」。持续的相同事物步调缓慢的反复出现;这是让想死的情绪膨胀的第一要素。

\subsection{Another Brick in the Wall}

1978年日本曾经发生一件「纸上迷宫自杀事件」。

一对住在日本富山县的高一孪生姊妹,被发现在树林内上吊自杀身亡。

其中一人的笔记本上,画有用四条直线和数条横线交错形成的纸上迷宫图案(一种沿着线寻找源头的游戏)。

图案下面,分别写着「日本人的×」、「自杀」、「ROS」、「御三家」这几个令人无法理解的字眼。

从上面画的线是一直连接到「自杀」的地方。由于其父母怎么也无法想出她们自杀的动机,所以结论是两人因为「迷宫」而自杀。

据说「ROS」也许是滚动的石头(Rolling Stones)的意思。因为上面还写着「日本人的×」、「讨厌亚洲人」等字,所以说不定与此有关。不过,始终无法了解这些字的意思。

从前,有个法官说:「人的生命比地球还重。」然而,这是极无价值的误解。

正如同七十年 代两位高中女生早已察觉一般,人的生命很轻,和「日本人的×」以及「ROS」一样轻。

五十年代末期,美国大众社会学者说过:「大众就像陷入无力感的原。」

七十年代末期,英 国的摇滚乐团唱着:「我们是墙壁中的一块砖。」非常走红。即使是进入九十年代,日本的 这个状况依然没有任何改变。

老样子,我们仍旧是墙壁中的一块砖--证据是,假设我们之中有任何一个人死了,必定会 有某个人来取而代之。

没有一个人的存在是无可替换的,也没有一个政治家是足以被暗杀的。 

只少了一块砖,墙壁并不会因此而倒塌。 

我们每个人都充满无力感,存在不存在都无所谓,换句话说,生命是轻的;这是让想死的情绪膨胀的第二要素。

\subsection{Clockwork Orange}

抱 着 这 种 无 力 感 ,步 调 缓 缓 的 反 复 做 相 同 事 情 的 我 们 ,一 点 一 点 忘 掉「 真 正 活 着 的 真 实 感 」。

已经渐渐忘了自己是活着还是死了。

你有感觉自已是「活着」吗?现在,生与死之间,只被 一条细得几乎看不见的界线隔开而已。

因此,「生命很重要,所以不可以自杀。」、「只要能活着,一切都会有转机。」、「因为 周遭的人会难过,所以必须活着。」这类的话,已被打入冷宫,不再具有任何说服力。

制止 自杀的有效话语,已经消失;引导自杀的信号已经出现。

是的,要死也可以。如果上班或上学,活着很不舒服的话,很无趣的话,甚至还很痛苦的话, 是可以跨越细得快看不见的界线去寻求死亡,任谁都无法加以阻止。

前 面 也 说 过 ,反正 活 着 ,一 切 也 不 会 有 所 改 变 。

虽然 不 具 有 特 异 功 能 ,不过 大 致 上 可 以 知 道 , 今后的社会或自己会发生什么样的事。

「将来!将来!」,就算这句话再怎么有说服力都没 用。

你的人生,大概是在出生地念小学和中学,上补习班为联考读书,然后进入一所高中或 大学就读,浑浑噩噩玩了四年后进入某家公司上班工作。

男性的话二十来岁三十岁前结婚, 隔年生子,几次的工作异动或升迁,最高升到经理职位,六十岁退休,之后的十年或二十年过着享受自己兴趣的生活,最后死亡。
顶多就是这样。而且,令人绝望的,这竟是最能让人安心的理想人生。 

在这样的状况下,平凡活着已经再也没什么重大意义了。假使不是现在活着,或许只是像做 烤鸡用的嫩鸡一样,「被给予生命活着」而已。所以在适当的地方为人生画上休止符,并不 是「悲伤不已」、「不会发生第二次」、「担心会出现波及效应」这类的问题。

自杀是相当积极的行为。

\subsection{Angel Dust}

我有一个朋友,他有一种叫「天使尘埃」,吃了之后头脑会变得昏昏沉,可以毫不在意地从高楼往下跳的强烈药物,装在金属小囊中,作成项链,形影不离的带在身上。

他说:「必要的时候,可以吃下这个来寻死。」我的朋友没有固定工作,每天游手好闲,过得非常愉快。

希望这本书可以成为那条金属小囊项链。

\section{Taking Medicines 药物}

痛苦\ —————

麻烦\ ▼▼▼▼▼

死状\ ▼▽▽▽▽

牵连\ ▼▽▽▽▽

冲击\ ▼▽▽▽▽

致命\ ▼▽▽▽▽

如果能成功的话,吃药是最佳的自杀手段。不过自杀未遂的情况很多,及准备时间过度费时
是一大问题,痛苦的程度因药物不同而异。

桌上散放着红、黄、蓝等各种颜色漂亮的胶囊和药片,或是一小堆雪白的粉末。当把这些吞
服下去的时候会慢慢地睡着了,而且再也不会醒过来。这是在平静睡眠延长线上的死亡,也
是最理想的自杀手段,而且是只有透过药物才有可能实现的手段。

\subsection{Lobotomy}
某种药物会控制头脑。

脑部是担负人体之自我组织性的系统控制装置。不论是呼吸功能或心脏跳动等,凡是维持系统用的功能全部都是由大脑控制的。

在此要说明的是,透过药物使这 一控制装置逐渐失去协调,而用来破坏整个「自我」体系的技术。这也可以说是一种自己动手进行的脑白质切除术\footnote{脑白质切除术(Lobotomy)是一种神经外科手术,包括切除前额叶皮质的连接组织。脑白质切除术主要于1930年代到1950年代用来医治一些精神病,这也是世界上第一种精神外科手术。含有非法军事用途。}的方法。

\subsection{致死量}
不论什么药物,基本上都有「会产生作用的量」和「中毒量」及「致死量」。

也有专家说,「不会致死的药不算是药。」

不管什么药都有致死的可能。只不过是作用量和致死量之间的
差距较大,要达到致死量就必须吞服相当大的量,比较难掌握罢了。现在的药物,这种差距
都放宽了,换句话说是安全药物。这里要介绍的,是差距较小的。

不过,此问题就在于「致死量」。本书所述药物服用量的标准,是各种文献所载不同药物的 致死量,但也因文献而各有不同,幅度相差也很大。同时,致死量又因个人的体质以及是否 经常服药而大有差异,因此,也有虽大大超量服用却被救活,而服用少量的却死亡的例子。

本来应决定生与死界限的「致死量」,却有因剂量不同、幅度参差、因人而异而未能致死的 情况。于是有的研究工作者就说,真的想死的话就服用致死量的三倍吧。这样一来,致死量 已经不是表示「致死的量」了。本书暂且将最少量与最大量之间作为「致死服用量」(未特 别注明的均为成人服用量),这也只是大体的标准。这种致死量的含糊不清不但表明了生与 死的模糊界限,也是使药物自杀产生困难的最大原因。

不过,这里还要再啰嗦几句。仅仅把手腕割开一道口子是绝不会死去的,但这里所举出的药 物,服用了就真的会死去。如果只是想体验一下自杀的滋味(这也并不是坏事),还是在手 腕上割开一道口子较好。

\subsection{呕吐}

加强消化器官的吸收以避免反胃 最应该避免的是将服下的药物吐出来。服用药物而死,换句话说就是急性药物中毒。只要设 想 一 下 喝 酒 的 情 景 就 行 了 。当然 ,酒 精 也 不 是 不 能 致 死 。常 常 就 有 大 学 生 因 酒 精 中 毒 而 死 亡 。 

可是,一般情况是当身体无法再承受时就会吐出来。这只是将酒精换成药物而已。由于大幅 地超量服用,呕吐出来也是很自然的事。所以,掌握适量是非常困难的。

阻止呕吐当然也有方法,只是加强消化器官的吸收,就像对症疗法那样。对因生理的拒绝反 应而呕吐出来的,则无方法对应。总之,只有朝「致死量」努力服用,哪怕只是再多吃一片。

\subsection{服用……是不会死的}

其它应注意的,就是了解「致死量」之后再服用药物。可以致死的药物有很多。

譬如,当今成为话题的安眠药——\qquad \qquad 。睡 意 来 得 快 ,服用 后 会 丧 失 记 忆 力 ,所以 看 起 来 非 常 危 险 , 但 却 是 很 安 全 的 药 。一 位 喝 酒 同 时 吞 服 了 $80$ 片 的 二 十 七 岁 女 性 ,睡 着 后 又 像 梦 游症患者似的爬了起来,向双亲说了句「到罗森去」就倒了下去。被送到医院的第二天恢复 知觉,经检查的结果一切正常。仅仅看上去危险是不会死人的。

本 书 对 不 明 了 致 死 量 的 药 物 一 律 不 加 以 介 绍 。没 有 计 划 的 行 动 ,不但 会 失 败 而 且 还 要 尝 到 不 必要的痛苦。一位吞服氰酸钾的家庭主妇,整个晚上都抓挠胸,被发现时胸部一片血肉模糊。

顺便提一下,有关药物自杀的传闻,有很多都是虚假不实的。常有人说,会对内脏或脑子留 下 终 生 不 治 的 严 重 后 遗 症 ,但 仅 从 本 书 所 举 对 神 经 系 统 产 生 作 用 的 药 物 来 看 ,有的 医 生 说 几 乎都是没有这种副作用的。

也有「反自杀论者」宣扬说,健康保险不承担自杀的治疗费,所以要花费钜大的金钱,但这 也是不真实的。一位二十六岁的女子吞服了虽然我们不知道成分的剂量,但在「医院里也认 为药性最强的药」的镇静剂$20$片片剂和$20$包粉末,而被送进医院整整昏迷了两天,但她 却使用保险卡支付了医疗费。所以,不被谎言所欺骗,也是需要注意的一点。

\subsection{静脉注射}

首先,防止呕吐有各式各样的对策。

最简便的方法是,将药物放在铝类的东西上用火烘烤使其融化,或用热水溶解变成水溶液,然后用针筒向静脉注射。这样的话,当然不会呕吐,而 比肠胃吸收的作用更大,需要量也少得多。如果针筒内混入空气,脑血管会发生空气栓塞而 死亡。还有因血管内混入血液以外的异物,引起休克而死亡的。由于异物的混入,血管会有 相当程度的疼痛。多次注射的情况,疼痛得很厉害,可对手臂进行冷敷,总之,需要想些办法。

\subsection{最后的晚餐}

不想注射而想服用药物去死的人,就有必要事先调整一下胃的状态了。为了更快地吸收药物,
有效地引起急性中毒,胃里就不应留有食物。但如果过于空腹,会产生反应过敏而呕吐,掌
握它的分寸是很难的。

在 决 定 自 杀 前 的 一 小 时 左 右 应 该 喝 杯 红 茶 吃 一 片 吐 司 ,服用 一 点 防 醉 药 亦 可 。片 剂 应 事 前 碾 碎,放入粉碎机使其成粉末。胶囊则应去掉,留下粉末,或者放入热水里使其溶成液状。
然后将药物混在布丁或酸奶里,加些蜂蜜,迅速吞服。但也有专家说,将药物混在布丁或酸
奶里并不会加快吸收。或许将它认作是帮助吞服的一种方法较好。

同时喝点酒是必要的条件。据说,酒对中枢神经产生作用的药物具有相辅相乘的效果,对任
何药物都可加速溶解速度,并用时的效果可提高百分之五十。饮料除了酒之外,最好再多准
备一点。

再 者 就 是 一 定 要 选 择 别 人 找 不 到 的 地 方 ,至少 要 保 证 八 小 时 之 内 能 单 独 停 留 的 地 方 ,例 如 旅 馆就是最合适的。如果在数小时以内被人发现,就成为自杀未遂,要尝到反复灌肠的痛苦。

一 位二十五岁的女子因企图自杀而被灌肠,她说「早知那么痛苦的话就不自杀了。」(不过她 还说了一句「或许会再自杀一次。」)

\subsection{大众药}
准备工作完成后,就开始介绍自杀药吧。

市面贩卖的药物,其毒性较低。专业书上写说:「(市面贩卖的药物)单剂大量服用的话几
乎是不会致死的。」

这一叙述虽不能认为是正确的,不过,服用大众药去死亡确实是难的。如前所述,药物有各
种产生作用的量和引起中毒的量,所以,与中毒量保持一定的距离而以低毒性制造的就是大
众医药。因此,一般的药房都有出售。

可是,自杀药的基本点是「容易到手」和「服用即睡」。无法到手的话,即使介绍也无意义。 有必要知道服用大众医药物是难以致死的,但之前专业书籍却写道「几乎是不会致死的」。 也就是说,并不是百分之百地不可能,这里就介绍这种例外。

\subsubsection{利斯隆S}
\begin{flushright}
「有来历的安眠药」
\end{flushright}

在战后的五十至六十年代里,出现过第二次安眠药自杀热。以二十多岁的年轻人为主,服用
安眠药自杀大大地超过上吊人数。溴戊酮尿素(Bromovaleryl)在当时以「布罗巴林(ブロバリン)」的商标名出售,是自杀热的主要药物。
在1926年芥川龙之介服用安眠 药自杀而引起第一次自杀热时,「布罗姆」以「卡尔摩汀」之名出售被广泛使用。

太宰治也用「卡尔摩汀」多次情死未遂。因此说「布罗姆」是日本的安眠药自杀由来已久的 药品。 现在,「布罗巴林」被指定为「须经医生指示使用药」,没有医生的处方是买不到的。「利斯隆S(リスロンS)」(佐藤制药公司)在市面上出售的则与「布罗巴林」的成分和剂量相同。显然这并 不违法。因为,每片中的「布罗姆」的分量在$500$毫克以下是可以出售的。这是八毫米大 小的片剂,白色味苦。

条件成熟后只要服用致死量,中枢神经系统被控制,脑和脊髓神经的刺激作用失灵,迅速
丧失意识,快的话一小时左右主要就会因呼吸停止而死亡。有人服用$50$片「布罗巴林」而
死掉的(案例2),但还是服用$200$片为适当。$200$片是满满的两只手的量,如果是这些总量的话,
也不必要混在酸乳酪那么麻烦服用了(案例1)。

此外,在一片药品中含有$100$毫克「布罗姆」的药物中有「姆尔蜜锭 - ルミン」(小林 药品工业)、「日夜(デイナイト )」(大正制药工业)等,「利斯隆S」是纯由「布罗姆」 制成的,但其它的却都含有若干咖啡因等一些成分。由于含有多余的成分,服用量就随之增 加,呕吐的可能性也增大。 「姆尔蜜锭」之外,后述的「Acetaminophen」也在一片中含有一$50$毫克的「布罗姆」。 服用相当于「布罗姆」致死量的$100$~$200$片的话,也达到「Acetaminophen」的致死 量区。这是双倍致死量,更加危险。

「利斯隆S」的说明书上写着「对不安、紧张有镇静作用的药品」,只要说明症状即可买到,
但最大的问题是怎样能买到十到二十盒。

多数药房无货,制药公司批发给药房的量也不多,
一次是买不到多少的。有时也有因药房的判断而不予出售的情况。

大车站附近的药房一般都
有供应,寻找五家左右,每半个月去买一次,两个月就积下二十盒。这种药,每日服用二至
三片,盒内只有十二片,所以每隔十天去买一次也不足为奇。

不过,为了防止滥用,药房有
时不出售,或者询问症状,或者建议前去医院。

某药房工作人员说:「看上去不对劲的人不
卖给他。」这种场合,你就解释说「不想去看医生。」、「要是让同事(同学)知道精神不
佳可不得了。」这对唯恐滥用药物也是一样的。

当找不到出售的药房时,可打电话给制药公司,他们会告诉你那几家有出售。

\subsubsection{阿塔拉克斯}
\begin{flushright}
「吃下后立刻睡着」
\end{flushright}

这种药,一般药房都有出售。买到六十到六百个胶囊,即三到三十盒是很容易的。前面说
过,「弄到手,服用即睡」是安眠自杀药的条件,而这个「阿塔拉克斯P(アタラックスP)」(ファイザ-制 药公司,通称「阿陀P」)就是条件最好的药品。致死量是体重每公斤需要$25$~$250$毫 克,幅度太大,而且又是含糊的推测,所以六十公斤体重的人可服用$600$片。 仅 仅 将 $600$个 绿 色 小 胶 囊 放 在 面 前 ,就会 引 起 一 种 幻 觉 ,但 一 个 胶 囊 中 含 有 的 成 分 即 使 全 部服用也不过是二十毫克。胶囊里有黄色粉末,集中起来也只有小碗的四分之一,这点量是 用不着混在酸乳酪里也很容易服用的。

盐酸羟胺具有抑制中枢神经,稳定自律神经的效果。大量服用时会出现睡觉、疲倦、头晕, 最后因呼吸停止而死亡。服用时一定要并用酒类。中枢神经抑制作用会增强,如在此时把身 体弄凉的话效果则更好。 成分完全相同的药物还有「吉斯隆P」(帝国化学)、「波布斯尔( ボブス-ル)」(加藤翠松堂)等。「阿塔拉克斯P」有$100$片装的。 不过,有一位服用了「阿陀P」$100$片并喝了一瓶啤酒和一杯威士忌的四十岁妇女,在十 七小时后丧失知觉并呕吐,被发现后送往医院,两天以后痊愈出院。

总之,要再三考虑,并 应服用多量。

\subsubsection{巴布隆S、阿涅通颗粒}
\begin{flushright}
「止咳剂」
\end{flushright}

八十年代曾一度流行过「一口气喝下止咳液后狂跳」的游戏。因为止咳液里含有作用于脑内
咳中枢而镇咳的「可待因(Codeine)」,和扩张气管的麻黄素(Ephedrine)等麻药性物质
的缘故。可待因是鸦片成分,麻黄素的致死量虽不清楚但它是兴奋剂的原料,所以很危险。
理 应 禁 用 的 ,但 不 使 用 就 止 不 了 咳 ,因此 ,现 在 出 售 的 止 咳 剂 ,大多 都 含 有 这 两 方 面 的 成 分 。

「巴 布 隆 S 」(大 正 制 药 )也 是 这 类 药 物 之 一 。八十 年 代 使 用 较 多 的 是「布 隆 液 」(S S 制 药 ),
但现在这种药不含麻黄素,所以,这里举出与可待因的构造几乎相同的磷酸二氢可待因
(Dihydrocodeine)$30$亳克和麻黄素的姊妹成分--盐酸甲基麻黄碱
(Methylephedrine)的「巴布隆」。大量服用磷酸可待因会造成睡意,而麻黄素则有觉醒
作用。整体看来,与其说是睡眠,无宁说兴奋作用方面较强,导致心跳数的增加和呼吸加速
因循环不全而死亡。此外,含有$30$毫克二氢可待因(Dihydrocodeine)的止咳剂有「新德
尼(トニン)
止咳液」和「新メトロンSコ-ワ液」(兴和)。

「阿涅通颗粒」(ファィザ-)的dl--盐酸甲基麻黄碱的含有量为$10$毫克。关于可待 因《最终的出路》一书中列举了具体的$2.4$ 毫克这个数字,所以就相信它而服用$160$包 以 上 吧 。这样 就 要 服 用 $240$ 毫 克 的 颗 粒 ,从量 来 说 只 是 一 小 碗 罢 了 。

某 专 家 的 意 见 是 :「混 在酸乳酪一类的东西」白色颗粒,味甜。

\subsubsection{拉克摩亚}
\begin{flushright}
「止吐剂」
\end{flushright}

拉克摩亚( ラックモア) 止 吐 剂 ,意 外 地 含 有 很 多 危 险 成 分 。某 位 二 十 四 岁 的 妇 女 为 了 堕 胎 而 服 用 迪 门 希 得 里 纳 德 $7.5$克后,出现呕吐、目眩、抽搐,陷入昏睡状态,九十分钟以后因呼吸不全而死亡。「拉 克摩亚( ラックモア)」(藤泽药品工业)除了迪门之外还含有盐酸吡哆醇$7$亳克,无水咖 啡因$14$毫克,比纯迪门希得里纳德的药物容易弄到手。这是白、蓝、白三层结构的大一点 的 少 许 片 剂 。含 有 迪 门 希 得 里 纳 德 $50$ 毫 克 的 药 物 有「卡 拉 克 斯( カ-ラックS)」(共荣)、 「迪门希得里纳德」、「摩德( モ-ト )」(摩德制药)、「Drive Soft」(长生堂)等好几种,药房也有出售。加上服用至少$30$片就能死去的话,对自杀是
再好不过的了。
服用这种药物的问题在于是否能很好地睡觉。前面提到这种药对中枢神经产生作用导致呼吸
停止,副作用为有睡意、头痛等,但由于不是精神药物,也就不能迅速睡着了。要是一定要
睡的话,可并用市面上卖的安眠药。

不 过 ,也有 服 用 $800$ 毫 克 迪 门 希 得 里 纳 德 ,虽 产 生 了 幻 觉 但 未 死 去 的二十 岁 和 二 十 二 岁 的 男子的事例。 

\subsubsection{旅行明尼亚}
\begin{flushright}
「著名的『旅行明尼亚』也会死人吗」
\end{flushright}

或许有人会产生疑问,「旅行明尼亚(トラベルミンシニア)」的成分是磷羟基苯甲酸(ジフェンヒドラミン)$40$毫克,和二羟丙基茶碱(Diprophylline)
$26$毫克。其余的成分不多,危险成分磷羟基苯甲酸所占的比例最大。

其它的,含有咖啡
因等预防睡意成分的也很多。 「磷羟基苯甲酸」作为抗组胺剂(Histamine)
而 产 生 作 用 ,中枢 神 经 系 统 的 抑 制 和 兴 奋 的 症 状 出 现 混 合 。产 生 剧 烈 的 脑 刺 激 和 严 重 的 知 识 丧 失 。

可是 ,抗 组 胺 剂 具 有 催 眠 效 果 的 作 用 ,禁 止 在 服 用 后 驾 驶 车 辆 ,如大 量 服 用 会 睡 着 的 。


\subsubsection{仙巴亚}
\begin{flushright}
「谁都不会用那种东西去自杀」
\end{flushright}

防吐防晕药的优点,就在于人们认为「谁都不会用那种东西去自杀」。你去购买时可以说「因
为要出远门,想多买点预防晕车的药。」,药房也不会起疑心的。「仙巴亚」(大正制药)
一盒只有六片,要买$200$片只需要三十三盒,是不必担心的。

溴化氢东茛菪碱具有抑制副交感神经的作用,大量服用会抑制呼吸。副作用为有睡意,同时
有错乱、幻觉,吞咽困难等,服用后是否立即睡着是有疑问的。
含有溴化氢东茛菪碱的药物以防止晕车船的较多,目前市面上供应的药物中,「仙巴亚」东
茛 菪 碱 的 含 量 最 多 ,而且 不 含 其 它 多 余 成 分 。直 径 一 公 分 的 大 片 剂 ,要$200$片 就 需 要 弄 碎 , 无苦味。


\subsubsection{其它}

大正东普库( トンプク)、隆三宝胶囊A 「大正东普库」(大正制药)和「隆三宝胶囊A」(三宝制药)含有 $300$ 毫克的 Acetaminophen 具有镇静作用和解热作用。大量摄取可产生睡意、目眩、头痛、恶心、呕吐,陷于昏睡。最 后危及肝脏而死亡,不能说是安眠药。不过,「大正东普库」的一包中含$200$毫克的 Bromovaleryl 尿素,所以容易睡着。药品的注意事项也写着「容易引起睡意」而禁止服用 后驾驶汽车和操作机器等。白色小颗粒,味苦但服用到口内有清凉感。

镇痛剂「诺信( ノ-シン )」( アラクス)等,在一袋中也含有 $300$ 毫克的Acetaminophen 。其它 配 合 物 也 很 多 ,但 这 有「一 下 子 可 买 很 多 」的优 点 。如果 要 列 举 出 含 有Acetaminophen 的解毒镇痛剂的话那就不胜枚举了,市面上出售得很多,不妨到处采购可达致死量的药量。 举例来说,一个四十九岁的妇女服用了$30$片雪德丝A(含 Acetaminophen $80$ 毫克、Bromovaleryl尿素 $100$ 毫克、
$200$ 亳克、无水咖啡因 $25$ 毫克)引起了肝、肾衰竭,经过了十六天的兴奋和昏睡,因呼吸
停止而死亡。只服用$30$ 片(Acetaminophen $2.4$
克)而死去的这个妇女是幸运的。这种「雪德丝」的成分中产生最厉害的副作用的是
Acetaminophen。

现 在 出 售 的「新 雪 德 丝 片 」(盐 野 义 制 药 )用 尿素 $30$ 毫克、
咖啡因 $40$ 毫克代替布罗姆,其它成分无变化。

有人认为「经常在喝咖啡」而不太在乎咖啡因是不好的。咖啡因会刺激脑内的血管运动中枢
和呼吸中枢,加强心脏的肌肉收缩力,是刺激性很强的药剂。大量饮用可引起兴奋、血压上
升、心室颤动(心脏下部的心室部分收缩的现象),心肺停止直至死亡。

阿司匹林刺激呼吸中枢等的中枢神经系统和代谢系统,导致呼吸过急、代谢异常、高热,因 呼吸衰竭或休克等死亡,对肾脏也会造成损伤。 
两者都具有兴奋作用,服用绝对不会睡眠,但因容易到手和少量即可致死的原因加以介绍。

对 想 安 然 死 去 的 人 是 无 缘 的 药 品。

试 举 一 例 ,一个 服 用 了$500$毫 克 的 阿 司 匹 林$100$ 片($50$克)的二十一岁女学生,经历了倦怠感、恶心、兴奋、呼吸加促、昏睡等各种症状后,约 十九小时后因呼吸停止而死亡。

从上述例子来看,服用阿司匹林到死亡的时间是相当短的。

\subsection{案例研究 1}

\subsubsection*{向报纸投稿预告自杀的青年销售员}

1956 年 8 月 2 日 ,晚 间 十 时 左 右 当 目 白 一 家 电 影 院 连 续 放 映 两 部 影 片 ,领 位 小 姐 发 现 一 个青年还在睡觉,于是打算叫醒他,却发现此人已经冰凉。

他是喝着酒慢慢地吞服了$200$ 片Brovarin 和$30$片Adorm 。

在立即被送往医院后,整整昏睡了一天一夜,四日清晨死去。

他出生在东京,二十三岁。从水产大学中途退学,在制药公司工作过四年半,主要从事打包 和送货工作。

其兄也是制药公司的药剂师,估计从那里得到了药物的致死量和服用方法。

在前一个月的7月28日,因「私人原因」辞去了制药公司的同一天,又开始某百货公司 「分期付款订货」的推销。但是,他对该公司所规定的达不到目标业绩,则全部佣金归公司 所有的办法极为不满,竟然在自杀前一天的8月1日向朝日新闻的投稿栏寄出了一封写着 「当我离开这个世间之时,大声疾呼这个公司的不合法做法。」的一封信。

也就是说,他以 投稿方式预告自己的死亡。

他在给百货公司的股长遗书中这样写道,

「当我决意进行早已下了决心的事之前,只不过是 很 偶 然 地 遇 到 了 百 货 公 司 这 一 段 经 历 。
我并 没 有 把 这 个 社 会 存 在 弱 肉 强 食 的 责 任 ,推 给 机 构 基层的头上……」

这说明不是因指责公司不公而自杀的。

人很开朗,又有爱好哲学的一面,关在房间里写过侦探小说的这个年轻人的自杀动机,至今 不明。

\subsubsection*{检验死因}

在这里需要检验的,并不是他在死之前所采取的奇怪行为,而是他吞服了 $200$ 片 Brovarin,
$30$ 片为 Adorm 等大量安眠药而居然没有吐掉,完成了自杀。

一般情况下,吞服这么大量的片剂都会吐出来的,这也是服毒自杀的缺点之一,他恐怕是在电
影上映的两个多小时里一点一点地吞服,以及证明了「一点一点吞服就不会吐出」的对应做
法。

Brovarin 销售时包装就只有 $100$ 毫克的,他吞服了 $20$ 克,就是说足以达到了 $10$~$30$
克的致死量。据说,和这些安眠药一起喝下酒的话,效果就会达到十倍,他大概是从其兄那
里听来的。

\subsection{案例研究 2}

\subsubsection*{服用 Brovarin 50 片,留下「死亡记录」的学生}

1972 年 5 月 ,一 个 二 十 岁 的 学 生 在 房 间 里 从 椅 子 上 滑 倒 而 死 去 。

他 房 间 里 的 桌 子 上 放 着 遗书,题为《最后时刻》的崭新笔记本,开头就写道「我并不是受所谓厌世观的影响而逃避 这个社会的。我之所以要现在消失,是感到现在正是时机才决定自杀的。」 而且更奇怪的是,他把自己吞服药物起到失去知觉为止的经过,详细地写在大学笔记本里。

「现在是 1972 年5月19日下午7:07,我在两三分钟前吞服了 $50$ 片 Brovarin。

……我不知道再过几分钟或几小时后会睡着,并与这个社会告别,但我将观察我本人 的死到最后一刻。

……10:15,一点都不想睡。打开收音机听了一会儿,又关上了。 我 想 保 持 冷 静 但 还 是 兴 奋 的 ,平 常 所 爱 好 的 音 乐 今 天 却 感 到 厌 烦 。

……我想 仔 细 地 回 顾 一 下 自 己 的人生。

(关于双亲和友人有五张纸的记录)

……11:30,从刚才起就一直在打呵 欠,看来最后的时刻即将来临了。
但我要看着死的心情到最后一刻,所以不断地告诫自己要 坚持。我不知道自己的尸体将在什么时候被发现,但我自己想看看自已的欲望不断涌现上 来。

……12:05。」

笔记到此结束,大概就在这个时候他神志不清了。

\subsubsection*{检验死因}

Brovarin $50$ 片相当于 $5$ 克 Bromovaleryl 尿素,其量比致死量低得多,但我不认为该青年在撒谎,所以就可以知道 吞服这么多的量也会死去的。同时也知道了吞服$50$片 Brovarin
就会慢慢地增加睡意,大约两小时睡着以后就不会再醒过来了。而且到失去知觉为止完全没
有痛苦,神志也很清楚。

当然,会因本人的体力和健康状态而大为不同,如果马上就倒下的
话大概会在更短的时间里睡着的。

他的自杀真的可说是「实验」。关于留下记录一事,他写道:「什么也不留下就死去,很容 易被误解为在我这个年龄常有的忧郁结果。我无论如何也不想被误解,因此,我在这里很冷 静地留下我在仔细看看自己的证据。」可是,我们完全搞不清楚他自杀的动机,所以只有认 为他是「为了试试死去时候的心情而死掉的」。

不管怎样,到即将失去知觉为止一直握着笔杆写出的记录,是极其珍贵的。在最后的「五分钟」之后,恐怕是应该接着写「然后就该睡着了吧」的字句。

此外,作为服用 Bromovaleryl 尿素的例子,有一个吞服$50$ 片 五 种 布 罗 姆 系 列 的 药 物 的 二 十 二 岁 女 性 ,在 服 用 后 十 二 小 时 被 发 现 并 治 疗 ,在 半 睡 状 态 中 延续了十天后,因心脏衰弱和并发肾脏炎而死去。

还有,分别吞服了 $100$ 片 Brovarin 即 $10$ 克的两个十五岁女学生,走路时摇摇晃晃地被
发现,用救护车送进医院,五小时后恢复意识,现在已痊愈了。

\subsection{限制医药品}

对于有可能被滥用做为麻药,或被用来自杀的药物,厚生省将其规定为剧药、毒药、要指示
药、指定医药品等,不让一般人取得。在此要介绍的就是这类的「限制医药品」。

不容易取得并不等于弄不到,或许有人已经有了处方亦未可知,所以加以介绍。

\subsubsection{歇尔信、吉亚结巴姆、苏奈孔}

近来,一些精神科医生乱开处方。当你推开精神料的门扉,告诉医生说「睡不着」、「焦躁
不安」的,立即会开出「歇尔信」、「吉亚结巴姆」等抗焦虑剂的可能性很大。有时内科也
会开出此处方的。在精神科拿到抗焦虑的处方后,你就说「没作用」而要求调换药物。因为, 究竟哪一种对患者更适合,医生也需要试试看。

然而,正因为医生们首先会开出这类 Benzo Diazepam系药物,也说明它们的安全性很高。

即使你很幸运拿到这类处方,但因作用少而与致死量的距离太大,为了弄到致死量就不得不经常跑医院。

所开的处方,一天至少也不过是$15$毫克以内,所以至少要积存一个月的药量。而且,医生也曾担心所开的药物被累积而用于自杀。

正因为容易到手所以不适合于自杀。

有过估计吞服了$450$~$500$毫克的吉亚结巴姆的两个患者,在四十八小时都先后恢复了健康的例子。

「雪雷那敏」「克拉西那」等商品名虽异,但都属「吉亚结巴姆」的药物。有时你不明白医生开的是什么药的时候,可以参考「医生所开药物指南」一类书籍。 

\subsubsection{太妃拉尼尔}

「太妃拉尼尔」等抗忧郁症剂,对那些抗焦虑剂不起作用的,极度的忧郁、失眠等病是适合
的。比抗焦虑剂难弄到手,但你可向医生表明难以入睡、抗焦虑剂不起作用等,就可能得到
处方。甚至可以说明想自杀。如果真的陷入忧郁状态的话,有时会开朗,有时活泼,一旦药
物中断,即又回到了原状。

这是所列出的抗忧郁剂,其性质都相似,大量服用后会出现头痛、目眩、想睡觉。有时也会 引起精神错乱、幻觉,一般很快进入睡乡。

但手足的痉挛则是常见的中毒作用。虽认为不会 对内脏造成损害,但也有对肝脏造成障碍的例子。

如同时并用中枢神经抑制剂和酒类,效果 增强。

\subsubsection{因斯敏}

「因斯敏」虽不是抗焦虑剂和抗忧郁剂,但作为安眠药则较普遍。当然,如大量服用会形成兴奋状态,但因抑制中枢神经,很快就会入 睡,因呼吸停止而死亡。

同样地,如并用酒类和其它抑制中枢神经的药物即可加大其作用。

\subsubsection{Wintermin、抗特敏}

作用效果非常大。一般人只要服用一片,不久就会感到强烈 的睡意。

一个二十一岁的大学生在听课时服了一片「Wintermin」后竟然酣睡,当他醒过来 时下一堂课已开始。就是说,他对周围人的走动毫无察觉,足足睡了五个多小时。 「Wintermin」、「抗特敏」就是这种。

这种药是给症状严重的,不是神经症而是精神分裂症患者使用的,所以更难弄到手,不过致死度很高。这种药同样抑制中枢神经,死因是呼吸停止。

重要的是在睡意来临前须吞服够致死量的药物。

不过,也有人吞服$9.75$克后获救的例子。

中毒症状为:持续高热、意识障碍、呼吸困难、循环虚脱、脱水症状出现之后,结果因急性肾衰竭而死亡。

\subsubsection{巴比妥、伊苏米塔尔}

这些是吸了会死人的。其作用量和致死量之间的幅度极小,据说服用一点点就会入睡而死亡。

欧美安乐死协会也大力推荐该药。不过,百分之百地弄不到手的。因为年份已经很久了,目
前日本几乎都不使用,即使得到处方,这也只给症状极其严重的人使用。

不过,在海外则有 可能弄到,所以在此加以介绍。

这些都是属于巴尔比士酸(Barbituric)系的安眠药,大量服用即可使心脏和呼吸停止而死 去的极端危险药物。

中毒症状有头痛、痉挛、精神错乱等,如大量服用会一下子陷入昏迷。 如果能弄到手,则是没有比这更好的安乐自杀药了。

打算在海外购买的话,因商品名不通用,所以应记住一般名称。据说,荷兰的规定是比较松的。

不过,也有下述的例子。一个五十一岁的医生,估计吞服了$20$ 克的苯巴比妥普来(Phenobarbital)
的 粉 末 ,整整 四 天 陷 于 昏 迷 状 态 ,后来 通 过 人 工 透 析 回 复 了 知 觉 ,继 续 不 断 出 现 妄 想 、幻想 、 兴奋之后,经过一个月,精神上没留下任何障碍的出院了。

对这类几乎弄不到手的,又知道其致死量的药物,在本章末列表供作参考。

\subsubsection{麻药}

麻药是最接近死亡的药物。与其它的医药品相比,其致死量非常的少。

但是,推荐麻药做为自杀药物却多少有所顾忌。

本来我就不知道怎样弄到手。即使有人告诉我说「外国人常去的俱乐部就有」,可是,一般
还是无法知道是哪个俱乐部,又是谁会卖给我。

即使弄到手,但这又含有多少不纯物。再说,这又是哪一种麻药也完全不懂。

同 时 ,也因 为 身 体 是 否 习 惯 ,其中 毒 作 用 和 致 死 量 又 大 不 相 同 。

本 来 致 死 量 就 是 因 人 而 异 的 , 麻药的话,更为明显。慢性中毒者是不适用一般的致死量的(有的记录说是十倍),这种致 死法,除大量摄取之外别无他法了。这就是说,麻药是不适合有计划性的自杀者使用的。

正因为如此,下面所介绍的是,只是给那些能弄到手或已经在手的,而且知道其成分的人看的。

\subsubsection*{兴奋型}

曾经有个问题说:「是要停用兴奋剂,还是要了结人生?」,正确的回答应该是「使用兴奋
剂了结人生。」

第二次大战刚结束时,民间流通商品名叫「希罗朋」的掀起一场兴奋剂热的药 物,今天则称做「夏布」、「斯皮德」,被流氓以至于俱乐部青年人广泛使用,至今仍占日本麻药界的王者宝座的就是这种麻药。

\subsubsection*{安非他命}

安非他命(Amphetamine)主要在欧美使用,而美他非他命(Methamphetamine)(希罗朋)的发明者是日本人,后来又在全世界广为流传成为代表日本的麻药。不妨说一句,希罗朋在希腊文中是「爱好工作」的意思。

兴奋剂刺激中枢神经系统,特别对大脑刺激较大,提高集中力和活动欲望。

第一次在静脉注射致死量的美他非他命的时候,在数秒至数十秒内心跳急剧,自律神经出现异常,出汗、呼
吸深而加速、瞳孔放大,因循环器官衰竭而死亡。

有时会引起脑出血。如饮用则效果降低,所以必须在静脉里注射。

不过,已经习惯了的人,即使一天摄取$0.5$~$1$克也不会致死。一个二十七岁的女子服用
了$2$克美他非他命后陷入昏迷,瞳孔放大、面部肌肉剧烈抽搐而住进医院,十一天后痊愈出
院了,可以认为她是很禁得起该药物。相反地,也有仅一‧五毫克就死去的例子。

估计不会有人去尝试的,但要避免和「歇尔信」或「Wintermin」并用,因为会降低作用的。

\subsubsection*{古柯碱}

据1984年美国妇女杂志所作的调查,二十五岁以下的女性有三分之一使用过古柯碱。

这是 现 在 人 们 的 兴 趣 还 在 上 升 之 中 的 美 国 麻 药 。

目 前 尚 未 有 古 柯 碱 死 亡 的 报 告 ,致死 量 比 其 它 麻药高得多,较不适合自杀。

对中枢神经产生刺激作用,使呼吸深,其药理作用被认为与兴奋剂大致相同。

一般都是从鼻子吸入粉末,但静脉注射的效果大得多。据说一般急性中毒的话,使用之后三小时之内就会
死亡。濒死状态的意识,与幻觉剂以外的麻药相同,即使有些头痛但神志清楚,会相当冷静
地思考「这不太好」、「不能这个样子死去」。

常用者每天摄取$1.5$~$2$克也不至于死的。

古柯碱是古柯树树叶的主要成分,如想在海外购入,则应在原产地的哥伦比亚购买。

\subsubsection*{抑制型}

鸦片、吗啡、海洛因。一般这三种被认为是完全不相同的麻药,事实上是同祖同宗的。

鸦片是从罂栗未成熟果皮所得乳液干燥后制成粉末的,由可待因等二十多种植物盐基组成。

其中,含量最多的占$4\%$~$20\%$成为主成分的吗啡,再经化学处理而成强力的是海洛因。

当然其烈性强度顺序是海洛因、吗啡、鸦片,海洛因的强度被认为是吗啡的十倍。

中毒作用大致相同,如果能弄到手,还是海洛因的致死最可靠。

这些兴奋剂和古柯碱相反,能抑制中枢神经,造成恍惚的快感,被称为抑制系列禁药。

如过 量地吸入或注射时,数十秒内就出现目眩、恶心想吐、血压和体温逐渐下降,呼吸被抑制, 陷 入 昏 迷 满 六 至 十 二 小 时 就 因 呼 吸 停 止 而 死 亡 。

海 洛 因 引 起 的 肾 衰 竭 是 致 命 的 。

\subsubsection*{混合剂}

混合剂(Cocktail) 快速球(Speedball) 与其它麻药混合使用会形成相乘效果,致死度更高,如安眠药等,一般认为把抑制中枢神经 的药物混在一起服用,就比较容易致死,但在麻药世界,把抑制系列的古柯碱和兴奋系列的 海洛因混合后吸入或注射的通称「快速球」,特别具有危险性。 使 用 这 类 混 合 剂 ,就会 交 叉 出 现 中 枢 神 经 的 刺 激 作 用 和 抑 制 作 用 ,刚 出 现 心 脏 激 烈 跳 动 忽 又 感到心脏快要停止,这种现象数秒钟内反复呈现。死因是心脏停止,对自杀来说是绝妙的混 合剂。

此 外 ,刺激 系 列 的 兴 奋 剂 和 可 待 因 ,兴 奋 剂 和 LSD 的 混 合 剂 ,据说 其 刺 激 性 是 相 当 厉 害 的 。

\subsubsection*{LSD}

这 是 以 $20$~$250$ 微 克( 百 万 分 之 一 克 )的极 微 量 就 作 用 极 快 的 麻 药 ,致死 量 少 得 无 法 比 。

在本书中所列举的所有药物中,致死量是最低的。

那么是不是很快就能死呢?并非如此。平时市面上的LSD是把原来稀释数百倍浸到纸里,
这就需要大量服下这种纸片。

一般来说,产生幻觉作用基本上都是刺激性的麻药,大量使用会产生心悸亢进。同时,脑袋
形成慌乱状态,有时还会因此而跳出窗外致死。虽则如此,若是按一般的使用量使用,对人
体的危害要比其它麻药要少。

\subsection{医药品以外}

这里介绍的是在你身边到处都有的「毒」。

当然有不少人会有一种愿望,哪怕不能安乐死,总之想马上死掉。

为了这些人,本书也介绍
了绝不是安乐死法的「撞车自杀」和「自焚自杀」的方法。关于药物也是同样的,即使不是
睡眠而死,只要经历剎那间的痛苦就死也行。

对于打算了结一生的你,去弄那些不易弄到手
的药物是太麻烦了。如果明天你打算去上班或上学的话,那就吞服本书所列这些药物好了。

在这里大体上也叙述了药理作用,但某中毒专家却认为「这类东西没有任何药理作用」。

就是说,大部分人吞服后立即会「哇--」地作声、喉咙、胃发生溃烂,经由痛苦而死。

吞服方法,除例外地写明含有量之外,并不一一说明。也不是一小时用餐,混在酸乳酪里的
一类的做法。

但 是 必 须 知 道 致 死 量 。

这类 产 品 的 有 害 成 分 含 有 量 是 各 不 相 同 的 ,虽 不 注 明 服 用 量 但 想 知 道 各商品的成分量,可打电话询问出售处。

从致死量计算,不要吞服过少的量。

先举出一些看上去可能致死,实际上毒性很差不宜自杀的,有干燥剂硅胶(Silica
gel)、合成洗涤剂、洗发乳、头发润丝精、发油(Pomade)、墨水、保鲜膜、蚊香、捕蟑
垫、防臭剂、脱臭剂、对二氯苯(Para-dichloro-benzene)系列的防虫剂等、家庭用的漂
白剂、作为干燥剂使用的生石灰。香豆素(Coumarin)系列、杀鼠剂等也都因毒性差是不能
用的。此外,理所当然地口红、牙膏、沐浴剂等有「由口入体」可能性的都是安全的。牙膏
把一管全吞下去也是无害的,即使你有股冲动想自杀的时候也绝不可用。不论吞服多少,只
不过感到难受罢了。

\subsubsection{香烟}

香烟中所含的尼古丁毒性比一般想象的要厉害得多。

一个婴儿,有时一两支就可能致死。

专业书籍写道「与氰酸相匹敌的毒物」。

现在出售的香烟,短支HOPE含 1.6 毫克、HI-LITE含 1.6 毫克,短支PEACE含 2.7
毫克的尼古丁。比雪茄烟的含有量更多,不过,吸烟时因尼古丁燃烧而没有效用。当然也可以把烟叶吃掉,但把烟叶浸在水中溶出尼古丁,吸收快而效果更好。

烟叶浸水后经一小时就会溶出 $50\%$~$70\%$ 的尼古丁。溶出$50\%$,打算摄取$60$毫克的尼古丁时,可用短支 PEACE 四十四支浸在水里,一小时后饮用溶液即可,放在锅内煮沸,可溶出近$70\%$~$100\%$,时间亦短。而一百度的热量是不会破坏尼古丁的,放在酒精里溶出更快。香烟的溶液非常之苦,加些砂糖也不错。

尼古丁在开始刺激中枢神经如运动神经时,使其兴奋,随后产生抑制作用。服用后喉咙立即 感到被灼烧似的疼痛、恶心、反胃、伴之以头痛,不久知觉麻痹、神志不清。因为呼吸停止 而死。

在服用了致死量后,在没有兴奋状态下出现麻痹,陷入虚脱状态,剎那间就会断气。

最大问题是尼古丁具有强烈的恶心作用,最好与防止恶心的药物一起服用。

当然,若将这种溶液注射到静脉内,则效果更强,而且不必担心会呕吐。

\subsubsection{杀虫剂}

Naphtalin、樟脑、杀蟑剂、杀虫剂能杀虫的药物当然对人体也会有害。

中毒症状有头痛、呕吐、错乱、倦怠感等,严重时 陷入昏睡。肝脏、肾脏也会受伤。

现在的防虫剂以比较安全的对二氯苯(Para-dichlorobenzene)为主。

杀蟑剂中「大地杀蟑硼酸球」、「蟑螂天虫杀」等都含有害成分硼酸。硼酸抑制中枢神经,
大量服用时会出现头痛、呕吐、无力、嗜睡、昏睡等症状,因循环不全而死。对肾脏、肝脏
也 会 引 起 障 碍 ,担 心 产 生 后 遗 症 的 要 防 止 自 杀 未 遂 就 应 大 量 服 用 ,还有 漂 白 剂 等 也 含 有 硼 酸 。

杀虫剂等商品很危险。这些商品所含有害成分「甲酚(Cresol)」只有$1\%$~$10\%$,所以不服用相当的量是达不到致死量的,所含$40\%$~$70\%$成分O-二氯苯(O-dichlorobenzene)的毒性很强,所以药物本身的毒性是相当大的。

甲酚(Cresol)会引起粘膜腐蚀、血管收缩等,O-二氯苯(O-dichlorobenzene)有抑制
中枢神经、刺激粘膜等作用。两者均对肝、肾造成危害。服用后会导致上部消化器官疼痛、
知觉障碍、血压降低、循环不全而死亡。

\subsubsection{有机溶剂}

煤油、汽油、苯(Benzene)、稀释剂(Thinner)等。

把煤油、汽油等浇在身上点上火也是可以的,但服用的话,可以少量而没有多大痛苦即可死
去。这两种服用后都会刺激粘膜,抑制中枢神经。出现恶心、睡意、胸部灼热感、错乱等,
最后会因呼吸停止而致死。  

有时也会有心室颤动而突然死亡的情况。不过,曾有喝了$250$
毫升汽油而复原的例子。

苯可在药房随时买得到。最低致死量是$10$毫升,相当于玻璃杯的五分之一的量,但至少应该
服用 $100$ 亳升。但是这也不能算是很大的量。

服用后中枢神经被抑制,全身成麻醉状态。其症状,首先是头痛、目眩,及短时间地出现陶醉感并陷于昏睡,因呼吸衰竭或突然丰心室颤动而死亡。

中毒症状并不严重,最适合于自杀。

吸入苯的挥发成分比服用的毒性还大。

但因还没弄清吸入的致死量,所以不作介绍。

稀释剂本来是「稀释涂料的液体」,多半是甲苯(Toluene)和甲醇(Methyl alcohol)的调合物。大多托别人弄到手,成分的含有量不固定,所以不适于有计划的自杀,
但毒性很大。服用后嘴巴、食道、胃有灼热感,抑制中枢神经,损害造血机能,导致死亡。
稀释剂的使用是以吸的方法容易死去。 $1$~$3$ ppm 就能立即陷于麻醉状态。在洗面盆里倒满香蕉水,盖上被子,就能失去知觉,缺氧而死。

这是医药物以外的唯一安乐自杀方法。

使用时应该使用纯度极高的稀释剂。

当然,在利用其它方法自杀之前,为加快速度而先吸几口也是不错的开始。

\subsubsection{家庭用品}

除莠剂「清洁锈」、「最佳清洁帮手」(Best Clean)等商品含有约 $40\%$ 的浓度和 $89\%$ 的
磷酸。除莠剂的致死量为 $22.5$ 毫升。

如服了如此多量,与其说是磷酸本身的毒性起作用不如说嘴巴、食道、胃等产生组织
损坏导致急性死亡。

消毒剂的毒性大,在药房亦可任意买到,是绝好的自杀药。

在医药物以外,恐怕没有超过它的。

有位药剂师还说:「这种东西最能简便地致死。因为能大口地喝下去」。

碘能腐蚀消化器官,引起头痛、神精错乱和休克,导致昏睡、死亡。甲酚,一般认为一个体
重 $60$ 公斤的成人推测致死量为 $180$ 毫克,服用可使中枢神经兴奋,之后显示麻醉作用,导致知觉障碍和痉挛,从神志不清、呼
吸 麻 痹 到 心 跳 停 止 。快 则 五 分 钟 内 ,迟 则 三 十 分 钟 以 内 即 可 丧 失 知 觉 。

对 肾 脏 、肝脏 有 危 害 , 摄 取 后 的 二 十 四 小 时 是 关 键 ,喝了 约 $80$ 毫 克 的 甲 酚 陷 入 昏 迷 状 态 的 十 四 岁 国 中 生 ,曾 一 度 恢复了知觉,但肝脏、肾脏损坏,第三十六天引起了空气滞留肺膜的「气胸」,因心跳停止 而死亡。

\subsubsection{化妆品}

指甲去光油、染发剂、烫发液
指甲去光油是化妆品中最危险的产品。含有丙酮(Acetone)$20\%$~$25\%$,醋酸戊酯(Amyl
acetate)$30\%$~$50\%$,有麻醉性,大量服用会刺激粘膜,经历头痛、兴奋状态、疲劳后至昏迷。
其含有量因产品而各有不同,以此表作参考而确认致死量。如自杀未遂,有可能损伤肝脏和
肾脏。

指甲去光油(Nail Lacquer)也含有丙酮 $30\%$,醋酸戊酯 $30\%$,喝上 $500$~$750$
毫克即可达致死量。有严重的恶心,但只要吸收就有头痛、兴奋状态直至昏迷。

由于毒性不算大,缺点是必须大量地饮用。

染发剂的成分中的对苯二胺(P-phenylenediamine)使用于将头发染黑,毒性极大。

「比根彩发」,「漂王」(山发产业)、「维拉通」,( 化妆品)等都含有约
$2\%$ 的对苯二胺。粉末的含有量更高,「漂王」粉末剂(一盒中六克)含有 $6\%$;只喝
十四克就够了。当然,在饮用的时候,先溶化在水中以提高其吸收率较好。

大 量 服 用 会 引 起 急 性 肝 障 碍 ,从 循 环 衰 竭 到 呼 吸 困 难 直 至 死 亡 。

烫 发 液 的 第 二 液 中 和 剂 含 有 $2\%$~$6\%$ 的溴酸钾(Potassiumbromate)。

溴酸钾刺激中枢神经和胃的组织,有恶心、呕吐和胃部灼热感,陷于昏迷状态。有时会造成听觉障碍,被认为毒性很高。

不过,由于含有量较少,至少要$80$克,为了万无一失就必须喝上一公斤的烫发液。

因此我不打算推荐。

\subsubsection{其他}

含有 Paraquat\footnote{剧毒除草剂“百草枯”,对人和动物有剧毒。急性毒性虽低,但会造成进行性肺纤维化,最终导致呼吸衰竭死亡。大鼠口服LD50剂量为100mg/kg,人摄入3毫升即可致死,中毒死亡率通常在45%~90%,其中口服中毒死亡率高达90%~100%。} 的除草剂,一般人虽然无法轻易买到,但农村仍在广泛使用,所以还是可以弄到手。这是剧毒药,目前仍不清楚解毒方法。

Paraquat 和吉克华特具有相似的结构和性质,两者都与体内的酵素产生及使细胞膜的 脂质变质。

饮用后出现厉害的呕吐、口、食道、消化器官腐烂危害肝脏、肾脏、循环器官、 肺等终至死亡。特点是,内脏虽已损坏但知觉却清楚,所以痛苦较大(案例 3)。

除草剂的「葛拉莫奇松」含有 Paraquat $24\%$ 。

普利葛罗克斯L、「麦节特」含有 Paraquat $5\%$,吉克华特 $7\%$,「雷葛罗克丝」则含 $30\%$ 的「吉克华特」。

「雷葛罗克丝」的致死量为 $20$~$40$毫升。除此之外,含有两成分的除草剂还有很多。

有案例是一名四十岁的妇女,喝下约$10$毫升的 Paraquat 却自杀未遂。估计喝了「葛拉莫
奇松」除草剂约 $100$毫升的二十八岁的男子,在十四天后导致肺障碍而死亡。

同样,喝了$250$毫克「葛拉莫奇松」的五十岁妇女在三十小时后因血液循环衰竭而死亡 。

喝了「雷 葛 罗 克 丝 」 约$100$毫克的六十七岁男子,六小时后因呼吸停止而死去。
总之,想要早点死,就必须大量地喝下去。

数年前,因用于杀人事件而引起人们注意而且大为畅销的乌头所具有的乌头碱(Aconitine),
在植物性毒中以毒性大而闻名的,根部的含有量特别多,刺激中枢神经系统和末梢神经、心
脏等,尤其是破坏体髓使呼吸肌产生麻痹,约两小时就能使呼吸机能停止。

但是,普通花店所出售的是毒性较差的花乌头。乌头主要生长在日本本州的山林中,可采取。

事实上,有人误以为是野菜食用而引起中毒的。

恐怕没有吃盐去自杀的人吧。

可是,盐也是会致死的,因为很有意思,这里就介绍一下。

打算用盐自杀就必须喝下 $300$ 克(一茶碗)不可。

曾有喝下一升(约 $1.8$ 公升)酱油而死掉的人,就是因为这些酱油中含有约百分之十四食盐的缘故。

食盐中毒的症状,有目眩、错乱、呼吸急促、发热、无力等,大量摄取时循环系统、肝脏、
肾脏等引起障碍,会成为致命伤。

看到这里,你或许会感到纳闷,为何没有出现常听到的氰酸钾和河豚毒的名字。

这是因为,目前的管理、废弃的一系列规定极为严格,实在难以弄到手,因此即使列出来也是毫无意义
就未列于此。

河豚的卵巢、肝脏含有剧毒,但因为鱼体本身和季节差别,毒性也会有一定的差距,就算是 弄到手也不一定适合于自杀。关于这些,作为将难以到手的医药物及致死量列于本章之末。 如果能弄到手,倒是希望大家利用。

\subsection{案例研究 3}
\subsubsection*{喝下Paraquat八天后死亡的少年}

1985 年10月在群马县,外出回家的家属发现中学二年级的少年(当时十四岁),在房间
内捂着肚子乱打转。那天早晨和平时一样出了门,在偏离上学的山林里喝了农药
Paraquat 约$40$~$50$毫升后又回到了家。

在山林里留下有上学用的自行车、书包之外,还有草草写完的遗书、呕吐的痕迹、药瓶等。
被送往医院时,他的嘴、喉已经溃烂,每当叫喊时口中就流血。之后虽然相当痛苦,仍在说
「我想早点吃饭」、「我想看漫画」、「爸爸、妈妈谢谢啦」等,过些日子开始说些产生幻
想的话「我的巴士开走啦」、「UFO来啦」等等。

有时也会大吵大闹,需四个大人制服,但八天后还是死了。

遗书中写有同班的三个学生的名字和「到了天国我也一直恨你们」等字句。他是篮球队的成员,大约从自杀的前三个月起,在练习中经常被其它人责骂说:「你在磨蹭什么!」、「别 偷懒!」等,并被殴打、脚踢了胸部和腹部,十月间退出后也在下课后在体育馆后面被殴打, 自杀的原因就是来自以上的欺负。

他 是 个 计 算 机 迷 ,研 读 有 关 计 算 机 的 书 籍 ,还喜 欢 动 画 片 ,房 间 里 挂 着 高 桥 留 美 子 原 作《 星 》\footnote{《福星小子》(うる星やつら),是日本著名漫画家高桥留美子的第一部长篇连载作品,也是受欢迎的代表作之一。}的大幅剧照画。

\subsubsection*{检验死因}

在案物自杀项目中一直使用了「恶心」、「粘膜腐蚀」、「错乱」等字眼,事实上就是这样
的症状。

Paraquat 致死量为十五毫克,该少年喝下的农药究竟含有多少比例的 Paraquat
不得而知,加上立即呕吐,所以实际吸收到体内的量大概恰好达到致死量。

他因为曾是运动员,具有相当好的体力所以维持了八天,但一般来说,服毒自杀的方式,在病床上过了好几
天之后才死去的例子并不少。

这一事例是由于被欺负而造成的少年自杀,像他所属的严格训练、集体竞赛的运动,所以有 几 个 家 伙 欺 负 人 也 可 说 是 理 所 当 然 的 。

究 竟 施 加 到 了 什 么 程 度 的 暴 力 虽 不 得 而 知 ,但 也 是 常 见的事。

事实上,不仅限于俱乐部的活动,这个世上就是这么回事。

\begin{table}[]
\small
\begin{tabular}{lll}
成分通称              & 致死量                    & 中毒反应       \\
Meprobamate       & $15$~$20$g & 倦怠感        \\
Pethidine         & $3$g~$6$g                 & 倾眠、昏睡      \\
Aminophylline     & 推断 $5$~$30$g           & 麻痹、昏睡      \\
Acetaminophen     & $0.2$~$1$g/kg          & 目眩、四肢无力    \\
Mephobarbital     & 2g                     & 腹痛、嗜眠、意识不清 \\
Metharbital       & 2g                     & 错乱、休克、心跳停止 \\
Neostigmine       & 约 $60$mg               & 目眩、四肢无力    \\
Ammonium          & $60$mg                 & 目眩、头痛      \\
Chloride          & $300$mg                & 多梦         \\
Digitoxin         & 最少 $3$mg               & 感觉迟钝       \\
Digoxin           & $10$~$20$mg            & 嗜眠、幻觉      \\
Phenol            & $8.5$~$60$mg           & 痉挛、昏睡      \\
Isoniazide        & 推断最少 $3$g              & 心室颤动、肝障碍   \\
Hydrogen Cyanide  & $500$mg                & 意识障碍       \\
Potassium Cyanide & $150$~$200$mg          & 呼吸困难       \\
Sodium Cyanide    & $200$~$300$mg          & 心跳停止       \\
Tetrodotoxin      & 约 $2$mg                & 全身运动障碍     \\
Pyrethrin         & 推断 $1$~$2$g/kg         & 恶心、头晕      \\
Chlordane         & 推断 $10$g               & 痉挛、呼吸困难   
\end{tabular}
\end{table}

\newpage

\section{Hanging 上吊}

痛苦 ▼▽▽▽▽

麻烦 ▼▼▽▽▽

死状 ▼▼▼▽▽

牵连 ▼▽▽▽▽

冲击 ▼▼▽▽▽

致命 ▼▼▼▼▼

确实、简单、无痛苦三部曲,不分男女老少压倒性的广受喜爱,堪称为「自杀之王」。

好像我是在说毫无依据的结论似的,但确实没有比「上吊」更能安然地、可靠地而且简便地进行自杀。完全没有必要考虑其它任何方法。

可 能 你 不 大 相 信 ,但 经 过 仔 细 调 查 还 是 没 找 到 比 上 吊 更 好 的 方 法 。

下 面 还 将 详 细 叙 述 上 吊 为 何优于其它方法,甚至可以说上吊是人类所想出来的艺术品。

正因为如此,日本每年的自杀 者中一半都选择了这种方法,而且不论古今中外都被广泛采用。 上吊的最大优点就是「未遂率」极低。只要绳子不断,套绳子的树枝不折断,而且在上吊后 十几分钟内不被发现,可以说成功率是百分之百的。

有那么一个人,服毒后还剖了腹但没死, 在轨道上等电车也没等到,没法子了去跳崖也没死成,最终在断崖的树上终于吊死了。

「想自杀就上吊」,打算自杀的人应该牢记这一点。

\subsection{准备}

简便得只需一根绳子准备的东西只需一根绳子就可以了。

电线、皮带、绳子,只要能缠到脖子上,什么都可以。

不过,要尽量挑选柔软而能贴在脖颈上的。

如百货公司包装用的塑料带等,即使对一百多公斤的人来说也是很足够的了。

如果使用钢丝等一类具有切断力的东西,就会发生割断头颈的情况。有人在桥栏杆上拴住了
车辆牵引用的钢绳一端,又把另一端打结套在脖子上,之后跳到河里,结果头部被割下流到
河里,成为「怪死事件」猛闹了一阵子。也有同样地将车辆牵引用钢索栓在树上,另一端套
在脖子上后发动了车子,把头割下来自杀的一位很有勇气的公司职员。

坐着也可以套绳子,只要不是容易折断的细枝,那里都可以。也有中学一年级的少年在书架上钉了一根
五寸钉上吊的例子。但是,找不到合适的套绳子地方也是常有的,特别是医院和看守所。

不过,上吊并不是非要把绳子套在比自己身长还要高的地方才能进行。

虽然脚部或臀部碰到地板,也是能死的。

在病床上上吊的人并不少,这从理论上也是可能的。

在说明这一理论之前,先说说勒颈的死因和吊颈的死因不同之处。

前者是以气管被堵塞造成的窒息为多,后者以输往脑部的血液被堵断所造成的脑内缺氧状态而死的占大多数。

向脑部供血的动脉有两种:颈动脉和在脊椎旁边被骨头保护着的椎动脉。勒脖子的方式,即使颈动脉被堵住,但被骨头保护的椎骨动脉是堵不住的。

但上吊的方式,脖子被斜上方吊起形成了角度,因此内外同时堵死,向脑部的供血剎那间就停止了。两者的差异从 尸 体 上 即 可 明 了 。

勒 颈 的 尸 体 由 椎 骨 动 脉 向 脑 部 的 供 血 虽 继 续 不 断 ,但 相 反 地 由 脑 部 输 送 血液的颈动脉却被堵塞了,因此变成紫色而瘀血;上吊的方式则不见瘀血。

现在弄清一些基本问题后,再回到上吊的高度问题。

上吊的时候,血压在$170$毫米汞柱的人,那么他的颈动脉用$3.5$公斤,椎骨动脉用$16.6$公斤的力量即可堵塞。

如果脚部着地而膝盖弯曲的程度,有全身体重的$70\%$~$90\%$, (膝盖着地也有体重的$20\%$的重量)压到脖子上。

譬如说,当体重$60$公斤的人膝盖 着地上吊的方式,压在脖子上的力量为$12$公斤,颈动脉也自然地被堵住了。

此时,对椎骨 动 脉 而 言 还 没 达 到 完 全 堵 塞 的 地 步 ,所以 有 微 量 的 血 液 流 向 脑 部 ,但 这 也 只 不 过 是 时 间 的 问 题,比单纯地勒颈还要好上几倍。

丧失知觉可能稍为晚点,但不会有以未遂而告终的事。

也就是说,压在颈部的力量哪怕只有体重的$20\%$,只要颈部形成角度,不完全把身体吊 起来也能简便地死亡。

事实上还有臀部和背部着地的例子,也有人说只要有$30$公分的高度就可以死的。

在欧美,甚至有人说脚部着地的情况较多。

打算在自己的房间死去的你,没有必要觉得高度不够而放弃。

房门的把手也足够了,利用楼梯斜面也是个办法。

其 它 应 注 意 的 是 ,固 然 要 找 不 易 被 人 发 现 的 地 方 ,但 不 像 采 用 煤 气 或 药 物 的 自 杀 需 要 数 小 时 或数日不被发现的地方才行。

仅仅在十来分钟之内不被发现就救不活了,如果有几十分钟更 是大功告成。

不 想 让 亲 属 和 朋 友 们 看 到 自 己 尸 体 的 话 ,最好 选 择 离 开 自 己 家 及 平 日 上 班 、上 学 路 途 稍 远 的 地方。

当身份被查明,双亲知道了的时候,你已经在医院或警察局里了。

\subsection{经过}

一瞬间丧失意识,没有痛苦脖子上套住绳子,踢掉脚凳子后悬在半空中时,你的知觉会是怎样的呢?
据法医学者的研究,一上吊知觉便朦胧,手脚想动也动弹不了,而且在此一过程中是完全没有痛苦的,这在医学界已经是常识了。

在东京,一个演员当着观众表演上吊,当他说「像这样蹲下腰……」的瞬时便失去知觉,并死在观众眼前。

有个法医学者想体验一下上吊的痛苦,把脚凳子放在随时可站立上去的位置,并请同事们在场,双脚刚离开脚凳子时便突然失神,幸亏被同事们救了下来,这种体验的例子并不少。

一个用电线上吊而被救下的自杀未遂者说:「脑袋嗡地一下什么都不知道,知觉也没有了,甚至连无法呼吸而难过或者疼痛的感觉都没有。」

并非只是痛苦,也有相当舒畅之说。

例如,在柔道中被使了勒技而摔得不省人事的快感,在浴室里用毛巾玩勒脖子游戏的少年就这样缢死的事例。

同时,如果一下子用力过猛吊住的话,颈关节会脱臼,立即导致心跳停止和呼吸停止。

提到上吊,许多人会认为是喉咙被扼住,在痛苦中窒息而死。

事实上在感到喘不过气之前,已引起脑部缺氧而失去知觉,所以是没有痛苦的。

假如怀疑这种说法,那你本人不妨用绳子套在脖子上吊吊看,是否真的在剎那间失去知觉。

大 概 你 刚 套 上 绳 子 、压 上 一 点 体 重 ,就会 因 为 觉 得 比 想 象 中 要 紧 张 而 害 怕 而 立 刻 停 止 试 验 吧 ! 十来分钟心脏停止也有学者进一步分成几个阶段而详细进行研究。

首先是第一阶段,开始是头脑嗡地发热,出现耳鸣。接着感到眼前有闪光,头发热、耳鸣、眼前闪光是刚上吊就立刻出现的,同时,知觉开始模糊起来。

进入失去知觉后的第二阶段,全身引起痉挛。

据说,手部作划水动作,脚部作走路的动作,然 后 双 手 双 脚 的 肌 肉 开 始 抽 筋 后 又 全 身 挺 直 抽 筋 。

但是 对 这 种 痉 挛 ,现 在 认 为 那 是 全 身 小 颤 动。

这一阶段是一分到一分半钟。

令人不解的是,男性在此时性器会勃起,并射精。 第三阶段,已经是假死状态,大小便和精液溢出、眼球突起、呼吸停止。

这期间约一分钟, 所以到达这种地步只需三分到三分半钟。

在此阶段心脏仍在跳动,如被发现也有可能救下一 命。

此后,心脏会跳动约十分钟,心脏一旦停止跳动就没有获救的希望了。

有一个美国杂技演员在观众面前模仿上吊,平日的安排是当知觉开始模糊时马上给助手递暗号。

有一天大概突然失去知觉没能及时递出暗号,就在那里吊了十三分钟,送到医院时已经 回生乏术。

这就是说,在短短的十来分钟里,未被人发现的话,你的自杀就成功了而且没有痛苦。

大概没有比此更适合的自杀方法!

\subsection{尸体}
 
人们常提到的上吊缺点,就是尸体的形象不好看。的确会有失禁和射精的情况。

有人说,由于喉咙被朝上扼住,舌头会伸出,面部因淤血而发紫,眼球会突出等。不过,到这种地步的可怕例子是不多的。

死后好几天眼球才会突出,舌头碰到牙齿也不会伸出的。

就是说,尸体的情况只不过是这样。

从上吊尸体的照片上看,大部分只不过是「耸拉」在那里,和活的时候并没有什么不同。

如前所述,由于供给头部的血液很快被切断,脸部一般都不会出现淤血。

虽说看上去的形象不大好,但和跳楼、撞车的尸体相比,可以说是体面的尸体了。

要想防止失禁,事先可去趟洗手间,不想射精,那就先做手淫。

为了追随昭和天皇,在驾崩 当天自杀的八十七岁老人,口衔纱布,再戴口罩,在橘子园里上了吊。

他或许是为了不想死 得太难看吧。

\subsection{注意:勒颈很痛苦}

前面已说过,上吊的优点在于因从斜上方拉住脖子的姿势,造成勒住椎骨动脉。手扼脖子的场合,当然遮断不了该动脉,因而脑部供血不可能一下子失去知觉。

上吊的话如前述第一阶段就失去知觉,扼脖子的场合在有知觉的情况下体验第一、第二和第三阶段,随后才好不容
易地失去知觉。

结果是气管被堵塞而变成窒息死亡,加上体验抽筋,伴之而来的则是相当的痛苦。

虽则如此,自己扼住脖子窒息自杀倒是很多。我忍不住要向他表示同情心,为什么不肯花点工夫制造三十公分的高度呢!?
四十二岁的大学副教授,用领带一点一点地勒紧,最后口中流血而死。

参 加 过 东 京 奥 林 匹 克 游 泳 比 赛 ,以后 就 过 着 平 庸 的 主 妇 生 活 的 四 十 三 岁 妇 女 ,充 分 利 用 了 过 人的肺活量,在嘴、鼻、颈上缠贴多层胶布窒息而死。

因杀死儿子被关起来的四十九岁家庭主妇,在看守所里在胃和鼻子里塞足了卫生纸而窒息死
亡。

也有在警察局监护室内,吞下约$100$ 克(一卷)卫生纸,后因支气管闭塞而自杀的四十六岁
建筑工人。

当然,用这种窒息的办法也是可以死的,但伴随而来相当的痛苦,我是不愿意推荐的。
关于上吊,还要提及的是:因脑内部形成缺氧状态,细胞被破坏,所以自杀未遂的时候会留
下严重的脑后遗症。脑细胞的特征与其它细胞不同,被破坏了就不会再生的。因此,有必要
做不被发现的周密安排。

\subsection{案例研究 4}
\subsubsection*{上吊未遂的悲剧:法兰克永井}

1985 年 10 月 ,歌 手 法 兰 克 永 井\footnote{法兰克永井(フランク永井,1932年3月18日-2008年10月27日),是日本的情境歌谣(ムード歌谣)歌手,本名永井清人,宫城县志田郡松山町(今大崎市)人。以独特的低音吸引了许多人,在歌谣界留下伟大的足迹。} ( 当 时 五 十 三 岁 )在晚 上 十 点 钟 左 右 演 出 结 束 回 到 家 里 , 和妻子喝了一点啤酒,十点半左右睡觉。

可是到了半夜,走出卧室后长时间没有回来,妻子出来寻找时,发现在一楼和二楼中间的螺旋楼梯的扶手上,套上四五根领带上吊了,那是半夜三点钟左右。

救护车赶到时,永井氏没有了呼吸和脉搏,昏迷倒在那里。送到医院时已经瞳孔放大,处于「即将脑死」的状态。
可是经过四五天后却奇迹般地恢复知觉。

一个月后会说几句话,给他看《恋君》曲名时,他还认得出「君」字。后来又恢复到会写「永井」两个字。

后来他又多次入院出院,现在已能做些散步,也能用卡拉OK唱熟悉的歌曲,显示了进一步的恢复。

尽管如此,他还不能辨认前去探病的妻子和朋友,有点近似老年痴呆症的状态,1989 年又成为禁治产者。

他把这次的体验讲给友人丹波哲郎氏时,是这样说的: 「刚上吊就呼吸困难,视野瞬时变成红色后又成为漆黑一团。

看到在空中自己的脸歪了,逐渐听见奇妙的声音。声音逐渐变大,消失在黑暗的隧道里。

自己忽然上升,浮游着穿过墙壁 和 门 ,可以 看 到 下 界 的 情 况 。突 然 注 意 到 自 己 站 在 平 地 上 ,从前 面 的 花 园 听 到 了 优 美 的 音 乐 , 还听到了已故的友人和亲属的声音,被怀念和相会的心情所驱使向前走了过去。那里有条三岔的河流,但一股不知什么力量把我拉了回来。」 

永井氏之所以企图自杀,是因为从前的情人提出她所生的孩子是他们俩人的,要求抚养费而苦恼的缘故。

\subsubsection*{检验本例}

极为轰动的上吊未遂案例,恐怕也仅此一例。

即使被认为既遂率是百分百的上吊,要是很快就被人发现也就不会成功。

上述例子若从停止脉搏来看,可能已上吊了十分钟以上。但有的专家认为「(从恢复到现在
的地步来看)被发现的时间是刚刚上吊三至四分钟,脚部和臀部应该是着地的。

不过,即使性命得救,被损坏的脑神经,不会再恢复。
它将留下怎样的后遗症,永井氏的情况已经说明。
脑子的神经细胞,自血液不再循环之时的瞬间就开始损坏。
这是非常可怕的,上吊未遂是可怕的!

据其本人所说,在剎那间(通常是水平的)失去知觉,视野通红后又变成漆黑。

以后的奇妙的体验,可称为「濒死体验」,据说许多濒临死亡的人都经历过。

\subsection{案例研究 5}
\subsubsection*{高度 $91$ 公分的上吊}

联合赤军最高干部森恒夫1973年1月1日,在东京拘留所的森恒夫(当时二十八岁)于单人牢房里上吊自杀了。

其本人是作为1972年1月组成的武装革命集团。
他的「唯枪主义」提倡只有依靠武器革命才能成功,在群马县山岳地下司令部进行武装训练,1972年2月被捕。他供认了自己以十二名「赤军战士」、「行动不是革命的」为理由,而进行杀害的事实,九月被移送东京拘留所,等待开庭审理。这个私刑拷打杀害了十二个人的事件,比起逃亡的赤军士兵占据浅 间山庄事件更令人感到「闭塞集团中的疯狂」,一时震惊了日本社会。 在第二年的1月1日,和平时一样森恒夫把早饭和午饭都吃光了。

下午1时38分,巡逻 的看守还看见他坐在塌塌米上看书。
但在$14$分钟后的1时52分,看守再次探视单人牢房时,他在高$91$公分的铁栏杆上用$60$公分长的毛巾打了结,把头套进去上吊了。

医生赶来打了强心针、做了人工呼吸,都无济于事,约一小时后断了气。
据说,他用身边的 衣衫绑住了双脚吊在那里,也有人说死去时的形状好像是「将要坐下来的样子」。

对这一自杀的分析是,当时自杀时「将绑住的双脚用力向空中踢出,于是 重力加到毛巾上勒住了脖子。」

发现时虽已失去知觉,但静脉则与活的时候一样没有变化,也没发现出血和失禁。 
他 留 下 了 两 封 遗 书 ,据说 内 容 是 承 认 了 唯 武 器 主 义 的 错 误 。
又 听 说 他 从 自 杀 的 前 几 个 月 起 就 爱读圣经,对基督教起了很大的兴趣。

\subsubsection*{检验死因}

他是身高$163$公分,体重$54$公斤的小个子,使用的毛巾为$60$公分,也很短。在栏杆打结套上脖子,那么脖子就紧紧靠在栏杆上。

事情的巧合是,他的腰部特长,加上毛巾收紧勒住脖子,他的臀部可能还不曾着地的。

就是说,一个体重$54$公斤的男子脚着地,背靠着墙上吊的方式,在十四分钟内是不可能苏醒的,这是证明上吊的简便和短时间可断气的有力证据。

话虽如此,为了革命而进行的不懈努力和磨难,最后都放弃坚持很久的信念,且又归依了神明,然而毕竟没能获救而自杀,该是多么可怜的事。

不 过 ,虽 说 是 在 二 十 年 前 的 事 ,相 信 也 能 引 起 革 命 本 身 就 远 比 对 十 二 个 同 伙 人 私 刑 杀 死 的 事 要疯狂得多。

\subsection{案例研究 6}
\subsubsection*{在精神病医院上吊获救的妇女}

1982 年 5 月 17 日 下 午 4 时 45 分,在某 精 神 病 医 院 ,一 个 住 院 中 的 妇 女 陷 入 兴 奋 状 态大喊「把我杀了吧!」于是被送进保护室。

$40$分钟后的5时25分,爬上用好几条被 子叠上的窗格后用毛巾上吊了。

发现时呼吸、心脏都已停止,被救了下来进行了人工呼吸, 用了强心剂,五分钟以后出现了自发性呼吸和微弱的脉搏。 尽管如此,整整两天没有知觉,第三天好不容易对刺激有了反应。

第四天起恢复了知觉,一 周后能进食流汁,第二周起进行了「恢复自我训练」,再过两周后能和其它病患一起过集体 生活了。

但记忆力却减退,丧失了从前的积极性,总是躲在人后,对看护者的依赖也多了。

这 个 妇 女 在 她 二 十 六 岁 后 ,在 这 个 医 院 里 住 了 十 三 年 ,在 此 以 前 还 在 其 它 精 神 病 医 院 住 过 三 次。

1979年出院后到附近的综合医院工作,因与单位的男员工产生摩擦而放弃了工作, 并与另外的男子同居,不久开始不吃东西,大喊「杀死我吧!」,过了两年又住院了。
上吊 是在半年以后的事,这是第三次的自杀未遂。

她的经历是,中学毕业在毛巾工厂工作六年,二十三岁结婚,第二年离婚,生了个男孩但不 久小孩死去。

还有,她姊妹五人中包括她在内的四个人都分别有自杀未遂、企图自杀的行为。

\subsubsection*{检验本例}

就像护士们所说的:「是少数因发现的同时医疗人员迅速处置,起死回生的稀有例子」;也是虽然心脏一度停止,但未到死亡的极少见例子。

从空白的$40$分钟中,减掉堆放被子的时间,大概是上吊$15$分钟后,心脏刚刚停止的时候被发现。

虽然有$40$分钟和足够的高度但未能致死,这和森恒夫的例子形成了鲜明的对比。在医院或 牢房因有监视所以及时被发现,自杀是困难的,尤其是医院因急救设施齐全,更加困难了。 从 这 些 未 遂 例 来 看 ,或许 会 想 上 吊 也 并 非 是 那 么 简 单 的 。

但是 ,毕 竟 这 些 都 是 极 稀 有 的 例 子 。 从每年有一万多人死于上吊,而且既遂率几乎是百分之百的效果来看,上吊仍旧是荣登自杀 手段的宝座。

可是,这个人的后半生究竟是什么呢?它已远远超越了不幸或倒霉。在旁人看来这才是「活地狱」。

从留下的后遗症来看,或许不去救她还来得好。

\subsection{自杀地图 1:树海}

如 果 你 对 工 作 和 人 际 关 系 感 到 厌 烦 ,要想 在 绝 对 不 被 他 人 知 晓 的 情 况 下 悄 悄 自 杀 的 话 ,我 就 劝你毫不犹豫地踏进「树海」里去。

没有比树海那样既不容易发现尸体、又容易走进去的自 杀地点了。 

你会去向不明,久而久之从人们的记忆中消失。

可是,要达到这一目的,那就必须研究进入树海的方法,这份地图可作为参考。

(1) 从红叶台眺望西湖方向所看到的树海

(2) 往树海的入口「青木原自然步道」

(3) 命运车站「风穴」

(4) 第一岔路;这里还是安全地带

(5) 第一警告

\subsubsection{历史}

在树海里的自杀为「每年三十人左右,没有什么增减」(富士吉田警察署)。

其它的「自杀胜地」大半都已成为过去,但树海作为自杀地点而优于其它地点,从上述一点已经表明了。

本 来 树 海 就 是 自 杀 胜 地 ,但 一 下 子 出 了 名 而 每 年 都 有 数 十 人 在 此 自 杀 ,主要 是 1957 年 以 树海为背景的松本清张小说「波浪之塔」,被改编为电视剧的原因。

\subsubsection{找不到尸体的路线}

一般来说,进入树海后立刻就会失掉方向感而走不出来,因此不管从哪里走进去都是可以的,但这里要介绍的是始终找不到尸体的路线。

最 一 般 的 是 从「 风 穴 」附 近 进 去 。
在 富 士 快 速 巴 士 的 风 穴 车 站( 如 何 去 法 参 照 交 通 栏 )下 车 , 即 可 在 国 道 139 号 线 的 南 侧 看 到 进 入 风 穴 入 口 的 道 路 。

往 前 走 就 是 风 穴 的 售 票 处 。但不 要 走进风穴,售票处前有两条路,要走左侧的青木原自然步道。 

往前走约三百公尺,又有岔道,要走左侧写着「冰穴‧红叶台」的岛。在正面有一块牌子写着「只有一次生命,珍惜它」,但不要把它当回事。

(6) 进入富岳林道的路标

(7) 最后的警告标示;有遭人破坏的痕迹

(8) 富岳林道;阻断前进道路的绳子

(9) 原生林内部;来到这里就可以安心赴死了

再往前走三百公尺左右竖有路标,左侧写书「红叶台‧冰穴」,右侧写着「山道」。

山道一侧又有一块铅皮板倒在那里,写着「生命是双亲赋与的宝贵财产!重新想想双亲、兄弟和孩
子们吧!」。

不要理它。

走到这里不免会产生一点动摇,你可以决定是否返回去,因为指南针仍起作用,再往前就走入富岳林道。

这条山路是极普通的山路,对有登山经验的人来说没有什么特殊。不过,四周都是浓密的森林,因景色一直没有变化逐渐会怀疑不知走了多少路,走在什么地方了。山路大体是向南延伸的。

再走十五分钟左右,路也不像路了,四周的树木有点像原始林了,此时再向前走十五分钟。

从此以后,不论从哪一个方向走进密林都可以,但以向右走最好。

\subsubsection*{绝对找不到尸体的地方}

此外再介绍两处绝对找不到尸体的地方。

一处是在富士快速巴士「红叶台入口」下车。附近有消防队和旅馆「珍木馆」。

沿着其间的柏油道路往前走约一公里,就渐渐看到一片密林,再往前走约一公里就是原始森林了。

沿着这 条 路 走 入 左 面 叉 道 ,尽 量 朝 深 处 走 。

接下 来 一 直 往 前 走 就 可 以 了 。

这一 带 ,不要说当地人 , 就算自杀志愿者也都不大会来的。

当地人说,叉道是「最难找到尸体」的地区。

再一个就是在富士快速巴士「赤池」下车,沿着柏油路精进湖登山道,走过一至二公里后向
右进入叉道。在这里,不论当地人还是旅游者都不会找到尸体的。

这两个方法都需走相当长的路,不过从风穴进入找不到尸体的可靠性比较高。

由于一般的旅游者也不来此地,往里走入时千万不要引起人们的怀疑。

\subsubsection*{再往前走一百公尺就无法回头了}

在原始森林里,到处都是盘根错节的树根,青苔和落叶盖住洞坑,所以没法直线前进。
自认为是记住来路的,走了一百多公尺后也完全搞不清方向。如果带着指南针的话,就把它扔掉,反正都不能再回到原地了。

在树海里最合适的就是采用死亡率很高的上吊,只要准备一根绳子就够了。
不过,这里的树 木 都 很 高 大 ,要找 一 棵 合 遇 的 树 相 当 困 难 。
为 了 要 找 树 ,你就 会 更 远 离 山 路 ,更难 被 人 找 到 。 
于是,你就这样永远地从人们的记忆中消失了。

\subsubsection{注意}

每年十月,当地的消防队、警察等共六百多人对图一的Z道进行尸体大搜索。

就是说,如果在国道和西湖南侧的道路中间部分进行自杀的话,尸体还是会被发现的。

即使在国道两侧,沿着自然步道走进不过五百公尺的话也是一样的。

最近几年来的大搜索发现的尸体是1989年三具(其中女性一具),1990年无,1991年五具(其中女性二具),1992年因风穴附近发生杀人事件未进行搜索。
总之,绝对不要进入这一区。

顺便提一下,当家属委托搜索的场合,一天一个人的搜索费用是一万日元。就是说,如果动
员五十人搜索两天,就需要一百万日元了。

每年三月,从风穴入口附近往干德道场一带,自卫队在密林中进行一列纵队的步行训练。
据 说 在 大 搜 索 中 未 发 现 的 尸 体 大 部 分 都 是 这 个 时 候 被 发 现 的 。

因此 ,从 风 穴 的 南 面 到 干 德 道 场 一带,同样还是不选择为好。

在当地长时间居住的人,一眼望去就能分辨出自杀者。
据认为,仅仅带一个小包,或者不拍照就是特征,但更重要的是靠直觉。当你想走进树海,当地人就会对你打招呼说「你知道这里是什么地方吗?」,并进行劝说。

尽管你说「让我死好了」他们也不会听的。
经过四十多分钟的磨嘴皮子,有的最终屈服而被拉回去。

所以,首先不要引起这种气氛。然而,临死关头还装做挺开心的样子也有点胡闹。
在进入树海之前不要拖拖拉拉、踌躇不前,很大方地走进去,当地人也不至于怀疑跟了过来。

因为即使是当地人走进一百公尺也难以回来的,这是真话。

\subsubsection{交通·住宿}

从 JR  三 岛 站 乘 富 士 巴 士 到 富 士 吉 田 需 两 个 小 时 。
再换 富 士 巴 士 到 红 叶 台 入 口 、风 穴 需 三十分到四十分钟。

巴士每小时有一班。要注意的是冬天的车子班次会减少。

当然没有住宿的必要,但为了备用所以介绍一下。

西湖的南面有很多民宿,全年营业,利用这里最方便。从这里到风穴车站,乘富士快速巴士约一个小时。
可在山道入口处的珍木馆住宿。



\subsection{案例研究 7}
\subsubsection*{在树海中过着流浪生活的男子}

有 个 三 十 一 岁 的 男 性 公 司 职 员 ,在 1983 年 11 月 在 树 海 里 徘 徊 了 十 六 天 后 ,被警 察 保 护 起来了。

他 是 九 月 下 旬 ,因公 司 和 个 人 的 烦 恼 而 开 车 离 开 横 滨 的 家 。

他 开 车 在 自 已 的 故 乡 爱 知 县 以 及 东 北 地 区 流 浪 了 一 个 多 月 后 决 心 自 杀 ,给家 属 寄 了 遗 书 ,10 月 26 日 从 鸣 泽 村 的 红 叶 台 进 入了树海。

最初的一个星期,到处去找合适的自杀地方,有时刚想上吊却遇到了采蘑菇的人,在一下子 死不掉的情况下逐渐对死的欲念淡薄起来。

于是,有时到国道去买些面包,有时在汽车停车 处吃饭,然后又回到树海,就这样过着像流浪汉的生活,最后难以抵挡寒冷的大雨,11 月 10 日住到西湖畔的一家民宿,并打了电话给家里。 当 地 的 警 察 局 和 消 防 队 受 到 家 人 的 委 托 ,动 员 了 一百五十余 人 进 行 了 三 天 的 搜 索 ,事实 上 这 个 男子知道在进行搜索。

他被保护起来后说:「我不会再想自杀了。」

\subsubsection*{检验状况}

即使再踏进一步就无法再退回的「死亡的原始森林」,如果进入的路线不妥的话,也会造成
这样的结局。

大概他是在西的南面道路和同8国道线之间的红叶台、龙宫洞一带活动的,这
里是远足路线,有休息处和牧场。遇到采蘑菇的人也会很自然。

如果真正要想死的话,这一 带是不行的。 早就淡薄了死的欲念,可是仍旧继续进去树海,度过了十来天的树海生活,听起来确实是个笑话。

是危险的树海生活,或者是愉快的亦未可知。

这对树海自杀志愿者来说,他的行动说明了「即使进入树海,一星期不吃不喝也能生存下来」的珍贵资料。


\section{Leaping 跳楼}

痛苦 ▼▽▽▽▽

麻烦 ▼▽▽▽▽

死状 ▼▼▼▽▽

牵连 ▼▼▼▽▽

冲击 ▼▼▼▼▽

致命 ▼▼▼▼▽

骤然跳下毫无痛苦。致死度高,是最高级的自杀方法。

跳楼及跳崖自杀是不痛的。

没有疼痛,没有不安,更没有恐怖。

不仅如此,甚至还可以算很痛快。这并不是在打比方,事实上确实如此。

听上去有点胡说八道,但如果把那些「掉下来的」人们的话综合起来,也只能这么说了。

关于这一点,下面还要详细叙述,不过,这样一来,跳楼及跳崖自杀可以说是与上吊自杀相媲美的极佳手段。

跳楼及跳崖没有像其它自杀手段所伴随的阴影。

彻底改变了跳楼及跳崖自杀观念的是一位叫做佐藤佳代\footnote{冈田有希子(本名:佐藤佳代,1967年8月22日-1986年4月8日)是1980年代的日本女性偶像艺人。}的少女。

距今九十年前的1903年,留下「万物真相,一言以蔽之,日『不可解』」,而从华严瀑布纵身跳下的青年——藤
村操\footnote{藤村操(1886年7月 - 1903年5月22日)为北海道出生的旧制一高的学生。于华严瀑布投水自杀。自杀现场所遗留下来的遗书“巌头之感”造成当时的媒体及知识分子极大的冲击。},是他对自杀本身赋予了与以往不同的哲学高尚形象。

就好像不断出现的后继者那样,佐藤佳代也给予了坠落自杀某种神圣的形象,招致了一个又一个的追随者,甚至还出现「她是掉下来的,是跳跃的」这种神话。佐藤佳代就是红牌歌手冈田有希子的真名。

或许由于这个缘故,不管男女老幼采用跳楼或跳崖法的仅次于上吊法,特别是青年和少女乐 于此法,而女性则更多。

此法出现上升趋势,现在包括十几岁的女孩在内的女性自杀者中百分之五十是坠落自杀,遥遥领先其它自杀方法。

最早的跳楼者,据说是1935年某公司男职员从银座松阪屋百货大楼七楼屋顶向银座大街 跳下来的。

从以往的华严瀑布或锦浦转向高岛平社区所代表的高层大厦,与都市化有着密切 关 系 ,也是 都 市 现 代 化 的 现 象 之 一 。

冈 田 有 希 子 自 杀 后 的 第 二 天 ,即1986年 4 月 11 日 , 十八岁的女孩子拉着妹妹的手,留下「为了能领悟自己在前世的超能力」而从社区的屋顶跳 下身亡。 

\subsection{准备}

需要距地$20$公尺的高度从高楼往下跳的时候,事前应查看一下必要的高度和落下地点的状况。

自杀未遂者比较多,大都是因为未能确保必须的高度。

如果你真的想死,那就应该从地面$20$公尺以上的高度,即从大约七至八层楼以上跳下。地面若是混凝土,那就无法得救了。

从四楼往下跳,成功率只有百分之五十左右。附带告之, 每增加一层就高出$3$公尺,不妨作为换算的依据。

查 看 落 下 地 点 也 是 非 常 重 要 的 ,楼 下 有 小 树 丛 是 不 行 的 。

从高 约 $18$公 尺 的 五 层 楼 校 舍 屋 顶 跳到楼下树丛的十六岁少女,也只落了个重伤。

在美国,有人从$28$公尺高处落到花坛, 也只是受到断了一根肋骨和左手腕骨折的轻伤 楼 下 有 树 木 或 路 灯 也 是 不 行 的 。

从$35$ 公 尺 高 的 十 四 楼安全梯 跳 下 的 十 七 岁 女 高 中 生 ,只 受了住院治疗六个月的重伤,结果是自杀未遂。

楼下有块绿化地且不说,在落下时穿在制服 外面的风衣因空气而鼓胀起了降落伞的作用,半空中又撞到了枫树,因而免除了灾祸。

从新 宿的出差旅馆七楼跳下的某通信社记者,因在半空中碰到路灯而脚先着地,左肩和骨盆骨折 没有死掉。

他当时没有失去知觉,一面自言自语地在骂「他妈的、他妈的」,一面回答赶来 的警察问话。

车 辆 有 时 也 会 起 衬 垫 的 作 用 。
在名 古 屋 市 ,一 个 由 $33$公 尺 高 的 百 货 大 楼 屋 顶 跳 到 马 路 上 的四十岁男子,以俯卧的姿势掉到车辆的引擎盖上,受到颜面挫伤和右肩骨折的三个月伤。

此外,下面是自行车停放处的锌皮板屋顶也不行。为了家里不准饲养捡回的小猫,紧抱着小 猫 从 十 一 楼 跳 下 的 十 四 岁 小 女 孩 ,使自 行 车 停 放 处 的 锌 皮 板 屋 顶 破 了 一 大 块 ,断 了 三 根 肋 骨 , 跌成重伤,但她和小猫却都保住了性命。

这 些 虽 是 特 殊 的 例 子 ,但 在 美 国 的 一 次 飞 机 事 故 中 ,有人 从 $370$公 尺 高 处 落 到 了 一 片 雪 地 上,除腰和几根肋骨折断之外,性命却保住了。所以,对积雪也应加以注意。

要寻找不显眼的地方 为了不被很快发现而送往医院,找个不显眼的地方也是首要考虑。

从丸之内大厦和毗邻大厦 中 间 跳 下 的 大 学 女 生 ,一 年 后 才 被 发 现 ,所谓 不 大 显 眼 是 指 要 隔 了 一 段 时 间 之 后 才 会 被 发 现 的地方。

在上智大学七号馆下的石墙旁边,当收拾从屋顶跳下的学生尸体时,偶然又发现半 年前就不知去向的学生跳楼尸体;当然这种例子是极少有的。

跳崖就选胜地跳崖有时是不可靠的。
两个国中女学生从$60$公尺高的崖上跳下自杀,一个死掉,另一个却得救了。
在同样的条件下跳崖,其结局居然也有这么大的差异。

再者,不论怎样陡的悬崖峭壁,落入海中就不一定会死。
所以,这种情况下就选择名胜吧。
名胜其来有自,所以才成为名胜。
如果选悬崖,有四国的足折岬、热海的锦浦,瀑布则有华严瀑布可供选择等。

\subsection{经过}

最令人好奇的是,在落下的过程中是什么感觉,落地时有无疼痛这个问题。

从大楼四楼跳下而得救的五十四岁男子说:「没有害怕的感觉,很自然地整个身子越过了阳台栏杆,跌到地面时是否有疼痛的感觉就记不清了,但却知道自己还跌倒在地上。」对正在半空中还在继续往下落时的感觉,他说:「从跳下到地面的过程中,虽是理所当然的事,但还 是 想 到 头 先 着 地 还 是 脚 先 落 地 的 问 题 ,大概 是 不 想 让 脸 碰 地 吧 ,很 自 然 地 就 用 双 手 捂 住 了 脸。」

从高处跌落到冰河岸上而生还的男子说:「好像是坐在巨大的翅膀上慢慢地在下降,很平静 地 想 到 自 己 ,想到 家 属 的 将 来 ,许许 多 多 的 回 忆 像 闪 电 般 掠 过 了 脑 海 。
落 到 地 上 后 呼 吸 不 乱 , 在没有任何痛苦的情况下失去了知觉。

虽然,头部和手、脚碰撞到崖石或冰块,多处撞伤, 但也没感觉到。可以说,没有比这一瞬间更痛快的时候了!」 这种因事故而摔下的事例是不胜枚举的。

这些体验的共通点是,开始是缓慢地往下跌落知觉 非常清楚,完全没有不安和恐怖,简直就像在做梦似的。在这时候,孩提时代的记忆常常会 像走马灯似地在脑海里翻滚,有时还会看见神秘的光线,有时还会从上面往下看正在下落的 自己。

着地时在安稳的心情中丧失了知觉。 
跳楼及跳崖自杀者,绝大部分不发出惨叫声和大声叫喊,大慨是由于这种缘故吧。某跌落者 说:「真想强调从高处跌落死亡是最无苦痛的死法。」

还有的跌落者甚至说:「那真是完美 的死,什么痛苦也没有,比起来打针要痛得多。」因此,不妨说跳楼及跳崖自杀是没有疼痛的。 

\subsubsection*{叫喊着「痛!痛!」,哭泣而死去的少女}

当然,也是有例外的。

「现在我要自杀啦,拜拜!」刚才还坐在学校四楼窗沿上的女高中生突然跳了下去。

当她被抱起来的时候,小声地一再说「痛死了,痛死了!」的哭泣着。她跌断了头颈,送到医院后
不久就死亡了。

原因当然很清楚,在于四楼这个高度不够高的。
为了不留下疼痛的感觉,应选择可以当场死亡或者至少能昏死过去的高度。

还有,从新宿的住友三角大厦的三十五楼,距地$140$公尺高处跳下的三十岁左右的女子, 双手伸向水平,以高空跳水姿势落下,脸的半部和头部摔成粉碎,当场死亡。这时,目击者 听到「哇——」的尖叫声。也有一面惨叫一面从十一楼跳下的十七岁女高中生。这两个人大 概跳楼时感到了恐怖亦未可知。 从物理学角度看,由$20$公尺高度(约七层楼)跳下的时候,一秒钟可降落$4.9$公尺。

这 时的速度为时速$35.5$公里,约两秒钟后即可着地。
就在这短短的两秒钟里,以各式各样的姿势翻滚着,着地时的时速约$70$公里。
不妨设想一下骑摩托车或乘车时,用这个速度撞 到墙壁时,并不是所想象的那么大的冲撞。

不过,从更高的地方摔下,其着地时的冲击是很 大的。
有一个令人难以相信的事例,例如从十一楼公寓屋顶跳了下来的女高中生,竟然在着地时把下水道铁盖撞成了两半。

跳崖从悬崖峭壁往下跳的时候,由于周围自然环境的缘故,会出现各种不同的情况,虽然有点说
不准,但还是以激烈碰撞死亡的较多。
掉到海里或瀑潭里的中途,因碰撞岩石而死亡的例子不少。在热海锦浦自杀的相声演员中田治雄,在跌落中猛烈撞断崖岩石,因而内脏破裂死亡。

同样在锦浦跳下的五十三岁男子,全身都是挫伤,当场死亡。
当然,如果跌入海里或湖里,也有可能溺死的。

\subsection{尸体状况}

如同五十四岁的男子在跳下以后所想的那样,人究竟哪个部位会着地呢?

某医学专家的研究认为,跳楼自杀时,脚先着地的情况好像最多。脚先着地的时候,$60\%$的人头部负有外伤,$30\%$的人脊椎骨折,肝脏和肺的损伤分别为$20\%$,心 脏破损为$25\%$。

其次是头朝地摔下,这时出现头盖骨骨折,脑损伤以及肋骨骨折等,手臂、脊椎的骨折和肺部损伤的情况较多见。再其次就是臀部着地和横卧摔下。

所以,不论哪个部位先着地,多处都会受伤,从跳楼及跳崖自杀者来看,头、腹、手和脚等 三处以上受损伤的情况近$70\%$。

总之,身体到处都会受伤,而以头部和胸部受伤的情 况 最 多 ,$70\%$ 以 上 的 致 命 伤 都 是 由 此 引 起 的 。

心 脏 则 因 为 人 体 落 下 时 的 惯 性 作 用 振 动 很大,大动脉发生断裂的情况也多。

于是,因头盖骨破裂、全身挫伤、内脏破裂,出血过多 等原因死亡。 
或 许 有 不 愿 被 人 看 到 那 样 惨 不 忍 睹 的 尸 体 ,但 一 旦 被 发 现 时 救 护 车 迅 速 赶 到 ,转 眼 功 夫 就 把 尸体收拾了。

从臀部着地的情况来看,有时尸体几乎看不到任何受伤的痕迹(案例9)。所 以,跳楼自杀并不是那么难看的死法。

\subsection{注意:当心行人!}

着地的地方有人的话,会引起许多麻烦的问题。 因 为 落 到 行 人 身 上 而 得 救 的 例 子 也 有 好 几 个 ,被压 在 下 面 的 伤 患 会 索 赔 巨 额 赔 偿 金 。

埼 玉 县 的通信学校高中生从百货大楼屋顶跳楼时,落到停在下面的车子上,本人固然死了,但坐在 车子里的男子也折断了颈骨引起了胸部以下的瘫痪,死者家属赔偿二亿日元。

最近的一个例子是,1992 年11月从公寓八楼跳下的男子落到正在下面与女友谈话的高 三学生身上,跳楼人不久死亡,而那学生也在四天以后死去。

跳楼落到别人身上绝非好事, 这一点也要注意。 有 位 五 十 一 岁 的 公 司 理 事 想 从 公 寓 最 高 层 跳 楼 ,但 看 到 下 面 有 几 个 孩 子 在 玩 耍 ,于 是 一 面 背 着手拉住十四楼的走廊栏杆,一面大声喊「躲开!躲开!」等到孩子们散开之后才跳了下来, 圆满地达成目的。

既然是十四层楼的高度,叫喊的嗓门应该是相当大的。

不想给自家人造成麻烦的话,请特别留意。 
头朝下即使有相当的高度也会得救的,相反地即使高度相当低也会有死去的例子 。

有从六公尺高度跌落到河底因头盖骨跌死,也有从五公尺高的人行路桥摔到马路上跌破了头,一个半小时后死去的。
不想失败,就应该头朝地,摔得巧的话,五公尺高度也会当场死亡。

还有,虽然跌撞的不是致命部位,但也会因特殊原因而死去。
一位五十岁的妇女从公寓四楼阳台摔下,腰骨多处折断,但因全身性的挫伤症而死亡。
一位男子从宿舍三楼跳下,二十天 后因肺部瘀血、急性肺栓塞而死掉。

\subsection{案例研究 8}
\subsubsection*{跳楼未遂大学生,从十五楼跳下过程中的感觉}

1986年 10 月 ,二十 一 岁 的 大 学 三 年 级 学 生 从 埼 玉 县 浦 和 市 十 五 层 公 寓 的 最 顶 楼 跳 楼 ,随 着一声「噗咚」的响声,跌到了自行车棚的铁质棚顶上。
他立即被送往医院,但除了左脚受点轻伤之外,无其它伤处。

他跌倒的自行车棚顶却留下了大字形的大破洞。

这个人很平淡地叙说在半空中的感觉:「一面往下落,一面却感到鞋子和眼镜慢慢地掉了。

跌到棚顶后过会儿,忽然想到,啊!我还活着啊。」
他多少留下了口吃这一语言障碍,担心 第二年就业面试可能通不过。 

\subsubsection*{检验本例}


这是从高处跳下的自杀未遂者所谈的「在空中时的心境」,非常少见生还的事例。

十五层楼约有$40$公尺高,从这个高度跳下,会有缓慢而降的感觉,可能非常冷静,一点都没有恐怖
感和着地时的痛苦。
从十五楼坠落而几乎没有受伤,真是个奇迹。所以,坚决跳楼自杀者自行车顶棚也应避开它。
他坠地时「噗咚」很大的声音,那么,自杀者掉下来时会出现什么样的声音呢?不妨查证一下。

当然,这是依据从什么样的地方掉下来而各有不同的,如果地面是水泥地,前述从十五楼公
寓屋顶坠落的女高中生发出了「噗咚」的声音。
从四楼公寓窗户掉下的男子是「仿佛泄气的气球似的声音」。
前述从住友三角大厦跳下的女性则发出「噗啦」的响声。
如果落到别人身上时,就像前述横滨高中生摔下来时发出一种「好像皮球在地上碰地一声弹了一下的声音」。
冈田有希子从$20$公尺楼顶掉下来时,据说发出了「咚咚」的一声大响声,可能是因为头盖骨激烈撞到水泥地的缘故吧。

\subsection{案例研究 9}
\subsubsection*{「活着反正也是无聊」的漫画家山田花子}

1992 年5月24日,漫画家山田花子(当时二十四岁)从东京多摩市自宅附近的公寓十一楼跳楼自杀。
因为腰部着地,尸体比较完整,出血也很少,连其父母都很吃惊「真的死了吗?」

她读小学时就内向,爱待在家里,读中学二年级时因被欺侮而曾企图用煤气自杀过。
到了高中也一再受到欺侮,读了一年就退学了。后来成为漫画家,在《青年杂志》上有作品连载,但画坛未予以好评。不久,连载作品发表不了,最后在不支付稿酬的「\qquad \qquad 」发表,单靠漫 画无法维持生活,于是她做茶艺馆的侍者。

但是她一下子记不住客人们很多的点菜单,加上 做事不够俐落,相继被辞退,而且在工作场所也遭欺侮。
好不容易持续工作了半年的饮食店 最后也不行了,受到这一连串打击以后精神有点失常,跑到这家深夜营业的饮食店哀求「再 雇用我一次吧!」,并强行上班,每晚都到天明。

实 在 无 法 忍 受 的 店 方 在 半 个 月 以 后 报 了 警 ,父 母 把 她 领 回 去 了 。
在回 家 路 上 的 出 租 汽 车 里 说 : 「大家都欺侮我」时好像在哭泣,事实上她是在笑,她患了精神分裂症。不久她进了精神病 医院,两个月后出院了,但对未来丧失信心,就在出院的第二天从自家附近的公寓坠楼死亡 了。

她在自杀前两天的日记中写道:「和别人相处不好。自己性格孤僻,一个朋友都没有。…… 看不到将来,也找不到工作。(被人欺侮)……什么也不想再干了。一切都那么吃力,没有 力气,疲倦得很。」事实上这就是她留下的遗书。

\subsubsection*{尸体状况}

跳楼自杀的尸体是惨不忍睹的,尤其是头部落地的情况。
像前面那种腰部着地的例子,面部会是完整的。不过,腿部先落地时可能由于骨折而扭曲,确实不忍卒睹。

\subsubsection*{被「视线恐怖」所杀的漫画家}

这里我们应注意的是,山田花子到处受的「欺负」。
不管到哪里,被欺负的「家伙」总是被欺负的,她本身证明了这个事实。
还有,她描绘的「日记漫画」绝大部分内容,都表现了她特别留意在学校或工作场所「别人是怎样看待自己的」。
从此亦可看出,她是天生就有一种叫做「视线恐怖症」的症状。

她本人说自己有「对人恐惧症」,就像在外出时一定要戴上太阳眼镜那样,她的一生就是在永无休止的惧怕他人目光中度过。

加上又遭欺负,最后终于患了精神分裂症而自杀。
有谁能料到她的烦恼有多大吗?那些性格不开朗、内向的、不能干净俐落处理事务的人,是不适合 在像日本这样的社会中生存下去的。

十九世纪意大利的自杀研究家就说过:「自杀是在自然 界的生存竞争中,让身心不大健全的那些人自然淘汰的一种手段。」真是说得一点不错,山田花子也是在平静中「淘汰」了的一个。 

她生前的自杀观也是值得一提的。

在其漫画中曾引用了电影《天堂里的快乐》下列这些句子—— 「 没 有 任 何 长 处 又 讨 人 嫌 的 话 还 是 去 死 吧 !不 说 那 些 坏 话 / 活 着 没 出 息 / 与 其 混 不 如 大 肆 渲 染一番/活着反正是无聊的。」

此外,在其它杂志还这样写道--「这个世界本来就是残酷的。」
「残疾者们,笑吧!喊叫吧!诅咒命运吧!你的人生就是这么回事。感到讨厌就去自杀吧。」

对束手无策的不幸发自肺腑的话,也只有在不幸的苦海中活过的她才能说出来的。
这个世上确实有无法克服的不幸。她一语道破了这个事实,与本书的目的也是相通的。


\subsection{案例研究 10}
\subsubsection*{因被欺负而跳楼自杀的中学生}

1979 年 9 月 9 日 ,在埼 玉 县 福 冈 市 的 公 寓 庭 院 内 ,发现 了 中 学 一 年 级 学 生( 当 时 十 二 岁 ) 穿著空手道衣,摔成「大」字形死亡。他是当天上午八点多,从离家约两公里的公寓十楼跳 到$20$公尺下面的水泥地。

自杀的原因是受欺负。
在上小学时还相当开朗的这个学生,进了中学不久因小事情跟别人打 架。
身高只有一百四十二公分的他是班级内最矮小的,被同学们说成「人虽个子小,倒挺蛮 横的」而被排挤在外。
没有人可说话,天天沉默不语,结果被人起了「墙壁」这个绰号。经 常被人嘲弄说:「你是墙壁,面壁就行了。」

同年6月18日傍晚,该少年把遗书贴在书桌上就不见踪影了。
遗书上写着「每日受欺负,不想上学了。也不想再活了。我要自杀。」

可是,这天晚上八点多,该少年满身大汗地回到了家。
后来听说他本想从公寓顶楼跳下的,但感到害怕就跑了回来,满身大汗就是因为跑得太快。

自 杀 未 遂 的 消 息 走 漏 了 出 来 ,班 级 的 那 些 专 门 欺 负 人 的 淘 气 鬼 们 越 发 欺 负 得 厉 害 了 。

他又有了一个新的绰号「自杀小子」,加上他父亲曾是清洁车的驾驶员,被嘲为「脏得很」、「臭 得很」、「像个乞丐的家伙」,有时把他当成小偷,也曾被满脸涂上蛋黄酱。

他实在无法忍受欺负,自杀前一天第一次旷课,第二天是星期日,他自杀了。这天穿的空手 道衣,是自杀未遂后开始练习空手道穿的,这也是他第一天穿。

知道他自杀了的那些捣蛋孩子竟然喊了「万岁」,至于他们欺负的理由,说是「无聊」、「好玩」。


\subsubsection*{检验死因}

该少年第一次企图自杀时,害怕得跑回家。一般来说,决定自杀的人即使站到高处心情也是平静的,并不害怕,但也并不一定都是如此。

一位二十四岁的女子想要自杀爬上$1713$公尺的高山,但快到山顶的悬崖处时害怕了,不敢跳了,又下不来,整整三天三夜在风雨中不吃不喝地伏在那里等待救助。

这是八月底发生的事,夜晚的气温降到六到七度,而她只穿了一件罩衫。看来当时下决心去死掉可能会舒服些。

\subsubsection*{被欺负的人不管做什么总是被欺侮}

 山 田 花 子 的 例 子 也 是 这 样 的 ,只 能 说 被 人 欺 负 是 没 有 办 法 的 。
被 欺 负 的 家 伙 不 管 做 什 么 总 是 会被欺负的。
不管空手道也好,自杀未遂也好,不但没有产生任何作用而且反而使事态恶化 了 。
没 有 共 同 目 标 而 只 有 人 与 人 之 间 的 关 系 的 班 级 里 ,能够 做 的 也 只 有 模 仿 恋 爱 的 游 戏 和 欺 负人了吧。

据说父亲曾劝过中学一年级的儿子说:「还有两年半,忍一忍吧。」,但没有人能保证中学 毕 业 以 后 就 会 有 幸 福 ,升 入 高 中 后 并 不 一 定 就 出 现 变 化 。

而且 想 到 还 要 再 忍 耐 两 年 半 的 时 间 , 那么他所做的选择可以说是正确的。
还不如第一次爬上公寓时就摔死要好得多。 

早点自杀也是很重要的。

\subsection{自杀地图 2:高岛平社区}

现在,那个高岛平社区是什么样子呢?事实上,因1980年前后的跳楼风,对超过十一 层 的 $37$栋 高 楼 ,自 三 楼 以 上 的 外 走 廊 和 外 楼 梯 都 装 上 了 防 止 跳 楼 的 围 栏 ,窗 户 则 装 上 闭 锁。

上屋顶的楼梯装了上了锁的铁门。
现在可以说是完全不可能跳楼了。
曾被称为自杀名胜地的新村管理部门,因被装了完全防止自杀装置,反而成了望上去一片铁牢的异样景致。

\subsubsection{历史}

社区完工是在1973年,当时曾被誉为日本规模最大的,现在看上去还是宏大规模的社区。 
当 年 就 发 生 了 五 起 跳 楼 事 件 ,但 使 社 区 出 名 的 是 1977 年 母 子 三 人 跳 楼 自 杀 事 件 。

由此 而 被称为「自杀圣地」的社区,自杀者每年超过十人,1980年突破了二十人,到了1982年累计达到了一百人。 
自杀者中的八成以上是来自静冈、新泻等外地的「远征自杀者」,有的人为了翻越屋外围栏而特意买来梯子,自杀旺季时每隔三天就有一个自杀者。
为此,1981年花费七亿日元装上钢铁栅,并在社区内安装了「救命电话」等,着实大费周张了一阵子。虽然如此,仍然有用长凳砸破玻璃窗,爬到屋檐跳楼自杀的十九岁和十八岁的男女。不过,或许这一对引起了更大的因应措施。

有关人员说:「(自高岛平警察署设立以来的)最近七年间每隔两三年才发生一件,因为装了围栏几乎听不到有人自杀了。」

\subsubsection{跳楼方法}

现在不会再有人特地到高岛平社区去跳楼了,不过为了那些在附近怎么也找不到合适的地
方 ,或者「 不 管 怎 样 还 是 高 岛 平 !」的 人 们 ,我在 这 里 悄 悄 地 告 诉 你 高 岛 平 的 一 些「好 地 方 」。

看 来 完 全 封 堵 了 跳 楼 的 社 区 ,不知 为 什 么 3 - 1 1 街 区却 完 完 全 全 地 毫 无 防 备 。 
尤其是3-11的一号楼,外人也能随意进出一至十四楼的外走廊,只有一公尺高的低栅, 毫 无 防 备 的 情 况 大 概 连 居 住 者 都 会 害 怕 。
3 - 1 1 的 二 号 ~ 六 号 楼 ,楼 梯 转 角 平 台 上 的 窗 子 没有闭锁装置,只要爬上$150$公分左右的矮墙,就能顺利地跳下去。

虽然如此,我还是推 荐可以展望都营线对面板桥区街景的3-11的一号楼十四层外走廊。
下面不用说是水泥 地,行人也少,摔下去肯定必死无疑,而且这栋楼的外楼梯也没有装设栏杆。

\subsubsection{交通}

乘都营三田线在新高岛平站下车,朝三、四丁目方向然后左拐前进。3-11的一号楼临电车轨道,是整个社区中唯一没安装铁栅的大楼,一下子就可找到。

要想观看铁栅的异样景致,不妨在提前一站的高岛平站下车为最佳视点。

\section{Cutting Wrist and Carotid 割腕、割喉}

割腕

痛苦 ▼▼▽▽▽

麻烦 ▼▽▽▽▽

死状 ▼▽▽▽▽

牵连 ▼▼▽▽▽

冲击 ▼▽▽▽▽

致命 ▼▽▽▽▽

虽然是「刀子割破手腕」,但也可以致死。这是最平静的死法之一;不过要有未遂的心理准
备。

颈动脉

痛苦 ▼▼▽▽▽

麻烦 ▼▽▽▽▽

死状 ▼▼▼▽▽

牵连▼▼▼▼▽

冲击▼▼▼▼▽

致命 ▼▼▽▽▽

想尝试一次血喷天花板的感觉?但是太恐怖了,而且未遂的机率很大,所以不是好的方法。

切腹

痛苦▼▼▼▼▼

麻烦 ▼▼▽▽▽

死状▼▼▼▼▽

牵连▼▼▼▼▽

冲击▼▼▼▼▼

致命 ▼▼▽▽▽

这种有百害无一利的手段也很新奇,但是为什么不断有人用这种方式自杀?真是不可思议。

\subsection*{割腕、刎颈}

在半夜,单独一个人在房间里拿把刀子放到手腕上比划比划。

死也好,不死也好。如果你真的想到过死的话,恐怕至少有过一次这样的经验吧。假如把刀子横在脖子上,也可体验一下死亡的惊险了。

割手腕,是最 简 便 而 又 能 当 场 体 验 到 自 杀 气 氛 的 方 式 。

同时 ,既 能 用 自 己 的 身 体 和 眼 睛 观 察 , 又可了解疼痛和死去的全部过程。

六十年代曾被称为「割腕综合症」,而在美国极为流行的
这种自杀方法,随后又波及到欧洲,最后火花飞溅到了日本。

最近,歌手中森明菜割腕后, 在六本木一带的迪斯科舞厅出现了年轻女性相继割腕的「明菜综合症」,成为时下最时髦的 自杀方法之一。

不过,对真正想自杀的人我是不大想推荐这种方法。
因为,有一种意见认为割腕死亡率只有 $5\%$,方法虽受欢迎但未遂率却较高。

话虽这么说,那种坚持「割手腕是绝对死不了的」说 法 也 并 不 正 确 。

假如 相 信 前 述 的 数 据 ,每二 十 人 就 有 一 个 人 死 于 这 种 方 法 。

某 前 检 察 官 也 说 : 「仅仅割腕而因出血过多死亡的并不多。」

这一章就是为了那些希望能够成为$5\%$的人所撰 述的。

当然,一开始就没打算去死而只是想体验一下自杀情绪而割腕,是不要紧的。
只要不对他人造成麻烦,也就没有被责备的道理。

再者,如同割腕一样,用带刃的东西伤残身体使其大量出血致死的方法,还有割颈动脉、刺
胸,或切腹等。这些自残行为的自杀也都是因为出血而死亡的,所以在本章一并加以介绍。

\subsection{准备}

\subsubsection{手腕——割断它!}

不论割手腕还是割颈动脉,只要有锋利的刀具就够了。
菜刀、剃头刀、裁纸刀等,只要利就都可以。

割腕的场合,最好先喝点酒,然后洗个澡,以使血液加速循环。
还有,为了不使出血停止,割后应把手浸在温水里或放置在面盆里,否则血液会凝固堵住伤口。

不过,手腕只要割断动脉就会死的,这是医生们的一致意见。
有的专业医生说:「只要割开动脉的一半以上或完全割断就会死的,割了一大半就不会再堵塞了。」

事先确认好切割的位置也是很重要的。
要想割手腕的人,应该把平日较不灵活使用的那只手的手掌朝上,细看手腕。

手关节的内侧有一根可摸到纵向的手动脉,就割这根手动脉。
在手动脉和皮肤之间有斜向的两根静脉,恰好在手腕皱纹处与动脉交叉。
为了更看清相互位置关系,不妨用力按住腋下使静脉突起。
单单割静脉而不想点法子,血流约$200$至$300$c.c.就会自然地停止,不至于死去的。

可是,就是这样也流了相当大量的血,常常就会因这个景象而到此为止不想死了。
把目标瞄准在没被静脉遮盖的皮肤下面的动脉,对准它横向地割上一刀,也只不过割了静脉 罢了。
动脉在皮肤下六至七毫米,比看上去还要深些。

必须刺透动脉,而且还要割几下。
即使是这样瞄准,当真正地去割也会割到旁边的正中神经,是很痛的。
沿着动脉 纵向去割或许可避开,但因太近而有困难。自杀未遂,反正也可再接上,所以可忍受疼痛而 把神经扣腱一起割开。

在摸脉处反面也有同样粗细的动脉,把它也一起割开较好。
总之,如果没有打算把手腕割下来的决心是不会死的。 
还有,不妨考虑到万一自杀未遂而留下伤疤,不妨确认手表表带的位置。

\subsubsection{颈动脉——刺透再拔出来}

颈动脉,只要一割就会立即死亡,不必担心未遂以后的事情。不过,这也是确确实实地割断的情况。
切割颈动脉的时候,最容易割断的是平日较灵活的手那侧的耳朵下面的外颈动脉。

颈动脉在颈部的高度处,分为内颈动脉和外颈动脉,靠近肩处的血管既粗也比较深。到耳朵下面里边深度达到三公分以上,四周粗厚的肌肉不少,割断也有一定的难度。

这种情况,也不要横向割上一刀,而是狠心地刺透然后再拔出来才好。
有时割 得不顺当,要割上好几刀(案例11),可是也有当倒下去时砸碎了饭碗,碗片割断了颈动 脉一下子就死掉了的人。
割颈动脉也非常地难,但上述事例也并不是没有的。

\subsubsection{其它}

心脏位置比一般想象的还要靠近中央刺扎心脏的方式,应在事前用手摸胸以确认位置。
心脏比想象还要偏向中心,深度虽因人而异但也有九公分深。

东 条 英 机\footnote{二战甲级战犯,因为战争罪行被谴责并判处死刑,于1948年12月23日执行绞刑。}向 医 生 请 教 了 心 脏 的 位 置 ,经常 抚 摸 胸 部 以 确 认 位 置 ,甚 至 还 让 人 画 了 靶 子 用 手 枪 击 穿 ,但实 行 时 还 是 偏 了 ,以自 杀 未 遂 告 终( 有 人 说 因 为 他 是 独 裁 者 中 难 得 的 左 撇 子 的 缘 故 , 他是用右手开枪的)。

没有刺中心脏而多次连续刺扎的例子也很多。

切腹的方式,刀刃至少要有十五公分长。
如果不能刺穿到脊背,那就没法进行切腹自杀。

\subsection{经过}

\subsubsection{手腕——让它流掉一公升血}

「人这个东西,那怕是用足了劲也割不动」,说这话的是用薄小刀在手腕上割了长五公分,
深四至五公厘的二十七岁的女子。不错,肉和血管都比想象得难割,手动脉的边上还有腱,更难割。

下了决心割开手腕,伤口张开很大看到血管和白肉,紧接着一股热血涌了出来。即使只割开静脉,敷上的毛巾也立即会染成血红的。

连动脉一起割开的话,将会以同样粗细静脉六倍的力量喷上两三米高。

但是,随着血的流出血压徐徐下降,流血会减少。 

至 于 疼 痛 ,只 割 了 静 脉 的 二 十 二 岁 的 女 大 学 生 的 话 是 值 得 作 参 考 的 。

她 一 面 用 毛 巾 捂 着 手 一 面这样说:「看着流血在想,就这样死掉吗?还是重新生活下去?两者必须选择其一。
大概 你们不一定相信这是真的,一点都不痛。从前,在中学二年级的时候曾割过手腕,那时也没感到痛,原来这样做是能死去的。」 割到神经会感到相当疼痛是很自然的,如果是静脉的话也不过是一般切伤程度的疼痛。

可是,割过以后的问题是,如果顺当地割断动脉,那么体内的血液流出了三分之一,人就会 死的。在人的体内,每公斤体重计算男性有约 $80$ 毫升,女性有约 $60$ 毫升的血液流动着。

因 此 ,如果 你 是 位 体 重 五 十 公 斤 的 女 性 ,体内 就 有 三 公 升 的 血 液 ,其中 的 三 分 之 一( 即 一 公 升 ) 流出后你就会死掉的。

这只相当于捐血时所采取的$400$c.c.的$2.5$倍而已。孩童和老人 会 以 更 以 少 量 的 出 血 而 死 去 。

用 割 腕 的 方 法 去 死 确 实 是 困 难 的 ,但 只 要 确 实 地 割 断 而 且 确 实 地流血的话,还是很简单。

\subsubsection{颈动脉——喷血十二秒钟}

完全割断左右任何一根颈动脉的方式,会怎样呢?

血会迸发出来喷溅到天花板或壁面上,约过五秒就神智不清,脑的功能停止,十几秒钟后因流失大部份血液而死去(某项研究认为是十二秒)。

有的人认为当割断一面的颈动脉时,另一面的颈动脉还向脑部输送血液,约有三分钟的时间还是有意识的。

但立即死亡的说法可靠性较大。

马拉松长跑运动员圆谷幸吉\footnote{运动员圆谷幸吉(1940–1968),生于福岛县须贺川市。1968年1月9日在宿舎用双刃刀片将颈动脉切断自杀身亡。}用割断颈动脉的方法自杀,是广为人知的事情。他是躺在床上, 用双面安全刀片割断又颈动脉的。

还有和手一样,割断位于比动脉浅的地方的颈动脉的方式,也会大量出血,随着动脉流动顺 势吸入空气,因进入血管的空气阻塞肺部而死亡。

同样地,空气进入气管就会窒息而死。
不 过,即使割了颈动脉,如手静脉一样因为出血停止而不至于断气的,也不在少数。

\subsubsection{其他}

切腹时,首先会割断许多不足以致命的腹部细小血管,然后刀刃伤及小肠,大便溢出而引起腹膜炎死去。

当然不会马上死去的,据说要在三、四天以后才能咽气。职业摔跤运动员力道山\footnote{力道山(1924年11月14日-1963年12月15日),在日朝鲜人,是二次世界大战后日本最具代表性的职业摔角选手,也是将摔角引进日本的先行者,被誉为“日本职业摔角之父”。因与极道发生争执,被刺杀,负伤后饮食无度,因腹膜炎而死,时年四十。}的死,也是因为刺伤后的腹膜炎。
当刀子插入腹部时会引起腹膜休克,一般情况下痛得不能再继续下去。

可是,如深刺可达背部的话,那就割了背骨前面的大动脉,会大量出血当即死亡。
不分男女 老幼,用这种方法死亡的还真不少。

这时,刀子不要一直插在里面而应立即拔出。
否则,刀 子会起堵塞血管的作用。 

刺穿心脏的场合,刺得准的话会当场死亡,但心脏的组织结构是相当坚韧的。
用厚刃菜刀自 杀的一个家庭主妇的心脏就有三个窟窿,这就是说,不是一下子就死了的。

企图用三根五寸 钉 子 钉 死 心 脏 自 杀 的 一 个 木 匠 ,因钉 子 堵 住 了 伤 口 而 避 免 了 大 量 出 血 ,送 到 医 院 后 在 手 术 中 因出血而死掉。

不管哪种方式,一般来说靠出血自杀是有一定的难度的。

\subsection{尸体状况}

\subsubsection*{「血海」只是形容词}

割断颈动脉的场合,有些书籍形容为「周围一片血海」,其实不过是有数公升的血流到地板,有时溅到天花板或墙壁上,至少是一滩血,但地板上全都是血的情况是没有的,圆谷运动员的房间也没成为一片血海。

切腹的场合,用刀横向剖腹的话,会有一桶之多的肠子流出来,样子极度的难看。

\subsubsection*{注意}

割手静脉是死不掉的。如果一定想割断手静脉而慢慢地愉快死去的话,最好喝些酒洗个澡,使血液循环畅通,在浴缸里把手放在心脏以下的位置,注意不要让血凝固,而坐等死神的来临。

稍为昏迷就有可能溺水,这时又恢复了知觉,这并不是什么安乐死,反而尝到呛水窒息的痛苦。
因此,要注意保持不要溺水的姿势。

还有,在割脖子的时候最好取侧仰卧姿势,使心脏的位置保持在上面。
当 血 液 缓 慢 流 出 的 场 合 ,在 此 期 间 内 与 维 持 生 命 无 直 接 关 系 的 其 它 内 脏 会 补 充 一 些 血 液 ,是 不流掉近百分之七十的血液就死不掉的情况。

这就是说,靠割手静脉死亡是不可能的。 
万 一 因 割 手 静 脉 而 死 亡 的 话 ,大都 是 静 脉 吸 入 的 空 气 堵 塞 了 脑 或 肺 的 血 管 ,乃 是 因 为 空 气 栓 塞而死的。

不过,这种情况极其少见。 

\subsubsection*{自杀骗局的费用}

原先就想到未遂的后果而去割腕时,担心的就是伤疤。
只割了静脉时尽管很深,但只不过一 道白痕,看上去和手腕皱纹差不多。
但割断动脉的话,也会割到下一层的腱和同一层的手腕 中央的正中神经。
虽然割断了腱和神经,但做手术可接好的(案例 12)。
不过,这种情况, 会留下纵向切开的痕迹,手术后的疤痕是显眼的。
整形外科手术会有帮助,但医疗费却需数 万至十几万日元。

割 断 手 腕 静 脉 的 缝 合 ,需 要 的 费 用 是 负 担 三 成 的 医 疗 保 险 约 需 三 万 日 元 。
作 为 舍 命 游 戏 的 代 价,还算是便宜的。

\subsection{案例研究 11}
\subsubsection*{割手动脉和神经的女高中生的感想}

1985年,一个十六岁的女高中生用小刀割腕。

伤口伤及动脉,出了一公升血,但以未遂告终。

对当时的情况她是这样说的:「动手割了到底是痛的,血流了不少,但一直是清醒的,所以又割了好几下。
真痛啊!割到粗血管时血就喷了出来,这时还有一种『嘶嘶』的声音,我 还 以 为 就 这 样 会 死 的 但 还 是 不 行 ,于是 我 又 割 了。

这时 好 象 割 到 神 经 似 的 ,感 到 麻 酥 酥 的 。 
被送往医院接受治疗时在想最好给我打针麻醉药,因为,对疼痛已经反感了。」

她从中学二年级时起,在大家面前还是有说有笑的,但只剩下和朋友两个人时却没有了话题,总感到自己不会说话而苦恼着。曾经想去神经科检查,但始终没能说出口。

当决定自杀时的状况是这样:「吃完晚饭,因有作业我就在想『不做不行,不做不行』的过程中忽然想到『死掉』的话不就没有这样的事吗?在这种情况下逐为变为『不死不行,不死不行』的心情了。」

同时,她说:「想到动手是剎那间的事」后,继续说:「好久以前就想过,真的死去痛苦就 没有了。虽也想应该活下去,但死掉了就不必做那些不愿意做的事,不必想那些痛苦的事了。」 当知道死不掉时:「我在想是不是有人太早发现我啊!」

结果,她在精神科病房住了四个月,进行了身心的调养后终于出院了。

\subsubsection*{检验案例}

看来她是割了手动脉和中枢神经,割到这种地步是非常痛的。

她的「死去的话痛苦也就没什么了」的想法很有洞察力。

\subsection{案例研究 12}
\subsubsection*{割了肘内侧的罕见自杀未遂者:中森明菜}

1989年7月1日下午四时半左右,走红歌星近藤真彦回到公寓时,发现爱人,同样是走红歌星的中森明菜\footnote{中森明菜(1965年7月13日-\qquad \qquad ),出生于日本东京都大田区,于东京都清濑市长大,日本知名女歌手与女演员。}(当时二十四岁),用剃腋毛的剃刀割了左肘关节内侧,倒在血泊里。

在她倒下的浴室流了约五大杯的血,其人神志模糊不清,看来她是在被发现前不久割腕的。

她立即被送往慈惠医科大学医院,伤口长达八公分,深两公分,因割了静脉和正中神经,缝 合手术进行了六小时之久。
手术施行全身麻醉,血压一度降到六十左右,输了$600$c.c. 的血液。手术后为了不使静脉和神经再度断裂,胳膊用石膏绷扎了一段时间。 

后来她在中伊豆的温泉进行调养,现在完全恢复了。

当初所担心的手指不能弯曲及大姆指、食指、中指丧失知觉的后遗症未出现,伤疤也看不出来。

自杀未遂的动机,人们认为是与近藤之间的感情摩擦。

\subsubsection*{检验本例}

这 是 割 肘 部 内 侧 血 管 的 自 杀 未 遂 例 子 ,很 容 易 被 认 为 是 无 聊 的 假 自 杀 骗 局 。

但八 公 分 的 长 度 是肘内侧的一端到另一端的长度,而且深度有两公分,是相当程度的重伤。
这个伤口肯定张 开得很大,如果割到动脉的话,绝对会危及生命的。 

掌管手的运动和感觉的正中神经在皮肤下面约一公分之处,附近还有一根较粗 的 动 脉 也 在 皮 肤 下 一 公 分 处 ,对自 杀 来 说 是 再 好 不 过 的 地 方 。

她 未 割 到 这 一 动 脉 可 真 算 是 个 奇迹,有的医生说:「应该不是纵向割的?」

不过,一般却认为,说是深两公分也只是开口部分有两公分,实际上没有割到两公分深处。

两公分,那就是可达到骨头的深度。

\subsection{案例研究 13}
\subsubsection*{刀砍全身也未死的日商岩井岛田常务}

因道格位斯‧格鲁曼难于判决的案件而榜上有名的商社——日商岩井的岛田三敬常务(当时五十六岁)的尸体,于1979年2月的一个早晨,在岛田任社长的该社子公司日商岩井大楼下面被发现。所谓道格拉斯‧格鲁曼难于判决案件就是,当时和洛克希德事件一起把政界、财界卷了进去而成为话题的案件。

岛田常务的尸体,在右颈和左手腕上有用刀子割过的痕迹,胸部有数处用锥子刺过的痕迹。 
死 因 是 出 血 及 外 伤 性 脑 机 能 障 碍 ,脑机 能 障 碍 是 因 为 从 大 楼 七 楼 社 长 室 跳 下 来 时 造 成 的 。

穿著衬衣和过膝衬裤,上罩西服外衣,袜子则只有右脚穿著。 
他在前一天的晚上十时半左右,对留下加班的社员说了句「可以走啦」后,整理了房间,脱 了衣服,用刀子割了颈部和手腕,又用锥子在胸部扎了几处。可是都没达成致命伤。

经过数 小时的痛苦之后,最后用足了劲爬上七十公分高的窗台,从七楼窗口跳了下去。

左脚的袜子 留在窗边,大概是爬过窗台时因血而滑倒就脱落下来的。 
距社长室五公尺之遥的洗手间里也留下了血迹,这是由于大量出血所引起休克所造成的失禁状态,才去了洗手间的。

岛田与死神的搏斗是极其凄惨的,因出血而几次昏迷,每当恢复知觉时又在身上乱砍,如此重复多次,看来花费了很长时间。

房 间 内 的 地 毯 自 然 被 血 染 红 了 ,刚 打 开 的 山 得 利 威 士 忌 的 酒 瓶 和 茶 碗 散 放 着 ,桌 子 上 好 象 表 明其生前性格似地整理得很干净,留下了九封遗书。在沙发中间整齐地放着裤子、外套、围 巾、衬衫等,这说明他对自杀作了周密的准备。

他被作为当时事件的重要证人,为后来被捕的海部八部前副社长的左右手,因此,无疑是与这次事件有关的自杀。
自杀的前两天,海部还举行记者招待会。

留给社员的遗书写着「公司的生命是永存的,为了永存我们应该奉献。」

\subsubsection*{检验本例}

这是生动地告诉人们靠自残进行自杀是困难的例子。

事实上,自残自杀和服毒自杀一样,未遂率特别高。

这个人先割手腕,再割脖子,又改用锥子刺胸部,其实他一开始就选择跳楼的话,就没有必要吃那么多不必要的苦头。

有一点不应忽略的,就是当他用刀乱砍自己的身体而快要昏迷时,去了洗手间。在其它自杀手段中,常有因休克但神志还清楚时出现脱粪现象。

既然能有整理房间、摆好遗书的周密准 备 ,那么 也 就 应 该 先 去 趟 洗 手 间 。尽管 是 自 残 ,但 也 不 该 盲 目 地 乱 刺 乱 砍 ,这样 是 死 不 掉 的 。

但是,不愿别人看到失禁的痕迹,而在自杀过程中去了洗手间,也真是稀奇。

\newpage

\section{Crashing 撞车}

痛苦 ▼▼▼▽▽

麻烦 ▼▽▽▽▽

死状 ▼▼▼▼▼

牵连 ▼▼▼▼▼

冲击 ▼▼▼▼▽

致命 ▼▼▼▼▼

尸体会撞的血肉模糊;电火车停驶会造成很大的妨碍。不过这个方法在决心想死的时候,确实可以死亡,所以越来越受欢迎。

拖着极端疲劳的身子离开公司或学校,只要想到明天,心情就格外沉重。
这时你站到了月台上,听到电车到站的广播。不愿再想以后的事了,真想在这里做个轮下鬼把一切都结束掉……
脑 海 中 产 生 这 样 想 法 的 人 该 是 不 少 。

明 明 知 道 将 把 凄 惨 的 尸 体 暴 露 在 人 们 面 前 ,但 撞电车而死的人却不断出现,恐怕是这种方法对冲动的自杀欲望是再合适不过的缘故。 提到撞车自杀,最简便的就是铁路自杀,其次就是在马路上撞卡车了,但据1991年警视 厅的统计,在车站内自杀者为$179$人,铁路卧轨为$787$人,马路上的自杀者仅为四人。

换 句 话 说 ,除跳 海 、跳 湖 以 外 ,百 分 之 九 十 九 以 上 不 是 撞 一 般 车 辆 而 是 撞 电 车 和 火 车 。其中 , 不在车站而在铁路上卧轨的居多。

铁路卧轨要比一般撞车多,可能是因为这种方法的「致死率」高的缘故,与上吊、跳楼一样是必死无疑的手段。
要想撞车而死,那就毫不迟疑地去撞电车吧!

根据旧国铁首都圈本部的统计,铁路卧轨最多的月份是七月和八月,时间是傍晚的六点到七点,与其它自杀手段的统计略为不同。

性别中男性占$77\%$。还有一点有趣的是,选阴天的自杀者较多。

\subsection{准备}

\subsubsection*{在快车经过的车站等候}

在车站的月台跳下时,应选特快或直快车辆的「不停靠车站」。
正在减速的电车,其致死率也会降低的,要是停靠站,那就应选月台的最远程电车进站处。

突 然 冷 不 防 的 跳 下 会 给 周 围 的 人 造 成 影 响 。
虽然 不 坏 ,但 时 间 掌 握 稍 迟 就 会 碰 撞 到 车 头 而 被 弹出,留下一条生命,跳的力量过大时,还有可能落到轨道另一边。

没有慌张的必要。驶入车站的电车,来到你面前一百多公尺时,即使急煞车也是来不及的。 

电车是一面煞车一面驶来的,所以紧急制动器起不了作用。
慢慢地跳到轨道上横卧在那里时, 不 会 有 人 来 阻 止 的 。

一 个 四 十 二 岁 的 公 司 职 员 从 月 台 下 到 轨 道 ,并俯 卧 在 那 里 等 待 电 车 的 到 来,最后电车辗过了他,造成头部、右腿轧碎,当场死亡。 

一辆山手线电车以时速五十七公里的速度驶入东京站,因三十公尺前方一个男子跳轨而紧急煞车,但由于惯性仍向前驶过了一百多公尺,男子浑身碾碎,后脑脑浆迸出,当场身亡。

这时应该记住,电车的警笛会特别地响,简直要把耳朵震聋了。

\subsubsection*{选择夜间行事}

在离车站一段距离的轨道上卧轨的时候,要尽量选在驾驶员不易看到的转弯处。
事先乘坐电车,在车头观察驾驶员不大留意的地方也是个办法。

找到了合适的地方,可安心地卧在轨道上等待电车的到来。不过,电车迫近的时候会听到隆隆的轰鸣声,据说以一般的精神状态是无法继续留在那里的,所以最好服用一点酒类、安眠药、镇静剂等较好。

可是也有面对时速$200$公里的「光」号列车\footnote{“光”(ひかり)是一种在东海旅客铁道(JR东海)东海道新干线和西日本旅客铁道(JR西日本)山阳新干线运行的特别急行列车(特急列车)班次的名称。从东海道新干线正式通车时就已经设立这列车等级。曾经是全线最快的列车,今已被希望号所取代。},从静冈站上行线月台跳到轨道伫立不动,被火车辗过的高中三年级女学生,真是令人甘拜下风的有胆量例子。

在 一 般 路 线 上 电 车 还 更 难 停 下 来 的 。
一 辆 以 时 速 $90$ 公 里 行 驶 的 电 车 ,当发 现 前 方 $200$ 公 尺 处 有 人 影 而 紧 急 煞 车 ,但 电 车 停 下 来 的 地 方 却 是 辗 了 那 个 人 后 又 行 驶 了 $200$公 尺 。

据资料说,电车紧急煞车之后到停下来的距离是煞车时的速度。

从时间上说,绝对应选择夜间。周围的人也好,驾驶员也好,都不大容易发现。 
进入路线区有困难时,或从平交道口迅速跑入,或从人行路桥上跳下亦可,从平交道口跳入 的情况也不少,进入线路区的方法其实很多。(案例 16)

\subsection{经过}

\subsubsection*{确实立刻死亡}

一开始就横卧轨道的话,凡是横在轨道上的头、腹、胸、手脚某部分都会被辗断,只要胸部和头部被辗断就会当场即死亡。

如果衣服被车辆钩住的话,会被拖上十多公尺,然后身体多 次 翻 滚 而 各 部 分 也 被 多 次 辗 过 ,粉 碎 的 尸 体 散 见 在 数 百 公 尺 内 。从月 台 跳 下 或 从 平 交 道 口 闯 入时,多半会出现这种情况。 电车的动能极大,有时鞋子或皮包会碰到在月台上的人,该情况显示人已死亡了。

一个男子飞身去撞时速$85$公里的电车,结果双手、双腿、躯体被压得粉碎,分散在一百公尺范围内,头部更是不知去向了。

被新干线火车压断的情况更惨,好像被搅拌机搅过似的。
在相模平野附近,一个男子突然在时速$200$公里行驶的「光」号列车的$300$公尺前面,背对火车而蹲在路轨上,只见一阵尘 土 飞 扬 似 地 鲜 血 四 溅 ,大体 完 整 的 肝 脏 、部 分 头 皮 和 头 发 、三十 公 分 左 右 的 连 骨 盆 的 脊 椎 、 脖颈和下颚的一点皮、右手腕、三颗牙齿等散落在四五百公尺之内。

其它部份则成为肉片和 骨片飞散到远处了。
火车是在煞车以后又向前行驶了三、四公里才停了下来。 

\subsubsection*{如雨般的骤下红雾}

前述那位站立在轨道上被「光」号列车撞死的少女,她的上半身撞到三十公尺外的月台墙壁 后 又 弹 回 月 台 ,接 着 衣 服 和 太 阳 眼 镜 也 飞 了 过 来 ,周 围 呈 现 一 片 血 雾 。

下 行 月 台 的 贩 卖 部 说 , 只听到轰隆一声沉闷巨响,顿时眼前一片红色。

那么,究竟是什么力量使身体的碎片或物品 被弹出去的呢? 碰撞到普通电车的前部,如果是行驶中就会被弹出五六公尺,这时因心脏瞬时停止而死亡的 例子很多。
这种情况下如果被弹出后仍落到路轨上,身体又会被压断,假使摔到路轨外面, 不幸的是仍有可能保住生命。

\subsection{尸体}

在所有的自杀手段中这是最惨的一种,对此应有心理准备。

不过,不在车站或平交道口撞车, 尸 体 是 不 会 被 一 般 乘 客 和 看 热 闹 的 人 看 到 的 。
被 撞 飞 出 去 的 尸 体 ,有时 表 面 上 也 没 有 什 么 损 伤。

这种场合,内脏的破坏却很厉害。 
车 站 工 作 人 员 在 十 分 钟 左 右 就 处 理 好 尸 体 ,把周 围 洗 刷 干 净 ,但 经 过 三 十 分 钟 却 还 找 不 到 被 压 断 的 手 或 脖 子 。
某 自 杀 者 的 脸 部 贴 到 货 车 上 ,从 福 岛 县 到 北 海 道 行 驶$900$ 公 里 后 才 被 发 现。
也有在山口县撞上的女子手臂,经过$900$公里的距离到横滨才被发现的例子。 

虽然尸体运走了,电车也恢复行驶了,但零碎的东西散见各处,枕木和铺路道上依然血迹斑斑,留下凄惨的痕迹。

在车站区以外的地方卧轨压死的话,较不会留下被压成两、三段的尸体。

总之,对想完完整整死的人来说,这是绝对不适合的方法。

\subsection{注意}

\subsubsection*{轨道和尸体应成直角}

有 时 运 气 不 好 ,手 、脚 都 被 压 断 但 未 达 到 致 命 ,而靠 义 肢 过 后 半 生 情 况 也 是 有 的。(案 例1 5 )

所以一定要想法让脖颈或胸部卧到轨道上。

有时身体会夹到轨道中间,奇迹般地只擦伤了一点皮毛(案例 14)。

因此,使身体和铁轨 成直角交叉,是卧轨自杀的基本要领。 

\subsubsection*{要做支付巨额赔偿金的准备}

铁 路 自 杀 会 带 来 巨 额 的 赔 偿 费 。

据 一 九 八 五 年 旧 国 铁 的 统 计 资 料 ,电 车 的 一 次 紧 急 煞 车 会 造 成$6000$~$8000$日元的电力损耗和约一万日元的车轮磨损费。

电车停一次,以乘客较多的东京山手线而言,每十五到三十分钟就损失数十万日元,如果是尖峰时间就会造成一百到两百万日元的损失。

新干线的损失费用更大,耽误一小时就是数千万日元。

此外 还要加上死者收容费。

也有连车撞上特快电车而被索取三千万日元赔偿费的例子,和同样连 车 撞 入 东 海 道 线 而 被 索 取 一 亿 四 千 万 日 元 赔 偿 费 的 例 子 。

了 解 内 情 的 人 流 传 着 说 ,因考 虑 到 舆论而不索赔的。
但日本铁路公司也不是好说话的。
虽然不提起公诉,但诉讼还是进行的。 
只是死者家属的联系地址不清楚的情况较多,索赔不一定能成功。 

这样看来,卧轨自杀并不是划算的方法,不过如开头所言,当你忽然想到自杀,眼前呈现的 就是这个方法。

今后的事情就让它去吧,现在马上想死,对上述这种人来说,当然是再好不 过的手段了。

\subsection{案例研究 14}

\subsubsection*{在撞车中奇迹般生还的OL}

1991 年12月29日下午五点多钟,一个二十四岁的女性从西船桥站前方五百公尺处
的天桥,跳向武藏野线行驶中的电车企图自杀。

她是千叶市内某计算器公司工作的独身OL,当时是在下班途中。

天桥高七公尺,只要跌得巧就是没有电车通过也会死亡的。她是在电车驶 进 至 靠 近 自 己 十 公 尺 左 右 处 跳 到 路 轨 上 ,接 着 电 车 驶 过 ,驾驶 员 也 感 到「这 下 可 压 着 了 !」 电车紧急煞车后停了下来,她在倒数第六节车的车底下,但她的身体刚巧夹在两根铁轨中间,没有受伤。经医生诊断,头部和腰部受了需治疗一周的挫伤。

她之所以得救,是因为体格较小,背部未被车辆钩住和电车进站前减速的缘故。
否则的话,身体会被卷入而被压死。

她虽不讲自杀的动机,但据其友人说,最近与一个男人的不正常关系有关。

\subsubsection*{检验本例}

又跳楼又撞车,这是同时利用两种高度致死的手段,却未死亡的稀有例子。
本来从七公尺的高处跳下来也会受到相当严重摔伤,再加上电车又在上面通过,最后竟只有仅以轻度挫伤,这确实是个奇迹。


\subsection{案例研究 15}

\subsubsection*{辗断单手双足而活下来的铁路自杀未遂女性}

T 于 1955 年 2 月 企 图 撞 车 自 杀 而 失 败 了 ,当时 她 是 十 六 岁 的 高 中 二 年 级 学 生 。

晚 上 十 点 多钟,她冲向从小田急线新宿车站刚发车的快速电车,由月台前端「像被铁路吸引似地跳了 下去」。

电车急速煞车后停下了。

T的右脚踝、左脚膝下部分、左手肩部十五公分以下,右手的无名 指和小拇指都撞断了。
剩下来的只有左手的拇指、食指和中指三个手指。

当时T拼命叫喊: 「热啊,热啊!」,不久即丧失知觉。
被救护车送进医院,昏迷了一个星期,等恢复意识并 知道自己情况,则是事故发生后的第十天了。

T曾绝望的想:「只有再死一次」,后来相信了基督教牧师的话和圣经而决定生活下去。
八 月份出院,双腿都装了义肢。第二年与这位牧师结了婚,有了两个孩子,现在她正从事宣扬生命可贵的讲演活动。

\subsubsection*{检验本例}

从 站 台 的 前 端 向 刚 起 动 不 久 的 电 车 冲 撞 是 她 的 失 策 ,但 脖 颈 、躯体 等 会 造 成 致 命 伤 的 部 位 未 被轧断,恐怕也只是偶然现象。

但从该例亦可知道电车速度应该是愈快愈好,但是,即使是 刚发车的电车仍具有撞断手脚的威力,这是告诉人们「铁路自杀失败的话就是这样」的稀有 例子。

\subsection{案例研究 16}

\subsubsection*{青函隧道中压死的女性}

1991 年 3 月 ,在 青 函 隧 道\footnote{青函隧道(青函トンネル)是位于日本津轻海峡的海底铁路隧道,为世界上最长的海底隧道。} 内 的 吉 冈 海 底 车 站 附 近 发 现 了 一 具 二 十 六 岁 女 性 尸 体 。双 手 双 腿 前 额 以 上 完 全 被 切 掉 。

可是 ,吉 冈 海 底 车 站 是 青 函 隧 道 中 两 个 无 人 车 站 中 靠 近 隧 道 一 侧 的 防灾用车站,电车是不停靠的,不是一般人可随意走进的地方,她怎么会在这样的地方死掉呢? 调查结果表明,这位女性自大前天起就去向不明,十八日决定回东京的家里,事情发生于从 札幌回到家里的途中。

她不久以前就有点神经衰弱,不知在什么地方下了决心要死,乘上同 日 二 十 二 时 发 车 的 快 车 ,第二 天 三 时 五 十 分 左 右 通 过 吉 冈 海 底 车 站 时 ,趁 列 车 减 速 之 际 爬 进 了车掌室,打开窗子跳出,再卧到另一侧的铁轨上,被四时左右通过的货车压死。 

十八日她给家里打电话,告诉父亲说:「就要回家了。」

\subsubsection*{检验死因}

横卧在铁轨上等死,尤其在都会区域是非常困难的。但她所发明的「隧道自杀」,在都会区是可以采用的。到处去寻找都市的死角,也是为铁路自杀作准备的乐趣,地下铁看来也有许多不为人知的好地方。

要想走进行驶路线内,主要条件是周围一定要黑暗。在都会区进入平交道口也不要让人看到, 晚上可能性就大些,驾驶员也难发现。

事实上这个女性尸体,是一个半小时以后的五时四十分左右才发现的,也就是说驾驶员完全没有意识到。 

前 额 以 上 完 全 切 断 也 是 重 要 的 一 点 。
可以 想 象 到 列 车 车 轮 所 具 有 的 巨 大 切 割 力 。
人 们 推 测 她 是横卧在路轨上,这是从尸体状况得出的。

如果飞身撞车的话,被辗得粉碎的情况较多,横 卧 的 话 唰 地 一 下 就 割 成 两 段 了 。
对 企 图 在 车 站 以 外 的 地 方 卧 轨 自 杀 的 人 来 说 ,这有 很 大 的 启 发。

在西村京太郎的推理小说中,也有假设吉冈海底车站发现女性尸体的情节,或许她看过这部
小说也说不定。

话虽如此,在地面下$150$公尺深的海底,在刺骨寒冷黑暗中,能够横卧在路轨上面等待十
来分钟,令人难以想象,超过一般惧怕的极端恐怖经历,对平常人来说是难以想象的。

自杀前的心理,实在是难以捉摸。


\subsection{自杀地图 3:三原山}

与其它自杀名胜一样,现在特意跑到三原山去自杀的人恐怕是不会多的,可是,火山却有吞没尸体的好处。

火山口内却是一种秘境,而且不像树海那样进行尸体大搜索,从这一点来说还是有其利用价值,故加以介绍。

\subsubsection{历史}

位 于 伊 豆 诸 岛 的 大 岛 三 原 山 ,怎 么 会 成 为 自 杀 胜 地 呢 ? 现 在 知 道 的 人 还 是 不 多 ,起因 要 推 溯 到六十年前的神秘案件。

1933年2月24日,二十一处的实践女子高等学校的两个学生登上了三原山。
其中一个人说了一句「向大家致意」就跳进了火山口,而另一个人正在犹豫不决的时候被人救了。

两个人原是准备一起自杀的,仅此一点就够使周围震惊的了,经过调查知道这个女学生在一
个月以前的一月九日也曾和另一朋友登上三原山,同样地使朋友只身自杀而她本人则返回来
了。

\subsubsection{死亡引路人}

事件轰动了社会,不知是何原因,以青年男女为主的自杀志愿者涌向三原山,三个月的时间里产主了自杀者$32$人,未遂者$67$人的惊人事件。

真实的情况是,这个成为见证人的女生在一个月以前受人委托陪人自杀,一个月后,有人偶然向她吐露了 想 自 杀 的 念 头 ,于 是 她 介 绍 了 三 原 山 的 事 ,岂 知 那 人 再 三 逼 她「 做 个 引 路 人 」, 不得已只好带友人去火山口。

结果就在这一年,三原山出现了男$804$人,女$140$人,合计近千人的自杀者,甚至还出 现在山顶见面后感到意气相投而一起跳进的,也有对游览者说声「大家再见啦」而跳下的男 子。

于是,三原山一下子就成了自杀胜地。

\subsubsection{跳入方法}

从 火 山 口 四 周 的 休 息 处 或 纪 念 品 店 的 山 顶 口 朝 内 轮 山 走 去 。

火山 口 四 周 ,只 有 内 轮 山 山 顶 禁 止入内,无法窥视火山口内部,想跳的话,也只是翻过栅栏后剎那间的事。 

跳火山口的时候,在半路上挂在岩棚上或者即使摔到火山底但达不到岩浆地区的情况也很多(案例 17)。

不过也会因吸入硫酸而昏迷,加上翻滚掉下时的碰伤,数小时便会死亡的。 

虽不是三原山的事,1948年追随太宰治\footnote{太宰治(1909年6月19日-1948年6月13日),本名津岛修治,生于日本青森县、无赖派小说家。1948年6月13日深夜与崇拜他的女读者山崎富荣跳玉川上水自杀,享年38岁。}之后留下「把我也带去吧」的遗书而投身到阿苏山的青年,悬在$150$公尺的岩石上,最后被拉了上来。

事实上,据调查数据显示,从阿苏 山口往下$240$公尺处,岩石的温度是$100$度,空气$是65.6$度。 往下跳的时候要注意,不要让山顶上瞭望台上的人发觉。

\subsubsection{交通·住宿}

从东京、横滨、热海、伊东有船去大岛,从羽田、调布还有飞机直达,如到三原山山顶口,可利用巴士或出租汽车。

住宿则在三原山第八段的大岛温泉旅馆,港口的元町和冈田也有大小旅馆,还有农家客栈,住些日子再自杀也不错。


\subsection{案例研究 17}

\subsubsection*{从三原山火口壁爬上来的男子}

1956年12月3日下午一时左右,一对年轻男(29 岁)女(26 岁)从三原山双双跳下 约 六 十 公 尺 深 的 火 山 。三 原 山 是 在 当 年 一 月 间 曾 大 规 模 地 喷 火 ,这两 个 人 跳 下 的 时 候 还 可 看 到岩浆喷出造成的新火山口。

接 到 消 息 的 救 助 员 冒 着 喷 发 着 难 以 忍 受 的 热 气 和 亚 硫 酸 气 下 去 营 救 。

女 子 的 腿 陷 进 了 三 处 冒 着溶岩火炎中的一处,动弹不得,已经没有营救的可能,男的则稍离火炎,满身是血在呻吟 着。

由于他还有知觉,营救人员就背起了这个腿部受伤的男子,设法搬到十公尺以上突出岩 石比较安全的地方。

到这里为止,已经到深夜,所以就折回到火山口岸,这时已是三点十五 分了。

可是第二天清早再次前去营救时,男子已经靠自己的力量爬上了火山口岸,倒在那里。

完全无法使用腿的他,撕裂了围巾包扎了头部和手,用手在火山壁上挖出两个洞,把膝盖放到洞 上后爬上,再挖两个爬上,不断地重复这个动作,完全爬上余下的倾斜度约七十度的火山口壁,时间是上午四点左右。从跳下开始经历了十五个小时的死斗。

他的面部因血液凝固而变黑,左眼青肿,但只受了点轻伤,放在身旁的围巾已变黄而破碎不堪。 

死 亡 的 女 子 于 上 午 十 一 点 多 钟 被 拉 了 上 来 。

左 腿 部 和 左 膝 以 下 烧 得 没 有 了 ,腿部 因 充 满 气 体 而鼓胀,面部和手则没有变化。

自杀的动机,据说是为了清理数年来的三角关系。

\subsubsection*{检验本例}

生命力是惊人的。当他被营救队员留在岩石时,「曾想再跳一次,但这时孩子的面孔浮现在眼前,要活下去的信念涌了上来。」事后他这么说,而这种人从一开始就不该去尝试自杀的。

跳下倾斜度七十,深六十公尺的断崖,也只受点伤却是个意外,但投身火山口直接跳进岩浆里的是极稀有的,大半都在半路上撞到岩石上再坠落到火山口底,因热气和瓦斯而死亡。

关于火山喷火口的可怕,人们谈论得够多了。他形容攀登火山壁时的情景:「由于猛烈的热气和喷上来的硫黄烟雾,面部发烫,喘不过气来。」

女子圆满地触到岩浆,达到了目的,但尸体是惨不忍睹的。

这个故事说明了,在火山口投身自杀时,跌落的位置和想死的意志是非常要紧的。

\newpage

\section{Gas-Poisoning 瓦斯中毒}

痛苦 ▼▼▽▽▽

麻烦 ▼▼▼▼▼

死状 ▼▽▽▽▽

牵连 ▼▽▽▽▽

冲击 ▼▽▽▽▽

致命 ▼▼▼▽▽

从汽车排气口接一条管子通到车窗内,并将缝隙堵住,相当费时费事,不过可以死得轻松, 死状也不太难看。 
为避免出差错我要先讲清楚的是,靠城市煤气是不大可能那么痛快地死去。

在房间里扭开煤气开关,然后倒下,逐渐逐渐地昏迷……这已经是过时了。现在的城市煤气,因煤气公司已 经将过去的煤气变换为不含引起中毒的一氧化碳的天然气,除部分地区外是不会发生中毒死 亡 的 。

液 化 气 也 不 含 一 氧 化 碳 。即 便 是 死 ,也不 过 是 因 缺 氧 窒 息 而 死 罢 了 。
中 毒 死 和 缺 氧 死 , 感受到的痛苦是截然不同的。

一般来说,中毒死还是比较舒服的死法。
能在房间里轻而易举中毒自杀幸福的七十年代,已经一去不复返了。 
不过含有一氧化碳的汽车排放气,还是具有致命力。
实际上,现在的「煤气自杀」大部分都被这种排放气体所代替了。

如果想用煤气或丙烷气,那只有准备承受比死还难受的痛苦,采取用嘴吸住煤气管,在狭小的室内放满气体,最终因缺氧而死的方法。
运气好的话,就像十七岁的女孩那样,在管子打开的瞬间达到窒息死。
话虽如此,有的家庭装有安全装置,大量排放数小时后自动停止排放,加上还有爆炸的危险性,因此,最好不要考虑在家里进行煤气自杀。

如你要想因缺氧而死的话,完全没有必要在室内充满气体,从头上罩个塑料袋就行了。
看起 来虽然有点原始,但每年有一百人以上是用这个方法自杀的。

塑料袋之外,再同时并用药物 自杀等其它方法,则效果更好(案例 19)。
 
同样地,虽不是中毒死,但也有在室内放满气体而点火炸死的人。

该方法简便,适合于冲动 型自杀的人。但是,炸死固然有当场死亡的效果,事实上全身烧伤、痛了好几天再死的例子 也不少,对周围造成的危害较大,是否顺利完成也没把握,有时仅仅引起一场火灾而已,所 以不值得推荐。

\subsection{准备}

在 汽 车 里 的 自 杀 方 式 ,首 先 要 准 备 三 至 四 公 尺 长 的 橡 皮 管 和 胶 布 。

橡 皮 管 应 根 据 排 气 口 的 粗 细,在杂货店购买即可,粗细不对时可用胶布多缠几层,这样既不会漏气也不会脱落。

然后 将橡皮管接到排气口,用胶布牢牢地加以固定,再将橡皮管从车窗插入,窗缝亦用胶布严密 封柱。

这样车内便成为密封状态,坐上去关上门发动引擎,再把座椅放倒,听车内音响播放 的音乐,而慢慢地走向永久睡眠的路。 一定要把汽油加满。

如果近邻不注意的话,汽车停在自己的车库也可,但在人们看不到的森 林中较好,过了季节的度假地因乏人问津也是极好的地方。

或者罩上汽车车套,这样即使在 听音乐也不会被人发现音响的指示灯。

在 房 间 自 杀 时 ,首 先 要 确 认 所 使 用 的 气 体 是 否 含 有 一 氧 化 碳 ,同时 也 要 查 看 一 下 有 无 安 全 装 置。

剩下的就是不论一氧化碳中毒死还是缺氧死,要把窗缝严密封死,在窗缝和门缝处多贴 一些胶布,准备妥当后打开开关睡下就可以了。

丙烷气的方式,因为比重大于空气,因此必须在地板上横卧。相反,煤气比空气要轻,因此尽量在高的地方较好。

当然具有把橡皮管塞在嘴里的勇气,那是再好不过了。利用这种绝断手法的人意外地多,作家川端康成也是这样自杀的,也有把头塞进煤气炉里死掉的例子。

头上罩住塑料袋的方式,不会漏气就万无一失了,在脖颈处绑一道绳子也是好的。

\subsection{经过}

痛苦极少随着空气中一氧化碳浓度的提高、吸入时间的增加、目眩和心跳加剧,不久就丧失意识直至死亡。
这是因为血液中输送氧的血色素,以氧的二百乃至三百倍的强度与一氧化碳结合,在血液中形成「一氧化碳血色素」,而使体内细胞供应的氧气锐减的缘故。

汽车的方式:排气中所含一氧化碳浓度大约是$0.4\%$~$0.5\%$,血中一氧化碳血色素的浓度达到$30\%$时,首先感到目眩、头痛、虚脱、疲痨、判断力降低。

达到$40\%$时,恶心、对想做的事情无力去做且有乏力感。

达到$50\%$时,皮肤因为一氧化碳血色 素的关系而呈红色,体温降低。 

达到$60\%$时呼吸急促,出现失神、失禁、痉挛。

达到$70\%$时呼吸停止很快就死 去。 

在汽车内的中毒自杀方式,半小时到一小时左右就会不省人事。

\subsection{最漂亮的死法}

在所有自杀尸体中,人们说煤气自杀尸体是「最漂亮的」。

由于含有一氧化碳血色素的血液使皮肤呈现粉红色,你的尸体将以被染成粉红色的状态被发现。

出现痉挛现象时衣服可能会紊乱,但没有关系。

不过,常常会发生失禁情况。在浴室以一氧化碳中毒死亡的男性家中,在厨房和床上都有粪 便。
因为死者的直肠没有粪便,所以知道是他的。

当时因空烧洗澡水而产生一氧化碳,在厨 房和床上都憋不住而排粪后,再到浴室之后死亡的。

一般而言,即使是神志清醒也会出现失禁。
要给人看到干净尸体的人,事前去趟厕所绝不可少。

\subsection{注意}

当心爆炸在汽车里自杀的方式,因一氧化碳比空气要轻,因此气体有时会从窗缝泄出去。在这期间被发现,或者出现汽油用完的情况,所以奉劝你们把窗缝堵严实点。

虽说中毒死亡痛苦少,但在较长时间里没有失去知觉的话,头痛、恶心也会持续,所以,同时服用安眠药较好。
把气体引入车子,堵好窗缝,用毛毯挡住前窗玻璃使人看不到里面,再服下六十片安眠药的 男子,居然一直清醒着,被人发现后通知警察。

虽然他大喊「我有权利去死!」,但还是被强行拉出了车子。

并不是所有的方式都是缓慢地丧失知觉的。 

在房间灌满气体时,特别要注意的就是爆炸。
即使日光灯点光器的闪动,也会引起爆炸。

如果造成大爆炸,会背上巨额赔偿金的,而往往这种处身在气体包围中却未死的情况较多(案 例 18)。

就像有的人先贴上纸条「关紧煤气开关,打开门窗排出气体,因有爆炸危险不要 使用电气开关和火柴」以提醒他人注意,再进行煤气自杀的那样,对爆炸要有足够的注意。

在公寓中用一氧化碳中毒自杀的方式,也要留意住在楼上的人。

1976年,二十岁的女子 在 公 寓 家 中 用 煤 气 自 杀 ,结 果 本 人 未 死 ,而在 楼 上 睡 觉 的 二 十 一 岁 女 子 却 因 一 氧 化 碳 中 毒 死 亡了。这是由于一氧化碳比重轻的原因。

可是死者的家属,追究自杀未遂者的双亲、公寓房 东、警察和消防单位的管理责任,向法院起诉请求$2037$万日元的赔偿费。

在室内的煤气自杀,应该意识到是最容易给人造成麻烦的方法。

万一得救的话,和上吊一样造成脑神经组织因缺氧而损伤,可能终身留下严重的后遗症。

不过,常常所说的「一生中留下严重后遗症」这句话,不仅限于煤气自杀的。
上吊、跳楼、撞 车、药物等所有自杀手法都是一样,要想自杀,这点后果也是理所当然应该想到的。


\subsection{案例研究 18}

\subsubsection*{因煤气自杀未遂而被判刑的男子}

1978 年 4 月 ,家 住 福 冈 县 町 营 住 宅 的 四 十 七 岁 无 业 男 子 企 图 用 煤 气 自 杀 ,上午 十 点 左 右 他 陷 入 了 兴 奋 剂 中 毒 的 幻 觉 症 状 。

他 因 为 情 妇 出 走 苦 恼 而 想 到 了 自 杀 ,把放 在 屋 外 丙 烷 桶 橡 皮管拉到浴室,使室内充满气体。

可是过了很长时间也没死掉,两小时后想要吸烟而点打火 机的瞬间,气体爆炸了。

爆炸造成了住在町营住宅的二十八户人家总额达$1240$万日元 的损失,九个居民受伤,他本人则住院一个月治疗烧伤。 

他 以「 气 体 泄 漏 罪 」被起 诉 ,次 年 二 月 福 冈 地 方 法 院 饭 冢 分 院 以「 使 多 人 受 伤 造 成 严 重 后 果 」 而判他八个月的徒刑。

\subsubsection*{检验本例}

这是选择了用气体爆炸自杀的例子,但如果未遂并给周围造成损害也会被判刑的。
因为,在室内里灌漏气体本身相当于「气体泄漏罪」的。

日本刑法第一百一十八条第一项规定:「因使瓦斯、电或蒸气泄漏或流出、或阻断而对人的生命、身体或财产造成危险者,处以三年以下徒刑及十万元以下的罚款。」与此相类似的例子是在1976年,大阪某公寓的一个二十八岁公司经营者,同样在室内企图用煤气自杀未遂,导致了大爆炸,一户烧毁、约七十户受害,以「气体泄漏罪」和「重大过失失火罪」(因失火对他人住屋等造成损害罪)起诉。即使他是自杀未遂者,也不能逃脱法律责任。

从 上 述 事 例 亦 可 知 道 ,在 家 里 充 满 两 小 时 的 丙 烷 也 可 能 没 死 掉 ,却对 二 十 八 户 人 家 造 成 灾 害 的爆炸,本人却只是受了治疗一个月的伤。

在室内充满气体,马上就要断气的时候,竟然因想抽烟而点燃打火机,看来他是个糊涂虫,但实际上这种事情却还不少。


\subsection{案例研究 19}

\subsubsection*{照自杀指导手册自杀的男子}

1983年7月,一个身穿游泳裤浸在浴缸的冷水里,头上套着黑色垃圾塑料袋,脖子上扎
着绳子并系在窗台上的奇怪男子(当时四十六岁)的尸体,被人发现。

从他死亡的公寓房间里找到了四十片止痛剂「雪德丝」的空袋,看来他是服了药以后再用这种方法自杀的。

同 时 还 在 房 间 里 找 到 同 年 日 本 出 版 的 指 导 手 册「 自 杀——最 能 安 乐 死 的 方 法 」(德 间 书 店 ), 并在参考的一页夹上了一条带子。

在184页,作为能安乐死的自杀手段的搭配,介绍了四 种组合:「一、用止痛剂和塑料袋的窒息;二、因可达到中毒程度的大量药物和汽车排出气 体引起的中毒;三、止痛剂和异常低温症(长时间浸泡冷水中);四、止痛剂溺死。」

他是 选了一和三项搭配而自杀的。 

他 曾 是 东 京 室 内 球 场 的 前 身——后 乐 园 球 场 的 售 货 员 ,一 年 前 失 踪 ,后来 又 离 开 了 妻 子 和 孩 子,在一间小小的房间里过着酒鬼的生活。失踪的原因,据说是背了一身债,被债主追逼而 逃走的。

\subsubsection*{检验死因}

用塑料袋的窒息自杀虽有痛苦,但像这样是非常简单的。止痛剂「雪德丝」是无法达到致死量的,但药物的作用和冷水造成的低体温症,对降低呼吸机能倒是有用。

自 杀 手 册 的 作 用 之 大 也 是 不 容 忽 视 的 。
这本 书 的 原 著 是 1982 年 在 法 国 发 行 ,很 快 就 有 六 个人受影响而自杀,因此引起很大回响,成为最畅销的手册。

这个人所参考的部分,来自「英国安乐死协会」对会员限制发行的手册,所以记述是正确的。

母庸讳言,本书也从该书引用 了一些材料。 

有趣的是,他特意为了让别人明白是「看这一段死的」而夹了一条带子。

在法国也有翻开所参考书页而服毒自杀的人,试图通过这种方法引起世人注目的意图是再清楚不过的了。

\subsection{案例研究 20}

\subsubsection*{用塑料袋进行自杀实验的失败者}

1974年11月,在川崎市发现了一名十八岁男子,在被窝里用塑料袋套住鼻子和嘴,再用橡皮筋缠住而窒息死亡的事件。

书桌上有两张四百字的稿纸,写着试验各式各样的自杀方法的结果,标题则为「实验中」。

由此可知,这个青年是把塑料袋套住嘴巴在做自杀实验当中,窒息而死的。

他的笔记是这样写的:「(1)实验开始十五分钟以后,呼吸加速、手脚麻木。二十五分钟以后,全身瘫痪、呼吸增加三倍、心跳$120$下。如果在床上不动可以生存三十分钟。(2)约一小时发生脑障碍,约两小时死亡。在这之前应取下塑料袋,按摩心脏。」

\subsubsection*{检验死因}

看来用塑料袋自杀,是经过「气闷、手足麻木、全身瘫痪、心跳加速、失神」等顺序至死的。

那么,他为什么没能取下塑料袋呢?

所能考虑到的理由是:(1)马上要窒息时慌忙想取下,但着急之中没料到很快就不省人事了:(2)慢慢失去知觉时,身体也瘫痪而不听使唤就昏迷了。

如果第(2)项是正确的话,那么我们一直写窒息自杀是痛苦的,但事实上也可以说是相当舒服的方法。

可是,这种关于濒死关头的感觉是无法实验的,只能靠医生的推测才能知道。 

因此说,这几起「死亡实验」在医学上也是极为宝贵的资料。

\newpage

\section{Electrocuting 触电}

痛苦 ▼▼▼▽▽

麻烦 ▼▼▼▼▽

死状 ▼▽▽▽▽

牵连 ▼▽▽▽▽

冲击 ▼▽▽▽▽

致命 ▼▽▽▽▽

缺点是数秒钟的电击与剥开电线时的麻烦;比一般所想象中还要温和的方法。
日本厚生省1991年的统计显示,当年$19875$个自杀者中,有$56$人采用了人数最少的触电自杀法。而且,奇怪的是男子为$53$人,占全部的$95\%$。

可以说是几乎只有男子采用的自杀手段。
并不是整体数字低的缘故,即使在超过$100$人的统计里,$90\%$以上还是男子,有人说可能是女性特别害怕电的缘故。

触电自杀的死亡,是由于剎那间的呼吸停止、心脏停止、休克等所造成,痛苦真的只是瞬间 的事。有位医生说,触电自杀是舒服的死亡方法之一。

办法非常简单。
把电线的一头剥开露出铜丝,插上电源后触碰到胸口或背上就可以了。

冲动想死的人,这是最佳的手段。
缺点是未遂的情况较多,但没有特别的后遗症,所以是一种不 妨重新加以考虑的好方法。

\subsection{准备}

\subsubsection*{给心脏通电}

偷 偷 地 溜 进 变 电 站 去 触 碰 高 压 电 流 部 分 是 最 简 便 的 作 法 ,但 毕 竟 是 可 怕 的 。

还是 在 自 己 的 屋 子里接根电线,两个端子贴在身上,装上定时器后睡下。 

一根铜丝贴在胸口,另一根贴在背上,如果不愿贴在胸口,贴在两只手腕上也行。

一个十六 岁女高中生把一根电线缠在右手大拇指上,另一根贴在背心自杀了。

但是要这样做的话,电线应该缠到离心脏较近的左手拇指较好。也有把电极的一端含在嘴里,另一端插入肛门使用定 时 器 自 杀 的 核 子 科 学 家 ,但 他 是 怎 样 不 使 铜 丝 从 嘴 中 脱 落 而 入 睡 的 仍 是 个 谜 ,这种 方 式 方 法还是不去模仿为好。

铜丝用胶布贴住就可以了,这时把与身体接触的部分弄湿,电阻就少,效果更佳。一般情况 下 电 流 不 通 也 是 降 低 准 确 率 的 因 素 ,不妨 用 湿 脱 脂 棉 或 纱 布 从 上 按 住 。

弄 湿 脱 脂 棉 时 可 用 食 盐水,如果能弄一些做心电图使用的油膏,电阻就会更小。
也有用拾圆硬币做诱导体的例子。

当然,把两根铜丝都缠到胸口上是最可靠的方法。 
不管怎样,要紧的是要让心脏通电。

用变压器把电压提高,致死度就更高了。 
在 这 种 情 况 还 能 够 熟 睡 的 人 是 不 多 的 ,因此 要 喝 点 酒 。

定 时 器 设 定 在 睡 后 两 小 时 左 右 最 熟 的 时间。
设定在上午三点钟的例子较多,应该就是这个道理。

总之,尽量减少电阻,提高电压 是最重要。

沐 浴 中 把 开 动 的 电 热 器 或 电 动 刮 胡 刀 扔 进 浴 缸 里 也 会 导 电 的 ,不过 可 靠 性 极 低 ,为 了 提 高 导 电性在热水中加些食盐效果会更好。

\subsection{经过}

\subsubsection*{瞬间的刺骨疼痛}

这里的最大的问题是当电流通过的瞬间,人的意识是怎样的呢?使用定时器的方式,如果按照 预 定 计 划 顺 利 进 行 的 话 ,在 你 入 睡 数 小 时 后 $100$ 伏 特 的 电 流 便 会 通 过 全 身 ,立即 引 起 心 脏 收缩,数秒钟后失去知觉,心脏血液循环停止而死亡。

据某医生观察:普通状态下会产生瞬间的刺骨疼痛,筋肉松弛剂因为不易弄到手,所以不可 避免地要经验肌肉痉挛,如大量饮酒或服用安眠药可使神志模糊不清,就在睡眠之中安然死去。

这时虽丧失意识,但全身皆会抽搐着。
还有另一种说法:丧失知觉只是数秒钟或剎那间的事,但等到心脏停止跳动则需要三分钟左右。
一个触电昏迷后恢复知觉的男子,就「瞬间」这一点是这样说的:「突然我感到被一股力量控制便不省人事了。」

那么,当电流短路后会不会跳闸呢?电力公司认为「这种情况并不是没有」。

某位验尸官说:
「至今为止好像没出现过跳闸」。

因此,关于这一点,看来是不必担心的。

既然有坐上$2000$伏特的电椅上仍没有立即死去的死刑犯人,那么家庭中的$100$伏特的电压够吗?有人会感到不放心。可是,电压也并不是愈高愈好。

人体皮肤的电阻,在干燥状态下是$1000$欧姆到$2000$欧姆(女性的阻抗比男性低),
继续保持这一状态能通过身体的电流强度最大也只到$100$毫安。这种强度是能引起可使心
脏停止的心室颤动的最小电流。

当然,如果弄湿皮肤的话阻抗可少十分之一,事实上,在低电压的场合,对心脏来说弱电流倒容易引起心室颤动,反而危险。
还有,对人体最危险的电 气 周 波 为 $50$~$60$赫 兹 ,与家 用 电 流 的 周 波 是 一 致 的 ,从 插 座 来 的 交 流 电 要 比 直 流 电 更 容 易引起心室颤动。这就是说,只要在发生心室颤动的三秒钟里向心脏通电的话,家用$100$伏特电压也能完全电死人。

在浴缸内死去的方式,其死因为心脏停止或休克。对脑通电则呼吸中枢麻痹而呼吸停止,不过这种情况不多见。

\subsection{尸体}

\subsubsection*{几乎无损伤}

电极触及的部分会留下灼伤痕迹,但是尸体几乎没有损伤。皮肤的电阻小时,有时连灼伤痕
迹也没有。就好像是处在睡眠状态,这也是非常漂亮的尸体之一。

\subsection{注意}

\subsubsection*{瞄准心脏!}

目标就是心脏。确实想死的话,其它部分是不行的。

一定要设法使电流通过心脏。如想用定时器在睡眠中死去的话,不要忘记牢牢固定铜丝以免在被窝中脱落,即使缠在身上,一旦在睡眠中脱落也是达不到目的。

本 人 已 经 死 掉 了 ,但身 体 上 仍 有 电 流 的 时 候 ,这个 时 候 ,赶 来 的 人 有 触 及 尸 体 而 触 电 的 危 险 , 因此要留神,但这一点自杀者就无能为力。

假使不想给别人添麻烦,那就贴上「不要碰,危 险!」的纸条吧。




\subsection{案例研究 21}

\subsubsection*{在赛马场厕所触电自杀的男子}

1981年4月某日的下午六点在船桥市中山赛马场,全部赛程结束后两小时发现一个年约
五十岁,身高$160$公分,运动员发型的男子尸体。

他把电线接在厕所的换气扇上,另一端用胶布贴在胸口而触电死亡,发现时尸体已经僵硬了。

死 者 身 穿 新 的 三 件 头 西 装 ,刚 浆 洗 过 的 蓝 直 条 衬 衫 ,系 着 蓝 底 白 点 的 领 带 ,披 着 草 黄 色 风 衣 , 脚穿黑短统皮靴,相当整洁。

身边只有现金$8350$日元,短支HOPE香烟,百元打火机,写有遗书的赛马报纸,但是却始终未弄清死者的身分。

他用红笔在报纸上留下的遗书写道:「我是个混蛋,为了马而人生失败。对不起啦!愚蠢的人留。实在对不起,我太累啦!」。

\subsubsection*{检验死因}

这 是 轻 易 实 现 触 电 自 杀 的 好 例 子 。

大概 他 露 出 铜 丝 电 线 贴 到 胸 口 后 ,再 把 插 头 插 入 插 座 的 同 时心脏通了电,引起了心脏停止。

遗书中的「为了马而人生失败」是句名言,第二天报纸的社会版用这句话做标题报导此事。


\subsection{自杀地图 4:自杀村熊取町}

一 九 九 二 年 的 六 月 到 七 月 之 间 ,在 大 阪 的 熊 取 町 每 周 有 合 计 共 五 名 男 女 自 杀 了 。最 早 死 的 无 职业的少年A君(17 岁)。

发现的是六月四日(星期四)。接着十日(星期三)建筑业的 B君(18 岁),十七日(星期三)旅馆服务员C君(18 岁),二十五曰(星期四)邻市的 市公务员D氏(22 岁)上吊自杀。

七月二日(星期四)町内体育大学女学生E向自已胸口 猛刺三刀自杀。 

不知是何缘故,全部都在星期三或星期四死去的。

自杀的地方也异常接近,集中在半径五百 公尺的范围内。

加上前三个人又都是一年前结成暴走族(飞车队)的不良少年们。

全员自杀的动机完全不明。

这一连串的动机不明自杀震动了全日本。
其神秘程度,在日本自杀史上可以和三原山的「死亡引路人」事件媲美。 

在当地他杀说占绝对上风,如果是他杀的语,那就等于有那么一个每周杀掉一个人共杀五人
的杀人魔存在,当这种古怪的自杀出现时必然会引出他杀说的。

不过,不要太相信为好。

\subsubsection{死亡地点}

据说人口不足四万人,但熊取町并不是个乡村城镇。车站前有很大的巴士站大楼,商业街行人来往频繁。

可是,町西端的车站搭乘出租汽车沿国道170号线向东行驶十分钟,就可看见右边已有的农村土地了。

就在这附近,五个人自杀了。在这附近下车走走。

到处有杂木林和农作小屋。自杀地点任你挑选。

事 前 查 看 一 下 连 续 自 杀 者 的 死 亡 地 点 ,不妨 怀 念 故 人 而 浮 起 连 续 去 死 也 是 不 错 的 念 头 。

A 君 和 B 君 自 杀 的 洋 葱 小 屋 和 小 仓 库 已 经 拆 掉 了 ,D 君 上 吊 的 粟 树 也 被 砍 倒 了 ,只 有 C 君 上 吊 的 做农活小屋还留在那里。

夏天也有点阴森的,屋顶有几根梁的小屋是最好不过的自杀处。

\subsubsection{方法}

要想在这个城镇自杀,那么除上吊之外别无他法了。上吊,还是用绳子比较符合周围气氛。

把自己反绑后上吊,在高处放下绳子,模仿连续自杀者们演出的神秘色彩也是有趣的。

或许在当地再度兴起他杀说,说不定你也成了宣传媒介的注目人物。

最好不写遗书,那么更增添神秘色彩。

\subsubsection{交通‧住宿}

由大阪乘 JR 环城线到东天王寺后换乘阪和线,三十分钟可抵达熊取町站。
熊取町内没有住宿设施,夜晚抵达,或者打算休息一夜再去自杀的人,可到距车站只有五分钟距离的泉佐野市的HOTEL NEW ‧ yutaka。
这是最近建成的高级旅馆,内部装潢也很好,单人房$6800$日元,双人房$14000$日元,是合理的价格。

贸然到了熊取町,你会搞不清地理环境的,最好是前一天到达,预先查看一下环境后住在这家旅馆里,仔细酝酿方法并决定地点,第二天再去自杀。


\newpage

\section{Drowning 投水}

痛苦 ▼▼▼▼▽

麻烦 ▼▼▽▽▽

死状 ▼▼▼▼▽

牵连 ▼▼▼▽▽

冲击 ▼▽▽▽▽

致命 ▼▼▼▼▽

只要有水的话,在任何地方都可以进行。

不过窒息的瞬间很痛苦,死状也很惨。

综合来看,并不是很好的自杀方法。

是不是有人会误认善泳的人是不会投水自杀的呢?不管你泳技多么高明,也会因为以下将要说明的由于「某种作用」而莫名其妙地淹死的。

至今为止,游泳高手淹死的并不少。

投水自杀主要是窒息而死。

因为要体验几秒钟的呼吸困难、窒息状态,不能算是安乐自杀。

尽管这样,这一古老而有情趣的自杀手段,不论古今中外都被采用。

这一点,从古希腊女诗人采用过亦可得知。

现在在日本,高龄者和女性特别喜爱这种方式,从整体来看利用率有所下降,但每年按采用方法分类统计显示,仍然在前五名。
既遂率也高达百分之八十,难怪很 有吸引力。 

基本上,投水或溺死不仅限于海、湖、河、池,只要有「水」哪里都可以的,没有把全身浸 在水里的必要。

喝醉了酒而溺死在水沟的人也不少,甚至还有在山林小道上醉卧,一场大雨 使他肺部进水而死亡的「山中溺死者」。

也 有 把 头 伸 到 脸 盆 或 洗 衣 机 自 杀 的 例 子 。

因名 作 家 太 宰 冶 投 水 自 杀 而 出 了 名 的 玉 川 上 水 ,一 个自卫队员在水深只有四十公分的地方淹死。

\subsection{准备}

\subsubsection*{捆绑身体}

捆绑身体的做法有点原始化,但却是非常有效的手段。

捆绑双脚,然后在背后绑上手后,一股气跳入水中的话,不论你的游泳术高明与否,都会淹死的。
(当然更高明的即使在这种情况下,还可仰面浮在水上。)

不过,单独一个人捆绑手脚是困难的。这里介绍一种谁都能做的方法,首先捆住双脚后,再 把 左 手( 惯 用 的 手 或 相 反 的 手 )绑 在 左 大 腿 上 跳 入 。假如 把 船 划 到 深 海 或 湖 中 心 而 跳 入 的 话 , 就会很快地沉下去的。

这时在口袋里放入石块一类的东西,更可加速淹死。 

喝过酒感到疲倦就容易失去平衡感,可靠性大,大量服用安眠药就更可靠了。

要想简单而方便地去死,那就连车子一起驶入水中,只要不设法爬出车子就不会得救。

更简便的方法是乘上大型客船,在半路跳入水中,那就会被轮船激起的水流漩涡而淹死。

据说,在濑户内海半数被打捞上来的尸体就都因为如此。

作为跳入或下水的地点,还是以海和湖比较理想。

选择的时候,就应选人迹稀少湖流湍急的海岸,时间当然选在不易被发现的夜晚。

\subsubsection*{在浴室也OK}

在浴室死亡也是很简单的。

浴缸内放满水,捆绑双脚,再把双手反绑趴下把头伸进水里就行了,某主妇就是用此一方法成功地自杀了。

更为特殊的方法,是在固定好的汽油桶内放满水,从头部入水的话,便可以因无法摆脱而淹死。

此 外 ,作为 特 别 适 合 溺 死 的 地 点 ,就是 濑 户 内 海 鸣 门 的 涡 流 。

朝 着 漩 涡 游 去 时 可 能 被 推 出 来 , 但一旦被漩涡卷入的话就再也出不来了,致死率为百分之百。

季节以冬天为佳。

水温低的时候,或者休克死或者引起心脏麻痹,疲劳出现也快,所以能安然地死去。

就是说,也有不是溺死而是「低体温症」即冻死的情况。

要想冻死,则必须喝些酒或安眠药为好。

\subsection{经过}

\subsubsection*{游泳高手是这样溺水的}

现在来介绍一下会游泳的人投水自杀的方法。

最好是在晚上出发,一直往前游去就会顺着潮流游向深海。

不久就会感到疲劳,但无法停下来。

外海的波浪很大,喝水的次数增加,气管里进水呛得厉害。

在不断地重复喝水、吐水、呛水过程中,连接嘴和耳朵的耳管里也进水了,这是关键。

有的人说:「耳朵有鼓膜,水应流入三半规管。」,可是水还从嘴里流入的。

投入水中时的溺水经过:首先由于皮肤突然受到刺激,会做一次深呼吸(第Ⅰ期),接着会 有 $30$ 秒到$1$分钟时间呼吸停止(第Ⅱ期)。

不久后由于血液中的二氧化碳增加,会有 $1$至$3$ 分钟时间出现激烈的痉挛性吸气吐气(第Ⅲ期),之后失去意识约有$1$分钟的痉挛(第Ⅳ期),沉没死亡(第Ⅴ期)。\footnote{摘自《小法医学书》金芳堂。}

流入耳管的水产生活塞运动而使覆盖三半规管的部分出血,通过急性循环不良而引起三半规管的机能障碍。

这种机能受到损伤,就失去平衡感,上下左右前后分辨不清。

愈是想吸气,气管进的水就愈多而引起咽喉的痉挛,最后呼吸停止、失去知觉而沉入水底。

会游泳的人淹死的情况大体如此。到达神志不清的过程,大部分不是因大量喝水窒息,而是 在 呛 水 中 间 把 水 吸 入 气 管 所 造 成 的 。

这样 的 话 ,即 使 你 是 游 泳 高 手 也 会 对 投 水 自 杀 抱 有 信 心 了。

而且有人说,在即将失去知觉的时候,以往的记忆会像闪电般清晰地浮现出来,就像在上吊 和跳楼项目中所介绍的奇妙体验。

失去知觉以后,先在水中大口呼气,接着大口吸进水到肺里,呼吸停止而死亡。

从呛咳开始到失去知觉大约是一至三分钟,淡水需四至五分钟,海水则需八至十二分钟,到心脏停止则需二十分钟至半小时。
话虽如此,在水中经过两分钟以上就难以得救了。


\subsection{尸体}

\subsubsection*{你能成为萝拉帕玛吗?}

过了数日,浮上来的尸体,其身体和脸都肿胀得连双亲都辨认不出来(专业者形容为「巨人样」),呈现皮肤剥落,阴囊膨胀得像气球,身体生出青苔,有时手脚被鲨鱼乱咬,留有船只推进器切割的伤痕,肉则有被鱼、螃蟹等吃过的惨状。

在一度沉下水底的尸体中,$20\%$~$30\%$不久就会浮上来。

被打捞到陆地上的尸体,从 口和鼻会吐出大量的小泡。

如果尸体没有很快浮上来,夏天是两三天,冬天是两三个月后, 因体内产生的腐败气体而浮上来。腐败气体的上浮力是很大的,连栓有近十公斤重物的尸体也都浮上来过。

不过下沉深度达到三、四十公尺以上,因水温低而腐败气体蓄 积不起来,水压又压缩气体,大都分的尸体是不会浮上来的。 再说,四国最南端的足折岬,是尸体绝对不上浮的投水自杀胜地。「尸体拒绝被发现者」不妨选择水很深的地方,像是足折岬。



\subsection{注意}

\subsubsection*{不会游泳的人比较有利}

投身到海和湖里的时候,绝对不要被人看到。尤其是跳海时,会造成花费很大的海岸救助部队的出动。

会 游 泳 的 人 也 有 可 能 这 样 自 杀 ,但 不 可 否 认 ,这种 方 法 是 适 合 那 些「不 会 游 泳 」和年 迈 者 的 。 实际上,二十和三十岁左右的男子几乎很少投水自杀的。

虽然下了决心而跳进水里,但总也 得游几下,在噗通的过程中被就上来的例子也很多,会游泳的人还是选择别的方法更好些。





\subsection{案例研究 22}

\subsubsection*{把头埋进马桶而淹死的女星}

1944 年,曾经是好莱坞电影界的红星\footnote{María Guadalupe Villalobos Vélez,出生于墨西哥,从影前在墨西哥登台,1926年进入哈尔·罗奇制片公司,参演喜剧短片,次年在《高卓人》中的表演而走红。银幕下的卢普个人生活起伏很大,在经历几次感情风波后,1944年服过量安眠药自杀。}把头埋进马桶而淹死了。
她是 1908 年出生在墨西哥,渴望当一位名演员,于十岁那年离墨西哥城来到了好莱坞,在电影《鹅》中扮演女主角。后来爬上了名演员的宝座,和许多著名男演员交往,过着奢靡的生活。由于她的任性和嫉妒的性格,闹出离婚事件,而且还出现丑闻。

为了这些丑闻,她的声名下降了,不久落到了R级喜剧电影的地步。

但 是 忘 不 掉 昔 日 光 荣 的 她 ,借 了 难 以 还 清 的 债 务 而 过 着 奢 侈 豪 华 的 生 活 。

最 后 她 怀 了 一 个 年 轻男演员的孩子而被对方要求打掉,极度心灰意懒的她觉得「与其打掉孩子而活着,倒不如自己死掉的好」。

她决意自杀,邀请了许多朋友出席她那赊帐而举办得点有数十根大蜡烛的豪华宴会,当天晚上服用了一手掌的安眠药睡到床上。

可是半夜发生呕吐而未能断气,在极 端痛苦的情况爬到浴室,把脸伸进马桶淹死了。

成 为 尸 体 的 她 ,固 然 不 复 当 年 银 幕 美 女 的 形 貌 ,但她 的 面 部 是 祥 和 的 。

她最 后 留 下 的 遗 言 是 : 「对人生感到太累了,已经厌烦竞争了。
虽然从孩童时代起一直想竞争下去。」


\subsubsection*{检验死因}

马桶里那么一点点水就能死人,这是个证明那里都能淹死人的例子。

同时这个例子也让我们看到了用药物自杀的典型失败范例。 

尽管如此,一蹶不振的女演员在马桶内了却一生,也太过份了点。


\subsection{案例研究 23}

\subsubsection*{即将溺死前的濒死经验}

美国的十七岁少女,有一天和其兄一起去湖里游泳。湖中有很多年轻人在游泳,不知是谁喊了一声:「游到对岸吧!」,于是大家开始向对岸游去。

这个少女曾多次横渡过,但却遭到差点溺毙的情况。

少女被救了上来,后来她谈了当时的情景:「我在半沉半浮中。突然有一种我和自己的身体分割开来的感觉 ,独自一 人呆坐在一片空虚之中 。

我在那里一动也不动 ,但我的身体却在三 、 四英呎前面的水中浮沉。

我是从右斜后方看到自己的身体。

就算我的身体在一段距离外,我还是感觉到我是有着完整的身体。

……心情十分轻松,好像变成了羽毛似的。」

\subsubsection*{检验本例}

这 并 不 是 自 杀 ,但 却 谈 了 即 将 溺 死 时 发 生 的 濒 死 体 验 。
这种情况是意识脱离身体观看自己的 灵魂脱离,身体变成羽毛似的轻松经验。

而且,不管曾横渡过多少次湖水的游泳高手,有时也会因「不知是何原因」而淹死,这一点
是值得注意的。


\newpage

\section{Self-Burning 自焚}

痛苦 ▼▼▼▼▼

麻烦 ▼▼▼▽▽

死状 ▼▼▼▼▽

牵连 ▼▼▽▽▽

冲击 ▼▼▼▼▼

致命 ▼▼▼▼▼

冲击最强烈,绝对可以致死,也有在历史上留名的可能。

不过,痛苦也最强烈,死状凄惨。
自 焚 自 杀 是 痛 苦 的 。
皮 肤 百 分 之 百 地 烧 伤 却 未 立 即 死 亡 而 被 送 往 医 院 ,在 那 里 折 腾 了 半 天 才 断气的情况不少。

未死而造成的后遗症,比任何其它自杀手段都更为悲惨。对想普普通通死去的人,我是决不劝他采用的。

不过,如果你对这个社会想控告什么而死的话,再也没有像自焚自杀这么悲壮的方式了!
曾 在 越 南 战 争 中 反 对 政 府 而 死 的 僧 侣 ,最近 在 韩 国 同 样 为 反 政 府 运 动 而 自 杀 的 学 生 ,在 日 本 为 争取改善基层工人阶级待遇的釜崎共斗会议干部,他们都采用了自焚自杀而引起了社会的注 目。

不错,在被火焰包围之中叫喊着自己的主张的形象确实是深具影响力。

用自焚表现坚强意志和死的决心是再好不过的了。1986 年追随已故「真理之友教会」教主而集体自杀的七个女教徒,也是自焚自杀。

有人可能以为采用这种自焚自杀的人不会太多,可是人数却年年在增长,目前一年间约达七百人之多,要比触电自杀多得多。照现在医疗诊断标准,在全部自杀者中有三分之一的人被认为是「狭义」的精神病患者,而企图自焚自杀的人中间,被认为患有精神病的居多。

\subsection{准备}

五公升是安全地带不用说是用汽油或煤油的,但浇几公升好呢? 

全 身 皮 肤 烧 伤 $20\%$ 对 全 身 已 造 成 很 大 的 打 击 ,但 为 了 死 去 就 只 有 全 身 浇 满 汽 油 了 。

虽 然有用两、三公升煤油而体无完肤的例子,但为了保险起见,还是准备至少五公升吧。

汽油、煤油是可以一点一点浇上的,衣服易燃,所以可以浸透。煤油进入眼睛非常痛会睁不开,
不过,过多地考虑这些是无法进行自焚自杀的。

如果害怕火灾,不妨在下面铺上阻燃地毯,这样即使在房间里躺着燃烧也不会造成火灾。

也有浇上汽油后钻进焚烧炉去自杀的四十八岁主妇,不过完全没有这种必要,反而成为妇女
周刊的好材料。

浇上汽油之后就可以用打火机或火柴点燃宿命之火。

打算发表声明的人要事先牢牢地记在脑子里,免得一入惊慌状态而忘得干干净净。

还有在房间内浇洒汽油,连房子一起焚烧而死的方式,采用的人也不少,这样的话倒不如选少点痛苦的方法,让人看到在燃烧才是自焚自杀的最大特点。

\subsection{经过}

\subsubsection*{冒出三公尺高的火柱}


火 苗 比 想 象 中 厉 害 得 多 。

汽油 、煤油 一 经 点 火 ,会发 出 声 响 而 一 下 子 冒 起 两 三 公 尺 高 的 火 焰 。

被火焰笼罩的身体会有剧烈痛疼和灼热感,因难以忍受会在地上打滚,四周有人肉烧焦的臭 味。

但大致上知觉还都是清楚的。

气管因吸入热气而烫伤,但声音还是能够发出来。 

当 衣 服 烧 得 干 干 净 净 之 后 火 也 就 灭 了 ,但有的人还能靠自己的力量站立在那里 ,继续 喊 叫( 案 例 24)。 

火灭了以后,如果皮肤有三分之一以上坏死的话,有百分之五十的人会死去,如果坏死三分之二以上,几乎毫无例外地都会死亡的。

到达死亡的时间是各不相同的,全身皮肤被烧得像 炭一般焦黑时会当场死亡的,火被扑灭后用救护车送往医院,身子动弹不了,便在痛苦的情况下迎接死亡的到来。

有的经过半天或一天,甚至还有经过五天或十天才好不容易断气的。

高呼「反对越南战争」 而 自 焚 自 杀 的 一 位 八 十 二 岁 美 国 妇 人 ,神志 清 醒 地 生 活 了 十 天 。她 的 毛 病 恐 怕 是 浇 上 了 燃 烧 不良的酒精性洗涤剂的缘故。 

焚身自杀所导致的死因有休克,缺氧及高热引起的重要器官热凝固等。


\subsection{尸体}

浇上五公升以上的汽油后点上火,当没有人扑灭火的时候,皮肤表面的一部分或全部碳化,
头发全部烧光。
皮肤剥离,露出了赤红色的肉。
本来人体基本上由水形成的,所以即使皮肤碳化,肉也很难烧。

这时的姿势,由于肌肉的收缩成为拳击姿势是其特征。尚未坏死的皮肤出现水泡,颜色由黄
色变为茶褐色,最后变为黑色。眼角膜白浊,伸出舌头的情况较多。


\subsection{注意}

\subsubsection*{千万别陷入「癜痕瘤」的地狱!}

最最可怕的是,留下了一条生命,但在脸部或全身留下烧伤的痕迹而继续活下去。

一位女性浇了一身煤油点燃了火柴但自杀未遂,经历了三年的病床生活,满身都是癜痕瘤,虽然施行过五次整容手术也没有消除癜痕瘤,没有了嘴唇,嘴也无法张大。

而且 ,与采用其它自杀未遂者企图再次自杀的情况相比较,自焚自杀或许是因为采用了过激的手段而从濒死状态恢复过来,而得到了感情平衡的缘故想再次自杀的比较少,可是皮肤移 植将花上巨大的治疗费。

为了不至于弄到这种地步,所以至少浇上五公升汽油,四周如有人来救火你就大喝一声:「不要靠近!」即使是这样你还是担心未遂的话,可与其它方法同时并用。

男子杀掉害了他的女人(二十四岁),跑到母校大学的十层楼顶,浇上汽油后跳楼自杀了。

可是,完全没有必要做到这种地步,自焚是致死度很高的自杀手段。

总而言之,自焚自杀是充满痛苦的自杀方法。
不过要记住的是,不会立即死亡的。

对于死的 方法是各有所爱的,一个人静悄悄地死的话,我就劝你采用别的方法了。

不过,假如你认为已经过了默默无闻的人生,但死的时候可要轰轰烈烈一番的话,就随你的便了。


\subsection{案例研究 24}

\subsubsection*{为要求改善劳动条件而自焚的韩国青年}

1970年11月,在恶劣劳动条件下工作的裁剪师青年(当时二十二岁)要求改善劳动条
件而在汉城和平市场自焚自杀。

在 此 以 前 他 一 直 在 从 事 争 取 改 善 劳 动 条 件 的 运 动 ,履 次 遭 受 挫 折 ,终 于 在11 月 3 日 在 和 平 市场内强行展开被阻止的示威游行,当参加示威游行的工人们和警卫队、警察部队激烈冲突 的时候,他比同伴们迟些时候出现在市场,并要求同伴们「点着火柴向我扔过来」。

火柴扔到他身上时全身立刻成了一根火柱。然后又变成火球,他吸着火焰挤进人群中,高呼:「遵守劳动基本法!我们不是机器!还我星期天!反对残酷驱使工人!」,最后惨叫一声倒了下 去。

可是他又艰难地站了起来,大叫一声:「不要让我白死!」。眼睛、鼻子都烧得模糊不清了。

他倒下去后失去了知觉,三分钟后同伙们把火扑灭了。被送到医疗中心的他,嘟嚷了 一句:「肚子有点饿啦!」,过了九个小时后的夜晚十点多,他断了气。

他的自焚自杀,后来被命名为「人类宣言」,至今还流传着。

\subsubsection*{检验死因}

虽 然 在 失 去 知 觉 后 三 分 钟 时 火 被 扑 灭 了 ,过了 九 个 钟 头 后 死 亡 的 严 重 烧 伤 ,但 不 停 地 喊 叫 是 可能的,而且只要有力气跌倒一次还可站起来喊叫。

当然会因人而异,当变成一团火球时, 要不是因为热得打转的话,他的神志可是非常清醒。 

这 可 以 说 是 最 有 效 地 利 用 了 自 焚 自 杀 这 种 手 段 的 例 子 。

他 的 自 杀 确 实 具 有 很 大 效 果 ,对死 去 的他来说也遂愿了。

\subsection{案例研究 25}

\subsubsection*{为近亲通奸所苦恼而自焚的中学女生}

有一个十二岁的中学女生,在叔父家院子里自焚自杀了。

她 出 生 后 不 久 双 亲 离 婚 而 失 去 了 母 亲 ,由 祖 母 养 大 ,因父 亲 的 工 作 关 系 在 小 学 和 中 学 期 间 各 转过两次学。

她生长的家庭环境非常复杂的,双亲离婚的原因是由于母亲和叔父有了性关系。
 
可是,促使她走上自杀道路的原因更为复杂,她本人和祖父、父亲也都有性关系。

对此一直 感 到 烦 恼 的 她 当 时 还 在 中 学 一 年 级 ,在 暑 假 前 两 个 月 就 开 始 长 期 缺 课 了 。终 于 在 刚 过 中 午 的 某一天,她到造成双亲离婚起因的叔父家后院,全身浇上汽油,进行了自焚自杀。

在学校里她经常玩排球,给人的印象是看上去和普通的学生没有什么两样。

\subsubsection*{检验死因}

十二岁的少女自焚自杀,这是令人震惊的例子,但首先应注意的是死的地点。她对叔父的仇恨可能相当大的,所以特意到叔父家后院变成火球,对唤起复仇之念也是非常有效的。

十二岁的女孩竟然和祖父、父亲两个亲人发生性关系,更是使人惊讶不已的,太可怕了!她本人大概已感觉到为发泄恐怖、怨恨、绝望的压抑而自杀,如采用跳楼、撞车这些方法也太不显眼了。

这种情况还是自焚最合适。



\newpage

\section{Freezing 冻死}

痛苦 ▼▼▽▽▽

麻烦 ▼▼▼▼▼

死状 ▼▼▼▽▽

牵连 ▼▽▽▽▽

冲击 ▼▽▽▽▽

致命 ▼▼▼▽▽

如果可以找到很好的地点,就很轻松了。不过要注意有可能经过大规模的搜索后,自杀未遂,却已冻坏手脚必须截肢。

冻死和上吊一样,大概都是很舒服的。说这种话恐怕要被怀疑它的真实性,但这是生还者所说的,只好相信了。

虽然如此 ,把冻死作为自杀手段而采用的人绝对是少数,在统计上也被分在「其它」一类里,或许因为特意走到雪山太麻烦亦未可知。

但冻死并不是只有在雪山和寒冷地区才发生的现象,只要条件齐备,也有在房间内冻死的,当然气温也没有必要一定要在冰点以下。

有报告显示,事实上在东京每年都有十几个人冻死,在更暖和的地方也会有冻死者。

报纸上经常有 流浪汉冻死的消息,对东京、大阪的流浪汉来说,如何度过寒冬是个很实际的愿望。

说 了 半 天 还 是 一 篇 大 道 理 ,毕 竟 不 会 有 人 想 在 房 间 里 冻 死 的 ,也从 未 听 说 过 在 房 间 内 冻 死 自 杀的。

在美国,有人进入冰箱里冻死自杀,这只能算是例外。

要想冻死,虽然麻烦但还是到雪山的好。

这方法的优点很多,可以弥补哪些麻烦呢?

下面将要详细加以介绍,例如尸体无损伤,痛苦也少等。

一个人对寒冷的耐受性是重要的因素,因此对不怕冷的人是不适合这种自杀手段。


\subsection{准备}

\subsubsection*{选定山上休闲地作为目标!}

要想去雪山,首 先 买 一 本 寒 冷 的 山 岳 休 闲 地 的 导 游 手 册 以 了 解 情 况 。
即 使 没 有 登 山 经 验 和 体 力,谁也都能到寒冷得足以冻死人的地方。

在去滑雪的时候,不妨到那些人们很少光顾的地方。
就连滑雪练习场的旁边,说不定也能找到合适的地方。

当然,喜欢登山的人可以很周到 地决定路线和地点。 
需 要 准 备 携 带 的 东 西 :有 药 房 可 买 到 的 镇 静 剂 、止痛 药 等 使 情 绪 稳 定 而 能 入 睡 的 药 物 以 及 酒 类。

饮酒固然能使体温上升,但为了克服痛苦喝点也是允许的。反正,到了雪地里是很难抵御外部的寒冷。

脱掉衣服,一个晚上就能死的,但是有些难度,所以至少找个两、三天别人找不到的地方蹲着,如果能够入睡那就更OK了。

\subsubsection*{在房间冻死:要挑选最寒冷的冬季}

真的想在房间里死,那就非选隆冬的日子不可,还要看天气预报,选择最寒冷的夜晚 。

另外 , 还必须做些能使身体降温的准备:全身赤裸浇上水,然后面向电风扇和冷气机,窗子和冰箱都打开。
还要空腹、睡眠不足和疲劳,否则也很难实现。

如事先喝点酒,使身体表面血管扩张,能迅速降低体温,有催促快死的效果。
当然也需要一些耐心,这也是为了等待睡意来临前的忍耐。


\subsubsection*{在街头冻死的方式}

可在夜间穿著湿透的衣服到僻静的公园、空地和树林里去。

为了避免人们的猜疑,携带水壶在决定好地点时淋上也可。

尽量选择散热较快的水泥地,不过最为重要 的就是不要让人看到。 

在气温五度、无风、半裸体、空腹的状态下,有一天之内就能冻死的例子;也有在最低气温 五度的夜里,烂醉如泥的夜归者在路上冻死的例子。

在 水 中 ,气温 十 五 度 的 情 况 下 冻 死 的 危 险 性 大 ,如果 是 五 度 的 水 温 ,浸 泡 数 小 时 就 会 死 亡 的 , 而 且 这 样 的 水 温 有 时 也 会 有 在 瞬 间 引 起 心 脏 停 止 的 情 况 。

向 报 社 和 杂 志 社 编 辑 部 寄 上 自 杀 预 告信的四十三岁家庭主妇,就在隆冬季节的室兰市内,坐在路旁的公园水池内冻死。


\subsection{经过}

\subsubsection*{甘美的恍惚感}

不论是钻进雪堆里还是在房间里脱光衣服,死亡的经过是一样的。

开始时全身冻得直哆嗦,直肠体温到达三十五度时会产生疲劳感、倦怠感和睡意。连孩子们都会说的那句有名的「睡了会死的啊!」的俚语,就是指这种情形。

体温降到$33$、$34$度,思考力逐渐减弱,意识模糊不清,会被一种「甘美的恍惚感」笼罩。到了$30$至$32$度就失去知觉,直至死亡。

降低至$25$度以下时就无法救助了,所以比想象要爽快得多。

对 这 甘 美 的 恍 惚 感 ,某冻 死 自 杀 生 还 者 这 样 说 :「随 着 呼 吸 急 促 有 一 种 … … 神 智 不 清 ,… … 。」

此外,一位学者在访问而得的报告中,有以下的例子:「数千条光彩夺目的光线在她眼前闪 耀,数千台大炮的轰隆声在她耳边响着。

脚感到剧烈疼痛,仿佛在针山上跑动似的,不久出 现了睡意。

一种令人平静的倦意不断地涌现,这么一来好像从世上的担心和灾难中解放出来。

空气清新,仿佛是春风似的,优美的音乐再次在她耳边响起。

把身子靠在柔软的皮毛枕头里小憩时,各种彩光又在闪耀,不久就神志不清,直到获救以后才恢复了知觉。」 

也有的生还者说:「做了一场在盛大宴会上狂舞的梦。」

根据上述体验,可以说与跳楼自杀一样,冻死也是一种除了初期的寒冷以外,完全没有痛苦的自杀方法。

\subsection{尸体}

\subsubsection*{真的是「最美」的吗?}


在雪山冻死的尸体和煤气自杀的一样,有时被形容为「最漂亮的死法」,但这是因发现时期而有所不同。

如果因雪而尸体以冷冻保存的状态来看,皮肤因失去血气而呈现透明般的白色。

可是,到了春雪溶化时,尸体就腐烂。
尸班是红色,这是因为氧和血液中的血色素结合变为粉红色的氧化血色素的缘故。

在雪山等处,有时手尖和脚尖会冻伤而出现水肿的情况。

有时也会发生神经错乱,自己脱光衣服被人发现的情况。尸体被野生动物撕烂的也不在少数(案例 26)。

因此这样说来,并不能说是很美的。

\subsection{注意}

\subsubsection*{别被找到!}

不管在哪里死,不要在半路发现是最重要的。

在雪山的地方,万一在手脚被部份冻坏的情况下被救出,那就需要切除而成为没有手脚的人。

不过,气温不到冰点以下是不会出现坏死现象。

此外,在雪山自杀的场合,家属和朋友发现后要求搜索时就需要巨额的搜索费用。因此事先要做些手脚,不要留下暗示行踪的遗书,外出录音电话也照旧就如平时外出那样即可。

如果真的想自杀,还是朝雪山进行的好,在房间内或路旁的冻死,必须一切条件齐备而且还要不易被发现,倒是挺困难的。

\subsection{案例研究 26}

\subsubsection*{在雪山冻死的女剧作家}

1981 年 6 月 ,在 北 海 道 中 央 部 石 狩 山 地 大 雪 山 系 黑 岳 五 合 目 发 现 了 女 性 白 骨 尸 体 ,地 点 在山上站三百米左右的原始森林内。

这位女作家是曾参与《七个刑事》电影制作的杉江慧子\footnote{本命渡辺祥子。出生于东京,毕业于青山学院初级学院。 1981年6月21日,在北海道上川町的层云峡自杀。}(当时四十七岁)。

白骨散乱各处,只发现了头盖骨、右大腿和小腿骨。
冬天在这一带,狐狸、野鼠等野生动物出没频仍,有被动物咬啃的痕迹,大概在冬季里被这些动物享受过吧!

在尸体附近有安眠药的瓶子,里面还有剩下的药片。
估计服用的安眠药未达致死量。

袋里也有一大瓶威士忌,还剩下八成左右。

从她留下的遗物中找到的车票日期是前年的1980年10月24日。

大家认为,她从所居住 的 东 京 来 到 了 人 烟 已 稀 少 的 冬 季 大 雪 山 ,由 车 站 走 进 原 始 森 林 ,然 后 服 下 安 眠 药 睡 着 后 冻 死的。

杉江慧子约在五年前开始从事剧作写作,对不能出人头地感到非常烦恼,同时对年龄的不断 增长也感到恐慌。

所留下来的稿子上虽有数行笔迹潦草的文字,因墨水浸濡,无法辨认,结 果始终没弄清她的自杀动机。

\subsubsection*{检验死因}

这 是 到 山 岳 地 区 服 用 安 眠 药 后 在 雪 地 里 睡 着 的 典 型 雪 山 冻 死 自 杀 。
然而 ,遗 体 却 令 人 惨 不 忍 睹,不是什么最美的。
在雪山的自杀,过了十个月尸体就会成为白骨了,而且还要注意野生 动物的袭击。

此外,到雪山冻死自杀的时候,是有必要事先做一些实地调查的,杉江本人则因为编写已成 为遗作的《星期六剧场》(朝日电视)剧本是以北海道为背景的,所以对当地多少作了调查的。

如果对选择自杀地点感到太麻烦的话,我就劝你采用和她相同的路线。
为了慎重起见,再介绍一下去黑岳的方法,首先去旭川,再从旭川乘两个小时到层云峡温泉。
然后乘缆车,在换乘缆车的地方转向雪道,进入原始森林就行了。

模 仿 她 的 死 法 的 话 ,可买 些 在 药 物 章 节 中 介 绍 市 面 出 售 的 安 眠 药 和 镇 静 剂 ,再 带 一 大 瓶 酒 也 不错。即使尸体散乱,但起码会在散乱前完全死去。



\newpage

\section{Special Cases 其它手段}

自杀的方法,除以上介绍之外,还有很多。

例如,反正不易弄到手就没介绍的枪枝自杀。

电影和电视经常出现的用枪口对准眉间或太阳穴而扣扳机的画面,仿佛是枪枝自杀的典型手法。

可是,这种扣扳机的做法是不值得推荐的。

由于头盖骨比想象得要坚硬得多,万一角度不对就有可能使子弹弹飞,即使打进去但子弹裂开,碎片在头盖骨内击伤脑子,而又转了一圈从眼睛或另一侧穿出以未遂而告终的情况。

要用枪,那就衔在嘴里击穿后脑部的延髓,这才是正确的。

在本章里,让我们依据案例研究来学习这些其它的自杀方法。



\subsection{案例研究 27}

\subsubsection*{东京都足立区都营新村饿死自杀的姊妹}

1985年 8 月 ,在 东 京 都 足 立 区 都 营 新 村 的 某 一 房 间 内 发 现 了 已 腐 烂 的 两 具女 尸 。

这是 二十五岁和二十三岁的一对姊妹,死因是饿死。

死亡推定时间为1984年底至1985年2月左右。

在遗体旁边,有一个姐姐写的「死给你看」几个字的旧信封。

饿死

痛苦 ▼▼▼▼▽

麻烦 ▼▽▽▽▽

死状 ▼▼▼▽▽

牵连 ▼▽▽▽▽

冲击 ▼▼▽▽▽

致命 ▼▽▽▽▽

姐姐为支付母亲和妹妹的医疗费以及父亲留下的债务而奔波,她积极认真的工作态度在工作单位也获得好评,可是每月十一万日元的工资是无法偿还的,最后只好借贷。

不久所借贷的金额达到了三百万日元,讨债的电话也打到工作单位。

即使这样,姐姐还是勤奋地工作着。

姐姐是在1965年和双亲、妹妹一起搬到这个新村的,母亲病弱,父亲在外工作长年不在家 ,因此每月有十一万日 元的生活补贴。

本来就有点精神分裂症的妹妹,上中学后常遭欺负,常常不去上学,姐姐一面看顾母亲的病,一面鼓励妹妹,本人也进入商业高中。

然而,念高三时父亲背了一大笔债回来,次年因患癌症死亡。
就在这个时候,妹妹又得了甲状腺亢进症,姐姐也因就职而被取消了生活补贴。 

母亲终于在1983年病死了。

但是,亲戚们不同意她领回骨灰。

大约从这个时候开始,做 姐姐的也对生活感到厌烦起来。
她在阳台随意放置垃圾袋,讨债的电话不断,从1984年6月起开始旷职,终于在九月被开除了。

而且就在九月间因未按时付费而被切断了电和煤气, 第二年的一月,自来水也断了。

讨债的连日找上门来,两个人只好装做出门而不倒垃圾。

十月间曾到邻居那里要吃的东西,过了一阵子邻居们不放心来找过她们,但她们回答说:「不 要再管我们啦!」。事实上,就在这个时候或许已经坚定了自杀的念头。新村附近有地区的 福利事务所,但她们从没去谈过什么。

大概是妹妹先死以后,姐姐才死的。到了春天,由于苍蝇孳生,臭味冲天而被发现,当时妹妹身穿T恤衫和西裤偎在 姐姐身上,而姐姐则不知为什么光着身子罩了一件对襟毛衣,下半身却赤裸着。

房间内堆满了垃圾、被褥和衣服,其中还有一些「十七岁」\footnote{《Seventeen》(日语:セブンティーン),集英社发行以少女为消费对象的日本潮流杂志,毎月1号发行。}等杂志和赤川次郎\footnote{赤川次郎(1948年2月29日-\qquad \qquad),日本推理小说家,生于福冈县福冈市。}的小说等。

壁面上贴着大岛弓子\footnote{大岛弓子(1947年8月31日-\qquad \qquad ),日本漫画家。}和幻想漫画「绵之国星」的大幅贴画。

\subsubsection*{检验死因}

这是令人感到惊愕的不幸。

饿死这个手段,只有吃尽人生的苦头才会选择的。
因为,她们已经疲劳到极点,连去自杀的力气都没有了。

人与人之间的差异固然很大,但只要连水也不喝,那么在一至二周便可死去,只喝水则在三十至四十天饿死。能量的储存量也是个很大的因素,这对姊妹有些肥胖,大概比一般人花费了更长的时间才死去的。

一家四口都各有其不幸,但姐姐的不幸更加凄惨。

母亲的病,父亲的借债,妹妹的受欺负和病,所有的不幸都压到她的肩上。不管怎样去努力也得不到报偿,毋宁说越 是努力情况越糟。

她的一生仿佛就是站在自杀与否的十字路口上。
自然,要是想活下去那也是可能的,她却拒 绝了帮助而选择了死亡。

面对这种人生,难道还会有「活着就会有好事」、「要想死什么都 做得出来」、「自杀是弱者的事」等说出这种蠢话的人吗?

对她的「死给你看」这句话,恐怕是无言以对的。



\subsection{案例研究 28}

\subsubsection*{在鸟取砂丘流砂自杀的男子}

1988 年 11 月 在 鸟 取 砂 丘 发 现 了 一 具 男 子 白 骨 尸 体 。

流砂

痛苦 ▼▼▼▼▽

麻烦 ▼▼▼▼▼

死状 ▼▼▽▽▽

牵连 ▼▽▽▽▽

冲击 ▼▼▼▽▽

致命 ▼▼▼▽▽

当天 下 午 一 点 多 ,在 远 离 观 光 旅 游 路 线 几 乎 无 人 影 的 地 方 散 步 的 当 地 老 人 ,发 现 了 一 个 头 盖 骨 微 露 砂 面 ,报 警 后 挖 掘 是 一 具 左 手握着小型铲子,好像抱在胸前似地蹲在砂坑里的尸体。

这 名 男 子 是 1974 年 也 就 是 十 四 年 前 因 失 恋 和 患 病 的 失 踪 人 口 ,当时 三 十 二 岁 。

一 天 晚 上 他 来到了砂丘,用铲子在砂丘斜坡挖了约一公尺深的洞,蹲进去后用双手往自己身上堆砂,然 后就靠着砂子埋了头部而窒息死亡。 

他失踪后父母收到的遗书写道:「搜索也是白费的。」

\subsubsection*{检验死因}

这是一个决意把自己从社会中抹消掉的人所做的彻底例子。

鸟 取 砂 丘 是 和 树 海 齐 名 的 日 本 秘 境 ,但 把 自 己 活 埋 窒 息 而 死 的 流 砂 自 杀 是 极 其 痛 苦 的 。

十 四 年中未被发现,他的目的可以说是达到了,可是并不是值得推荐的方法。

因为,不被发现尸 体的自杀方法还是很多的。


\subsection{案例研究 29}

\subsubsection*{被幼熊咬死的女性}

1989年2月晚上七点多钟,一个六十七岁的家庭主妇跑到离家约一个半小时路程的熊本县阿花町的阿苏熊牧场十二支苑,跳到苑内的幼熊棚自杀了。

熊

痛苦 ▼▼▼▼▽ 

麻烦 ▼▼▽▽▽ 

死状 ▼▼▼▼▽

牵连 ▼▼▽▽▽

冲击 ▼▼▼▼▽

致命 ▼▼▼▼▽

听到「噗咚」大响声的饲养员赶了过去,只看到离栅栏$2.5$公尺下面的熊棚,六十八头熊挤成一团,好像一座小黑山似的。

用灭火机驱散熊后,白粉中浮现出人的身子。尸体已被幼熊撕烂了,内脏完全消失,从胃到肠子处开了一个大窟窿,手和脚等全身有数处被撕咬的痕迹。衣服被撕成碎片,好像祼体似的。

她是虔诚的佛教徒,以前就在谈论「自己将落入恶魔的世界」、「人死了但灵魂永生」等。

中午过后离家时还说:「叫我到那个世上去」,给丈夫也留有遗书,近邻也都认为她是个古怪的人。
 
有的人说这位女子以前曾说过:「我想被老虎吃掉」的话,也曾想到熊棚旁边的老虎笼子,但因没法进入铁栅栏,于是就跳进了熊棚。

\subsubsection*{检验死因}

用被动物吃掉的方法进行自杀,这是出人意料的方法。

不过,从动机来说有点宗教气息,过于脱离现实的死法,不能作为自杀方法的参考。

而且,这种方法痛苦大,尸体也惨不忍睹,死亡的可靠性也不大,因此我不想劝人使用。



\subsection{案例研究 30}

\subsubsection*{利用自杀装置死去的美国妇女}

美国病理学家杰克‧盖博坎博士以开发独特的「自杀装置」而闻名。

死亡装置

痛苦 ▼▽▽▽▽

麻烦 ▼▼▼▼▼

死状 ▼▽▽▽▽

牵连 ▼▽▽▽▽

冲击 ▼▼▼▽▽

致命 ▼▼▼▼▼

这个装置是把生理食盐液、喷妥撒(Pantothal)氯化钾分别装在三个瓶子倒挂的高三十公分的设计 。

喷妥撒是开刀时使用的麻醉药,氯化钾是毒杀死刑时所用的剧药。其结构是自杀志愿者按下揿键后定时器起动,食盐 水自动地变为喷妥撒,一分钟后氯化钾开始流入体内。

利用这一装置于1990年6月进行自杀的是美国家庭主妇贾耐特‧阿德金斯(当时五十四 岁)。她于1989年被诊断为早发性痴呆症而打算自杀时,听说了盖博坎博士的自杀装置, 取 得 了 联 系 。

盖博坎博士准备了一辆白色德国大众小客货车,并寻找了适合于自杀的野营车专用公园。

1990年6月,她和博士一起乘上车驶往公园,她躲在车后,博士首先把针头 插入静脉注射了食盐液,博士调节其流量情况时,贾耐特就按了揿钮,注入液变成喷妥撒, 二十秒钟后她入睡了,然后喷妥撒切换为氯化钾,四十秒钟后脸部发红,再过三十秒就发青, 很 快 出 现 红 斑 点 。

心电图在经过五分三十秒时已完全成为直线,其实她的心脏在此以前已经停止跳动了。

贾奈特是个英语教师,又是个酷爱古典音乐的严肃女性,大家认为她的这种性格使她难以承受早发性痴呆症。

\subsubsection*{检验死因}

这一事件,就是在全美掀起的尊严死自主性大论战的「帮助自杀事件」。

在眼前,真的要完全可靠而又安然死去的话,只有依赖盖博坎博士的帮助自杀,否则别无他法了。

他在1991年10月也用这一装置使两个女性自杀,找他咨询的络绎不绝。

他之所以能使用这一装置,是因他所居住的密西根州没有惩治帮助自杀罪的缘故。

在日本,刑法明确规定了帮助自杀罪的。

所以,贾奈特才特意从奥勒冈州到密西根州。

除了针刺以外没有其它痛苦,又可以在睡眠状态下死去,不论别人看见与否,不想在自己房 间内死去而又愿意去美国的话,不妨取得联系。不过,对没有被病魔缠身的人,他是否肯使 用这一装置就不得而知了。

日本也有尊严死协会,但该会所承认的只是以现在的医疗技术无法医治的癌症晚期痛苦患者,不愿采取白费的延长生命措施的消极安乐死,对一般的自杀是反对的。

日本尊严死协会的联系地点如下:

邮政号码113东京都交京通本乡 2-29-1,渡边大楼2楼,
电话:03-3818-6563


\subsection{案例研究 31}

\subsubsection*{企图骗取生命保险金而煤气中毒自杀的男子}

1971年10月,一个五十六岁男子被发现在公寓的一间屋子里用煤气自杀。

据调查得知,这名男子在自杀的前两年的五月开始至七月间,加入了「安田生命」、「第一生命」、「第百生命」等三个公司的保险,死后家属可支领$7620$万日元。

生前他对人说过,「我要为子女们留下保险金去死。」

这名男子患了四年的肺结核,后来又热爱于赛车而使家计不济,给家中造成很大不幸而感到羞愧苦恼,自杀的时候正和家人分居中。

他在一家不动产公司工作的同时,还兼做保龄球的夜警和生命保险公司推销员,甚至还向亲友借钱支付了总额达$550$万日元的保险费。

据不动产公司的同事们反映,他平时为人老实,不像做那种事的社员。

\subsubsection*{检验死因}

如果是利用普通煤气的自杀,想自杀的人也应该知道一点有关生命保险的事,所以介绍这个
例子。

生命保险金,即使你的死因是自杀,但要加入一年以后才可支付金额。

而且,保险金是领取越早越有利,所以,签订合同以后过了一年就马上死去为好。

不过,考虑这些而去自杀的,恐怕除这名男子以外就再没有别人了。


\newpage

\section{Statistics 自杀的统计}

从 前 的 资 料 对 了 解 采 用 什 么 手 段 ,在 什 么 地 方 自 杀 等 问 题 是 很 有 参 考 价 值 的 ,因此 这 里 就 简 要地加以叙述。

\subsection{方式}

首先是按方式分类的资料,至今为止的自杀者究竟用什么方法死的呢?

\begin{table}[]
\center
\begin{tabular}{llllll}
年份 & 1950  & 1960  & 1970  & 1980  & 1991  \\
上吊 & 6641  & 6560  & 7542  & 10968 & 11313 \\
药物 & 4540  & 8135  & 2211  & 1335  & 1360  \\
跳楼 & 152   & 281   & 562   & 1365  & 2119  \\
瓦斯 & 39    & 834   & 1693  & 2342  & 1251  \\
投水 & 2619  & 2029  & 1762  & 1543  & 1342  \\
撞车 & —     & 1816  & 1142  & 1166  & 865   \\
其他 & 2320  & 488   & 816   & 1823  & 1625  \\
总数 & 16311 & 20143 & 15728 & 20542 & 19875
\end{tabular}
\end{table}

第一名是上吊,其次是跳车,其它排名如下:

据厚生省 1991 年的统计,「缢首、绞首及窒息」(上吊、绞死、塑料袋等)$11313$ 人,以悬殊的优势独占鳌头。

其次是「从高处跳下」的$2119$人,
第三是「固体或液体」(服毒)$1360$人,
第四是「投水(溺死)」的$1342$人,
第五是「气体或蒸气」的$1251$人。

再往下就是跳火山的$865$人,「热伤(自焚)」的$783$人,「刃器及利器」(即割手腕等)的$616$人,触电则格外地少,为$56$人。

上吊曾于1955年至1960年把第一名宝座让给了服毒,但在1980年又回升破了一 万 人 大 关 ,首 位 的 宝 座 坚 如 盘 石 。

跳楼 也 于1986 年 红 歌 手 冈 田 有 希 子 的 跳 下 而 一 下 破 了 当年两千人大关,次年起上升到第二名后未曾再少于两千人,趋于稳定。

服 毒 以 1960 年 达 到 顶 峰 的 安 眠 药 自 杀 浪 潮 而 曾 一 度 升 到 第 一 名 ,后来 有 所 下 降 ,但 不 知 为何到了1985至1987年间又有了回升达到两千多人,但是目前因药品管理日趋严格所以 略 有 下 降 趋 势 。

投 水 在 1960 年 只 有 两 千 人 以 上 的 自 杀 者 ,可是 现 在 却 下 降 到 $1400$ 人 左右。

天然气在1975年前后因煤气普及而出现风潮时,曾有过三千人的记录,自从都市 煤 气 已 变 换 为 不 含 一 氧 化 碳 的 天 然 气 以 后 ,使 用 车 辆 排 放 气 体 的 也 只 有 一 千 人 左 右 ,排 位 顺 序大致来说没有变更。


女子常采用的投水、塑料袋男女间的最引人注目的差异,首先是投水自杀。

自杀者中男人居多在全世界都是一致的,不论从哪一种自杀手段来看,基本上是$3:2$到$2:1$,男人居多。

可是 ,为 什 么 日 本 的 投 水 自 杀 自 古 就 以 女 性 为 多 ,不论 投 水 自 杀 的 总 数 多 的 一 年 还 是 少 的 一 年,男人自杀者只有女性的三分之二。

用塑料袋的窒息自杀,每年不过一、二百件,但大多数是女性采用,其原因完全不清楚。

在正文中也提到过,男子采用的触电自杀却多得出奇。触电自杀每年也不过一百多件,其中 女性自杀者每年只有十多件,而男性从来不会低于全部的$90\%$。

这也是专家中间所未 解的「谜」,当然有人说女子因为缺乏电的知识,但其理由也不清楚。

跳楼对十几岁的女性很具影响力不论哪一个年龄层,上吊占绝对是多数的,但多以十几岁和二十几岁的女子为主。

自1986 年 起 跳 楼 超 过 了 上 吊 。

尤 其 是 十 几 岁 的 ,在 1990 年 里 跳 楼 是$74$ 人 而 上 吊 却 还 有$31$人,而在1986年则为$162$比$78$,跳楼超过了一倍以上。

在1985年,十几岁 女孩子的跳楼自杀只不过是$52$人。

无庸讳言这是受到跳楼自杀的冈田有希子影响的缘故,她在日本自杀史上所留下的影响,是无法估量的。

过去,作为超过上吊数的例子有,在安眠药顶峰期的1960年,十几到二十几岁中间上吊人数$1311$人,而安眠药的自杀竟然达到$3889$人之多。

不过,除去这种例外而言,不论什么时候什么年龄层上吊总是位于第一名。

据说在美国手枪自杀是占第一名,固然是因为容易弄到手枪的缘故。

可是,包括美国的其它各国在内,上吊是普遍被采用的,这就说明了作为自杀手段上吊是最受欢迎的。

\subsection{时间}

\subsubsection*{五月是自杀旺季}

先按月份来看,世界性倾向皆是春秋多而夏冬少。

同时,有的报告说,春天比秋天多,冬天比夏天少,其中最多是四月,其次是五月,最少的是十二月和一月。不过,无论哪一个月份,差别都不是很大。根据最近几年日本的统计,五月份最多。

据1991年的资料,这一年五月最多而二月极少。

\subsubsection*{星期天自杀也休息}

在东京市内,整理了六年救护科机动处理的自杀者两千余人的情况,自杀多少的顺序(包括未遂者)是星期二($15.3\%$)、星期一($14.9\%$)、星期五($14.5\%$)、星期天($13\%$)、星期六($13.8\%$)。

东京市内的某区调查结果是星期四最多,星期三、星期日最少。

从已自杀成功的看,星期三最少。

一 百 多 年 前 ,法 国 的 社 会 学 家 盖 尔 凯 堡 所 作 的 统 计 显 示 ,星期 二 和 星 期 四 多 而 星 期 五 至 星 期 日就减少。星期日较少这一点都是一样的,其它的日子没有多大的差别。

\subsubsection*{夜间自杀者多过白天}

据东京观察医务院自1955年以来三年间所处理的八千人的资料分析,夜间的$21$~$24$时最多,清晨最少。
当时大为流行的安眠药自杀,$21$~$24$时占$36\%$,而$56\%$则是在$21$~$3$时发生的。

在镰仓进行的未遂者调查,最多的也是在夜间,清晨和白天较少。

把一天照六小时分成四段在进都进行的调查显示,已遂中多的是$12$~$18$时之间($32.1\%$)和$0$~$6$时之间($27.4\%$),少的是$18$~$24$时($19.0\%$)和$6$~$12$时 ($21.4\%$)。

但是,在未遂者中反而是$18$~$24$时,占$31.0\%$最多,其次是$12$~$18$时的$29.4\%$。 这 些 调 查 结 果 显 示 结 果 不 一 致 ,所以 不 能 一 概 而 论 ,但 是 自 杀 在 夜 间 与 早 晨 至 少 还 是 可 以 取 信的。

但是,在新瀉县所作的调查的顺序则是$0$~$6$时($29.4\%$)、$12$~$18$时($23.5\%$)、$18$~$24$时($21.6\%$)、$6$~$12$时($17.6\%$)。$6$~$12$时较少,这一点大致相同,分析除此以外的结果是有困难的。

在农村中老人自杀较多的地区,其自杀时间可以说与都市的时间是不同的。

\subsection{排名}

\subsubsection*{自杀是他杀者的两倍}

在战后,1955年至1958年和1983年至1986年出现过两次高峰,一年里有两万多死亡者,从此以后一直减少,1992年略有回升而已。

不过,1983年之所以增多被认为是小额信贷的缘故,1986年则是受冈田有希子的影响。
从整体来说,大体上以两万人为标准上下浮动的。这是交通事故的$1.5$~$2.0$倍,他杀的$2.5$倍的数字。

\subsubsection*{自杀在 20至30 岁间名列前矛}

从死因类别来看,近十年来自杀跟随在癌、心脏病、脑溢血、肺炎、支气管炎、意外事故、衰老之后,占第七位。战后时期,因患结核和肠胃炎而死亡的很多,所以顺序还在后面。

可是,从年龄层来看,最近几年来的记录显示,自杀在二十多和三十多岁中占第一名或次于事故或癌症的第二名。在十五至十九岁、四十至四十四岁,也在第三名。

在自杀率很高的老人中间,随着年龄的增长自杀的顺位排列就下降。

厚生省统计情报部所说的「仅从数字来看自杀者中以老人居多的说法,未必是正确的」就是从这种原因。

处在第一位的癌,到了四十多岁常年的死亡者达到一万多件,但随着到了五十、六十、七十多岁,其死亡数则以二万、四万、六万地增加。同时,自杀的排位顺序却是第五、第六、第七。

具体地说,从1991年的数字看,二十至二十四岁的自杀者为$962$人,这个年龄层的每十万人的死亡率为$10.4\%$,排在第二。而七十至七十四岁则为$1188$人,死亡率为$30.5\%$,排名第九。的确,按人口比例来说自指数是高的。

可是,如果有人问到自杀中死的最多的是什么人的话,这个顺序排列就是重要的回答。

可以说,从十到三十多岁的年轻一代以自杀而死的是最多的。

\subsection{国际}

\subsubsection*{难以理解的自杀大国——匈牙利}

即使在日本的统计方法,厚生省和警察厅就在两万人中相差一千多人,以按国别进行比较是没有必要了,但却是个有趣的问题,因此加以比较。

根据WHO(世界卫生组织)的最新资料所载自杀死亡率(每千万人口中的自杀死亡数)来看,以自杀大国闻名的匈牙利独占鳌头,占$39.9\%$(1990年调查)。仅仅男人就占了惊人数字$59.9\%$。以下是斯里兰卡的$33.2\%$(1986年)、芬兰的$28.5\%$(1989年)、丹麦的$24.1\%$(1990年)、奥地利的$23.6\%$(1990年)、前苏联的$21.8\%$(1990年)、瑞士的$21.9\%$(1990年)。

相反地希腊极少,只有$3.8\%$(1989年),可以说少得可怕。

其它就是以色列的$6.8\%$(1988年)、阿根廷的$7.4\%$(1987年)、英国的$8.1\%$(1990年)、葡萄牙的$8.8\%$(1990年)、波多黎各$8.9\%$(1989年)、荷兰的$10.2\%$(1989年)、香港地区的$10.5\%$(1989年)等。

总括来说,统计是有点暧昧的,不必 深究。 主要国家中,日本是$16.4\%$(1990年)、美国是$12.4\%$(1988年)、法国是 $20.9\%$(1989年)、中国是$17.1\%$(1989年),中国的女性自杀者较多是个 特点。

加拿大是$13.3\%$(1989年)。

日本在第二次安眠药自杀高峰期间曾跃居第一名而被称为「自杀大国」,目前居于中间。

以枪自杀的美国,以上吊自杀的德国因国家不同而采用的自杀手段各有不同也是很有趣的。

按 照 目 前 的 说 法 是 ,美国 以 枪 击 ,拥 有 水 都 威 尼 斯 的 意 大 利 以 投 水 ,德 国 以 上 吊 都 具 有 特 色 。

据最近的资料得知,美国因枪和爆炸物的自杀是$47.1\%$,确实不少。

以下就是服毒。煤 气是$20.8\%$,上吊是$20.5\%$,其它手段则相当地少。在意大利、瑞士、加拿大、澳大 利亚,据说枪枝自杀最多。

由西德的1969年的资料显示,上吊占绝对多数,为全部的$39\%$。

以下就是服毒。
安眠 药是$23\%$、煤气是$13\%$、投水是$9\%$。

当然也有地区差别,柏林以煤气、汉堡以安眠药最多,与日本相似之处就是跳楼只占$5\%$。同样在瑞典和丹麦,上吊是最多的。

英国的英格兰和威尔斯1965年的资料显示,男性是煤气占$34\%$最多,接下来是镇静剂。 
麻醉药$25\%$、上吊$13\%$。女性药物$49\%$和煤气$33\%$,不知为何这两种多得出奇,除此以外的方法都在个位数。

法国是药物。据1966年和1970年的资料,第一名是药物,第二名是利刃,第三名是煤气。女性用药物的特多。

印度也同样是药物。
1966年的统计,采样数虽只有$912$人,但顺序为药物$46\%$、投水$22\%$、上吊$14\%$。

在非洲的尼日利亚,据1962年发表的资料,上吊最多占$50\%$,以下顺序是枪枝、毒物、利刃。

上述情况,看来反映了各种不同的国情。

\subsection{自杀胜地}

与日本一样,海外当然也有自杀胜地。
最为著名的就是美国旧金山的金门桥。

就像很多自杀者所说的并不是为了自杀的目的而来的那样,可是目睹令人萌生自杀念头的绝景,纯属气氛。某位七十岁的老人留下了「为什么造就了这样容易自杀的风景呢?」的遗书就跳了下去。

同样地,美国的尼加拉瓜瀑布也被称为是自杀胜地。

英国的泰晤士河、印度的恒河等都曾被称为自杀胜地,在日本来说大概就是华严瀑布吧。



\newpage

\section{Postscript 后记}

在前言中,写了近似「关于现代社会与自杀」这样夸张的开场白,不过事实上的确是有点言过其实。
写这本书的最初理由,只是因为厌倦了「不可以自杀」这个怎么想都想不出有任何根据的观念,却受到大众非常单纯的信任。

小学老师用「生命的重要」做题目请学生写作文的情形到处可见;自杀的人随随便便被说是意志力薄弱等等。

「要活得坚强」这种话轻易就能说出口的社会,不仅封闭得令人喘不过气来,而且也令人活得痛苦。因此,让这本书流通,创出「万不得已时,也可以寻死」的选择,希望在封闭且走到末路的社会开辟一个通风口,使空气流通,使生活更容易些,这才是我真正的目的。

我不是在阐述「大家都来自杀吧!」这样无聊的事,想活的话就要活得自在,想死的话也要死得自在,生命应该就是这么一回事。

\newpage

在此,要向负责编辑的落合美砂、负责美术设计的铃本成一及插画的沙达希罗卡兹诺利,致上深深的谢意。


\newpage

\section{Appendix 附录}

中国时报 84 年 5 月 9 日\ 星期二\ 第五版

出版商紧急回收「完全」系列丛书

负责人陈明达提前自大陆返台善后

原订近期出版的新书也一并取消

\subsection*{【记者张企群台北报导】}

被台北地检署认定有煽惑他人犯罪之嫌的「完全复仇手册」,出版商茉莉出版社和代理发行的黎铭图书公司,八日已展开回收措施,至于同样引起争议的「完全自杀手册」也决定全部回收。

茉莉出版社负责人陈明达昨晚已搭机返台处理相关事宜。

据了解,这项回收作法是茉莉出版社与黎铭图书公司于前日作成的决议,茉莉出版社负责人陈明达在国外获如「完全复仇手册」被国内检察官认定有教唆犯罪之嫌,以电话紧急通知公 司将「完全复仇手册」全数回收,并委请代理发行商台北县三重市的黎铭图书公司办理回收 事宜。

昨 天 上 午 ,黎 铭 图 书 公 司 已 出 动 外 务 员 携 带 回 收 通 知 书 送 给 北 部 各 书 商 ,而中 南 部 的 书 商 则 以传真方式通知,希望在最短的时间内将全省市面的「完全复仇手册」悉数回收。

据了解,一般惯例书商在接获回收通知书的一个月内,会将书籍退还,如果逾期不退,发行
商便不再接受退书,因此,回收的关键仍视书商的配合度而定。

黎铭图书公司经理林俊言表示,今年三月才上市的「完全复仇手册」只印了一版,原本市场
反应平平,但在媒体曝光后又变得抢手,各地书商纷纷要求补书,目前市面上的书很可能大
部分已流入消费者手中,因此回收的数量可能有限,不过该公司还是会尽最大努力进行。

原本在大陆洽公的该公司负责人陈明达,昨晚也提前结束大陆的行程搭机返台处理后续事 宜,他表示,当初出版这一系列的书籍,是着眼于日本、香港等地都面临相同的问题,台湾 走在日、港的后面,势必会遇到相同的情形,才本着关怀社会的心理出书,先后出版「心目 中的自杀」、「完全自杀手册」、「完全病死手册」、「完全失踪手册」和「完全复仇手册」, 绝无煽惑他人犯罪、自杀的不良意图。

陈明达说,基于该系列书籍造成社会人士的非议,他返国后已决定除了回收市面上约三千余 本的「完全复仇手册」外,约万余本的「完全自杀手册」也一并回收,至于另二本「完全病 死手册」和「完全失踪手册」也考虑予以回收,一切损失由该公司自行吸收,此外,他也打 消原本该公司预定近日出版的「完全中毒手册」,今后对书籍的出版会将更审慎地研究评估。



\end{document}