%-*- coding: UTF-8 -*-

\documentclass[UTF8]{ctexart}
\usepackage{graphicx}
\usepackage{float}
\usepackage{color}
\usepackage{CJKpunct}
\usepackage{amsmath}
\usepackage{geometry}
\geometry{a4paper,centering,scale=0.8}
\usepackage[format=hang,font=small,textfont=it]{caption}
\usepackage[nottoc]{tocbibind}
\setCJKmainfont{STSongti-SC-Light}
\setromanfont{FandolSong}
\punctstyle{quanjiao} %使用全角标点	


\title{关于死的笔记}
\author{寺山 \ 修司}
\date{海带岛 \ 译}

\begin{document}

\centerline{\large{\textcolor{red}{WARNING}}}
\centerline{\large{\textcolor{red}{警告}}}

\centerline{\textcolor{red}{NOT SUITABLE FOR CHILDREN }}
\centerline{\textcolor{red}{OR THOSE WHO ARE EASILY DISTURBED}}
\centerline{\textcolor{red}{不适合儿童或那些情绪容易受到影响的人}}


\newpage

\maketitle

\newpage
\tableofcontents
\newpage

\section{自杀机器的制作方法}
我少年时代热衷于制作自杀机器。伊林认为,人类的历史是「工具的历史」,猴子之所以会从树上下来变成人,是因为它发明了工具。
但工具渐渐地文明化,又变成了机器。在人能够熟练使用它的期间,它是工具;但等到它反过来支配人的时候,就被称为机器了。

中学生物老师说过,人类与工具一同发展起来,大概会与机器一同灭亡。从此,机器和死亡在我心中就变得再也无法分开,我对发明「自杀机器」也越来越感兴趣。

我在旧书店找到一张查尔斯·亚当斯画的非常棒的自杀机器图纸,它是那本《亲爱的死日》书中的一页。
这种机器是让人自行了断的断头台,只要闻了药物在上面睡着,头上就会有把利斧落下来实施斩首。

我发明了几种自杀机器,但没有亚当斯的那么精巧。

先说说「双鸡式自杀器」吧:我得坐在椅子上读江户川乱步或其他什么人的书,预先固定好一把对准我心脏的上膛猎枪,并用绳子把扳机跟两只鸡的脚连起来。
然后我把沙袋顶在头上,再让两只鸡站在沙袋上。
由于沙袋是开了孔的,沙子一点一点朝外漏,两只鸡越来越站不稳,就会本能地飞下来。
这时它们脚上系着的绳子便会拉动扳机,让猎枪射出子弹将我打死。

另一种是「上海丽儿型浴缸自杀器」:先在荡漾着老电影主题歌《上海丽儿》的浴室中坐进浴缸,把剥掉外皮的留声机电源线缠在洗浴所需水位的地方,再拧开龙头向浴缸里放水。
然后我一丝不挂(最好戴上圆顶硬礼帽),心情愉快地边洗边听《上海丽儿》。
当浴缸水位一点一点上升到赤裸电源线的高度时,我就能够在一瞬间成功地触电身亡。

除了这些,我还发明了「螺栓式自杀桶」、「脱粒机型头顶振动自杀器」、「绕颈风琴自杀器」、「四〇式心脏破裂发动机」等等。

我心想:「为什么学校不让我们在上课时间做自杀机器呢?」现代的机器大都是「他杀机器」,而且汽车、轻轨、公害烟囱和污水都不是为了「杀人」来到世上的,它们的出现是源于其他目的,只不过正作为他杀机器的代用品在被人使用罢了。
至于人应该如何死去,我觉得首先必须恪守尊严,讲究方法,排除被动遭人杀害的死亡。

而死亡的自由,则希望由我们自己来创造。

\section{遗书的高明写法}

决定自杀之后,就得练习遗书的写法了。

遗书也是一种信,所以写的时候必须尽量考虑对方的心情。
不用说,字要写得整齐端正,如果潦草得看不清楚,那就等于白写。

不过,你就是跑到书店去买一本什么《美文书信写法》来也无济于事,那上面满篇都是什么贺年信、通知信、问候信、拜托信的各种写法,唯独没有教你如何写遗书的。

如果你辩解说「本来想自杀,但因为遗书写不好,所以只好作罢」,那你是永远没法自杀的。

因此,我建议你在做好自杀机器之后,专门来学习一下遗书的写法。

\subsection{任务}
首先,你必须明白遗书承担的两个任务。

其一,是选好措辞对死进行恰当描述,将自己的心情传达给对方;其二,是对相关事务进行处理。传达自己心情时,夸大其词也没关系,可以像藤村操那样「欲以五尺小躯测天地之大……不可解!」
也可以像我的中学老师那样,把「诸位日本同胞,再见了」的遗言留在一个空箱子内面上,不拘什么书写格式。

至于「对相关事务进行处理」,其实用不着解释,说的就是遗产分割的相关事宜。

除了上边举出的这两个任务外,遗书还可用来表达自己的情趣。譬如,太宰治在与自杀并无关系的场合,也写下了颇有情趣的遗书:

\begin{quotation}
不至于会有那种事吧?大概不会有的,对吧?为我造铜像的时候,必须让右脚朝前伸出一半,胸膛从容地挺直着,左手插在西装马甲的口袋里,右手攥着写糟了的稿子,但可别把我的头放在铜像上。别误会,我没有什么别的意思,就是讨厌麻雀把屎拉在我的鼻尖上。还有,给我把这些字刻在基座上头:

\begin{verse}
这里埋着一个男人。

他被生出来,后来又死去了。

这个人一生都在撕碎自己写糟了的稿子。
\end{verse}

\end{quotation}

\subsection{措辞}

遗书措辞可以使用自己平常习惯的口语来写。不必因为将死,就忽然郑重其事地改用敬语或文言。

\subsection{书写}

遗书应该亲笔书写。不管字写得多差,都比代笔的遗书更能打动人心。引用别人的遗书也没关系,但全盘照搬的剽窃就没意思了。

自杀的乐趣有一半是「写遗书的乐趣」,所以将这种乐趣让给别人太不划算。只有在写情书或有奖征文之类能带来实际利益的东西时,才用得着剽窃。

\subsection{笔迹}

遗书的字写得越工整认真越好。如果凌乱潦草,会被人推测你赴死之时已心慌意乱。不过,要是有个把错别字,反倒会让人印象深刻,久久不忘。

诗人雷蒙·拉迪盖写道:「高明的着装是随意,高明的文章也是如此……」

\subsection{检查}

遗书写完后不要立刻加封,一定要再看一遍。因为要是漏写了重要事宜,或是有说得过火、说得不周全的地方,死了以后就没法重写了。

\subsection{纸张}

遗书的纸要尽量不用稿纸(稿纸会使文章看起来像是虚构的)。用信笺纸没有问题,但写在没有行线的信笺纸上时,要用有行线的纸垫在线笺纸下面作为参照,以免写歪。在卷纸上用毛笔写也行,但你得记住,就算你草体书法发挥得淋漓尽致,死了之后也是没法解释别人看不懂的字的。

\subsection{信封}

遗书的信封过于花哨或带有水森亚土插画的,我觉得都不适用。把遗书放进信封时别弄错收信人。我少年时代就曾经把催母亲快点儿寄钱过来的信和写给酒吧女的情书给放错了信封,结果差点儿被逐出家门。所以必须牢牢记住,把遗书放错信封可是件能让你后悔莫及的大事。

\subsection{寄送}

明信片遗书可以寄给很多人。我念高中时收到的文艺兴趣小组学妹盐谷津子寄来的遗书就是一张明信片。她是从青函渡轮上投海自尽的,出发前给十二个朋友写了明信片遗书。不过,很难说标准明信片和图片明信片哪个适于写遗书。明信片一行能写十五六个字到二十来个字,遗书写十二三行就差不多了。如果用图片明信片的话,写得再短点儿也够了。

\bigbreak

以上内容是参照着《优美的女性书信写法》(花房恭一郎)一书写成的。实际上,遗书即使写在石块上或是墙壁上也没关系。遗书虽然是在对自己的死进行诗化表述,但书写时的空间并不需要师法文学创作。即便飞行员以航迹云为遗书,同性恋者以厕所涂鸦为遗书,广告撰写人以大报纸上的广告为遗书,也未尝不可。

十来年前,曾有一个九州大学工程系的学生用收音机自杀。他是在电台刚开始早晨播音时,让缠在身上的电线接通收音机电流来自杀的。致他以死命的收音机里当时正在播放着晨间宗教节目,那清脆爽朗的播音,就是他的「遗书」。

\section{动机也是必须的}

做好自杀机器,会写遗书之后,为方便起见,还是得先找个自杀的理由。因为对于「看到美丽的花朵后,我就想去死了」或「我早就想死死看」的心情,一般的社会理念是绝不会给予理解的。

在理性判断优先的社会里,「本质先于存在」被公认为天经地义的事。如果有凡事好为人师的解说家们信口开河地告诉你一些理由,你最好自己去琢磨琢磨哪个理由听上去最为合理。因为不管他们说的是否符合事实,那些理由都是你原来不知道的。

有个高中生出外修学旅行时,用浴衣的带子吊在旅馆拉门框上自杀了。「我怎么也控制不了手淫,」他在遗书上写道,「我其实已经不想再手淫了,睡觉时是先用浴衣带子把两手绑起来才躺下的。但手还是不知不觉地又开始动作起来。今天望着安艺宫岛美丽的大海时我心想,自己只有去死了。」

他的遗书既不是写给父母也不是写给老师,而是写给「上帝」的。时至今日还有高中生将手淫视为犯罪,这不能不使我感到吃惊。事情始末的真伪姑且不论,但他的自杀到底是可悲的。

与他相比,有个就在两三个星期前自杀的女中学生的情况更为悲惨,因为她没经过仔细考虑就在参加高中入学考试的前一天自杀了。

据同学说:「她说已经收到了第二志愿高中的合格通知,可是大家按她的中学成绩来看,原来都以为她能轻轻松松地进第一志愿高中的。」

她在遗书中写道:「给父母看过那份合格通知后自己就想自杀了。」想必她是突然起意自杀的,然而报纸上却给这条消息加了个「少女因苦于应考而自杀」的无聊标题。

\newpage

\begin{verse}
其实,

虽然无法大声说出来,

但漫长的过去、

漫长的未来,

都是一样的。

死死看吧,

我已经都明白了。

渊上毛钱
\end{verse}

就如这首诗中所说的那样,「死死看」有时也是一种经历。

如果将无法重复的体验、负面的体验理解为与「经历」无异,大概就能明白,死也是一种旅行。

总之,死的动机或理由都是编造出来的,它具有偶然性和虚构性。

所以,就像太宰治那样,「我原来一直想死,可是今年过年时得着一块和服衣料,算是压岁钱。衣料是麻纺的,上边织着鼠灰色的条纹,大概是夏天穿的吧。于是我想要一直活到夏天了。(《叶》)」预定是可以因一块和服衣料而改变的。因为假如自杀是美丽的,它就是虚构,是带有偶然性的。

有些中小企业的业主被债务逼得破产而自杀,这种表现形式上的自杀实际上是他杀。这种因资本主义社会膨胀过度的弊端造成的自杀,不管形式如何,都应该算是他杀,所以它已超出了我的《自杀学入门》的范畴。而我只想就自己能够揭示其内涵的偶然性自杀继续进行探讨,只想更为享受地来谈论自杀。

\section{选择适于自杀的场所}

决定自杀之后,就得考虑自杀的场所了。选择场所的重要度就像演戏时组装舞台布景一样(或者说比它更重要)。虽然日本国营交通公司发行的旅行指南没有登载自杀胜地,但不等于说那几个以往被称为自杀胜地的地方就消失了。不过,适于自杀的风景已经逐渐被政治化,并遭到了越来越严重的污染。

达米娅曾在歌中唱道:「死于大海的人会变成海鸥。」可是现在别提什么变成海鸥了,大概企业随意排放的污染物让那些死于大海的人都快腐烂光了。

蒸汽机车时代的卧轨自杀本来也不错,可是现在铁路正不断高架线路化,卧轨自杀也只能从车站的月台上往下跳了。

如今上吊只有尼龙绳可用,这也挺没情调的。

尾井基次郎描写的「樱树下埋着尸体」的情景也已无法重现,因为随着土地价格飞涨,人们开始感到,与其把樱树种在庭园中,不如将装配式房屋搭建在那里更为合理。

这一来,自杀者必须在场所选择上劳心费神,还不得不考虑自己死后尸体会被置于何处。

举个例子吧,很久以前就曾流行过在水边自杀。早在古希腊时代,女诗人莎芙失恋后就自杀在雷乌卡迪亚水中;埃俄罗斯的女儿亚克安娜也是在得知丈夫死后紧接着跳海自杀的。

据自杀研究家山名正太郎分析:跳入水中的自杀是出于「对母胎内羊水」的向往。还有一种观点则认为:因为母胎内漆黑一团,所以跳水自杀大多都发生在夜间。据说巴黎曾有人按照一个跳塞纳河自尽少女的面型做成石膏像出售,为其取名为「塞纳河姑娘」。但由于不断有人仿效那个少女跳进塞纳河自杀,最终那个「塞纳河姑娘」也被砸碎后扔进河里,完成了她的第二次跳河。然而今天的塞纳河也已经被严重污染,再别提什么女诗人莎芙了,甚至连塞纳河面也已经漂满了醉汉的呕吐物、坏掉的电灯泡和老鼠的尸体。我感到,那些想跳河自杀的人必须要有为了自杀另开一条河的心态。

演员为了几句台词都尚且需要特地制作舞台布景,如此想来,自杀者也有必要花费功夫「制作」出适合自己去死的布景。为了生存而创造世界的人应该也要为了死而再去创造另一个世界。那世界的布景要规划、设计、上彩——无论说它比任何舞台布景大多少倍都不过分。一台戏再长也不过演三个多小时就会结束,但死却会永远延续下去。而且,死亡也是一种虚构,其证据在于谁都无法触摸到自己的死亡。

\subsection{提议之一}

如果想要使自己死得别具一格,就应该单独为自己制作一套自杀用的布景。因此,从萨尔瓦多·达利到赤冢不二夫,自杀者必须挑选出自己喜欢的美术家,让他们来制作大布景和大道具(或者你自己来做)。

\subsection{提议之二}

适用于自杀的小道具也是需要的。一朵红花什么的只能使自杀变得感伤,因而更为唯美或幽默的小道具才与自杀相匹配。有一篇新闻报道给我的印象很深,说的是芝加哥哈莱姆有个黑人工人自杀了,他身旁放着一个「笑袋」,那「笑袋」陪在黑人尸体旁笑了整整一夜。

\subsection{提议之三}

大小道具备齐之后,当然还需要照明和音乐。即使无法尝试制作最能突显上吊效果的照明,无法从保罗·麦卡特尼的作品中编选出自杀的主题音乐,也希望自杀者能发挥个人优势,把适合自己个性的照明、音乐(可能的话还包括服装、化妆)提高到令人满意的程度。


\begin{verse}
樱花虽美艳,风来落满城;

转瞬繁华逝,唯余告别声。
\end{verse}

\section{自杀许可证}

但是,不能因为自杀机器做好了,得体的遗书写好了,自杀场所选好了,就以为任何人都可以自杀了。就像驾驶汽车要有驾驶证(许可证)一样,自杀也是需要许可证的。

一般来说,社会福利主义者会将一生保持童真的老神父、素食主义的道学家和人文主义当作招牌,然而我却不像他们那样认为什么「生存是至高无上的」。

但为了恪守自杀的价值,我希望你们避免将事故死亡、他杀、病死与自杀混淆在一起。精神障碍导致的上吊死亡属于病死;被贫困痛苦的生活逼得含着煤气管中毒身亡属于「政治性他杀」。
为了缺少什么东西而去求死的,都不能成为领取自杀许可证的对象。因为对「缺少的东西」进行思考之后,死的必然性就会完全消失了。

假如有这么一个人,他家庭幸福经济宽裕,又适逢天晴气爽花香鸟鸣,本没有任何不如意的地方,却突然想要去死。
这种靠充沛的物质和价值的替换都无法避免的不合逻辑的死才属于自杀。

从这个意义上来说,三岛由纪夫最完美地实施了自杀。

自杀是一件属于有钱人的极为奢侈的事,如果不从这一认知开始进行分析,就无异于颠倒了「被某人所杀」与自杀的不同性质。

那么,活得越心满意足,是否死后的世界就越有魅力呢?我们再次来探讨一下这个问题。

费铎在《死者仪礼史》这本书中指出,死与生一样,也是实际存在的。作为例证,他还介绍了自己对班巴拉族人影子的研究:死者会变成影子到别的地方去生活,能够在水中获得精灵的保护。

当想要重返生的世界时,又会找个刚出生的婴儿附体而复活……那么,人变成影子后能享受到什么样的快乐呢?

有人说「那是一种透明人的快乐」,死后变成透明人,就能免除现实社会中的烦恼,能到各种地方去,还能看到别人的生活;
也有人说「透明人可以不用上税」;
还有个叫兰斯顿·休斯的诗人说「坟场是最经济实惠的旅店」;
河内邦夫欢呼如愿以偿实现梦想的「万岁!万岁!我变成透明人啦!」也是一首自杀歌曲。

然而,所有这些对于死后的幻想,终究只属于逃避现实的思想,却并没有说明自杀的快乐。

如果不是「走向死亡获得自由」,而是「脱离苦海获得自由」,那自杀则无异于一种失败的自由。

我认为对于以下几种人是不能够发放自杀许可证的(即使他们擅自了断自己的生命,其实质仍然可以说是属于事故死亡或病死)。

没有自杀价值的人:

1.因早泄、性器短小而烦恼的人;

2.考大学失败的人;

3.听了摇滚乐后毫无反应的人;

4.因患痔疮而烦恼的人;

5.不知不觉变得厌恶人生的人;

6.因痴迷于扒金窟而一直挨骂的人;

7.「什么叫意义?什么叫无意义?体系化思想无非是意识的私有化,1920年以后的体系化理念不过是一直在历史性地充当体制的补充物。我们毫无目的地盲目追求具有极端无意义倾向的事物,感受到了自己小市民性的局限……(原文如此)」——喋喋不休地纠缠于这些问题的人;

8.童男,处女;

9.低收入工人;

10.还没尝过鱼翅汤的人;

11.还没听年轻姑娘对自己说过「我爱你」的男人;

12.看了高仓健演的电影后,心中羡慕不已的人;

13.有挪用公款、破产、生活困苦等沉重压力的人;

14.正在治疗脚癣的人。

自杀归根结底是一种使人生虚构化的形式,一种依托于戏剧理论之上的典礼,一种自我表现,一种神圣的一次性快乐。
为了达成生存自由与死亡自由同价化的目的,自杀还必须杜绝虚假仿冒,严格执行许可证发放规定,确保特权阶级对自杀的垄断。

\section{自杀绅士论}

到目前为止,我都是在总括性地谈论自杀,
这一章则想介绍几位「自杀绅士」。

以后想自杀的各位,可以将这些先行者作为参考范本,实施精彩的自杀。\footnote{太宰治《斜阳》:就让想活着的人去活吧。人类有生的权利,自然也该有死的权利。}

\subsection{\fbox{藤村操}}
一九〇三年三月二十二日自杀。
当时,作为一高\footnote{相当于东京帝国大学的预科,毕业学生基本都进入东大。}学生的藤村操从沐浴在阳光里的华严瀑布一跃而下,为日本自杀史带来「纯粹自杀」这一概念。

他在松树上刻下遗书《巌頭之感》,投身瀑布潭,用这种华丽的死法大大改变了自杀的意义。当时的报纸称藤村的自杀为「哲学自杀」,而「巌頭之感」也变得比任何诗歌都更广泛地流传开来。

\begin{verse}
    悠悠天壤,辽辽古今,五尺之躯想不透如此大哉问。
    
    赫雷修之哲学,值多少权威?
    
    万有之真相,一言以蔽之,不可解。
    
    怀抱胸中之恨,烦闷,最后选择一死,
    
    既已站在岩上,胸中了无不安。
    
    始知最大的悲观竟等于最大的乐观。
\end{verse}

因为藤村的自杀而成了「自杀圣地」的华严瀑布在一九〇三年到一九一一年之间就有两百多人投潭自杀(包括未遂者)。
印着《巌頭之感》的明信片也成了畅销商品,最后终于为了预防自杀模仿者再增多而停止发行了。

当然,也不是没有人批判这种风潮。「那哪里是什么哲学自杀,根本就是失恋自杀而已。」宫武外骨不仅这样揶揄,而且还模仿藤村的遗书写道「始知虚张声势竟等于沽名钓誉」。

但是,藤村的自杀毕竟不同于因为中小企业破产而走投无路以致全家用煤气自杀的情况。
那种死显露着「贫乏」,而他的死却是从「丰饶」中培育出的。在与自然与死的和亲行为中,蕴藏着对维新以来的近代化的批判。

在美学上获得合理性的这一自杀最终证明了,死与生一样,都是实存的方式。

\subsection{\fbox{原口统三}}
一九四六年自杀。藤村死于十八岁,而原口死于二十岁。

同是一高生,原口的死却比藤村蒙上了更多历史的阴影。战后的混乱,所有价值的崩溃带给原口要「去另一个世界旅行」的念头。

原口写过「诗人曾说,原口是在人生之初在降生之日就已经失恋的男人。」他无法爱他人。
你说他是太过自恋那也未尝不可,但这也同样映照了战后的社会氛围中隐藏在人与人关系中的无法信任。

原口共鸣于波德莱尔的一句话「爱情可以起于一种慷慨的情感:卖淫之瘾。但它很快就会被占有欲所侵蚀」。
他自己则写下这样的句子「爱无疑是我们的故乡。而我则丢失了家园」。

原口认为与追求人生之外的事物这种严肃感相比,人生本身所显露出的事物却都太过抽象\footnote{不真实。}了。

\begin{verse}
    养育我的家庭、父母、兄长、姐妹,连看惯了的家具都成了家人,娇惯着我。

    这藏身之地的温暖我无法忍受。

    我想品尝冷冽,要在精神的冒险之旅上前行。

    要拒斥一切的温暖,唯有以死为结点。
\end{verse}

\subsection{\fbox{圆谷幸吉}}

如果可以的话,很想像圆谷那样,奔跑着死去。这是对于一个长跑运动员来说,最美的自杀方式,因为那连接「最大的悲观和最大的乐观」。

一九六八年一月上吊自杀\footnote{查了很多资料都说是割腕,寺山应该是记错了。}的圆谷,是东京奥铃匹克马拉松项目的奖牌获得者。

圆谷本该在同年冲击墨西哥奥运会的奖牌,但却突然留下「幸吉累了,跑不动了」这样的遗言,选择了死亡。

圆谷的死,与其说是自杀,不如视作他杀。报纸写着「逼死圆谷选手的究竟是什么」,以暗示有一个犯人藏于背后,是在讨伐让这个孤独的长跑者背负奔跑使命的那被视作奥运荣光的「爱国思想」。

但圆谷的死又并非单纯的他杀。他在遗书中,像参加马拉松时一样均速地行走在自己的记忆里,那些正月里的美食变成了内心的风景被描写出来。那遗书变成了一首美丽的诗。

\begin{verse}
父亲大人,初三的山芋泥很好吃。柿干和年糕也是。

敏雄哥、嫂子,寿司很好吃。

克美哥、嫂子,葡萄酒和苹果很美味。

严哥,嫂子,紫苏饭和油炸鱼\footnote{正确的翻译应该是「洋葱醋腌油炸鱼」,为了行文流畅直接译成油炸鱼了。}很好吃。

喜久造哥,嫂子,葡萄酒和养命酒很好喝,谢谢一直为我洗衣服。

幸造哥,嫂子,谢谢让我搭便车,乌贼刺身很好吃。

正男哥,嫂子,很抱歉让你们为我操心了。

幸雄,秀雄,干雄,敏子,竹子,良介,明子,雪江,光江,彰,芳幸,惠子……

你们一定要成长为优秀的人啊。
\end{verse}

\subsection{\fbox{伊势滨}}

同样是运动员的自杀,但又稍显特别的是一九二八年的大关伊势滨。
他退役后上了年纪后,在相扑协会做总务。某天突然给儿子留下遗书「别像父亲这样懦弱,长成优秀的人吧」,自杀了。

特别的地方是他是用老鼠药自杀的。一个大男人喝下杀死老鼠的东西在痛苦的折磨中杀死,是否暗藏着自我惩罚的意味呢,或者说批判自己只不过是区区一只老鼠而已。

加东秀君说过「日本经济的工业化从明治中期开始,到大正始才完成,而昭和初期则出现了能反映这种工业化成果的自杀」。

比如芥川龙之介的巴比妥,东大学生川田的东莨菪碱和卡摩丁,千叶医科大学的助教青木所用的茛菪胺皮下注射,再从重铬酸钾的流行到老鼠药。

老鼠药是谁都能轻易买到的东西,但即便如此,大关和老鼠药的结合让这起自杀成了和工业化社会的同谋,要视作纯粹的自杀,还差那么一点。

\subsection{\fbox{调所五郎}与\fbox{汤山八重子}}

一九二三年五月九日,神奈川县大矶附近的草丛发现了调所五郎和汤山八重子殉情的尸体。当时的报纸以「坂田山殉情,结缘在天国」为标题对他们的死做了特写报道。

五郎是庆应大学的学生,八重子是富家千金,两人的尸体边上是北原白秋的《青鸟》和羽仁本子的《婴儿的心》,还有两人服毒所用的氯化汞的瓶子。

五郎穿着制服,八重子穿着淡紫色的和服。两人完全发生性关系,但身边放着《婴儿的心》这样的书,足见这场自杀是充分计划了的虚构物,是两个合作演出的剧作。

在我看来,这场殉情也和藤村操一样实践着「死的权利」,这种美学(虽说着实太过哀伤)已经是自杀的一种典型模式了。

五郎和八重子还被流行歌曲这样唱道:「死后愉快地升入天国,才好做你的妻子」。
五郎的父亲伤感地在杂志上发言说「八重子小姐,请称呼五郎为丈夫吧。五郎,也以妻子称呼八重子小姐吧。神灵啊,请福泽这对可怜的灵魂吧,赐予他们平静。」

但是呢,这位父亲的同情其实是非常「搞不清楚状况」的表现。「如果两个人没有死,正常地结为夫妇,一定比殉情这种唯美主义的作为要幸福」这种观点只不过是活着的人的解读而已。

五郎和八重子已经充分实现了他们各自的目的,留在这个世界上的人的所有解读和补充都是多余的。

\subsection{小王子}
文学中描写的自杀者实在不算少。对我们来说他人的自杀,究竟是真实还是虚构呢?实际上这样问的时候就已经昭示了自杀一定会被故事化的宿命。

也就是说「我们自身是无法体验这个自杀的」。

圣埃克苏佩里的小王子无法面对人类社会的虚无。文明、财富、欲望这些东西,对他来说都是噪音,他始终怀念着故乡那颗星球上与自己斗嘴的玫瑰花,后来故意让蛇咬了自己,灵魂得以回到故乡。

这位小王子的「故乡之星」自然成全了他现实层面上的自杀,但从另一层意义上,这是对另一个世界的期许,必须和单纯厌世所导致的自杀区分开来。

这不是「为了逃离生」而选择的死,是为了「奔向另一种生」而选择的死。

这本书的作者圣埃克苏佩里不仅写了《夜间飞行》《人间的土地》等飞行相关的作品,自己本身也是飞行员。他某次飞行途中突然失踪,也不见尸体,或许他是回到了他自己的「故乡之星」吧。

\subsection{狂人皮埃罗}

让•保罗•贝尔蒙多的脸上涂着油漆,身上缠上炸药的引线,看着大海,背出兰波的那句「找到了!什么?永恒。」,然后点燃引线,自爆而死。

戈达尔的电影《狂人皮埃罗》的最后场景里所描绘的这场自杀,因为其手法的惊世骇俗,自然应该给人留下深刻的印象。

名为费迪南的男主人公,也就是让•保罗•贝尔蒙多是个漫画爱好者,所有他爱的东西他都无法得到。虽说他始终带着淡淡的笑容,但毋宁说他是个具备自杀条件的人,对他来说,除了死别无他法。

\subsection{\fbox{森恒夫}}

我为森恒夫的死哀悼。这位出演了Nechaev\footnote{Sergey Gennadiyevich Nechayev(1847 - 1882),俄国革命家。陀思妥耶夫斯基的反否权主义小说《群魔》中的角色彼得·韦尔霍文斯基(Pyotr Verkhovensky)正是以涅恰耶夫为原型。}革命剧的美少年,被现代性的恶灵缠身,一月一日在拘留所上吊自杀了,他的政治主张用这种虚构的方式实现了。

如果他活下来,经历了法庭的争论,那么,浅间山庄的枪战\footnote{1972 年联合赤军与日本警方在浅间山庄进行的枪战。},树林里的人民审判,他所践行的一系列主张,这所有一切都会在日常性中被风化。

最终,官方的狡猾手段会把这变成简单的私刑行为和警民冲突。

但是,森选择一月一日这个日子,是想让整个事件在自己的死亡中再现,同时,对「联合赤军」所为之努力的革命做最后的总结。
这场自杀是和电视媒体混杂在一起传达给我的。

电视画面里飘着雪花,如梦似幻的女优藤纯子,这位「绯牡丹博徒」从桥上回眸。桔子滚下来。安静的离别,没有再会的约定。

这时,画面插进了临时新闻「联合赤军·森恒夫·拘留所上吊自杀」。他的死或许可以视为政治性的死,但实际上却不是。

他的自杀是在漫长又灰暗的革命之路上的穷途末路,因为他们这种人只信仰着唯一的现实。

我一边比较森恒夫和藤村操的,一边思考。藤村在松树上刻下了遗书,而森难道不是用炮弹用血用 Alberto Bayo\footnote{Alberto Bayo y Giroud(1892 – 1967),西班牙内战中失败的左派的军事领导人。}的著作,用数条人命写下了自己的遗书吗?如果用一般的社会道德,又该如何评判呢?


\begin{verse}
自然为人生规定了唯一的入口,

却告诉我们无数的出口。

蒙泰涅
\end{verse}

\subsection{后记}

我在思考自己的自杀的时候,感受到了将自己与他人割裂开的困难。「自己」无论如何也无法以独立的姿态存在,所谓自己也不过是他人的一部分而已。

杀死自己,或多或少都是在伤害他人,甚至是谋杀他人。
在这个时代,已经无法不将他人卷入而孤独地自杀了。

我边想着这些,一边不甘地望着铅笔。

\begin{verse}
夏日远天的船帆

那是我心中的帆\footnote{原文:「炎天の遠き帆やわがこころの帆」。}

山口誓子
\end{verse}
\end{document}