%-*- coding: UTF-8 -*-

\documentclass[UTF8]{ctexart}
\usepackage{graphicx}
\usepackage{float}
\usepackage{CJKpunct}
\usepackage{amsmath}
\usepackage{geometry}
\geometry{a5paper,centering,scale=0.8}
\usepackage[format=hang,font=small,textfont=it]{caption}
\usepackage[nottoc]{tocbibind}
\setromanfont{SourceHanSerifSC-Medium} %设置中文字体
\punctstyle{quanjiao} %使用全角标点	

\title{水中的男孩}
\author{孙建成}
\date{}
\begin{document}
\maketitle

\newpage

植物园的湖边,一棵虬枝伏地的雪松下,李渊弥漫在湿润的空气里,依附在树皮皲裂的树枝上,在微风的吹动下轻轻地摇晃。松针散发着清冽的略带苦味的清香,使他感到安全和放松。这是他的栖身之地。李渊的眼前是一片绿得耀眼的略有起伏的草地。草地的边缘便是那片平静的湖面。

小小的湖,也许只能算是一个池塘。环湖长着几簇高高的芦苇,在芦苇丛的间隔里,有一片水杉林、一片冬青、一片终日不见天日的竹林、几十棵枝干疏落遒劲的黑松。这些树的倒影从四面八方黑压压地倾倒在窄小的湖面上,湖水越发显得幽静黑暗、深不可测。这里已是植物园的边缘,几乎很少有游人涉足。

对李渊来说,植物园外面是陌生的世界。马路上,纷杂的人群和永不停息的车辆,使他寸步难行。他不知道回家的路在哪里。

自从那个春天,
一辆豪华大巴开了足足一个多小时,将他和他的同学拉到这里以后,他再也没有回家。

那一年的春天,湖面长满了绿色的浮藻,密密的浮萍遮盖了整个湖面,远远望去,就如一片茂盛的绿得魅人的草地。春游的小学二年级学生,在老师的一声号令下,开始自由活动。八岁的李渊透过密密的黑松林看到了这片美丽的绿草地,他离开自己的伙伴,从倾斜的坡面上狂奔而下,欣喜地张开两臂。八岁的李渊要在草地上打滚,翻斤斗,仰面朝天舒舒服服向上一躺。

草地一样的浮萍像玻璃似的被李渊踩碎,又悄然无声地把李渊包围。在下坡强大惯性的带动下,李渊无法收住自己的脚步。他顺着油一样细滑的河泥向湖泊深处沉入,水面的浮萍在他双手的扑动下,
一次次闪开又合拢,阳光透过被染绿的湖水在李渊的头顶上奇异地闪烁。八岁的李渊用力挣扎四肢,要从水的包围中突围,胸前的红领巾在水里火焰似的向上飘舞。可是他的两条腿怎么也踩不到湖底,身子老是打横,怎么也立不起来。

李渊透不过气来,每吸一口气,都有大量的水涌进他的胸腔,他再也无力挣扎了。李渊深深地吐出最后一口气,他看见自己正顺着喉咙里发出的气泡,向上腾升\ldots\ldots{}

李渊又一次看到了草地般的浮萍,湖面一如先前恢复了平静。透过浮萍的缝隙,李渊看见暗绿色的湖底躺着一个男孩。男孩两手紧紧捏着一把湖泥睡着了,男孩胸前的红领巾在水中微微晃动,像一棵红色的水草,与世无争地生长在自己的世界里。

浮出水面的李渊感到了透心彻骨的害怕,他向四周大声叫喊。没有声音,风在树林间掠过,树叶瑟瑟作响。李渊向有人群的地方狂奔,感觉像在空气中浮动、飘移。

李渊推撞着遇到的第一个人,拉他的手敲他的头,贴近他的耳朵拼命地叫嚷。那个男人毫无知觉,只顾在花丛中全神贯注地捕捉一只上下翻飞的黑蝴蝶。植物园里的游人三三两两四处可见,可是没有一个人听到李渊的呼唤,他们脸上的笑容透露出内心的喜悦和恬静。李渊的努力一次次失败,他只得去找他的同伴。

李渊的同学正围坐在曲尺形的长廊里,吃着各自从家里带来的午饭。李渊带来的面包、鸡蛋和橘子水,还在书包里沉在湖底躺在男孩的边上。年轻的班主任坐在孩子们的身边,眼睛望着莺飞草绿的景色呆呆地出神。她刚刚从大学毕业,做孩子王让她感到既幸福又困惑,他们顽皮的举动有时使她无所适从。

不好了,不好了\ldots\ldots 李渊来到老师身边,跺着双脚一遍遍对着她的耳边喊。

女教师突然抬起头来,目光吃惊地朝四周环顾,似乎在寻找什么,仅仅只是恍惚了一下,女教师又回到了原先的状态,再次沉湎于无尽的遐思冥想。不管李渊如何叫嚷推搡,都不能使她惊醒。李渊无能为力了,他知道那个男孩再也没有救了。

李渊一步一回头,离开那些熟悉的小伙伴,回到了湖边,静静地守着湖底的那个八岁男孩。李渊幻想男孩突然会在什么时候醒来,那时候,李渊就可以和男孩一起回家了。

天渐渐黑了,植物园里没有了游人,工作人员除了值夜班的也都走了。偌大的植物园里空空荡荡,像一片无人涉足的原始森林。男孩还是那么一动不动地躺在湖底,湖水,还有碧绿的浮萍,默默地覆盖着他,像包裹着一个天大的秘密。

李渊等得有点困倦了,或者是有点麻木了,于是蜷在湖畔的那棵雪松下,打了一个盹。

突然,植物园里热闹起来。狗吠,灯光,还有嘈杂的人声,在植物园内四处游动。栖息在树上的鸟,成群地飞起,在空中盘旋聒噪,各种小动物从它们的窝里跑出来,惶恐地乱窜。黑暗中走着一行人,手电筒的光束交织晃动。他们在那些很少有人涉足的地方搜寻。植物园园长陪着李渊的父母、班主任和校长,走在人群的最前列。他们身后还有前来帮忙寻找的热心人。

「李渊------小渊------」

母亲一声声叫着李渊的名字,叫得小心翼翼诚惶诚恐,生怕吓着了李渊似的,声音中的绝望和悲凉使所有的生命澹然动容。

「渊------渊------渊------」声音在夜空下、树丛间、空气中回荡。

「你是什么时候发现孩子不见的?
」校长口气严厉地问女教师。他需要有一个人和他一起承担这个沉重的责任。

「离开植物园的时候,清点了人数,就少了他一个。」女教师的声音像刚刚哭过。

年轻的女教师的确哭过了,别的教师一辈子也碰不上的事,她执教才半年就遇上了。今天晚上,她的男友还在老地方等她,说好不见不散的。想到这里,她的惶恐中又添了几分委屈。

校长又问李渊的父亲:「他肯定没有自己回家吗?上次有个同学和大部队失散以后,自己跑回了家,弄得一场虚惊。」

「没有,我刚才还和他奶奶通了电话。」李渊的父亲说,「这件事你们学校要负完全的责任。」李渊的父亲在政府机关里当科长,知道应该抓住责任人不放。

他们的谈话被一条半人多高的狼狗的出现打断了。巡夜的人牵着狗过来,对李渊的母亲说:「有没有那孩子用过的东西?
」母亲把手里的一件外套递了过来,她害怕李渊夜里在外面冷,出门时特意带上的。狼狗在李渊的衣服上嗅了片刻,直嗅得李渊的母亲一阵阵惊颤。离开了衣服,狼狗开始在四周搜寻相同的气味。

寻寻觅觅了几分钟,那狗突然兴奋起来,朝湖边的方向一溜小跑。

栖息在雪松下的李渊被惊醒了。他看到一行人亮着灯火,喧哗着向湖边蜿蜒而来。他还没有弄明白发生了什么事情,
一条深黄色的大狼狗已经蹿到跟前。那狗似乎看到了李渊的身子,跳跃着不停地向他扑击。李渊左躲右闪,还是难以逃脱大狼狗的追咬,他一步步后退,最后不得不漂移到湖而上。那狗还是不依不饶,四爪刨地伏下身子,冲着湖面不停地吠叫。

人们在湖边停了下来。值班人员按住了躁动不安的狼狗。湖面浮萍覆盖,悄然无声,平静如凝固的油画。大家的脸色顿时变得铁青,在暗淡的星光下,发出绿颜色的光。年轻的母亲双手捂住了脸。

「快,去找一张捞鱼的网。」植物园园长吩咐身边的人。

渔网抬来了,几个人向湖中撒出渔网。巨大的渔网罩住湖面缓慢地沉入水里,停留在湖面上的李渊穿过渔网的眼孔悬到空中,他看见渔网的一角覆在水下男孩的身上。随着渔网一点点收拢,男孩被拉动起来,在水中轻盈地上下浮动。隔着湖面,李渊在人群里看到了母亲、父亲,还有女教师。

母亲穿着黑色的风衣,长发两边的发夹紧紧地贴在耳旁,她下班后还没有换下衣服,就出来找她的儿子。父亲扶着母亲的身子,唯恐她支撑不住倒下去。他板着脸孔,表情一如往常的严肃。

妈------李渊喜出望外地喊了一声。

母亲放下了捂在脸上的双手,仿佛凝神倾听。

李渊不顾狼狗的扑击,欣喜地向他们奔去。他像往常一样,用手钩住母亲的脖子,两只脚围着母亲的腰,脸紧紧地贴着她的脸。母亲的身子在颤抖,她没有如往常那样亲吻李渊。她是那样的焦虑万分,她的心思全在那张在湖底一寸寸搜索的渔网上。李渊知道她在寻找那个躺在水底的男孩。

母亲心里在说:不会的,不会的,我的儿子不会在水里的。

只有找到了男孩,母亲才会理睬李渊。她是那样地爱着那个叫李渊的八岁男孩。每天临睡前,她都要靠在男孩的床边,给他讲一个故事,这些故事都是她自己编的,她入睡以前一定要把明天的故事想好了,她的故事常常没有讲完,男孩就睡着了,每天清晨起来,她轻轻地来到床边,抚摸着男孩的脸,把他唤醒,然后为他穿衣\ldots\ldots 她一直要送李渊到学校门口,看着他走进教室,才会离去。

今天她要为李渊讲一个什么故事?

渔网全部拉到岸上,男孩从水底捞了上来。那条崭新的红领巾血一样地贴在男孩的胸前,他们今天早上刚刚举行了入队仪式。

母亲还没看清男孩的面容,就昏倒在草地上。母亲倒在众人的脚下,像风衣盖着的土丘。李渊一遍遍地摇着母亲白纸一样的脸。父亲蹲在她的旁边,用指甲揿压她的人中。好不容易,母亲哇地哭出声来。

人们抬着死去的男孩,搀着悲恸欲绝的母亲,朝灯光灿烂的城市走去。年轻的父亲和母亲脑海里一片空白。校长对自己说,以后说什么也不许低年级学生外出活动了。女教师觉得自已完了,她的事业刚刚开始就要结束了。只有那个男孩永远地沉默了。

李渊跟随在人群的后面飘浮移动。李渊要回家。

可是,那条狼狗围着李渊不停地狂叫。黑暗中,狗的眼睛像两颗红色的珠子。李渊不明白为什么别人都无视他的存在,而这条狼狗却紧紧盯住了他。狼狗一次次向李渊扑来,有几次,狗的前爪差一点就抓到了李渊。李渊只得一步步后退,退回到了布满浮萍的湖面上。狼狗对着湖水一个劲地吠叫。

李渊眼睁睁地看着母亲他们一点点远去,远去\ldots\ldots{}

几年过去了。

在李渊的记忆里,自从那个黑夜那条凶狠的狗把他和母亲隔绝以后,每到春天,也就是学生们春游的季节里,母亲就会来到这个小小的湖塘边。头一年,父亲还陪着她,后来就是她独自一个人了。

母亲每年都穿着当年的那件黑色风衣,梳着同样的发式,
一步步向湖边走来。她的身前身后总有那么几只蝴蝶翻飞追逐,是她身上那一股永远不变的淡淡的清香吸引了它们。李渊飞奔着迎向她,树叶和青草微微地摇动。

母亲在湖边伫立,望着儿子沉没的地方。她的脑海里浮现出儿子的身影。一个和她齐腰高的男孩,白白的胖胖的,她已经抱不动他了,八岁的孩子不能再抱了,可是男孩还是缠着她,顺看她的手一点点爬上来,双手搂住了她的脖子,两腿钩在她的腰上。这时候,他就可以亲她了,红嘟嘟的小嘴在她的脸上一下一下用力地亲着,发出叭叭的响声,然后贴着她的耳朵说:我又学了一句英语,你听着,哎哟我怎么想不起来了\ldots\ldots{}

渊啊,母亲轻轻地问,你好吗?好吗?

李渊在她的耳边一遍遍地说:妈,我想回家,你带我回家吧。

母亲环顾四周。湖边静静的,看不到一个人影。自从溺死了一个二年级的男孩,湖面上再也看不到浮萍了,工作人员每年春天都要把它们打捞干净。鱼儿在水中悠闲地游着。

母亲卸下肩上鼓鼓囊囊的背包,从包里取出一件件东西:玩具手枪、游戏机、铅笔、簿子,还有李渊最喜欢吃的红肠面包和巧克力。母亲在草地上铺上塑料布,然后把这些东西轻轻地放在上面,之后在一边坐下来。

母亲说:渊啊,你先吃了红肠面包和巧克力,慢慢地吃,不够的话我再给你去买。吃完了再玩,你看这儿空地多大,你就尽兴地玩吧,我看着呢。我们再也不会丢失了。

过了一会儿,母亲又说:玩够了,咱们再做作业,有不懂的地方,你问我。你要记住,别的你都可以放一放,语文和英文不能放松,语文是你的母语,英文是将来吃饭的工具,你听懂了吗?你外公说了,迟早要回来带你去美国的,在外公那儿你不会英文连问个路也找不到人。

李渊说:妈,我不要吃,不要玩,也不要做功课,我要回家。这儿就我一个人,家里好。

母亲没有听见李渊的话。母亲看到的是儿子大口大口吃面包巧克力的景象,她那苍白的脸上露出陶醉的神态。她就这么呆呆地坐着,默默地注视着想象中的场景,什么也没有听到,什么也没有看到。李渊失望地哭了起来,平静的湖面上漾起了阵阵涟漪,树叶哗哗地响。母亲抬起头来,惊恐地看着四周。李渊停止了哭泣,他害怕吓着了母亲。

母亲开始收拾地上的东西,把那些东西一样一样地放进湖里,看着湖水把它们淹没。水中放养的鱼儿几乎是一下子涌过来的,聚成一团啄食着红肠面包。还有的鱼从四面八方赶来,清一色的青鱼和鲫鱼,黑色的背像犁一样划开水面,尾鳍甩动着,拍打得水面哗啦哗啦乱响。李渊从来没见过湖里有这么多的鱼,平常它们不知躲在了什么地方,即使在水里游过,也是悠然无声像散步街头的一个绅士。

母亲被眼前的景象惊呆了,对争食的鱼儿说:渊啊,你是饿的吧?你慢点吃,慢点,别噎着了呀。

李渊在湖里驱赶那些鱼儿。李渊说,你们就不能慢一点吗?等我妈走了再吃也不迟,不管怎么说,我们也做了这些日子的邻居,给我一点面子好不好?鱼儿并没有听李渊的,只顾自己抢食。李渊不由得有点气愤:到底是畜牲,一点修养也没有。他气得在湖里乱搅,水面浪花四溅。

面包、巧克力瓜分完了,鱼儿们像来时那样迅捷地退去,水面平静下来。

李渊回到草地上。草地上空无一人。母亲走了,就在李渊驱赶鱼儿的时候母亲离去了。十几岁的李渊趴在草地上,闻着母亲留下的气息,伤心地哭了。太阳被一片厚厚的云层遮蔽,天色顿时暗了下来,湖边阴风凄凄。

下一年的同一个日子,母亲又来到了湖边。李渊早早地就在植物园的门口等着。他紧紧地挽着母亲的左臂。母亲比上一年消瘦了,眼神中有了一种茫然和呆滞。李渊惊讶地发现,母亲的脑海里居然飘浮着一层薄纱一般的空白。就在这空白的底子上,他看到那个背着书包的李渊浮现出来,那是个只有一张桌子高矮的小男孩。

小李渊勾下母亲的头,在她的耳朵旁悄悄地说:告诉你一个秘密,你可不许对别人说,谁说谁就是乌龟。

母亲兴奋得涨红了脸说:好,咱们拉钩钩。

小李渊说:爸爸昨天和那个阿姨打架了,两个人扭在一起,不停地在对方的脸上咬着,两个人的手还在对方的身上用力地抓\ldots\ldots 我吓得从桌子下面钻了出来,朝门外逃跑。阿姨看到以后,猛地推开爸爸,举起巴掌在爸爸的脸上拍了一下\ldots\ldots{}

母亲的脸一点点变得惨白:哪个阿姨,你说的是哪个女人?小李渊说:就是那个经常到我家来的阿姨,你们都叫她小茱小茱的,还不明白啊,就是他们办公室里那个打字的女人\ldots\ldots{}

李渊感到迷惑,他从来也没有说过这样的话,也不认识那个叫小茱的女人。母亲是怎么啦?

母亲的嘴里发出丝丝冷笑:我就知道他们有这种事,想骗我不是那么容易的。我人住在医院里,但外面的事都知道。好几次,他来探望我,我在他的眼珠子里都看到小茱的影子。还说什么是在加班。骗谁呀,连八岁的孩子也骗不了。有一次我回家拿换洗的衣服,就闻到了家里有一股奇怪的香味。我们家没有人用这种香水。一定是那个女人来过了\ldots\ldots{}

李渊被母亲的这番话说得目瞪口呆。他不知道家里发生了什么事,但他担心母亲这样的身体状况。

母亲口中念念有词地走着,来到了湖边,在一块裸露的太湖石上坐下。看着平静的湖面,母亲的心情平静下来。李渊看见她浑身散发着忧伤的柔情,像蓝天下静止的白云。白云的阴影投在地面上,与周围灿烂的日光形成鲜明的对照。母亲的包就放在她的脚下。这天母亲没有从包里拿出什么东西,她只是默默地看着幽静的湖面,又高又密的杉树林投下的影子把她整个地笼罩起来。

李渊想,她一定有什么话要对他说。他静静地等待着。湖对岸的树梢上飞起了一只白鹭,又飞起一只白鹭,两只白鹭一前一后向远处的天空飞去,渐渐地不见了影子。

母亲的脑海里还是李渊的身影。星期天,趁母亲在卫生间洗衣服的时候,八岁的李渊骑在了阳台的栏杆上,一只脚在里面,一只脚在外面。二楼的阳台外面长了一棵枫杨树,树梢的杈枝上有一个鸟窝,枯草烂叶间,一窝刚出生没几天的小鸟在探头探脑。从阳台上看过去,只要伸出手去就可以碰到鸟窝了,当他骑在栏杆上伸出手去,才发现手指距离鸟窝还有一段距离。他尝试着探长身子伸展手臂,
一寸寸地接近鸟窝。这时,身后传来母亲的一声尖叫,随即从四面八方响起了惊叫声。他马上被搂在了母亲的怀抱里,母亲的手掌狠命地拍打着他的屁股,直到她再也没有力气打下去为止。与此同时,
枫杨树上的那只鸟窝被父亲用长竹竿捅了下去,
一窝还不会飞的小鸟摔在地上嗷嗷乱叫。

迟早会有这么一天的,母亲说,这孩子太顽皮了。渊啊,你怎么就不会汲取一点教训,打也打不怕的,你也要为做母亲的想想。妈只有你一个儿子。

李渊大声说:我知道了,妈,我现在知道了。

现在知道也没有用了,母亲自言自语,春游那天送你到学校时,我还对你说,不要到河浜旁边去玩,你就是不听。

我没有到河浜旁边去,李渊着急地申辩,我以为那是一片草地,一片绿得刺眼的草地。这样的草地谁见了都会喜欢。

你知道多少人被你连累了,母亲没有理会李渊,目光呆呆地说,你这一走,把多少人毁了。你们的校长被撤了职,班主任记了大过,她对我说她还只有
23 岁,刚刚从师范学校毕业。还有我,我什么也没有了。

李渊觉得母亲的口气里有一种仇恨的东西,她几乎是咬牙切齿地在说着这一切。她的眼前是灰褐色的,一切景物变了颜色,冷冰冰的,
像荒漠的月球表面。荒漠的月球景象被制作成图片,
就竖在植物园的科技馆里。李渊被母亲眼睛前的仇恨惊呆了。

妈,这不是我的错,我不是有意要气你的呀。李渊一遍遍地说。

孩子,妈知道不是你的错,谁也没有错。母亲冷冷地笑了,正因为谁也没有错,妈才觉得冤得难受,妈只有向自己讨债。

妈,你不要难受。李渊说,你看看我,除了没有实在的人形,我不是活得很好吗?说着,他飘浮在水面上,
一连翻了十几个斤斗,湖面上一圈圈的涟漪轻柔地向远处荡去,交织出美丽的花纹。

李渊停下来,看到母亲并没有在看他。她又陷入了沉思状态。

李渊回到母亲身边,听见母亲的嘴里发出喃喃的声音。仔细地听,她一遍遍地说着那个名叫小茱的阿姨。她在说,连小渊都看到了,你们还想抵赖,小孩子是不会说谎的,你们要好就好吧,我不在乎,你们只要还给我儿子,把小渊还给我。

李渊说:这是没有的事,我从来也没有见过那个小茱阿姨,我怎么会说这样的话呢?

母亲惊悚地抖动了一下,抬眼望了望四周。她的手放进了湖水里,轻轻地抚摸着,湖水滑腻腻的,像漂了一层油。母亲的眼光停在了湖边的雪松下,那里铺了一地黄黄的松针,干燥洁净,寸草不生。李渊说:这就是我每天睡觉的地方。母亲似乎很满意湖边的环境,或者说她听到了他的声音,脸上泛起了微笑,神色立即变得生动起来,点点头,然后站了起来。

渊,我走了,母亲说,你好好地歇着,明年我还会来看你的。

李渊一步不离地跟着母亲:妈,我们一起走,我要回家,我一个人在这儿太冷清了。

母亲走得很慢,脚步恍恍惚惚,身体左右摇晃。游园的人越来越多,母亲在人群里挤过来挤过去。好几次,李渊被游人冲撞得摇晃不定,不时离开母亲的身边。他只得蜷缩起来,一点点隐入在母亲的体内,只有母亲知道回家的路,只有这样,他才能回家。这一次他再也不能错过机会了。

母亲的脑海里升起了一片浓雾,比来时的空白更加可怕。她觉得眼前的一切都在晃动,头脑好像被什么紧紧地箍着,李渊的影子遮住了她的眼睛,疼痛从遥远的地方一步步逼近,他再也看不清眼前的路了。母亲头痛欲裂,她只得在花坛的石沿上坐下来。

这该死的头痛!这几年里,头痛总是纠缠着她。她捧着自己的脑袋想,她再也回不了家了,她要死在植物园里了。

李渊重重地吓了一跳,他不明白发生了什么事。他离开母亲的身体,站到了她的身边。母亲长长地舒了一口气,揉了揉太阳穴,好像轻松了很多。她站起身来,活动了一下麻木的腰肢,然后朝植物园的大门走去。

李渊也松了一口气,重新回到母亲的身上。

母亲又一次蹲到地上,两手捂头,痛得哼哼起来。渐渐地,她的身边围起了观望的人墙。

母亲的痛苦让李渊焦急万分。他离开母亲,心急如焚地对那些围观的人说:你们不要光站着看热闹,你们快想想办法救救她呀,你们快去叫救护车呀。没有人听从他的呼救,围观的人不解地看着这个蹲在地上的女人,不知道发生了什么事。正在这时,母亲站了起来,头痛没有了,神情轻松了许多,对周围关注她的人露出歉意的笑容。

李渊一下子明白了,由于他的缘故,母亲才痛不欲生。他不能随她而去,他不能因为自己想回家,而不顾母亲的生命安危。

母亲再一次迈开脚步,缓缓地向植物园大门外走去。这一次她没有头痛,尽管步履恍惚,她还是一步步地走出大门,走向车站,最终消失在来来往往的人流里。

李渊哀伤地站在原地,目送着母亲的身影。他想,她一定还会再来的。想到这里,他心里宽慰了许多。

新的一年,春天来临了。在那个日子到来的时候,李渊早早地来到了植物园的大门口,悬浮在高大的梧桐树上,看着每一个进门的人。耀眼的阳光照得他头昏脑涨。随着时间的推移,充盈着他内心的期待和喜悦一点点消失,失望和担忧把他笼罩。

直到暮色降临,植物园大门关闭,母亲也没有出现在李渊的视野里。

随后的日子,李渊等了一天又一天,炎热的夏天过去了,凉爽的秋天过去了,北风呼啸的冬天也过去了。一年过去了。母亲再也没有出现。

又一年过去了,母亲还是没有来。

李渊充满忧伤地想着母亲。她一遍遍地安慰自己:母亲一定是被什么事情耽误了,等她腾出时间,她会来看他的。

夏天,一阵大雨过后,积水还没有退净。李渊忧伤地眺望着空荡荡的公园。一个女人蹚水走进植物园,在几乎看不到游人的公园内,她显得十分地醒目。女人的黑发中已经有了白丝,眼神定定地望着前方,有些呆滞或者说看上去若有所思。她的衣着样式陈旧,好像是几年前的旧物。她目不旁视地向前走着,似乎急于要去找一件遗失了的东西。

女人身体的姿势和走路的样子使李渊感到似曾相识。他迎上前去。

走进以后,李渊大大地吃了一惊。从那件老式的黑色风衣上,他终于确认她就是母亲,只是她变化得让李渊不敢相认。离最后一次见到她,已经两年过去了。意想不到的相见,李渊的内心悲喜交加。他相信如果不是家里发生了什么重大的变故,母亲是不会拖到今天才来看他的。

他在母亲的身前身后来回地奔跑,想把她看个真切。

母亲蹚着积水,来到了湖边。湖水四溢,她只能坐到湖边高高垒起的太湖石上,远远地眺望儿子沉睡的地方。四周是那样的安静,看不到一个人影,一个人面对洪荒似的世界,让母亲心绪平静。

他们这样对待我是不公平的,母亲自言自语,他们应该让我到这里来,和李渊在一起,我就什么病都没有了。母亲看着李渊轻烟飘渺的身影,向他诉说内心可怕的经历。

他们说,嗨,我们送你去一个很好的地方疗养。就因为我老是独自一人对你说话,就像我们现在这样,他们认为这妨碍了他们的工作和生活,连你的父亲也这么说\ldots\ldots{}

他们用汽车送我去了一个很远很远的地方,一扇沉重的铁门把我与外面的世界隔绝了。

儿子,你现在明白了吧。那个地方原来是一所精神病院。那些穿白衣服的人把我绑在床架上,强迫我吃下大把大把的白药片。那些药片使我全身骨头像散了架似的酸痛,人整天昏昏沉沉想睡觉,一觉醒来,刚刚感觉清爽一点了,又要让我吃药了,重复循环,日复一日,无休无止。

我实在不想吃那种让人昏昏沉沉的药片,他们便强迫我吃下去,不吃就给我打针,我要是反抗,他们就把我捆起来,用电针把我击昏。

我不知道自己做错了什么,我只是愿意和你说说话罢了,他们居然要这样对付我。这样的日子一天天过去,最后连你的父亲也觉得厌烦了,她受不了没有女人和孩子的生活,终于离我而去。

前些日子,我看见你父亲了。他和她的妻子小茱还有他们的女儿走在商场里,他们一路走一路笑着,看上去她已经把过去的一切忘掉了。那个小女孩已经会走路了,看人的时候喜欢侧着脑袋,一副做作的样子,那双眼睛倒是很讨人喜欢的。

你父亲看到了我,我们站着说了几句话。他说:你现在身体怎么样?我说:你看呀,好不好你一看就知道了,不好的话医生也不会让我出院的。他就有点尴尬。他说:你今后打算怎么办?我看看他的家人,他们似乎都知道是怎么一回事。我觉得对他已经没有什么可说的了,就说,这是我自己的事,你就不用操心了。

其实,我说这些话丝毫没有怪她的意思,他有权利做出自己的选择,看到他现在幸福的样子我也高兴,怨只怨我自己没有这个福分。孩子,千万不要怪你父亲,男人看问题比较现实,他们有自己的生活目标,不像我们女人太情绪化,常常有一些不切实际的想法\ldots\ldots{}

李渊怎么也没有想到,母亲这些年来居然一直住在精神病院内。由于这个原因,她无法在那个日子来看望儿子。对母亲的失约,他曾经有过许多猜想:她很忙,工作和家务,没有时间年复一年来这儿凭吊;她又有了新的孩子,他希望那是个女孩,他就有一个妹妹了,小妹妹一天天长大,母亲也会把他淡忘;他最不愿意想的是家中发生了变故,或者母亲生了重病,或者她去了很远很远的地方\ldots\ldots{}

他从来没有想过,她会被囚禁在精神病院内,以致他和她两地相思一样情愁。现在好了,母亲出院了,又可以每年来看他了。

母亲望着涟漪荡漾的湖面,淡淡地笑了。孩子,我现在明白了,人死不能复生,想念只能放在心里,用一个小小的空间把它贮存起来,不能放在嘴上,更不能整天打开去看,如果那样的话,人就要生病了。我明白了,我的病也好了。现在我要带着心中的你,到一个很远很远的地方去,你外公已经在那里帮我办好了移民定居手续,这一去也许就再也不回来了。今天我是来向你告别的。你听明白了吗?

怎么会这样的呢?李渊一时无法接受母亲的这一决定。想到再也见不到母亲了,李渊感到无边的恐惧。他在她的耳边大声地喊:一切都没有过去,没有过去,没有过去\ldots\ldots{}

母亲和李渊相隔在两个世界,无论李渊怎么喊,她都无法听到。

李渊听到母亲在心里说:儿子我要去做我的那份事情了,我为你浪费了这几年,我没有丝毫的怨言,我有过你,你留在了我的生活里。我的意思你听懂了吗?我今生今世不会再有第二个儿子了,你就是我唯一的儿子。我还要告诉你,你留在家里的那些东西,我都收拾起来了,准备随身带着,就像你在我的身边一样。

母亲从口袋里拿出一包巧克力,撕去包装纸,把它放入湖里。还有橘子、香蕉和苹果。湖水把这些东西吞没,就像李渊收下了它们。

母亲看着湖水,就像看着她久违的儿子。他的眼神又一次变得呆滞起来,好像被什么牵引着,一直拉到很远的某一个点上。在这一刻,儿子慢慢地从水中升起,那个活泼的男孩,又蹦又跳地向她走来。她的身子摇晃了一下,差点从石头上掉进水里。

她浑身一个激灵,惊醒了。她想,她应该走了,再不走,她也要留在这儿了。

母亲留恋地最后望了一眼水面,站起身来准备回家。她的脑海里出现了那个遥远国度的美丽风光。李渊知道,他再也看不到母亲了。

李渊拦着母亲,寸步不离地挡在她的路上。母亲每向前走一步,就像在他的身上踩上一脚。李渊想,我还不如让你踩死的好。

在植物园门口,母亲的步子因为李渊的存在而躲躲闪闪,她的行动又一次变得恍惚起来,身子摇摇晃晃,连迎面开来的汽车也分不清了。她对自己说,你怎么了,怎么又出现了这种症状。他从口袋里取出几粒药片吞进嘴里。你可要挺住啊,再有几天就出国了,不要再出什么差错,再进精神病院,我这辈子就算完了。

李渊感到害怕,他怕由于他的阻拦,母亲再一次去那个牢狱似的精神病院。他不忍心毁了母亲的下半生。可是,他还是心存不甘,他知道只要他一让步,母亲就永远地离开了他。

母亲用手敲打着自己的头,恨恨地说:李渊呀,你就不要再缠着我了,我生了你养了你,虽然没能把你养大,可那不是我的责任呀,那是你自己的命不好,我已经想开了,我不能为了一个死去的孩子毁了自己的一生。

母亲敲打自己脑袋的样子让李渊感到内疚,他离开了母亲的身体,站在一旁,目送着她一步步远去。

这个晚上,李渊哪儿也没有去。她一直呆在母亲坐过的那块石头上,感受着久违了的母亲的气息。他想象着母亲在精神病院里的生活,石头变得湿漉漉的,分不清是泪水还是露水。他知道他再也回不去了,他真正成了一个孤儿。

雨后的湖面静静的。李渊惊讶地感到,自己的身体在一点点变大,长成了一个年轻的小伙子\ldots\ldots{}

\end{document}