%-*- coding: UTF-8 -*-

\documentclass[UTF8]{ctexart}
\usepackage{graphicx}
\usepackage{float}
\usepackage{CJKpunct}
\usepackage{amsmath}
\usepackage{geometry}
\geometry{a5paper,centering,scale=0.8}
\usepackage[format=hang,font=small,textfont=it]{caption}
\usepackage[nottoc]{tocbibind}
\setromanfont{SourceHanSerifSC-Medium} %设置中文字体
\punctstyle{quanjiao} %使用全角标点	

\title{茉莉香片}
\author{张爱玲}
\date{}
\begin{document}
\maketitle

\newpage

我给您沏的这一壶茉莉香片,也许是太苦了一点。我将要说给您听的一段香港传奇,恐怕也是一样的苦——香港是一个华美的但是悲哀的城。

您先倒上一杯茶——当心烫!您尖着嘴轻轻吹着它。在茶烟缭绕中,您可以看见香港的公共汽车顺着柏油出道徐徐地驰下山来。开车的身后站了一个人,抱着一大捆杜鹃花。人倚在窗口,那枝枝丫丫的杜鹃花便伸到后面的一个玻璃窗外,红成一片。后面那一个座位上坐着聂传庆,一个二十上下的男孩子。说他是二十岁,眉梢嘴角却又有点老态。同时他那窄窄的肩膀和细长的脖子又似乎是十六七岁发育未完全的样子。他穿了一件蓝绸子夹袍,捧着一叠书,侧着身子坐着,头抵在玻璃窗上,蒙古型的鹅蛋脸,淡眉毛,吊梢眼,衬着后面粉霞缎一般的花光,很有几分女性美。惟有他的鼻子却是过分地高了一点,与那纤柔的脸庞犯了冲。他嘴里衔着一张桃红色的车票,人仿佛是盹着了。 

车子突然停住了。他睁开眼一看,上来了一个同学,言教授的女儿言丹朱。他皱了一皱眉毛。他顶恨在公共汽车上碰见熟人,因为车子轰隆轰隆开着,他实在没法听见他们说话。他的耳朵有点聋,是给他父亲打的。 

言丹朱大约是刚洗了头发,还没干,正中挑了一条路子,电烫的发梢不很鬈了,直直地披了下来,像美国漫画里的红印度小孩。滚圆的脸,晒成了赤金色。眉眼浓秀,个子不高,可是很丰满。她一上车就向他笑着点了个头,向这边走了过来,在他身旁坐下,问道:“回家去么?”传庆凑到她跟前,方才听清楚了,答道:“嗳。” 

卖票的过来要钱,传庆把手伸到袍子里去掏皮夹子,丹朱道:“我是月季票。”又道:“你这学期选了什么课?”传庆道:“跟从前差不多,没有多大变动。”丹朱笑道:“我爸爸教的文学史,你还念吗?”传庆点点头。丹朱笑道:“你知道么?我也选了这一课。”传庆诧异道:“你打算做你爸爸的学生?”丹朱扑嗤一笑道:“可不是!起先他不肯呢!他弄不惯有个女儿在那里随班听讲,他怕他会觉得窘。还有一层,他在家里跟我们玩笑惯了的,上了堂,也许我倚仗着是自己家里人,照常的问长问短,跟他唠叨。他又板不起脸来!结果我向他赌神罚咒说:上他的课,我无论有什么疑难的地方,绝对不开口。他这才答应了。”传庆微微地叹了一口气道:“言教授……人是好的!”丹朱笑道:“怎么?他做先生,不好么?你不喜欢上他的课?”传庆道:“你看看我的分数单子,就知道他不喜欢我。”丹朱道:“哪儿来的话?他对你特别严,因为你是上海来的,国文程度比香港的学生高。他常常夸你来着,说你就是有点懒。”

传庆掉过头去不言语,把脸贴在玻璃上。他不能老是凑在她跟前,用全副精神听她说话。让人瞧见了,准得产生某种误会。说闲话的人已经不少了,就是因为言丹朱总是找着他。在学校里,谁都不理他。他自己觉得不得人心,越发的避着人,可是他躲不了丹朱。

丹朱——他不懂她的存心。她并不短少朋友。虽然她才在华南大学读了半年书,已经在校花队里有了相当的地位。凭什么她愿意和他接近?他斜着眼向她一瞟。一件白绒线紧身背心把她的厚实的胸脯子和小小的腰塑成了石膏像。他重新别过头去,把额角在玻璃窗上揉擦着。他不爱看见女孩子,尤其是健全美丽的女孩子,因为她们对于自己分外的感到不满意。丹朱又说话了。他摆着盾毛勉强笑道:“对不起,没听见。”她提高了声音又说了一遍,说了一半,他又听不仔细了。幸而他是沉默惯了的,她得不到他的答复,也就恬然不以为怪。末后她有一句话,他却凑巧听懂了。她低下头去,只管把绒线背心往下扯,扯下去又缩上去了。她微笑着道:“前天我告诉你的关于德荃写给我的那封信,请你忘记掉它罢。只当我没有说过。”传庆道:“为什么?”丹朱道:“为什么?……那是很明显的。我不该把这种事告诉人。我太孩子气了,肚子里搁不住两句话!”传庆把身子往前探着,两肘支在膝盖上,只是笑。丹朱也跟着他向前俯着一点,郑重地问道:“传庆,你没有误会我的意思罢?我告诉你那些话,决不是夸耀。我——我不能不跟人谈谈,因为有些话闷在心里太难受了……像德荃,我拒绝了他,就失去了他那样的一个朋友。我爱和他做朋友。我爱和许多人做朋友,至于其他的问题,我们年纪太小了,根本谈不到。可是……可是他们一个个的都那么认真!”隔了一会,她又问道:“传庆,你嫌烦么?”传庆摇摇头。丹朱道:“我不知为什么,这些话我对谁也不说,除了你。”传庆道:“我也不懂为什么。”丹朱道:“我想是因为……因为我把你当做一个女孩子看待。”传庆酸酸地笑了一声道:“是吗?你的女朋友也多得很,怎么单拣中了我呢?”丹朱道:“因为只有你能够守秘密。”传庆倒抽了一口冷气道:“是的,因为我没有朋友,没有人可告诉。”丹朱忙道:“你又误会了我的意思!”两人半晌都没做声。丹朱叹了口气道:“我说错了话,但是……但是,传庆,为什么你不试着交几个朋友?玩儿的时候,读书的时候,也有个伴。你为什么不邀我们上你家里去打网球?我知道你们有个网球场。”传庆笑道:“我们的网球场,很少有机会腾出来打网球。多半是晾满了衣裳,天暖的时候,他们在那里煮鸦片烟。”丹朱顿住了口,说不下去了。

传庆回过头去向着窗外。那公共汽车猛地转了一个弯,人手里的杜鹃花受了震,簌簌乱飞。传庆再看丹朱时,不禁咦了一声道:“你哭了!”丹朱道:“我哭做什么?我从来不哭的!”然而她终于凄哽地质问道:“你……你老是使我觉得我犯了法……仿佛我没有权利这么快乐!其实,我快乐,又不碍着你什么!”传庆取过她手里的书,把上面的水渍子擦了一擦,道:“这是言教授新编的讲义么?我还没有买呢。你想可笑么,我跟他念了半年书,还不知道他的名字。”丹朱道:“我喜欢他的名字。我常常告诉他,他的名字比人漂亮。”传庆在书面上找到了,读出来道:“言子夜……”他把书搁了下来,偏着头想了一想,又拿起来念了一遍道:“言子夜……”这一次,他有点犹疑,仿佛不大认识这几个字。丹朱道:“这名字取得不好么?”传庆笑道:“好!怎么不好!知道你有个好爸爸!什么都好,就是把你惯坏了!”丹朱轻轻地啐了一声,站起身来道:“我该下去了。再见罢!”

她走了,传庆把头靠在玻璃窗上,又仿佛盹着了似的。前面站着的抱着杜鹃花的人也下去了,窗外少了杜鹃花,只剩下灰色的街。他的脸,换了一副背景,也似乎是黄了,暗了。

车再转了个弯。棕榈树沙沙地擦着窗户,他跳起身来,拉了拉铃,车停了,他就下了车。

他家是一座大宅。他们初从上海搬来的时候,满院子的花木。没两三年的工夫,枯的枯,死的死,砍掉的砍掉,太阳光晒着,满眼的荒凉。一个打杂的,在草地上拖翻了一张藤椅子,把一壶滚水浇了上去,杀臭虫。

屋子里面,黑沉沉的穿堂,只看见那朱漆楼梯的扶手上,一线流光,回环曲折,远远的上去了。传庆蹑手蹑脚上了楼,觑人不见,一溜烟向他的卧室里奔去。不料那陈旧的地板吱吱格格一阵响,让刘妈听见了,迎面拦住道:“少爷回来了!见过了老太太没有?”传庆道:“待会儿吃饭的时候总要见到的,忙什么?”刘妈一把揪住他的袖子道:“又来了!你别是又做了什么亏心事?鬼鬼祟祟地躲着人!趁早去罢,打个照面就完事了。不去,又是一场气!”传庆忽然年纪小了七八岁,咬紧了牙,抵死不肯去。刘妈越是推推搡搡,他越是挨挨蹭蹭。刘妈是他母亲当初陪嫁的女佣。在家里,他憎厌刘妈,正如同在学校里他憎厌言丹朱一般。寒天里,人冻得木木的,倒也罢了。一点点的微温,更使他觉得冷的彻骨酸心。

他终于因为憎恶刘妈的缘故,只求脱身,答应去见他父亲与后母。他父亲聂介臣,汗衫外面罩着一件油渍斑斑的雪青软缎小背心,他后母蓬着头,一身黑,面对面躺在烟铺上。他上前呼了“爸爸,妈!”两人都似理非理地哼了一声。传庆心里一块石头方才落了地,猜着今天大约没有事犯到他们手里。他父亲问道:“学费付了?”传庆在烟榻旁边一张沙发椅上坐下,答道:“付了。”他父亲道:“选了几样什么?”传庆道:“英文历史,十九世纪英文散文——”他父亲道:“你那个英文——算了罢!跷脚驴子跟马跑,跑折了腿,也是空的!”他后母笑道:“人家是少爷脾气。大不了,家里请个补课先生,随时给他做枪手。”他父亲道:“我可没那个闲钱给他请家庭教师。还选了什么?”传庆道:“中国文学史。”他父亲道:“那可便宜了你!唐诗,宋词,你早读过了。”他后母道:“别的本事没有,就会偷懒!”

传庆把头低了又低,差一点垂到地上去。身子向前伛偻着,一只手握着鞋带的尖端的小铁管,在皮鞋上轻轻刮着。他父亲在烟炕上翻过身来,捏着一卷报纸,在他颈子上刷地敲了一下,喝道:“一双手,闲着没事干,就会糟蹋东西!”他后母道:“去,去,去罢!到那边去烧几个烟泡。”

传庆坐到墙角里一只小凳上。就着矮茶几烧烟,他后母今天却是特别的兴致好,拿起描金小茶壶喝了一口茶,抿着嘴笑道:“传庆,你在学校里有女朋友没有?”他父亲道:“他呀,连男朋友都没有,也配交女朋友。”他后母笑道:“传庆,我问你,外面有人说,有个姓言的小姐,也是上海来的,在那儿追求你。有这话没有?”传庆红了脸,道:“言丹朱——她的朋友多着呢!哪儿就会看上了我?”他父亲道:“谁说她看上你来着?还不是看上了你的钱!看上你!就凭你?三分像人,七分像鬼——”传庆想道:“我的钱?我的钱?”

总有一天罢,钱是他的,他可以任意地在支票簿上签字。他从十二三岁起就那么盼望着,并且他曾经提早练习过了,将他的名字歪歪斜斜,急如风雨地写在一张作废的支票上,左一个,右一个,“聂传庆,聂传庆,聂传庆”,英俊地,雄纠纠地,“聂传庆,聂传庆。”可是他爸爸重重地打了他一个嘴巴子,劈手将支票夺了过来搓成团,向他脸上抛去。为什么?因为那触动了他爸爸暗藏着的恐惧。钱到了他手里,他会发疯似地胡花么?这畏葸的阴沉的白痴似的孩子。他爸爸并不是有意把他训练成这样的一个人。现在他爸爸见了他,只感到愤怒与无可奈何,私下里又有点害怕。他爸爸说过的:“打了他,倒是不哭,就那么瞪大了眼睛朝人看着。我就顶恨他朝人瞪着眼看——见了就有气!”传庆这时候,手里烧着烟,忍不住又睁大了那惶惑的眼睛,呆瞪瞪望着他父亲。总有一天……那时候,是他的天下了,可是他已经被作践得不像人。奇异的胜利!烟签上的鸦片淋到烟灯里去。传庆吃了一惊,只怕被他们瞧见了,幸而老妈子进来报说许家二姑太太来了,一混就混了过去。他爸爸向他说道:“你趁早给我出去罢!贼头鬼脑的,一点丈夫气也没有,让人家笑你,你不难为情,我还难为情呢!”他后母道:“这孩子,什么病也没有,就是骨瘦如柴,叫人家瞧着,还当我们待亏了他!成天也没有见他少吃少喝!”传庆垂着头出了房,迎面来了女客,他一闪闪在阴影里,四顾无人,方才走进他自己的卧室,翻了一翻从学校里带回来的几本书。他记起了言丹朱屡次劝他用功的话,忽然兴起,一鼓作气地打算做点功课。满屋子雾腾腾的,是隔壁飘过来的鸦片烟香。他生在这空气里,长在这空气里,可是今天不知道为什么,闻了这气味就一阵阵的发晕,只想呕。还是楼底下客室里清净点。他夹了书向下跑,满心的烦躁。客室里有着淡淡的太阳与灰尘。霁红花瓶里插着鸡毛帚子。他在正中的红木方桌旁边坐下,伏在大理石桌面上。桌面冰凉的,像公共汽车上的玻璃窗。窗外的杜鹃花,窗里的言丹朱……丹朱的父亲是言子夜。那名字,他小时候,还不大识字,就见到了。在一本破旧的《早潮》杂志封里的空页上,他曾经一个字一个字吃力地认着:“碧落女史清玩。言子夜赠。”他的母亲的名字是冯碧落。

他随手拖过一本教科书来,头枕在袖子上,看了几页。他仿佛又回到了从前那不大识字的年龄,一个字一个字吃力地认,也不知道念的是什么。忽见刘妈走了进来道:“少爷,让开点。”她取下肩上搭着的桌布,铺在桌上,桌脚上缚了带。传庆道:“怎么?要打牌?”刘妈道:“三缺一,打了电话去请舅老爷去了。”说着,又见打杂的进来换上一只一百支光的电灯泡子。传庆只得收拾了课本,依旧回到楼上来。

他的卧室的角落里堆着一只大藤箱,里面全是破烂的书。他记得有一叠《早潮》杂志在那儿。藤箱上面横缚着一根皮带,他太懒了,也不去脱掉它,就把箱子盖的一头撬了起来,把手伸进去,一阵乱掀乱翻。突然,他想了起来,《早潮》杂志在他们搬家的时候早已散失了,一本也不剩。

他就让两只手夹在箱子里,被箱子盖紧紧压着。头垂着,颈骨仿佛折断了似的。蓝夹袍的领子直竖着,太阳光暖烘烘地从领圈里一直晒进去,晒到颈窝里,可是他有一种奇异的感觉,好像天快黑了——已经黑了。他一个人守在窗子跟前,他心里的天也跟着黑下去。说不出来的昏暗的哀愁……像梦里面似的,那守在窗子前面的人,先是他自己,一刹那间,他看清楚了,那是他母亲。她的前刘海长长地垂着,俯着头,脸庞的尖尖的下半部只是一点白影子,至于那青郁郁的眼与眉,那只是影子里面的影子。然而他肯定地知道那是他死去的母亲冯碧落。他四岁上就没有了母亲,但是他认识她,从她的照片上。她婚前的照片只有一张,她穿着古式的摹本缎袄,有着小小的蝙蝠的暗花。现在,窗子前面的人像渐渐明晰,他可以看见她的秋香色摹本缎袄上的蝙蝠。她在那里等候一个人,一个消息。她明知道消息是不会来的。她心里的天,迟迟地黑了下去。……传庆的身子痛苦地抽搐了一下。他不知道那究竟是他母亲还是他自己。至于那无名的磨人的忧郁,他现在明白了,那就是爱——二十多年前的,绝望的爱。二十多年后,刀子生了锈了,然而还是刀。在他母亲心里的一把刀,又在他心里绞动了。

传庆费了大劲,方始抬起头来。一切的幻像迅速地消灭了。刚才那一会儿,他仿佛是一个旧式的摄影师,钻在黑布里为人拍照片,在摄影机的镜子里瞥见了他母亲。他从箱子盖底下抽出他的手,把嘴凑上去,怔怔地吮着手背上的红痕。

关于他母亲,他知道得很少。他知道她没有爱过他父亲。就为了这个,他父亲恨她。她死了,就迁怒到她丢下的孩子身上。要不然,虽说有后母挑拨着,他父亲对他也不会这么刻毒。他母亲没有爱过他父亲——她爱过别人么?……亲友圈中恍惚有这么一个传说。他后母嫁到聂家来,是亲上加亲,因此他后母也有所风闻。她当然不肯让人们忘怀了这件事,当着传庆的面她也议论过他母亲。任何的话,到了她嘴里就不大好听。碧落的陪嫁的女佣刘妈就是为了不能忍耐她对于亡人的诬蔑,每每气急败坏地向其它的仆人辩白着。于是传庆有机会听到了一点他认为可靠的事实。

用现代的眼光看来,那一点事实是平淡得可怜。冯碧落结婚的那年是十八岁。在订亲以前,她曾经有一个时期渴望着进学校读书。在冯家这样的守旧的人家,那当然是不可能的。然而她还是和几个表妹们背地偷偷地计划着。表妹们因为年纪小得多,父母又放纵些,终于如愿以偿了。她们决定投考中西女塾,请了一个远房亲戚来补课。言子夜辈分比她们小,年纪却比她们长,在大学里已经读了两年书。碧落一面艳羡着表妹们的幸运,一面对于进学校的梦依旧不甘放弃,因此对于她们投考的一切仍然是非常的关心。在表妹那儿她遇见了言子夜几次。他们始终没有单独地谈过话。

言家托了人出来说亲。碧落的母亲还没有开口回答,她祖父丢下的老姨娘坐在一旁吸水烟,先格吱一笑,插嘴道:“现在提这件事,可太早了一点!”那媒人陪笑道:“小姐年纪也不小了——”老姨娘笑道:“我倒不是指她的年纪!常熟言家再强些也是个生意人家。他们少爷若是读书发达,再传个两三代,再到我们这儿来提亲,那还有个商量的余地。现在……可太早了!”媒人见不是话,只得去回掉了言家。言子夜辗转听到了冯家的答复,这一气非同小可,便将这事搁了下来。然而此后他们似乎还会面过一次。那绝对不能够是偶然的机缘,因为既经提过亲,双方都要避嫌疑了。最后的短短的会晤,大约是碧落的主动。碧落暗示子夜重新再托人在她父母跟前疏通,因为她父母并没有过斩钉截铁的拒绝的表示。但是子夜年少气盛,不愿意再三地被斥为“高攀”,使他的家庭受更严重的侮辱。他告诉碧落,他不久就打算出国留学。她可以采取断然的行动,他们两个人一同走。可是碧落不能这样做。传庆回想到这一部分不能不恨他的母亲,但是他也承认,她有她的不得已。二十年前是二十年前呵!她得顾全她的家声,她得顾全子夜的前途。

子夜单身出国去了。他回来的时候,冯家早把碧落嫁给了聂介臣。子夜先后也有几段罗曼史。至于他怎样娶了丹朱的母亲,一个南国女郎,近年来怎样移家到香港,传庆却没有听见说过。关于碧落的嫁后生涯,传庆可不敢揣想。她不是笼子里的鸟。笼子里的鸟,开了笼,还会飞出来。她是绣在屏风上的鸟——悒郁的紫色缎子屏风上,织金云朵里的一只白鸟。年深月久了,羽毛暗了,霉了,给虫蛀了,死也还死在屏风上。

她死了,她完了,可是还有传庆呢?凭什么传庆要受这个罪?碧落嫁到聂家来,至少是清醒的牺牲。传庆生在聂家,可是一点选择的权利也没有。屏风上又添上了一只鸟,打死他也不能飞下屏风去。他跟着他父亲二十年,已经给制造成了一个精神上的残废,即使给了他自由,他也跑不了。

跑不了!跑不了!索性完全没有避免的希望,倒也死心塌地了。但是他现在初次把所有的零星的传闻与揣测,聚集在一起,拼凑一段故事,他方才知道:二十多年前,他还是没有出世的时候,他有脱逃的希望。他母亲有嫁给言子夜的可能性。差一点,他就是言子夜的孩子,言丹朱的哥哥。也许他就是言丹朱。有了他,就没有她。 

第二天,在学校里,上到中国文学史那一课,传庆心里乱极了。他远远看见言丹朱抱着厚沉沉的漆皮笔记夹子,悄悄地溜了进来,在前排的偏左,教授的眼光射不到的地方,拣了一个座位,大约是惟恐引起了她父亲的注意,分了他的心。她掉过头来,向传庆微微一笑。她身边还有一个空位,传庆隔壁的一个男学生便推了传庆一下,撺掇他去坐在她身旁。传庆摇摇头。那人笑道:“就有你这样的傻子!你是怕折了你的福还是怎么着?你不去,我去!”说罢,刚刚站起身来,另有几个学生早已一拥而前,其中有一个捷足先登,占了那座位。

那时虽然还是晚春天气,业已暴热。丹朱在旗袍上加了一件长袖子的白纱外套。她侧过身来和旁边的人有说有笑的,一手托着腮。她那活泼的赤金色的脸和胳膊,在轻纱掩映中,像玻璃杯里滟滟的琥珀酒。然而她在传庆眼中,并不仅仅引起一种单纯的美感。他在那里想:她长得并不像言子夜。那么,她一定是像她的母亲,言子夜所娶的那南国姑娘。言子夜是苍白的,略微有点瘦削,大部分的男子的美,是要到三十岁以后方才更为显著,言子夜就是一个例子。算起来他该过了四十五岁吧?可是看上去要年轻得多。

言子夜进来了,走上了讲台。传庆仿佛觉得以前从来没有见过他一般。传庆这是第一次感觉到中国长袍的一种特殊的萧条的美。传庆自己为了经济的缘故穿着袍褂,但是像一般的青年,他是喜欢西装的。然而那宽大的灰色绸袍,那松垂的衣褶,在言子夜身上,更加显出了身材的秀拔。传庆不由地幻想着:如果他是言子夜的孩子,他长得像言子夜么?十有八九是像的,因为他是男孩子,和丹朱不同。

言子夜翻开了点名簿:“李铭光,董德基,王丽芬,王宗维,王孝贻,聂传庆……”传庆答应了一声,自己疑心自己的声音有些异样,先把脸急红了。然而言子夜继续叫了下去:“秦德芬,张师贤……”一只手撑在桌面上,一只手悠闲地擎着点名簿——一个经历过世道艰难,然而生命中并不缺少一些小小的快乐的人。传庆想着,在他的血管中,或许会流着这个人的血。呵,如果……如果该是什么样的果子呢?该是淡青色的晶莹多汁的果子,像荔枝而没有核,甜里面带着点辛酸。如果……如果他母亲当初略微任性,自私一点,和言子夜诀别的最后一分钟,在情感的支配下,她或者会改变了初衷,向他说:“从前我的一切,都是爹妈做的主。现在你……你替我做主罢。你说怎样就怎样。”如果她不是那么瞻前顾后——顾后!她果真顾到了未来么?她替她未来的子女设想过么?她害了她的孩子!传庆并不是不知道他对于他母亲的谴责是不公正的。她那时候到底是一个十七八岁的女孩子,有那么坚强的道德观念,已经是难得的了。任何人遇到难解决的问题,也只能够“行其心之所安”罢了。他能怪他的母亲么?

言教授背过身去在黑板上写字,学生都沙沙地抄写着,可是传庆的心不在书上。吃了一个“如果”,再剥一个“如果”,譬如说,他母亲和言子夜结了婚,他们的同居生活也许并不是悠久的无瑕的快乐。传庆从刘妈那里知道碧落是一个心细如发的善感的女人。丹朱也曾经告诉他:言子夜的脾气相当的“梗”,而且也喜欢多心。相爱着的人又是往往地爱闹意见,反而是漠不相干的人能够互相容忍。同时,碧落这样的和家庭决裂了,也是为当时的社会所不容许。子夜的婚姻,不免为他的前途上的牵累。近十年来,一般人的观念固然改变了,然而子夜早已几经蹉跎,灭了锐气。一个男子,事业上不得意,家里的种种小误会与口舌更是免不了的。那么,这一切对于他们的孩子有不良的影响么?不,只是好!小小的忧愁与困难可以养成严肃的人生观。传庆相信,如果他是子夜与碧落的孩子,他比起现在的丹朱,一定较为深沉,有思想。同时,一个有爱情的家庭里面的孩子,不论生活如何的不安定,仍旧是富于自信心与同情——积极,进取,勇敢。丹朱的优点他想必都有,丹朱没有的他也有。他的眼光又射到前排坐着的丹朱身上。丹朱凝神听着言教授讲书,偏着脸,嘴微微张着一点,用一支铅笔轻轻叩着小而白的门牙。她的脸庞的侧影有极流丽的线条,尤其是那孩子气的短短的鼻子。鼻子上亮莹莹地略微有点油汗,使她更加像一个喷水池里湿濡的铜像。

她在华南大学专攻科学,可是也匀出一部分的时间来读点文学史什么的。她对于任何事物都感到广泛的兴趣,对于任何人也感到广泛的兴趣。她对于同学们的一视同仁,传庆突然想出了两个字的评语:滥交。她跟谁都搭讪,然而别人有了比友谊更进一步的要求的时候,她又躲开了,理由是他们都在求学时代,没有资格谈恋爱。那算什么?毕了业,她又能做什么事?归根究底还不是嫁人!传庆越想越觉得她的浅薄无聊。如果他有了她这么良好的家庭背景,他一定能够利用这机会,做一个完美的人。总之,他不喜欢言丹朱。

他对于丹朱的憎恨,正像他对言子夜的畸形的倾慕,与日俱增。在这种心理状态下,当然他不能够读书,学期终了的时候,他的考试结果,样样都糟,惟有文学史更为凄惨,距离及格很远,他父亲把他大骂了一顿,然而还是托了人去向学校当局关说,再给他一个机会,秋季开学后让他仍旧随班上课。传庆重新到学校里来的时候,精神上的变态,非但没有痊愈,反而加深了,因为其中隔了一个暑假,他有无限的闲暇,从容地反省他的痛苦的根源。他和他父亲聂介臣日常接触的机会比以前更多了。他发现他有好些地方酷肖他父亲,不但是面部轮廓与五官四肢,连行步的姿态与种种小动作都像。他深恶痛嫉那存在于他自身内的聂介臣。他有方法可以躲避他父亲,但是他自己是永远寸步不离地跟在身边的。

整天他伏在卧室角落里那只藤箱上做着“白日梦”。往往刘妈走过来愕然叫道:“那么辣的太阳晒在身上,觉也不觉得?越大越糊涂,索性连冷热也不知道了!还不快坐过去!”他懒得动,就坐在地上,昏昏地把额角抵在藤箱上,许久许久,额上满是粼粼的凸凹的痕迹。

快开学的时候,他父亲把他叫去告诫了一番道:“你再不学好,用不着往下念了!念也是白念,不过是替聂家丢人!”他因为不愿意辍学,的确下了一番苦功。各种功课倒潦潦草草可以交代得过去了,惟有他父亲认为他应当最有把握的文学史,依旧是一蹶不振,毫无起色。如果改选其他的一课,学分又要吃亏太多,因此没奈何只得继续读下去。

照例圣诞节和新年的假期完毕后就要大考了。圣诞节的前夜,上午照常上课。言教授要想看看学生们的功课是否温习得有些眉目了,特地举行了一个非正式的口试。叫到了传庆,连叫了他两三声,传庆方才听见了,言教授先就有了三分不悦,道:“关于七言诗的起源,你告诉我们一点。”传庆乞乞缩缩站在那里,眼睛不敢望着他,嗫嚅道:“七言诗的起源……”满屋子静悄悄地。传庆觉得丹朱一定在那里看着他——看着他丢聂家的人。不,丢母亲的人!言子夜夫人的孩子,看着冯碧落的孩子出丑。他不能不说点什么,教室里这么静。他舔了舔嘴唇,缓缓地说道:“七言诗的起源……七言的起源……呃……呃……起源诗的七言!”

背后有人笑。连言丹朱也忍不住扑嗤一笑。有许多男生本来没想笑,见言丹朱笑了,也都心痒痒地笑了起来。言子夜见满屋子人笑成一片,只当做传庆有心打趣,便沉下了脸,将书重重的向桌上一掼,冷笑道:“哦,原来这是个笑话!对不起,我没领略到你的幽默!”众人一个个的渐渐敛起了笑容,子夜又道:“聂传庆,我早就注意到你了。从上学期起,你就失魂落魄的。我在讲台上说的话,有一句进你的脑子去没有?你记过一句笔记没有?——你若是不爱念书,谁也不能逼着你念。趁早别来了,白耽搁了你的同班生的时候,也耽搁了我的时候!”传庆听他这口气与自己的父亲如出一辙,忍不住哭了。他用手护着脸,然而言子夜还是看见了。子夜生平最恨人哭,连女人的哭泣他都觉得是一种弱者的要挟行为,至于淌眼抹泪的男子,那更是无耻之尤,因此分外的怒上心来,厉声喝道:“你也不怕难为情!中国的青年都像了你,中国早该亡了!”

这句话更像锥子似地刺进传庆心里去,他索性坐下身来,伏在台上放声哭了起来,子夜道:“你要哭,到外面哭去!我不能让你搅扰了别人。我们还要上课呢!”传庆的哭,一发不可克制,呜咽的声音,一阵比一阵响。他的耳朵又有点聋,竟听不见子夜后来说的话。子夜向前走了一步,指着门,大声道:“你这就给我出去!”传庆站起身,跌跌冲冲走了出去。

当天晚上,华南大学在半山中的男生宿舍里举行圣诞夜的跳舞会。传庆是未满一年的新生,所以也照例被迫购票参加。他父亲觉得既然花钱买了票,不能不放他去,不然,白让学校占了他们一个便宜,因此竟破天荒地容许他单身赴宴。传庆乘车来到山脚下,并不打算赴会,只管向丛山中走去。他预备走一晚上的路,消磨这狂欢的圣诞夜。在家里,他知道他不能够睡觉,心绪过于紊乱了。香港虽说是没有严寒的季节,圣诞节夜却也是够冷的。满山植着矮矮的松杉,满天堆着石青的云。云和树一般被风嘘溜溜吹着,东边浓了,西边稀了,推推挤挤,一会儿黑压压拥成了一团,一会儿又化为一蓬绿气,散了开来。林子里的风,呜呜吼着,像捌犬的怒声。较远的还有海面上的风,因为远,就有点凄然,像哀哀的狗哭。传庆双手筒在袖子里,缩着头,急急地顺着石级走上来。走过了末了一盏路灯,以后的路是漆黑的,但是他走熟了,认得出水门汀道的淡白的边缘。并且他喜欢黑。在黑暗中他可以暂时遗失了自己,脚底下的沙石嘁擦嘁擦响了。是谁?是聂传庆么?“中国的青年都像了他,中国就要亡了”的那个人?就是他?连他自己也不知道是不是。太黑了,瞧不清。

他父亲骂他为“猪,狗”,再骂得厉害些也不打紧,因为他根本看不起他父亲。可是言子夜轻轻的一句话就使他痛心疾首,死也不能忘记。他只顾往前走,也不知走了多少时辰,摸着黑,许是又绕回来了。一转弯,有一盏路灯。一群年青人说着笑着,迎面走了过来,跳舞会该是散了罢?传庆掉过头来就朝着相反的方向走。他听见言丹朱的嗓子在后面叫:“传庆!传庆!”更加走得快。丹朱追了他几步,站住了脚,又回过身来,向她的舞伴们笑道:“再会罢!我要赶上去跟我们那位爱闹蹩扭的姑娘说两句话。”众人道:“可是你总得有人送你回家!”丹朱道:“不要紧,我叫传庆送我回去,也是一样的!”众人还有些踌躇,丹朱笑道:“行!行!真的不要紧!”说着,提起了她的衣服,就向传庆追来。

传庆见她真来了,只得放慢了脚步。丹朱跑得喘吁吁的,问道:“传庆,你怎么不来跳舞?”传庆道:“我不会跳。”丹朱又道:“你在这儿做什么?”传庆道:“不做什么。”丹朱道:“你送我回家,成么?”传庆不答,但是他们渐渐向山巅走去,她的家就在山巅。路还是黑的,只看见她的银白的鞋尖在地上一亮一亮。丹朱再开口的时候,传庆觉得她说话从来没有这么的艰涩迟缓。她说:“你知道吗?今天下课后我找了你半天,你已经回去了。你家的住址我知道,可是你一向不愿意我们到你那儿来……!”传庆依旧是不赞一词。丹朱又道:“今天的事,你得原谅我父亲。他……他做事向来是太认真了,而华南大学的情形使一个认真教书的人不能不灰心——香港一般学生的中文这么糟,可又还看不起中文,不肯虚心研究,你叫他怎么不发急?只有你一个人,国文的根基比谁都强,你又使他失望,你……你想……你替他想想……”传庆只是默然。

丹朱道:“他跟你发脾气的原因,你现在明白了罢?……传庆,你若是原谅了他,你就得向他解释一下,为什么你近来这样的失常。你知道我爸爸是个热心人。我相信他一定肯尽他的能力来帮助你。你告诉我,让我来转告他?行不行?”

告诉丹朱?告诉言子夜?他还记得冯碧落么?记也许记得,可是他是见多识广的男子,一生的恋爱并不止这一次,而碧落只爱过他一个人……从前的女人,一点点小事便放在心上辗转,辗转,辗转思想着,在黄昏的窗前,在雨夜,在惨淡的黎明。呵,从前的人,……

传庆只觉得胸头充塞了吐不出来的冤郁。丹朱又逼紧了一步,问道:“传庆,是你家里的事么?”传庆淡淡地笑道:“你也太好管闲事了!”丹朱并没有生气,反而跟着他笑了。她绝对想不到传庆当真在那里憎嫌她,因为谁都喜欢她。风刮下来的松枝子打到她头上来,她“哟!”了一声,向传庆身后一躲,趁势挽住了传庆的臂膀,柔声道:“到底为什么?”传庆撒开了她的手道:“为什么!为什么!我倒要问问你:为什么你老是缠着我?女孩子家,也不顾个脸面!也不替你父亲想想!”丹朱听了这话,不由得倒退了一步。他在前面走,她在后面跟着,可是两人距离着两三尺远。她幽幽地叹了口气道:“对不起,我又忘了,男女有别!我老是以为我年纪还小呢!我家里的人都拿我当孩子看待。”传庆又跳了起来道:“三句话离不了你的家!谁不知道你有个模范家庭!就可惜你不是一个模范女儿!”丹朱道:“听你的口气,仿佛你就是见不得我似的!仿佛我的快乐,使你不快乐。——可是,传庆,我知道你不是那样的人。你到底——”传庆道:“到底为什么?还不是因为我妒忌你——妒忌你美,你聪明,你有人缘!”丹朱道:“你就不肯同我说一句正经话!传庆,你知道我是你的朋友,我要你快乐——”传庆道:“你要分点快乐给我,是不是?你饱了,你把桌上的面包屑扫下来喂狗吃,是不是?我不要!我不要!我不要!我宁死也不要!”

山路转了一个弯,豁然开朗,露出整个的天与海。路旁有一片悬空的平坦的山崖,围着一圈半圆形的铁栏杆。传庆在前面走着,一回头,不见丹朱在后面,再一看,她却倚在栏杆上。崖脚下的松涛,奔腾澎湃,更有一种耐冷的树,叶子一面儿绿一面儿白,大风吹着,满山的叶子掀腾翻覆,只看见点点银光四溅。云开处,冬天的微黄的月亮出来了,白苍苍的天与海在丹朱身后张开了云母石屏风。她披着翡翠绿天鹅绒的斗篷,上面连着风兜,风兜的里子是白色天鹅绒。在严冬她也喜欢穿白的,因为白色和她黝暗的皮肤是鲜明的对照。传庆从来没看见过她这么盛装过。风兜半褪在她脑后,露出高高堆在顶上的鬈发。背着光,她的脸看不分明,只觉得她的一双眼,灼灼地注视着他。

传庆垂下了眼睛,反剪了手,直挺挺站着。半晌,他重新抬起头来,简截地问道:“走不走?”

她那时已经掉过身去,背对着他。风越发猖狂了,把她的斗篷涨得圆鼓鼓地,直飘到她头上去。她底下穿着一件绿阴阴的白丝绒长袍,乍一看,那斗篷浮在空中仿佛一柄偌大的降落伞,伞底下飘飘荡荡坠着她莹白的身躯——是月宫里派遣来的伞兵么?传庆徐徐走到她身旁。丹朱在那里恋爱着他么?不能够罢?然而,她的确是再三地谋与他接近。譬如说今天晚上,深更半夜她陪着他在空山里乱跑。平时她和同学们玩是玩,笑是笑,似乎很有分寸,并不是一味放荡的人。为什么视他为例外呢?他再将她适才的言行回味了一番。在一个女孩子,那已经是很明显的表示了罢?

他恨她,可是他是一个无能的人,光是恨,有什么用?如果她爱他的话,他就有支配她的权力,可以对于她施行种种绝密的精神上的虐待。那是他唯一的报复的希望。

他颤声问道:“丹朱,你有一点儿喜欢我么?……一点儿?”

她真不怕冷,赤裸着的手臂从斗篷里伸出来,搁在栏杆上。他双手握住了它,伛下头去,想把脸颊偎在她的手臂上,可是不知道为什么,他在半空中停住了,眼泪纷纷地落下来。他伏在栏杆上,枕着手臂——他自己的。

她有点儿爱他么?他不要报复,只要一点爱——尤其是言家的人的爱。既然言家和他没有血统关系,那么,就是婚姻关系也行。无论如何,他要和言家有一点联系。

丹朱把飞舞的斗篷拉了下来,紧紧地箍在身上,笑道:“不止一点儿。我不喜欢你,怎么愿意和你做朋友呢?”传庆站直了身子,咽了一口气道:“朋友!我并不要你做我的朋友。”丹朱道:“可是你需要朋友。”传庆道:“单是朋友不够。我要父亲跟母亲。”丹朱愕然望着他。他紧紧抓住了铁栏杆,仿佛那就是她的手,热烈地说道:“丹朱,如果你同别人相爱着,对于他,你不过是一个爱人。可是对于我,你不单是一个爱人,你是一个创造者,一个父亲,母亲,一个新的环境,新的天地。你是过去与未来。你是神。”丹朱沉默了一会,悄然答道:“恐怕我没有那么大的奢望。我如果爱上了谁,至多我只能做他的爱人与妻子。至于别的,我——我不能那么自不量力。”一阵风把传庆堵得透不过气来。他偏过脸去,双手加紧地握着栏杆,小声道:“那么,你不爱我。一点也不。”丹朱道:“我从来没有考虑过。”传庆道:“因为你把我当一个女孩子。”丹朱道:“不!不!真的……但是……”她先是有点窘,突然觉得烦了,皱着眉毛,疲乏地咳了一声道:“你既然不爱听这个话,何苦逼我说呢?”传庆背过身去,咬着牙道:“你拿我当一个女孩子。你——你——你简直不拿我当人!”他对于他的喉咙失去了控制力,说到末了,简直叫喊起来。

丹朱吃了一惊,下意识地就三脚两步离开了下临深谷的栏杆边,换了一个较安全的地位。跑过去之后,又觉得自己神经过敏的可笑。定了一定神,向传庆微笑道:“你要我把你当做一个男子看待,也行。我答应你,我一定试着用另一副眼光来看你。可是你也得放出点男子气概来,不作兴这么动不动就哭了,工愁善病的——”——传庆嘿嘿地笑了几声道:“你真会哄孩子!‘好孩子别哭!多大的人了,不作兴哭的!’哈哈哈哈……”他笑道,抽身就走,自顾下山去了。

丹朱站着发了一会愣。她没有想到传庆竟会爱上了她。当然,那也在情理之中。他的四周一个亲近的人也没有,惟有她屡屡向他表示好感。她引诱了他(虽然那并不是她的本心),而又不能给予他满足。近来他显然是有一件事使他痛苦着。就是为了她么?那么,归根究底,一切的烦恼还是由她而起?她竭力地想帮助他,反而害了他!她不能让他这样疯疯颠颠走开了,若是闯下点什么祸,她一辈子也不能够饶恕她自己。他的自私,他的无礼,他的不近人情处,她都原宥了他,因为他爱她。连这样一个怪僻的人也爱着她——那满足了她的虚荣心。丹朱是一个善女人,但是她终究是一个女人。

他已经走得很远了,然而她毕竟追上了他,一路喊着:“传庆!你等一等,等一等!”传庆只做不听见。她追到了他的身边,一时又觉得千头万绪,无从说起。她一面喘着气,一面道:“你告诉我……你告诉我……”传庆从牙齿缝里迸出几句话来道:“告诉你,我要你死!有了你,就没有我。有了我,就没有你。懂不懂?”他用一只手臂紧紧挟住她的双肩,另一只手就将她的头拼命地向下按,似乎要她的头缩回到腔子里去。她根本不该生到这世上来,他要她回去。他不知道从哪儿来的蛮力。不过他的手脚还是不够利落。她没有叫出声来,可是挣扎着,两人一同骨碌碌顺着石阶滚下去。传庆爬起身来,抬起腿就向地下的人一阵子踢。一面踢,一面嘴里流水似地咒骂着。话说得太快了,连他自己也听不清。大概似乎是:“你就看准了我是个烂好人!半夜里,单身和我在山上……换了一个人,你就不那么放心罢?你就看准了我不会吻你,打你,杀你,是不是?是不是?聂传庆——不要紧的!‘不要紧,传庆可以送我回家去!’……你就看准了我!”

第一脚踢上去,她低低地嗳唷了一声,从此就没有声音了。他不能不再狠狠地踢两脚,怕她还活着。可是,继续踢下去,他也怕。踢到后来,他的腿一阵阵地发软发麻。在双重恐怖的冲突下,他终于丢下了她,往山下跑。身子就像在梦魇中似的,腾云驾雾,脚不点地,只看见月光里一层层的石阶,在眼前兔起鹘落。跑了一大段路,他突然停住了。黑山里一个人也没有——除了他和丹朱。两个人隔了七八十码远,可是他恍惚可以听见她咻咻的艰难的呼吸声。在这一刹那间,他与她心灵相通,他知道她没有死。知道又怎样?他有这胆量再回去,结果了她?他静静站着,不过两三秒钟,可是他以为是两三个钟点。他又往下跑去。这一次,他一停也不停,一直奔到了山下的汽车道,有车的地方。

家里冷极了,白粉墙也冻得发了青。传庆的房间里没有火炉,空气冷得使人呼吸间鼻子发酸。然而窗子并没有开,长久没开了,屋子里闻得见灰尘与头发的油腻的气味。

传庆脸朝下躺在床上。他听见隔壁他父亲对他后母说:“这孩子渐渐的心野了。跳舞跳得这么晚才回来。”他后母道:“看样子,该给他娶房媳妇了。”

传庆的眼泪直淌下来。嘴部掣动了一下,仿佛想笑,可又动弹不得,脸上像冻上了一层冰壳子。身上也像冻上了一层冰壳子。

丹朱没有死。隔两天开学了,他还得在学校里见到她。他跑不了。



\end{document}