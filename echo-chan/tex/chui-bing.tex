%-*- coding: UTF-8 -*-

\documentclass[UTF8]{ctexart}
\usepackage{graphicx}
\usepackage{float}
\usepackage{CJKpunct}
\usepackage{amsmath}
\usepackage{geometry}
\geometry{a5paper,centering,scale=0.8}
\usepackage[format=hang,font=small,textfont=it]{caption}
\usepackage[nottoc]{tocbibind}
\setromanfont{SourceHanSerifSC-Medium} %设置中文字体
\punctstyle{quanjiao} %使用全角标点	

\title{吹兵}
\author{三毛}
\date{}
\begin{document}
\maketitle

\newpage

那天上学的时候并没有穿红衣服,却被一只疯水牛一路追进学校。跑的开始以为水牛只追一下就算了的,或者会改去追其他的行人,结果他只钉住我锲而不舍的追。哭都来不及哭,只是没命的跑,那四只蹄子奔腾着咄咄的拿角来顶——总是在我裙子后面一点点距离。好不容易逃进了教室,疯牛还在操场上翻蹄子踢土,小学的朝会就此取消了。同学很惊慌,害怕牛会来顶教堂。

晨操播音机里没有音乐,只是一再的播着:「各位同学,留在教室里,不可以出来,不可以出来!」

我是把那条牛引进学校操场上来的小孩子,双手抓住窗口的木框,还是不停的喘气。同学们拿出了童军棍把教室的门顶住。而老师,老师们躲在大办公室里也是门窗紧闭。

就是那一天,该我做值日生。值日生的姓名每天由风纪股长写在黑板上,是两个小孩同时做值日。那个风纪股长忘了是谁,总之是一个老师的马屁鬼,压迫我们的就是她。

我偶尔也被选上当康乐股长,可是康乐和风纪比较起来,那份气势就差多了。

疯水牛还在操场上找东西去顶,风纪股长却发现当天班上的茶壶还是空的。当时,我们做小学生的时候,没有自备水壶这等事的,教室后面放一个大水壶,共用一个杯子,谁渴了就去倒水喝,十分简单。而水壶,是值日生到学校厨房的大灶上去拿滚水,老校工灌满了水,由各班级小朋友提着走回教室。牛在发疯,风纪股长一定逼我当时就去厨房提水,不然就记名字。另外一个值日小朋友哭了,死不肯出去。她哭是为了被记了名字。我拎了空水壶开门走到外面,看也不看牛,拼着命就往通向厨房的长廊狂奔。

等到水壶注满了滚水,没有可能快跑回教室,于是我蹲在走廊的门边,望着远处的牛,想到风纪股长要记名字交给老师算帐,也开始蹲着细细碎碎的哭了。

就在这个时候,清晨出操去的驻军们回来了。驻军是国庆日以前才从台湾南部开来台北,暂住在学校一阵的。

军人来了,看见一只疯牛在操场上东顶西拱的,根本也不当一回事,数百个人杀声震天的不知用上了什么阵法,将牛一步一步赶到校外的田野里去了。

确定牛已经走了,这才提起大茶壶,走三步停两步的往教室的方向去。也是在那么安静的走廊上,身后突然传来咻咻、咻咻喘息的声音,这一慌,腿软了,丢了水壶往地下一蹲,将手抱住头,死啦!牛就在背后。

咻咻的声音还在响,我不敢动。

觉得被人轻轻碰了一下紧缩的肩,慢慢抬头斜眼看,发觉两只暴突有如牛眼般的大眼睛呆呆的瞪着我,眼前一片草绿色。我站了起来——也是个提水的兵,咧着大嘴对我啊啊的打手势。他的水桶好大,一个扁担挑着,两桶水面浮着碧绿的芭蕉叶。漆黑的一个塌鼻子大兵,面如大饼,身壮如山,胶鞋有若小船。乍一看去透着股蛮牛气,再一看,眼光柔和得明明是个孩童。我用袖子擦一下脸,那个兵,也不放下挑着的水桶,另一只手轻轻一下,就拎起了我那个千难万难的热茶壶,做了一个手势,意思是——带路,就将我这瘦小的人和水都送进了教室。那时,老师尚未来,我蹲在走廊水沟边,捡起一片碎石,在泥巴地上写字,问那人——什么兵?那个哑巴笑成傻子一般,放下水桶,也在地上划——炊兵。炊字他写错了,写成——吹兵。后来,老师出现在远远的长廊,我赶快想跑回教室,哑巴兵要握手,我就同他握手,他将我的手上下用劲的摇到人都跳了起来,说不出有多么欢喜的样子。

就因为这样,哑巴做了我的朋友。那时候我小学四年级,功课不忙。回家说起哑巴,母亲斥责我,说不要叫人哑巴哑巴,我笑说他听不见哪,每天早晨见到哑巴,他都丢了水桶手舞足蹈的欢迎我。我们总是蹲在地上写字。第一次就写了个「火」,又写「炊」和「吹」的不同。解释「炊」的时候,我做扇火的样子。这个「吹」就嘟嘟的做号兵状。哑巴真聪明,一教就懂了,一直打自己的头,在地上写「笨」,写成「茶」,我猜是错字,就打了他一下头。那一阵,对一个孩子来说,是光荣的,每天上课之前,先做小老师,总是跟个大汉在地上写字。

哑巴不笨,水桶里满满的水总也不泼出来,他打手势告诉我,水面浮两片大叶子,水就不容易泼出来,很有道理。

后来,在班上讲故事,讲哑巴是四川人,兵过之前他在乡下种田,娶了媳妇,媳妇正要生产,老娘叫哑巴去省城抓药,走在路上,一把给过兵的捉去掮东西,这一掮,就没脱离过军队,家中媳妇生儿生女都不晓得,就来了台湾。

故事是在「康乐时间」说的,同学们听呆了。老师在结束时下了评语,说哑巴的故事是假的,叫同学们不要当真。

天晓得那是哑巴和我打手势、画画、写字、猜来猜去、拼了很久才弄清楚的真实故事。讲完那天,哑巴用他的大手揉揉我的头发,将我的衣服扯扯端正,很伤感的望着我。我猜他一定在想,想他未曾谋面的女儿就是眼前我的样子。

以后做值日生提水总是哑巴替我提,我每天早晨到校和放学回家,都是跟他打完招呼才散。

家中也知道我有了一个大朋友,很感激有人替我提水。母亲老是担心滚烫的水会烫到小孩,她也怕老师,不敢去学校抗议叫小朋友提滚水的事。

也不知日子过了多久,哑巴每日都呆呆的等,只要看见我进了校门,他的脸上才哗一下开出好大一朵花来。后来,因为不知如何疼爱才好,连书包也抢过去代背,要一直送到教室口,这才依依不舍的挑着水桶走了。

哑巴没有钱,给我礼物,总是芭蕉叶子,很细心的割,一点破缝都不可以有。三五天就给一张绿色的方叶子垫板,我拿来铺在课桌上点缀,而老师,总也有些忧心忡忡的望着我。

也有礼物给哑巴,不是美劳课的成绩,就是一颗话梅,再不然放学时一同去坐跷跷板。哑巴重,他都是不敢坐的,耐性用手压着板,我叫他升,他就升,叫他放,他当当心心的放,从来不跌痛我。而我们的游戏,都是安静的,只是夕阳下山后操场上两幅无声无息的剪影而已。

有一天,哑巴神秘兮兮的招手唤我,我跑上去,掌心里一打开,里面是一只金戒指,躺在几乎裂成地图一般的粗手掌里。那是生平第一次看见金子,这种东西家中没有见过,母亲的手上也没见过,可是知道那是极贵重的东西。

哑巴当日很认真,也不笑,瞪着眼,把那金子递上来,要我伸手,要人拿去。我吓得很厉害,拼命摇头,把双手放在身后,死也不肯动。哑巴没有上来拉,他蹲下来在地上写————

\begin{quotation}
不久要分别了,送给你做纪念。
\end{quotation}

我不知如何回答,说了再见,快步跑掉了。跑到一半再回头,看见一个大个子低着头,呆望着自己的掌心。不知在想什么。也是那天回家,母亲说老师来做了家庭访问,比我早一些到了家里去看母亲。家庭访问是大事,一般老师都是预先通知,提早放学,由小朋友陪着老师一家一家去探视的。这一回,老师突袭我们家,十分怪异,不知自己犯了什么错,几乎担了一夜的心。而母亲,没说什么。也因为老师去了家里,这一吓,哑巴要给金子的事情就忘了讲。第二天,才上课呢,老师很慈爱的叫我去她放办公桌的一个角落,低声问我结识那个挑水军人的经过。

都答了,一句一句都回答了,可是不知有什么错,反而慌得很。当老师轻轻的问出:「他有没有对你不轨?」那句话时,我根本听不懂什么叫做鬼不鬼的,直觉老师误会了那个哑巴。不轨一定是一种坏事,不然老师为什么用了一个孩子实在不明白的鬼字。很气愤,太气了,就哭了起来。也没等老师叫人回座,气得冲回课桌趴着大哭。那天放学,老师拉着我的手一路送出校门,看我经过等待着的哑巴,都不许停住脚。

哑巴和我对望了一眼,我眼睛红红的,不能打手势,就只好走。老师,对哑巴笑着点点头。

到了校门口,老师很凶很凶的对我说:「如果明天再跟那个兵去做朋友,老师记你大过,还要打————」我哭着小跑,她抓我回来,讲:「答应呀!讲呀!」我只有点点头,不敢反抗。第二天,没有再跟哑巴讲话,他快步笑着迎了上来,我掉头就跑进了教室。哑巴站在窗外巴巴的望,我的头低着。

是个好粗好大个子的兵,早晚都在挑水,加上两个水桶前后晃,在学校里就更显眼了。男生们见他走过就会唱歌谣似的喊:「一个哑巴提水吃,两个哑巴挑水吃,三个哑巴没水吃……」跟前跟后的叫了还不够,还有些大胆的冲上去推水桶将水泼出来。过去,每当哑巴兵被男生戏弄的时候,他会停下来,放好水桶,作势要追打小孩,等小孩一哄跑了,第一个笑的就是他。也有一次,我们在地上认字,男生欺负哑巴听不见,背着他抽了挑水的扁担逃到秋千架边用那东西去击打架子。我看了追上去,揪住那个光头男生就打,两个厮打得很剧烈,可是都不出声叫喊。最后将男生死命一推,他的头碰到了秋千,这才哇哇大哭着去告老师了。

那是生平第一次在学校打架,男生的老师也没怎么样,倒是哑巴,气得又要骂又心痛般的一直替我掸衣服上的泥巴,然后,他左看我又右看我,大手想上来拥抱这个小娃娃,终是没有做,对我点个头,好似要流泪般的走了。

在这种情感之下,老师突然说哑巴对我「不鬼」,我的心里痛也痛死了。是命令,不可以再跟哑巴来往,不许打招呼,不可以再做小老师,不能玩跷跷板,连美劳课做好的一个泥巴砚台也不能送给我的大朋友。

而他,那个身影,总是在墙角哀哀的张望。

在小学,怕老师怕得太厉害,老师就是天,谁敢反抗她呢?上学总在路上等同学,进校门一哄而入。放学也是快跑,躲着那双粗牛似的眼睛,看也不敢看的背着书包低头疾走。

而我的心,是那么的沉重和悲伤。那种不义的羞耻没法跟老师的权威去对抗,那是一种无关任何生活学业的被迫无情,而我,没有办法。终是在又一次去厨房提水的时候碰到了哑巴。他照样帮我拎水壶,我默默的走在他身边。那时,国庆日也过了,部队立即要开发回南部去,哑巴走到快要到教室的路上,蹲下来也不找小石子,在地上用手指甲一直急着画问号,好大的:「?」画了一连串十几个。他不写字,红着眼睛就是不断画问号。「不是我。」我也不写字,急着打自己的心,双手向外推。

哑巴这回不懂,我快速的在地上写:「不是我!不是我!不是我!」他还是不懂,也写了:「不是给金子坏了?」我拼命摇头。

又不愿出卖老师,只是叫喊:「不要怪我!不是我不是我不是我……」用喊的,他只能看见表情,看见一个受了委屈小女孩的悲脸。就那样跑掉了。哑巴的表情,一生不能忘怀。

部队走时就和来时一般安静,有大卡车装东西,有队伍排成树林一般沙沙、沙沙的移动。走时,校长向他们鞠躬,军人全体举手敬礼道谢。我们孩子在教室内跟着风琴唱歌,唱「淡淡的三月天,杜鹃花开在山坡上,杜鹃花开在小溪旁……」而我的眼光,一直滑出窗外拼命的找人。口里随便跟着唱,跟看军人那一行行都开拔了,我的朋友仍然没有从那群人里找出来。歌又换了,叫唱:「丢丢铜仔」,这首歌非常有趣而活泼,同学们越唱越高昂,都快跳起来了,就在歌唱到最起劲的时候,风琴的伴奏悠然而止,老师紧张的在问:「你找谁?有什么事?」

全班突然安静下来,我才惊觉教室里多了一个大兵。

那个我的好朋友,亲爱的哑巴,山一样立在女老师的面前。「出去!你出去!出去出去……」老师歇斯底里的将风琴盖子砰一下合上,怕成大叫出来。

我不顾老师的反应,抢先跑到教室外面去,对着教室里喊:「哑巴!哑巴!」一面急着打手势叫他出来。

哑巴赶快跑出来了,手上一个纸包;书一般大的纸包,递上来给我。他把我的双手用力握住,呀呀的尽可能发出声音跟我道别。接住纸包也来不及看,哑巴全身装备整齐的立正,认认真真的敬了一个举手礼,我呆在那儿,看着他布满红丝的凸眼睛,不知做任何反应。

他走了,快步走了。一个军人,走的时候好像有那么重的悲伤压在肩上,低着头大步大步的走。

纸包上有一个地址和姓名,是部队信箱的那种。

纸包里,一大口袋在当时的孩子眼中贵重如同金子般的牛肉干。一生没有捧过那么一大包肉干,那是新年才可以分到一两片的东西。老师自然看了那些东西。

地址,她没收了,没有给我。牛肉干,没有给吃,说要当心,不能随便吃。校工的土狗走过,老师将袋子半吊在空中,那些肉干便由口袋中飘落下来,那只狗,跳起来接着吃,老师的脸很平静而慈爱的微笑着。许多年过去了,再看《水浒传》,看到翠屏山上杨雄正杀潘巧云,巧云向石秀呼救,石秀答了一句:「嫂嫂!不是我!」

那一句「不是我!」勾出了当年那一声又一声一个孩子对着一个哑巴聋兵狂喊的:「不是我!不是我!不是我!」

那是今生第一次负人的开始,而这件伤人的事情,积压在内心一生,每每想起,总是难以释然,深责自己当时的懦弱,而且悲不自禁。而人生的不得已,难道只用「不是我」三个字便可以排遣一切负人之事吗?亲爱的哑巴「吹兵」,这一生,我没有忘记过你,你还记得炊和吹的不同。正如我对你一样,是不是?我的本名叫陈平,那件小学制服上老挂着的名字。而今你在哪里?请求给我一封信,好叫我买一大包牛肉干和一个金戒指送给你可不可以?
 
\end{document}