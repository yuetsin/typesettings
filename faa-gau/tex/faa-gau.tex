%-*- coding: UTF-8 -*-

\documentclass[UTF8]{ctexart}
\usepackage{graphicx}
\usepackage{float}
\usepackage{color}
\usepackage{CJKpunct}
\usepackage{amsmath}
\usepackage{geometry}
\geometry{a6paper,centering,scale=0.8}
\usepackage[format=hang,font=small,textfont=it]{caption}
\usepackage[nottoc]{tocbibind}
\setCJKmainfont{STSongti-SC-Light}
\setromanfont{FandolSong}
\punctstyle{quanjiao} %使用全角标点	

\title{花狗}
\author{萧红}
\date{}
\begin{document}
\maketitle

\newpage
在一个深奥的,很小的院心上,集聚几个邻人。这院子种着两棵大芭蕉,人们就在芭蕉叶子下边谈论着李寡妇的大花狗。

有的说:

「看吧,这大狗又倒霉了。」

有的说:

「不见得,上回还不是闹到终归儿子没有回来,花狗也饿病了,因此李寡妇哭了好几回……」

「唉,你就别说啦,这两天还不是么,那大花狗都站不住了,若是人一定要扶着墙走路……」

人们正说着,李寡妇的大花狗就来了。它是一条虎狗,头是大的,嘴是方的,走起路来很威严,全身是黄毛带着白花。它从芭蕉叶里露出来了,站在许多人的面前,还勉强的摇一摇尾巴。

但那原来的姿态完全不对了,眼睛没有一点光亮,全身的毛好象要脱落似的在它的身上飘浮着。而最可笑的是它的脚掌很稳的抬起来,端得平平的再放下去,正好象希特勒的在操演的军队的脚掌似的。

人们正想要说些什么,看到李寡妇戴着大帽子从屋里出来,大家就停止了,都把眼睛落到李寡妇的身上。她手里拿着一把黄香,身上背着一个黄布口袋。

「听说少爷来信了,倒是吗?」

「是的,是的,没有多少日子,就要换防回来的……是的……亲手写的信来……我是到佛堂去烧香,是我应许下的,只要老佛保佑我那孩子有了信,从那天起,我就从那天三遍香烧着,一直到他回来……」那大花狗仍照着它平常的习惯,一看到主人出街,它就跟上去,李寡妇一边骂着就走远了。

那班谈论的人,也都谈论一会各自回家了。

留下了大花狗自己在芭蕉叶下蹲着。

大花狗,李寡妇养了它十几年,李老头子活着的时候,和她吵架,她一生气坐在椅子上哭半天会一动不动的,大花狗就陪着她蹲在她的脚尖旁。她生病的时候,大花狗也不出屋,就在她旁边转着。她和邻居骂架时,大花狗就上去撕人家衣服。她夜里失眠时,大花狗摇着尾巴一直陪她到天明。

所以她爱这狗胜过于一切了,冬天给这狗做一张小棉被,夏天给它铺一张小凉席。

李寡妇的儿子随军出发了以后,她对这狗更是一时也不能离开的,她把这狗看成个什么都能了解的能懂人性的了。

有几次她听了前线上恶劣的消息,她竟拍着那大花狗哭了好几次,有的时候象枕头似的枕着那大花狗哭。

大花狗也实在惹人怜爱,卷着尾巴,虎头虎脑的,虽然它忧愁了,寂寞了,眼睛无光了,但这更显得它柔顺,显得它温和。所以每当晚饭以后,它挨着家是凡里院外院的人家,它都用嘴推开门进去拜访一次,有剩饭的给它,它就吃了,无有剩饭,它就在人家屋里绕了一个圈就静静的出来了。这狗流浪了半个月了,它到主人旁边,主人也不打它,也不骂它,只是什么也不表示,冷静的接得了它,而并不是按着一定的时候给东西吃,想起来就给它,忘记了也就算了。

大花狗落雨也在外边,刮风也在外边,李寡妇整天锁着门到东城门外的佛堂去。

有一天她的邻居告诉她:

「你的大花狗,昨夜在街上被别的狗咬了腿流了血……」

「是的,是的,给它包扎包扎。」

「那狗实在可怜呢,满院子寻食……」邻人又说。

「唉,你没听在前线上呢,那真可怜……咱家里这一只狗算什么呢?」她忙着话没有说完,又背着黄布口袋上佛堂烧香去了。

等邻人第二次告诉她说:

「你去看看你那狗吧!」

那时候大花狗已经躺在外院的大门口了,躺着动也不动,那只被咬伤了的前腿,晒在太阳下。

本来李寡妇一看了也多少引起些悲哀来,也就想喊人来花两角钱埋了它。但因为刚刚又收到儿子一封信,是广州退却时写的,看信上说儿子就该到家了,于是她逢人便讲,竟把花狗又忘记了。

这花狗一直在外院的门口,躺了三两天。

是凡经过的人都说这狗老死了,或是被咬死了,其实不是,它是被冷落死了。

\end{document}