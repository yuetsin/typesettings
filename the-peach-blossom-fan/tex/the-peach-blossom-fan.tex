%-*- coding: UTF-8 -*-

\documentclass[UTF8]{ctexart}
\usepackage{graphicx}
\usepackage{float}
\usepackage{CJKpunct}
\usepackage{amsmath}
\usepackage{geometry}
\geometry{a4paper,centering,scale=0.8}
\usepackage[format=hang,font=small,textfont=it]{caption}
\usepackage[nottoc]{tocbibind}
\setromanfont{SourceHanSerifTC-Medium} %设置中文字体
\punctstyle{quanjiao} %使用全角标点	


\title{桃花扇}
\author{續四十齣 餘韻}


\begin{document}

\tableofcontents

\newpage

\section*{桃花扇/餘韻}

\subsection{西江月}

(淨扮樵子挑擔上)放目蒼崖萬丈,拂頭紅樹千枝;雲深猛虎出無時,也避人間弓矢。建業城啼夜鬼,維揚井貯秋屍;樵夫剩得命如絲,滿肚南朝野史。在下蘇崑生,自從乙酉年同香君到山,一住三載,俺就不曾回家,往來牛首、棲霞,採樵度日。誰想柳敬亭與俺同志,買隻小船,也在此捕魚為業。且喜山深樹老,江闊人稀;每日相逢,便把斧頭敲著船頭,浩浩落落,儘俺歌唱,好不快活。今日柴擔早歇,專等他來促膝閒話,怎的還不見到。(歇擔盹睡介)(丑扮漁翁搖船上)年年垂釣鬢如銀,愛此江山勝富春;歌舞叢中征戰裡,漁翁都是過來人。俺柳敬亭送侯朝宗修道之後,就在這龍潭江畔,捕魚三載,把些興亡舊事,付之風月閒談。今值秋雨新晴,江光似練,正好尋蘇崑生飲酒談心。(指介)你看,他早已醉倒在地,待我上岸,喚他醒來。(作上岸介)(呼介)蘇崑生。(淨醒介)大哥果然來了。(丑拱介)賢弟偏杯呀!(淨)柴不曾賣,那得酒來。(丑)愚兄也沒賣魚,都是空囊,怎麼處?(淨)有了,有了!你輸水,我輸柴,大家煮茗清談罷。(副末扮老贊禮,提絃攜壺上)江山江山,一忙一閒,誰贏誰輸,兩鬢皆斑。(見介)原來是柳、蘇兩位老哥。(淨、丑拱介)老相公怎得到此?(副末)老夫住在燕子磯邊,今乃戊子年九月十七日,是福德星君降生之辰;我同些山中社友,到福德神祠祭賽已畢,路過此間。(淨)為何挾著絃子,提著酒壺?(副末)見笑見笑!老夫編了幾句神絃歌,名曰《問蒼天》。今日彈唱樂神,社散之時,分得這瓶福酒。恰好遇著二位,就同飲三杯罷。(丑)怎好取擾。(副末)這叫做『有福同享』。(淨、丑)好,好!(同坐飲介)(淨)何不把神絃歌領略一回?(副末)使得!老夫的心事,正要請教二位哩。(彈絃唱巫腔)(淨、丑拍手襯介)

\subsection{問蒼天}

\begin{verse}

    新曆數,順治朝,歲在戊子;九月秋,十七日,嘉會良時。
    
    擊神鼓,揚靈旗,鄉鄰賽社;老逸民,剃白髮,也到叢祠。
    
    椒作棟,桂為楣,唐修晉建;碧和金,丹間粉,畫壁精奇。
    
    貌赫赫,氣揚揚,福德名位;山之珍,海之寶,總掌無遺。
    
    超祖禰,邁君師,千人上壽;焚郁蘭,奠清醑,奪戶爭墀。
    
    草笠底,有一人,掀鬚長嘆:貧者貧,富者富,造命奚為?
    
    我與爾,較生辰,同月同日;囊無錢,竈斷火,不啻乞兒。
    
    六十歲,花甲週,桑榆暮矣;亂離人,太平犬,未有亨期。
    
    稱玉斝,坐瓊筵,爾餐我看;誰為靈,誰為蠢,貴賤失宜。
    
    臣稽首,叫九閽,開聾啟瞶;宣命司,檢祿籍,何故差池。
    
    金闕遠,紫宸高,蒼天夢夢;迎神來,送神去,輿馬風馳。
    
    歌舞罷,雞豚收,須臾社散;倚枯槐,對斜日,獨自凝思。
    
    濁享富,清享名,或分兩例;內才多,外財少,應不同規。
    
    熱似火,福德君,庸人父母;冷如冰,文昌帝,秀士宗師。
    
    神有短,聖有虧,誰能足願;地難填,天難補,造化如斯。
    
    釋盡了,胸中愁,欣欣微笑;江自流,雲自卷,我又何疑。

\end{verse}

(唱完放絃介)出醜之極。(淨)妙絕!逼真《離騷》、《九歌》了。(丑)失敬,失敬!不知老相公竟是財神一轉哩。(副末讓介)請乾此酒。(淨咂舌介)這寡酒好難吃也。(丑)愚兄倒有些下酒之物。(淨)是什麼東西?(丑)請猜一猜。(淨)你的東西,不過是些魚鱉蝦蟹。(丑搖頭介)猜不著,猜不著。(淨)還有什麼異味?(丑指口介)是我的舌頭。(副末)你的舌頭,你自下酒,如何讓客。(丑笑介)你不曉得,古人以《漢書》下酒;這舌頭會說《漢書》,豈非下酒之物。(淨取酒斟介)我替老哥斟酒,老哥就把《漢書》說來。(副末)妙妙!只恐菜多酒少了。(丑)既然《漢書》太長,有我新編的一首彈詞,叫做《秣陵秋》,唱來下酒罷。(副末)就是俺南京的近事麼?(丑)便是!(淨)這都是俺們耳聞眼見的,你若說差了,我要罰的。(丑)包管你不差。(丑彈絃介)六代興亡,幾點清彈千古慨;半生湖海,一聲高唱萬山驚。(照盲女彈詞唱介)

\subsection{問蒼天}

\begin{verse}

    陳隋煙月恨茫茫,井帶胭脂土帶香;駘蕩柳綿沾客鬢,叮嚀鶯舌惱人腸。
    
    中興朝市繁華續,遺孽兒孫氣焰張;只勸樓臺追後主,不愁弓矢下殘唐。
    
    蛾眉越女才承選,燕子吳歈早擅場,力士簽名搜笛步,龜年協律奉椒房。
    
    西崑詞賦新溫李,烏巷冠裳舊謝王;院院宮妝金翠鏡,朝朝楚夢雨雲床。
    
    五侯閫外空狼燧,二水洲邊自雀舫;指馬誰攻秦相詐,入林都畏阮生狂。
    
    春燈已錯從頭認,社黨重鉤無縫藏;借手殺讎長樂老,脅肩媚貴半閒堂。
    
    龍鍾閣部啼梅嶺,跋扈將軍譟武昌;九曲河流晴喚渡,千尋江岸夜移防。
    
    瓊花劫到雕欄損,玉樹歌終畫殿涼;滄海迷家龍寂寞,風塵失伴鳳徬徨。
    
    青衣啣璧何年返,碧血濺沙此地亡;南內湯池仍蔓草,東陵輦路又斜陽。
    
    全開鎖鑰淮揚泗,難整乾坤左史黃。建帝飄零烈帝慘,英宗困頓武宗荒;
    
    那知還有福王一,臨去秋波淚數行。

\end{verse}

(淨)妙妙!果然一些不差。(副末)雖是幾句彈詞,竟似吳梅村一首長歌。(淨)老哥學問大進,該敬一杯。(斟酒介)(丑)倒叫我吃寡酒了。(淨)愚弟也有些須下酒之物。(丑)你的東西,一定是山殽野蔬了。(淨)不是,不是。昨日南京賣柴,特地帶來的。(丑)取來共享罷。(淨指口介)也是舌頭。(副末)怎的也是舌頭?(淨)不瞞二位說,我三年沒到南京,忽然高興,進城賣柴。路過孝陵,見那寶城享殿,成了芻牧之場。(丑)呵呀呀!那皇城如何?(淨)那皇城牆倒宮塌,滿地蒿萊了。(副末掩淚介)不料光景至此。(淨)俺又一直走到秦淮,立了半晌,竟沒一個人影兒。(丑)那長橋舊院,是咱們熟遊之地,你也該去瞧瞧。(淨)怎的沒瞧,長橋已無片板,舊院剩了一堆瓦礫。(丑搥胸介)咳!慟死俺也。(淨)那時疾忙回首,一路傷心;編成一套北曲,名為《哀江南》。待我唱來!(敲板唱弋陽腔介)俺樵夫呵!

\subsection*{哀江南}

\subsection{北新水令}

\begin{verse}
    山松野草帶花挑,猛抬頭秣陵重到。
    
    殘軍留廢壘,瘦馬臥空壕;
    
    村郭蕭條,城對著夕陽道。
\end{verse}

\subsection{駐馬聽}

\begin{verse}
    野火頻燒,護墓長楸多半焦。
    
    山羊群跑,守陵阿監幾時逃。
    
    鴿翎蝠糞滿堂拋,枯枝敗葉當階罩;
    
    誰祭掃,牧兒打碎龍碑帽。
\end{verse}

\subsection{沈醉東風}

\begin{verse}
    橫白玉八根柱倒,墮紅泥半堵牆高,
    
    碎琉璃瓦片多,爛翡翠窗櫺少,
    
    舞丹墀燕雀常朝,直入宮門一路蒿,住幾個乞兒餓殍。
\end{verse}

\subsection{折桂令}

\begin{verse}
    問秦淮舊日窗寮,破紙迎風,壞檻當潮,目斷魂消。
    
    當年粉黛,何處笙簫?
    
    罷燈船端陽不鬧,收酒旗重九無聊。
    
    白鳥飄飄,綠水滔滔,嫩黃花有些蝶飛,新紅葉無個人瞧。
\end{verse}

\subsection{沽美酒}

\begin{verse}
    你記得跨青谿半里橋,舊紅板沒一條。
    
    秋水長天人過少,冷清清的落照,剩一樹柳彎腰。
\end{verse}

\subsection{太平令}

\begin{verse}
    行到那舊院門,何用輕敲,也不怕小犬牢牢。
    
    無非是枯井頹巢,不過些磚苔砌草。
    
    手種的花條柳梢,儘意兒採樵;這黑灰是誰家廚竃?
\end{verse}

\subsection{離亭宴帶歇拍煞}

\begin{verse}
    俺曾見金陵玉殿鶯啼曉,秦淮水榭花開早,誰知道容易冰消!
    
    眼看他起朱樓,眼看他宴賓客,眼看他樓塌了。
    
    這青苔碧瓦堆,俺曾睡風流覺,將五十年興亡看飽。
    
    那烏衣巷不姓王,莫愁湖鬼夜哭,鳳凰臺棲梟鳥。
    
    殘山夢最真,舊境丟難掉,不信這輿圖換稿。
    
    謅一套哀江南,放悲聲唱到老。
\end{verse}

(副末掩淚介)妙是絕妙,惹出我多少眼淚。(丑)這酒也不忍入唇了,大家談談罷。(副淨時服,扮皂隸暗上)朝陪天子輦,暮把縣官門;皂隸原無種,通侯豈有根?自家魏國公嫡親公子徐青君的便是,生來富貴,享盡繁華。不料國破家亡,剩了區區一口。沒奈何在上元縣當了一名皂隸,將就度日。今奉本官籤票,訪拿山林隱逸,只得下鄉走走。(望介)那江岸之上,有幾個老兒閒坐,不免上前討火,就便訪問。正是:開國元勳留狗尾,換朝逸老縮龜頭。(前行見介)老哥們有火借一個!(丑)請坐。(副淨坐介)(副末問介)看你打扮,像一位公差大哥。(副淨)便是。(淨問介)要火吃煙麼,小弟帶有高煙,取出奉敬罷。(敲火取煙奉副淨介)(副淨吃煙介)好高煙,好高煙!(作暈醉臥倒介)(淨扶介)(副淨)不要拉我,讓我歇一歇,就好了。(閉目臥介)(丑問副末介)記得三年之前,老相公捧著史閣部衣冠,要葬在梅花嶺下,後來怎樣?(副末)後來約了許多忠義之士,齊集梅花嶺,招魂埋葬,倒也算千秋盛事,但不曾立得碑碣。(淨)好事,好事,只可惜黃將軍刎頸報主,拋屍路旁,竟無人埋葬。(副末)如今好了,也是我老漢同些村中父老,檢骨殯殮,起了一座大大的墳塋,好不體面。(丑)你這兩件功德,卻也不小哩。(淨)二位不知,那左寧南氣死戰船時,親朋盡散,卻是我老蘇殯殮了他。(副末)難得,難得。聞他兒子左夢庚襲了前程,昨日扶柩回去了。(丑掩淚介)左寧南是我老柳知己。我曾託藍田叔畫他一幅影像,又求錢牧齋題贊了幾句;逢時遇節,展開祭拜,也盡俺一點報答之意。(副淨醒,作悄語介)聽他說話,像幾個山林隱逸。(起身問介)三位是山林隱逸麼?(眾起拱介)不敢,不敢,為何問及山林隱逸?(副淨)三位不知麼,現今禮部上本,搜尋山林隱逸。撫按大老爺張掛告示,布政司行文已經月餘,並不見一人報名。府縣著忙,差俺們各處訪拿,三位一定是了,快快跟我回話去。(副末)老哥差矣,山林隱逸乃文人名士,不肯出山的。老夫原是假斯文的一個老贊禮,那裡去得。(丑、淨)我兩個是說書唱曲的朋友,而今做了漁翁樵子,益發不中了。(副淨)你們不曉得,那些文人名士,都是識時務的俊傑,從三年前俱已出山了。目下正要訪拿你輩哩。(副末)啐,徵求隱逸,乃朝廷盛典,公祖父母俱當以禮相聘,怎麼要拿起來!定是你這衙役們奉行不善。(副淨)不干我事,有本縣籤票在此,取出你看。(取看籤票欲拿介)(淨)果有這事哩。(丑)我們竟走開如何?(副末)有理。避禍今何晚,入山昔未深。(各分走下)(副淨趕不上介)你看他登崖涉澗,竟各逃走無蹤。


\subsection{清江引}

\begin{verse}
    大澤深山隨處找,預備官家要。
    
    抽出綠頭籤,取開紅圈票,把幾個白衣山人嚇走了。
\end{verse}

(立聽介)遠遠聞得吟詩之聲,不在水邊,定在林下,待我信步找去便了。(急下)(內吟詩曰)

\begin{verse}
    漁樵同話舊繁華,短夢寥寥記不差;

    曾恨紅箋啣燕子,偏憐素扇染桃花。

    笙歌西第留何客?煙雨南朝換幾家?

    傳得傷心臨去語,年年寒食哭天涯。
\end{verse}

\end{document}