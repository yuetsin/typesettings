% XeLaTeX can use any Mac OS X font. See the setromanfont command below.
% Input to XeLaTeX is full Unicode, so Unicode characters can be typed directly into the source.

% The next lines tell TeXShop to typeset with xelatex, and to open and save the source with Unicode encoding.

%!TEX TS-program = xelatex
%!TEX encoding = UTF-8 Unicode

\documentclass[12pt]{article}
\usepackage{geometry}                % See geometry.pdf to learn the layout options. There are lots.
\geometry{a5paper}                   % ... or a4paper or a5paper or ... 
%\geometry{landscape}                % Activate for for rotated page geometry
%\usepackage[parfill]{parskip}    % Activate to begin paragraphs with an empty line rather than an indent
\usepackage{graphicx}
\usepackage{amssymb}

% Will Robertson's fontspec.sty can be used to simplify font choices.
% To experiment, open /Applications/Font Book to examine the fonts provided on Mac OS X,
% and change "Hoefler Text" to any of these choices.

\usepackage{fontspec,xltxtra,xunicode}
\defaultfontfeatures{Mapping=tex-text}
\setromanfont[Mapping=tex-text]{Hoefler Text}
\setsansfont[Scale=MatchLowercase,Mapping=tex-text]{Gill Sans}
\setmonofont[Scale=MatchLowercase]{Andale Mono}

\title{Information on Suicide Methods}
\author{lostallhope.com}
\date{}                                           % Activate to display a given date or no date

\begin{document}
\maketitle

\newpage

\tableofcontents

\newpage

\section{Home}

Welcome to Lost All Hope - one of the most comprehensive suicide resources on the web.

You may be reading this looking for information on methods to commit suicide. They are here. Maybe you'd like to know statistical information about suicide - you're in the right place. Perhaps you are feeling really low; part of you wants to end it, and part of you just wants to be happy. You'll find information and links that might help you.

When I searched on the web in 2002 to try and find ways to kill myself, I was overwhelmed by the amount of information I found; frustrated by how hard it was to find the information I really wanted; fed up of websites that tried to save me.

Lost All Hope has no angle. The site is non-profit. It has no religious affiliation. It has no political stand point on the whys and wherefores, pros and cons, of suicide or euthanasia. It offers no advice, and has no bulletin boards, chat or forums (although does link to some). It is here as an impartial resource, to help inform you, and make whatever choice is right for you. I hope you find it of use.

Before browsing the site, please read the Terms of use. Your use of this site is subject to those terms irrespective of whether you read them or not.

\section{Help me}

I am glad you are reading this, because if you are, it means that at least some part of you believes there is a chance you can be helped. I hope you can spare a couple more minutes to read to the end of this page. If you can't, please at least read Surviving today before you go back to looking at ways to kill yourself.

I know what it is like to feel there is no hope left. To feel like there is nothing worth living for. To not be able to face the rest of your life. I know because I felt it myself, and I am truly sorry you are in that place. It probably means you are past caring about anybody or anything, and don't believe it's possible to change how you feel right now.

Think for a minute. Have you always felt like you do right now? The chances are, there were times in your life when you did not. Which means something in your life changed to get you where you are now. But that also means that something can change to get you \textbf{away} from where you are now. Seriously - life is changing all the time - yours included. Surely there were other times you felt really low and something happened to make you feel brighter?

People think about committing suicide as a solution to a problem they see no solution to. But here's the thing - even if you don't think there is a solution to your problems right now, that does not mean it \textbf{does not exist}. It just means that you can't \textbf{see it} right now.

\subsection{What does it take to feel OK?}

Leaving aside for now those that are terminally ill, it is probably fair to say that most people who are considering killing themselves due to emotional problems, or intolerable life circumstances, have not always felt that way. People are not born suicidal.

It is probably also fair to say, that for these people, given a choice between feeling great about themselves and life generally, or dying, they would probably choose the former. We all want to feel great and happy. Death only becomes attractive once we lose hope that we can ever feel OK again.

For people that have been struggling with emotional problems for years, perhaps sought and had all types of treatments - from therapy to medications - it is easy to see how they could lose hope that anything will make a difference. I felt that way too. Not only when I was suicidal, but many times since.

Many people on this site have issues that have origins years or decades ago. These are tough sons of bitches to shift, and even if they are shifting, those shifts can be so small it is hard to notice them. So what can help?

I am re-writing this page 4.5 years after the site was started. I write it with the benefit of email and feedback from the millions that have visited the website. I believe there are common themes of what people need to live (over and above physical health, a subsistence wage, food, heat, light etc.):

\subsubsection{Emotional connection} The suicidal often feel alone. They may have a partner, a family, friends, but they still feel alone. Because having people around us does not mean we are connected to them. So what is this elusive connection? Being seen for how we truly are - all our bad stuff, and being accepted and \textbf{loved} despite it all. Our desire to feel love is strong, yet can be quite unconscious.

Of course, most of us are far too ashamed of how we feel to let anyone else see it, me included. And in that shame we hide who we really are, or parts of us anyway. Even if we are with people, they don't see the "real" us - we see that part as unlovable. So who we really are never connects with anyone else, and thus we are isolated - desperately needing connection, too ashamed to make connection.

\subsubsection{Physical connection} This could be sex (as part of our genetic programming to reproduce, or simply for pleasure), or be something much simpler. The need to be held, touched, hugged, stroked. Studies on monkeys showed massive detrimental effects on those that had no touch from others, and humans are no different - we need touch.

\subsubsection{Support} In today's world people have become more insular. It is easier to conduct a life from home in front of a PC. People don't rely on each other anymore - we are fiercely independent. People can't fall back on a community, a support structure. This can be especially true for those who are not part of a close family. That may be because their family are no longer living or spread over a wide geographical area, or because they simply aren't close to their parents or siblings.

We need this support desperately. So we have people to fall back on when times are tough. People that can hold us in that space. People we can be ourselves with. People we can really talk to. And people we can have a laugh with, do things with, discuss things with. People that can help us. Sometimes all the therapy and medications in the world just won't work if people can't feel part of a group. What constitutes a group is not set in stone - does not need to be a big group, but there is something about multiple people interacting that can be much stronger than just being with people one-on-one.

\subsubsection{Purpose} Everyone needs a reason to get out of bed in the morning. It might be to look after a family, to earn money for that family, to help friends, to overcome a challenge, to help those in need. There are lots of reasons, but we all need one. Theoretically, I have lots of reasons, but in my darkest moments there are only two I can fall back on. A promise I made to my sister not to kill myself (at least whilst she is still alive!), and one I made to myself to support my best friend during his poor health. Those can make me soldier on even when I don't really want to.    

If a magic wand could be waved and you had these four things right now, the chances of you still feeling suicidal would be quite small. Even if you had a major physical or situational issue in your life.

So is it possible to get these four things? Well, they aren't the most impossible things in life. They are available to pretty much anyone. The question is, whether you believe it enough to make some effort on getting them. Maybe committing suicide seems easy by comparison. But of course, if what we really want is to feel OK, happy, loving and loved, then death is not an option to achieve those.

\subsection{What next?}

So I write this on new year's eve 2009. In the next month, and every month thereafter, over 16 million people will do a Google search on the word suicide. Yes, 16 million, so trust me, you are not alone feeling like you do. In the next 12 months the World Health Organization (see Suicide statistics) estimates well over 20 million people will actually try and commit suicide. Over 1 million will succeed. A good chunk of the others will end up in hospital, in pain. Possibly with permanent health issues. Like brain damage.

It would be nice if big emotional problems had simple fixes, but that is rarely the case. As with all big problems though, the road to overcoming them starts with small steps. On this site, that step is to read Surviving today, so please do.

\subsection{Surviving today}

Assume, just for a second, you have a car. If it breaks down, what do you do? Normally, take it to someone to get it fixed. You wouldn't pour acid on it, blow it up, drive it over a cliff. It's just broken, and broken things can be repaired. You are no different to a car (assuming your issues are emotional, and not physical). You have a problem, and part of that problem is that people don't see ill health of the mind as an illness which is just as serious as any physical illness. But illnesses can be treated. The issue is you either don't believe you can be made better, or don't know how. Both can be addressed.

On www.metanoia.org, Martha Ainsworth uses a slightly different analogy, and talks about suicide being driven by a person getting to a place where their pain exceeds their ability cope with the pain. I think this is also true, and the trick is to either reduce the pain, increase the coping resources, or ideally both.

So, you are on the brink, you have no hope, but for some strange reason you are still reading. Is there anything that can make you feel worse than you do right now? I hope so...

1. Follow the three day rule. If you are ready to commit suicide, like really, really ready, wait three days, or better still a week, until you actually go through with it. If you are going to be dead for the rest of time, what's another few days wait? It may be that in a few days your enthusiasm to go through with it might not be the same, which will suggest that maybe suicide is not the only answer, and possibly something could change in your life circumstances, or how you view/feel about your life, that will change your decision.

Many suicides and attempted suicides are done on \textbf{impulse}, but this suggests that the same people would not have tried to kill themselves either days before, or days after, had they thought about their actions for longer. Suicide is a permanent solution to an often temporary problem. It is not a decision that should be rushed.

2. Speak to someone. \textbf{Now}. People who are suicidal are very often feeling lonely, isolated, depressed, hopeless. I understand that in that place you may not want to speak to anyone. I certainly didn't. After all, what difference can they make? Or maybe you are too embarrassed or ashamed to talk to anyone about how you are feeling, especially someone that knows you? I certainly was.

But I'll say this. However much you might not want to speak to someone, \textbf{do it anyway}. If you are going to spend the rest of time dead, what difference making a phone call now? Speaking to someone, and discussing how you feel, is possibly the single most powerful thing you can do right now. You don't have to speak to someone that knows you. Whatever the time of day, the organizations below have trained people available who will listen to you, without judgement. If you can't face speaking to someone, see further down this page for chat forums:

US: www.samaritansnyc.org (212) 673-3000

UK: www.samaritans.org 08457 90 90 90 (Republic of Ireland 1850 60 90 90)

Australia: www.lifeline.org.au 13 11 14

Worldwide www.befrienders.org (phone numbers dependent on country of residence)

If you have a friend or a family member you can trust to listen to you, without being angry or judgmental, consider using them as well as, or instead of, the above. Just chatting to someone who cares can make all the difference. Really, it can. Even just going round to a friend or family member whose company you enjoy can make a \textbf{huge} difference.

Or call a minster or rabbi. If you are already having psychotherapy, you should tell your psychotherapist.

It is also highly advisable to tell your doctor, as they should be able to point you towards some form of treatment (and see Addressing the problem).

If you don't want to speak to a real person, another option might be going to a chat room (see Chat), which should be an environment where you can share how you feel, and be supported by others who will understand exactly what you are going through. It is open 24/7, and some people might feel more comfortable communicating by chat than to a person. However, this is probably worse than actually speaking to someone.

3. Realize that statistics show the vast majority of people who are suicidal do not go through with it. You are not alone. And the odds of you getting though this and feeling better again are in your favor. So even if you think there is no hope, the statistics would point to there being lots of hope. Most people who are suicidal go on to lead a much longer life.

4. Think, very, very carefully, about the pain of killing yourself. Many people mistakenly assume that suicide is painless. In many cases though, suicide is not painless, and is positively very painful. Look through this site to see the possible dangers of whatever method you are considering.

To read more about longer term initiatives to help you feel better, read the section Addressing the problem.

\subsection{The point of living}

\subsubsection{The Script}

We all have a part of us that seems to take over our lives. That voice inside. Richard Wilkins, founder of the Ministry of Inspiration, calls it our script. It is the sum total of our life experience to date. Everything that has happened to us, that we have heard, seen, felt - and how we have interpreted those experiences.

Many people have a script that says "I'm not good enough", that started in their childhood. Some "I'm not attractive". Some "nobody will ever love me", "I'll always be broke", "I'll never find a partner" or "I'll never be happy", "life will never be OK", "I will always be miserable". And that script is \textbf{POWERFUL}. This second it has you by the scruff of your neck. It has been running your life for a long, long time. And you can't change it - your script is your script. However, what you \textbf{can do} is to recognize it for what it is - a script, written in your past. You can, and I really mean, you can - however fragile and on the edge you might be right now - start to ignore the old script, and start writing a new script, a better one, for the rest of your life.

\subsubsection{What's the point in living?}

In my dark days, and less dark days, there was one thing that I thought about often - what is the point in living? I just wanted to find an answer to that simple question! Maybe if I had a reason for that I'd decide I wouldn't need to kill myself after all. I suspect it might be a question that you have thought about too.

I actually researched that very question for this site, and found an interesting array of answers from philosophers, psychotherapists, religions and social commentators, including:

Many religions state the meaning of life is to serve god and obey his commandments.
 
Instincts states that every action is driven by a need to attract the opposite sex, thereby fulfilling the basic need of reproduction and thereby the continuance of the human race.
 
Some say life doesn't have its own meaning, just the meaning you ascribe to it. And when people are asked what is important to them, the answers generally come down to three things: love, learning and being happy.

My own view, is that there are three or four things that drive us, not necessarily all at the same time. Firstly, physical connection. We have genetic programming to reproduce, and that drives a lot of our behaviour, whether that be making ourselves more attractive to potential partners, or what we have to do to stay with them. When we aren't having physical connection, we are driven by what's required to get it. If we don't have one, it can make us sad, frustrated, or drive us to focus on other things so we are distracted from not having it.

For those that have children, especially younger ones, providing for children can be a really powerful reason to stay alive. Once they grow older this reason can start to reduce though - the kids don't need the parents anymore, and the parents need to find some other reason for living.

Next is emotional connection. Sex and intimacy are not enough any more. We search for deep emotional connection, which some might describe as love. Like when a parent loves a young child unconditionally. The child does not need to be clever, funny, rich or good at sport. They are loved for who they are. As we go through life, we spend so much energy doing things to be loved, whether that is by parents, partners, children or friends. But deep down, we just want to be loved and accepted for who we are - the bad and the good. But these days it seems people find it ever harder to express love, and/or if they get it coming their way, how to receive it. It's no wonder we live in a fucked up world.....

However, love alone isn't enough to keep us going - as Don Henley sung, "sometimes love just ain't enough". So the other thing we need is to have a purpose. make a difference to the world. We need to do something beyond ourselves for the good of our fellow human beings. Why, I don't know. Maybe it is simply genetic programming about making sure our race survives. And whilst we might not be able to control who loves us, or getting physical connection, this is one area we can take charge of our own destiny. Which begs the difficult question - "what do I have to do to make a difference"?

\subsubsection{What makes a difference?}

For me, it was always something more than whatever I did (read My story). If you donated \$10 to provide food or water to a starving person in a developing country, that would make a real difference to them. A huge difference. Maybe the difference between living and dying. Would that be enough reason for you to stay alive?

What about if you stayed alive to help a friend or family member, through some sort of personal crisis. And with your help and support that person went on to make the lives of 20 other people better. Would that be enough? Maybe if you hadn't helped, it never would have happened?

Maybe in staying alive you have a business that makes loads of money and helps find a cure to a disease that improves the lives of thousands of people. Or maybe you are the doctor that helps find that cure? Is that enough?

Or maybe by not killing yourself, you save a number of people you know the sort of unbelievable heartache that is 100 times worse than whatever you are feeling now? Is that alone enough reason? You might be thinking that those near to you will come to terms with what you are about to do; that your pain is much worse than theirs will be. But the legacy you leave those people is not only the grief of your death, but guilt for the rest of their lives that they couldn't help you feel better. They will feel responsible. Or else hurt, confused, angry or remorseful.

Make no mistake. Suicide will have a \textbf{devastating} effect on those around you, and the effects will stay with them the rest of their lives. If in your mind you belittle this impact (as I did), you are fooling yourself. And to test it, just ask someone close to you how they'd feel if you got knocked over by a bus.

Or maybe, like me, you find some small way to help just a few people feel happier in their lives. Even if it is just one person. And trust me, in a life we can all make a difference to at least one other person. Whatever you are suffering from. Whatever it is in your life that makes you desperate enough to find this site. I promise you. I absolutely promise you - YOU CAN MAKE A DIFFERENCE. Much harder is for you to \textbf{realize} how much difference you can make, and probably already do. I am not saying it is a magic spell to make you feel better. People can dedicate their lives to others and still feel miserable, but, coupled with emotional connection, physical connection, feeling part of a group, it is important.

\subsubsection{You are not alone}

I don't judge anyone who considers ending their life, tries to do it, or succeeds. It saddens me that economically the world has come so far, yet emotionally we have gone so far in reverse. Over the last 50 years the suicide rate has increased more than 50\%. So what difference would it make to the world if 20 million people per annum, spent just half the time they were going to spend planning their death, on doing something to help someone else? Even something small to a friend or member of their family? That would be huge. What difference would it make if 20 million less people tried to kill themselves - what impact would that have on those that know them? \textbf{Massive}. As Margaret Mead said:

"Never doubt that a small group of thoughtful citizens can change the world. Indeed, it is the only thing that ever has."

So I'm sad that we live in a world that drives so many people to consider they want to leave it. The reason I created this website is that it scares me how many people injure themselves trying to commit suicide. It scares me that after these people wake up in hospital, the majority will say they weren't that intent on killing themselves. I'm happy that by not succeeding in killing myself, I have a chance to help you. And maybe you will pass on that favor and help someone else.

A life with no love, and no making a difference, is going to be a struggle. We need at least one, and ideally both. And they both cost \textbf{nothing}. So here's my plea - that rather than trying to kill yourself, you read the rest of this section of the website. Try and find something that will make a difference to you. Just one thing, and  commit to doing it. If you've tried already - try again. If you've tried again. Try harder. There's always time to look at the methods of suicide later...

\subsection{Addressing the problem}

According to the American Association for Suicidology (AAS), nearly everyone at some time in their life thinks about suicide. Almost everyone decides to live because they come to realize that the crisis is temporary. However, people in the midst of a crisis often perceive their dilemma as inescapable and feel an utter loss of control. Frequently, they:

\begin{verse}

Can't stop the pain

Can't think clearly

Can't make decisions

Can't see any way out

Can't sleep eat or work

Can't get out of the depression

Can't make the sadness go away

Can't see the possibility of change

Can't see themselves as worthwhile

Can't get someone's attention

Can't seem to get control

\end{verse}

The AAS estimates that about two thirds of people who commit suicide are depressed at the time of their death. Not a big surprise. Perhaps what is a surprise is that the treatment of depression is effective 60\% to 80\% of the time. The risk of someone suffering from an untreated major depressive disorder trying to commit suicide is around 1 in 5 (20\%). However, the suicide risk among treated patients is around 1 in a 1,000 (0.1\%).

You'd think that if treatment was that effective, a lot less people would try and kill themselves. However, according the World Health Organization, less than 25\% of individuals with depression receive adequate treatment. Unfortunately believing your condition is "incurable" is often part of the hopelessness that accompanies serious depression. But what you believe is \textbf{different} to the facts. So the clear message is here:

\begin{quote}

Depression IS treatable, and in gaining effective treatment the chances of feeling suicidal go down significantly.

\end{quote}

Of course, depression is not the only reason people commit suicide. It might be for any variety of reasons - see What's behind suicide? for more information. But whatever your reason is, there is a high probability there is some form of effective treatment for it, so please keep reading.

\subsubsection{Suicide and medical conditions}

Even if you are reading this and suffering from a serious medical condition, that might not have a cure, I'd urge you to read on. I don't pretend to know what it is like living life in permanent physical pain. I can understand anyone in severe pain wanting to consider ending their lives. But you must be reading this for a reason. You must be open to hope. Even people in physical pain or with a disability can still find reason to live. To you, I would urge you watch the Jonny Kennedy and Christopher Reeve movies (see Watch this), and read Still Me by Christopher Reeve (see Books \& DVDs). For people in pain have done incredible and inspiring things with their lives, and maybe you will find inspiration in their stories.

\subsubsection{Loss of hope}

The thing that struck me about suicidal people is the loss of hope. That anything can be done that will make a difference to the rest of their life. Many people arriving at this website might have been on anti-depressants and/or had some form of therapy. But just because they have not worked does not mean there is nothing that can work. There is a good chance that there are other things you can try. As Thomas Edison, inventor of the light bulb, said, when asked about whether he had any success yet in inventing the light bulb:
\begin{quote}
"Results? Why man, I have gotten lots of results! If I find 10,000 ways something won't work, I haven't failed. I am not discouraged, because every wrong attempt discarded is often a step forward...."
\end{quote}
Or as the inventor of pasteurisation, Louis Pasteur said:
\begin{quote}
"Let me tell you the secret that has led me to my goal: my strength lies solely in my tenacity."
\end{quote}
Or, Mary Pickford (know as one of the pioneers of Hollywood, and founder of United Artists and the Academy of Motion Picture Arts and Sciences, that awards the Oscars):
\begin{quote}
"If you have made mistakes, there is always another chance for you. You may have a fresh start any moment you choose, for this thing we call "failure" is not the falling down, but the staying down."
\end{quote}
So whatever you have tried so far, whatever has not worked, is a step in the direction of finding something that will work. Remember that.

\subsection{Psychotherapies}

There are lots of different types of psychotherapy (called here "therapy" for short). If you have tried therapy before and it has not worked, that does not mean that another type of therapy won't have more positive results for you. A change of therapists may also have positive results.

Further information about different types of therapy is available by selecting the links below, or on the left hand menu. However, this is not a definitive list or categorization of therapies, and you may well come across something not listed here, although this does cover all the most popular forms. Further information on some of these types of therapy is also available at Psychcentral.com.

\section{Suicide}

Welcome to Lost All Hope. I’m sorry you are here reading this, but glad you have come across a site that is filled with information and resources designed to \textbf{help} people thinking of committing suicide.

People who are depressed and thinking of committing suicide often feel alone, isolated, hopeless; that the world will be a better place without them, and no-one will miss them. Well, you are not alone, because there are around 20 million attempted suicides every year (see Suicide statistics). Probably around 160m every year who have serious thoughts of committing suicide. So there are many, many people feeling just like you do now.

And whilst this might not make you feel better, perhaps be reassured that out of those 160m who think about committing suicide, only 0.6\% of them actually do. So for 99.4\% of people, their feelings of desperation reduce enough for them not to try and kill themselves. However you feel right now, a bit more reading before committing suicide can’t do any harm, so I would really urge to read Help me, especially Surviving today.

Possibly like you, for a number of years, I frequently spent time trawling the internet researching ways of killing myself. Feel free to read My story. I thought I had the means all sorted out. My will and financial affairs were in order. All I needed was something to push me over the edge. One day that thing happened. I really expected my suicide attempt to work. I was \textbf{very} disappointed when it didn’t.

\subsection{Painless suicide}

Years later I put together this site. In doing so I researched suicide forums, books, blogs, newsgroups and websites. I am no longer surprised I failed. Because the more I read, the more I see how many ways there are to screw up killing yourself. The research would point to painless suicide being the problem. People that hang themselves, shoot themselves, throw themselves off the top of a building or cliff, don’t seem to engage in so much chat about how to do it.

The people who discuss suicide the most, think about it for the longest time, attempt suicide the most, yet fail suicide most often, are those looking for peaceful and painless methods to go. This site was designed to help people thinking of killing themselves. That help might consist of informing of the dangers of particular methods. And there are, many dangers, in many methods. They aren’t on this site to dishearten you, or overwhelm you with information, it is just the facts. A successful, \textbf{painless suicide} takes a lot of research and preparation. And, if you read as much information as I have, you will realize it does take effort.

The help on this site might also consist of more practical advice, like that contained in Help me and Addressing the problem. Please have a good look around. My own experience was that it took a lot of trial, error and persistence to find something that made a difference to me. But that doesn’t mean that help is not possible.

I understand that people visiting this site feel hopeless. I really do know how that feels. And in that place it is hard to believe anything will ever get better. I was always prone to cutting myself off from people. If I did speak to people, I never told them how I felt. Not how I really, \textbf{really} felt. Who could possibly want to put up with someone who felt so miserable and hopeless?

I read in my research that speaking to someone was very powerful, and noticed there are all these suicide helplines. Who really wants to speak to some helpline when you want to kill yourself? What good will that do? Nine years later I learnt how much difference speaking to someone can make. Do not underestimate the power of truly expressing how you feel. It can be massive. It can be hard, but for reasons I can’t explain, really beneficial.

So, if you can reach out to someone, and somehow express what you really, really feel – do it. If you can’t reach out to a friend, then almost every country has some sort of suicide helpline. Speak to somebody. If speaking makes no difference then you have only lost the time it takes to make a phone call. Links to the main suicide helplines are below, and further information on these and other helplines is in Links.

Befrienders Worldwide (operating in 39 countries): www.befrienders.org

National Suicide Prevention Lifeline (USA): www.suicidepreventionlifeline.org 1-800-273-8255

The Samaritans (UK): www.samaritans.org 08457 90 90 90

Lifeline (Australia): www.lifeline.org.au 13 11 14

\section{Things to consider}

If one is intent on committing suicide, there are a few practical matters that need to be considered before making that attempt. This section aims to briefly outline those in a factual way, with no comment or opinion.

1. Be sure, be really, \textbf{really} sure, that there is no other way to get over the pain you are experiencing, other than suicide. You are recommended to read the section Help me.

2. Consider the likely responses of those you leave behind i.e. how family and friends will react to the news of your choice to end your life, what impact it will have on them; what impact it may have on your work, business, clients. According to the US Suicide \& Crisis Centre, each suicide intimately affects at least six other people, so consider carefully the impact on them. Read more in Help me.

Be wary if what is driving you to consider suicide is sending a message to others (which it is in many cases, often on a sub-conscious level). Your actions might be driven from anger, hurt or revenge, rather than a true desire to end your life.

3. Select a method of suicide. There is plenty of information on this website, and recommended further reading. Consider carefully how painful the selected method is, and whether your desire to die exceeds the likely pain, and risk of a failed attempt.

4. Prepare the means. Most suicide methods require some preparation.

5. If required, find someone to assist getting the means for the selected method, or help with the method itself. This may be particularly relevant for the infirm or terminally ill, and working out who can assist without judging you is key. Take great care that any assistance will not run foul of euthanasia laws in your country, as assisting people in their own suicide is generally illegal.

6. Decide who you tell. Many people will not tell anyone for fear of being judged, or attempts to be talked out of it. It could be argued that if being talked out of it is a fear, then that is a reason for telling someone, as the chances are you have some doubts that it is the right thing to do.

7. Have an up to date will. For anyone in the UK who has not yet written a will, or needs to replace their current will, quality, reasonably priced do-it-yourself wills (that can also be reviewed by a lawyer) are available online at www.desktoplawyer.co.uk, and in the US from www.smartlegalforms.com.

8. Create a living will. This could be handy in the event your attempt is not successful and significant physical or mental damage results. A living will sets out your wishes regarding healthcare and how you would like to be treated in the event you are seriously ill and unable to make or communicate your own choices. See websites above for how to create one.

9. Financial affairs. It makes sense to have all your financial affairs in order. It is also extremely helpful to whoever will be left to deal with your financial affairs, generally the executors of your will, to have a letter detailing bank accounts, life insurance policies, pension funds and details of any other assets that may be subject of your will.

10. Funeral. Consider whether you wish to be buried, cremated, or leave your body to be donated to medical science. Put this instruction in your will. Try and make sure someone else knows.

11. Follow the three day rule. If you are ready to commit suicide, like really, \textbf{really} ready, wait three days, or better still a week, until you actually go through with it. If you are going to be dead for the rest of time, what's another few days wait? It may be that in a few days your enthusiasm to go through with it might not be the same, which will suggest that maybe suicide is \textbf{not} the only answer, but possibly something could change in your life circumstances, or how you view/feel about your life, that will change your decision. After all, something changed to get you to this place - maybe it will change again? Read Help me.

Many suicides and attempted suicides are done on impulse, but this suggests that the same people would not have tried to kill themselves either days before, or days after, had they thought about their actions for longer. Suicide is a permanent condition. It is not a decision that should be rushed.

\section{What's behind suicide?}

In researching this website, I came across some interesting studies and comments about what drives people to suicide. I thought it might be helpful to share this information so people can understand better what might be lying behind how they are feeling.

Research in the UK and in the US seems consistent in identifying the principal factors that contribute to someone committing suicide. According to the American Association of Suicidology, major depression is the psychiatric diagnosis most commonly associated with suicide. The risk of suicide in people with major depression is about 20 times that of the general population. About two thirds of people who complete suicide are depressed at the time of their deaths. That's a very high percentage.

The risk of someone suffering from an untreated major depressive disorder trying to commit suicide is around 1 in 5 (20\%). However, the suicide risk among \textbf{treated} patients is around 1 in 1,000 (0.1\%). That would point to treatment for depression substantially reducing the risk of suicide, so maybe there is hope for feeling better. See Help me.

Research studies would point to the following being major factors triggering people to attempt to kill themselves. Note that more than 90 percent of people who die by suicide have the top two risk factors:

\begin{verse}
Depression (especially if exhibiting extreme hopelessness, lack of interest in activities that were previously pleasurable, heightened anxiety and/or panic attacks) and other mental disorders

An alcohol or substance-abuse disorder (often in combination with other mental disorders)

Relationship difficulties (either with an existing partner, or due to divorce, being widowed or a relationship break-up)

Prior suicide attempt (one study indicated that anyone who has previously attempted suicide is 100 times more likely to make a successful attempt compared to the suicide rate of the general population)

Family history of mental disorder or substance abuse

Family history of suicide, or exposure to the suicidal behavior of family members, peers, or media figures

Family violence, including physical or sexual abuse (especially for young people)

Firearms in the home, the method used in more than half of US suicides

Being in prison

Unemployment

Issues with studies (a major problem for those at university/college)

Financial problems

Legal problems

Social deprivation

Social isolation
\end{verse}

However, suicide and suicidal behavior are not normal responses to the factors mentioned above; many people have these risk factors, but are not suicidal. Research also shows that the risk for suicide is associated with changes in brain chemicals called neurotransmitters, including serotonin. Decreased levels of serotonin have been found in people with depression, impulsive disorders, a history of suicide attempts, and in the brains of suicide victims.

Depression, like other mental illnesses, is probably caused by a combination of biological, environmental, and social factors, but the exact causes are not yet known. For years, scientists thought that low levels of certain neurotransmitters (such as serotonin, dopamine, or norepinephrine) in the brain caused depression.

However, scientists now believe that the interplay of factors leading to depression is much more complex. Genetic causes have been suggested from family studies that have shown that between 20 and 50 percent of children and adolescents with depression have a family history of depression, and that children of depressed parents are more than three times as likely as children with non-depressed parents to experience a depressive disorder.

In some people, there can be underlying physical reasons for severe depression. For instance, those diagnosed with a terminal illness, or those living with a long term physical disability, especially if accompanied by pain that is never likely to go away. It can be much harder to treat depression for people in this category, as the underlying causes are physical issues that cannot be cured. That isn't to say though that even people such as this can't find a motivating reason for living. See Help me.

For more information on the factors that lead to someone feeling suicidal, see Risk factors for suicide.

\subsection{Risk factors for suicide}

There are a large number of reasons that can get someone to a place where they feel suicidal. Sometimes it can simply be a specific life event that gets a person so emotionally overwhelmed they do not feel they can face their rest of their life, and, on impulse, they try and kill themselves.

Appleby and Condonis (Hearing the cry: Suicide Prevention, 1990) wrote that the majority of individuals who commit suicide do not have a diagnosable mental illness. They are people just like you and I who at a particular time are feeling isolated, desperately unhappy and alone. Suicidal thoughts and actions may be the result of life's stresses and losses that the individual feels they just can't cope with.

For many others though, the seeds are sown well before the event itself, and in some cases, go back to childhood. Just being born in certain countries, of a certain race, or being of a certain age, may increase the suicide risk. These people have an above average propensity to commit suicide, but even in these cases it is likely there is some specific trigger for the suicide attempt itself.

The factors that increase suicide risk can be broadly broken down into three categories:
\begin{verse}
Life events \& circumstances

Mental or physical disorders

Historic \& demographic
\end{verse}

\subsubsection{Life events \& circumstances}

These are the things that are most likely to push someone over the edge from considering suicide, to actually trying it. The table below is from a study in Oxford (UK), where suicide survivors were asked what factor was causing them current distress and/or contributed to them making their attempt. It is interesting to note the variations between men and women:

\begin{table}[]
\begin{tabular}{llll}
\textbf{Problem}       & \textbf{Both sexes} & \textbf{Men} & \textbf{Women} \\
Other family member    & 42.1\%              & 35.7\%       & 45.6\%         \\
Partner                & 40.8\%              & 39.6\%       & 41.5\%         \\
Alcohol                & 33.9\%              & 45.9\%       & 27.4\%         \\
Employment/Studies     & 25.7\%              & 31.1\%       & 22.8\%         \\
Financial              & 23.5\%              & 30.0\%       & 20.0\%         \\
Social isolation       & 18.6\%              & 21.4\%       & 17.1\%         \\
Housing                & 17.8\%              & 22.2\%       & 15.4\%         \\
Friends                & 10.6\%              & 7.6\%        & 12.2\%         \\
Bereavement            & 10.0\%              & 9.5\%        & 10.3\%         \\
Childhood sexual abuse & 8.2\%               & 3.8\%        & 10.6\%         \\
Physical health        & 8.1\%               & 8.1\%        & 8.1\%          \\
Drugs                  & 7.9\%               & 13.2\%       & 5.0\%         
\end{tabular}
\end{table}

In a different study, of 4,391 self harmers in Oxford, 80.6\% reported multiple life problems, the most common being the relationship with spouse or partner. Females, but not males, with high suicidal intent, had more problems than those with low intent. Patients with high intent more frequently experienced psychiatric and social isolation problems than those with low intent. Females with high intent more frequently reported bereavement or loss and eating problems.

The study in Oxford is by no means a definitive list though. The American Association of Suicidology (AAS) published a factsheet on the risk factors for suicide, and had the items below as things that can increase the short term risk of suicide:

\begin{verse}
Recently divorced or separated with feelings of victimization or rage

Current self-harm behavior

Recent suicide attempt

Excessive or increased use of substances (alcohol or drugs)

Psychological pain (acute distress in response to loss, defeat, rejection, etc.)

Recent discharge from psychiatric hospitalization

Anger, rage, seeking revenge

Aggressive behavior

Withdrawal from usual activities, supports, interests, school or work; isolation (e.g. lives alone)

Anhedonia (inability to experience pleasure from normally pleasurable life events such as eating, exercise, social interaction or sexual activities)

Anxiety, panic

Agitation

Insomnia

Persistent nightmares

Suspiciousness, paranoia (ideas of persecution)

Severe feelings of confusion or disorganization

Command hallucinations urging suicide

Intense affect states (e.g. desperation, intolerable aloneness, self-hate)

Dramatic mood changes

Hopelessness, poor problem-solving, cognitive constriction (thinking in black and white terms, not able to see gray areas or alternatives), rumination, few reasons for living, inability to imagine possibly positive future events

Perceived burdensomeness

Recent diagnosis of terminal condition

Feeling trapped, like there is no way out (other than death)

Sense of purposelessness or loss of meaning; no reasons for living

Negative or mixed attitude toward help-receiving

Negative or mixed attitude by potential caregiver to individual

Recklessness or excessive risk-taking behavior, especially if out of character or seemingly without thinking of consequences, tendency toward impulsivity
\end{verse}

It should also be noted that according to the AAS, owning a firearm, or having easy access to one, increases the short term suicide risk for those living in the US. Clearly, for people that are suicidal and not thinking clearly, having the means for an effective suicide to hand makes it that much easier.

The AAS also mention the following heighten the risk in the short term of a suicide attempt:

\begin{verse}
Any real or anticipated event causing, or threatening: shame, guilt, despair, humiliation, unacceptable loss of face or status, or legal problems (loss of freedom), financial problems, feelings of rejection/abandonment

Recent exposure to another's suicide (of friend or acquaintance, or of celebrity through media)
\end{verse}

\subsubsection{Mental or physical disorders}

The information below is again based on information from the American Association of Suicidology, and lists various mental and physical disorders that would give a person an increased likelihood of committing suicide. It should be noted that these type of risk factors are all treatable/capable of being modified.

\begin{verse}

Psychiatric disorders, especially:

Mood disorder (particularly depression)

Anxiety disorder

Schizophrenia

Substance use disorder (alcohol abuse or drug abuse/dependence)

Eating disorders

Body dysmorphic disorder (an anxiety disorder where someone is excessively concerned about and preoccupied by a perceived defect in their physical features)

Conduct disorder (a pattern of repetitive behavior where the rights of others or social norms are violated - symptoms include verbal and physical aggression, cruel behavior toward people and pets, destructive behavior, lying, truancy, vandalism, and stealing)

More than one psychiatric disorder at the same time (depression and alcohol misuse being particularly dangerous)

A psychiatric disorder and an Axis II personality disorder (especially if the Axis II disorder is antisocial or borderline personality disorders) or an Axis III medical disorder

Axis II personality disorder, especially category B (Category A: paranoid, schizoid, schizotypal; Category B: antisocial, borderline, histrionic, narcissistic; Category C: avoidant, dependent, obsessive-compulsive). See Axis II personality disorders for more information.

Axis III medical disorders (brain injuries and other medical/physical disorders which may aggravate existing diseases or present symptoms similar to other disorders, especially if it involves functional impairment and/or chronic pain)

Traumatic brain injury

Low self-esteem/high self-hate

Tolerant/accepting attitude toward suicide

Lack of acceptance by someone or their family of their sexual orientation

Smoking

Perfectionism (especially in context of depression)

\end{verse}

\subsubsection{Historic \& demographic}

These risk factors (from the AAS) all exist by virtue of things that cannot be changed, as they either occur by birth, or as a result of past events:

\begin{verse}

Demographics (people in these categories all have statistically higher suicide rates): male (almost four times as many males as females die by suicide, although females make more attempts than men), older age (suicide rates for those aged 65 or over are 40\% higher than for the population in general), separation or divorce, early widowhood

History of suicide attempts (someone who has tried to commit suicide once has a much greater likelihood of trying again)

Prior thoughts about suicide (if someone has thought about suicide once, they are likely to think about it again, or act on those thoughts)

Family history of suicide or attempted suicide

Parental history of (if any of these happened to parents, then suicide risk is increased for the children):

Violence

Substance abuse (drugs or alcohol)

Hospitalization for major psychiatric disorder

Divorce

History of trauma or abuse (physical or sexual)

History of psychiatric hospitalization

History of frequent mobility (i.e. moving towns, cities or countries frequently)

History of violent behavior

History of impulsive/reckless behavior (suicide is often an impulse act, so those prone to acting impulsively are more at risk of committing suicide)

\end{verse}

\subsubsection{Elderly \& youth}

There are certain factors that are particularly prevalent for elderly or young people who attempt suicide. According to data presented at Suicide Lodge, suicide in older people is strongly associated with the below:

\begin{verse}

Depression

Physical pain or illness

Living alone

Feelings of hopelessness and guilt

\end{verse}

In youth, which account for a very high of suicide attempts, a number of studies have looked at the factors that are particularly influential, which include:

\begin{verse}

Alcohol and drug abuse: Alcohol and drugs affect thinking and reasoning ability and can act as depressants. They decrease inhibitions, increasing the likelihood of a depressed young person making a suicide attempt. American research has shown that one in three adolescents is intoxicated at the time of an attempt.

Unemployment: There is much debate over the role of unemployment in suicide and causal links have not been established. However the rate of attempted suicide amongst the short-term unemployed has been shown to be 10 times as high as for people in work.

Physical and sexual abuse: Young people who suffer, or have suffered, abuse in the past are often at increased risk of suicide or deliberate self-harm.

Custody: Within the prison population as a whole, young prisoners represent the largest group of at-risk individuals, particularly those under 21 who make up a third of the remand population in the UK. In 1995, 20\% of prison suicides were by people aged 21 or under.


\end{verse}

A study by the University of Oxford Centre for Suicide Research of actual suicides in Oxford by under 25 year olds cited that psychiatric disorders (most commonly depressive disorders) were the most popular reason for suicide, with it being diagnosed in 70.4\% subjects. Very few individuals were receiving treatment for their disorders.

The same study found that a third of subjects had more than one factor contributing to their suicide, and that the suicides were often the end-point of long-term difficulties extending back to childhood or early adolescence. In addition to mental disorders, relationship and legal difficulties were identified as relatively common contributory factors to the suicides.

The process leading to suicide in young people is often long term, with untreated depression in the context of personality and/or relationship difficulties being a common picture at the time of death.

\section{Suicide methods}

Lost All Hope has a library of information on methods to commit suicide, including dangers of individual methods and their reliability, and statistical information on which methods are most successful.

In researching this site I was struck by how much has been written on suicide methods. Numerous bulletin boards with long discussions over different methods (often by people who had tried various methods of poisoning and failed), books and websites. I found it strange that there was so much discussion, in the face of very easy to obtain statistics on the most sure-fire ways of committing suicide.

The holy grail seems to be painless methods of suicide, and people will go to great lengths to find a method that might achieve that. The perversity being that huge numbers of these attempts end up unsuccessful, as the clean, relatively painless methods are often not done in such a way so as to make them lethal. And many of these people end up in hospital requiring treatment for their failed attempt.

The figures presented in the section Suicide statistics would indicate that for every successful suicide attempt, there are 33 unsuccessful ones. For drug overdoses, the ratio is around 40 to 1. In fact, if attempting suicide, there is a much greater chance you'll end up in hospital alive, with either short or long term heath implications, than dead.

So if you are reading this more worried about finding a pain free method than an effective method, you'd be well advised to read carefully the information on the dangers of the suicide methods mentioned on this site, then perhaps look at Help me, because you are much more likely to hurt yourself by attempting suicide than to succeed killing yourself.

For anyone committed to killing themselves, achieving the goal can be straightforward if a reliable method is chosen. The major problems are often not with the logistics, but rather the internal self analysis, and thoughts of how others will react (our behavior is affected by how we think others will react). For many people contemplating suicide, they have a real desire to end their lives, but it is the fear of what their death might feel like that keeps them from doing it.

To read more about individual methods of suicide, please select from the left hand menu. Yes, there are methods not mentioned there. In almost every circumstance because they are either not reliable or very difficult to execute. Therefore, those considering unusual methods should be warned - there is a reason why they are not written about much.

\subsection{Most lethal methods of suicide}

For information on the most lethal methods of suicide, a good starting point is the statistics on the number of successful suicides by method. There is also a much published study from 19951, where 291 lay persons and 10 forensic pathologists rated the lethality, time, and agony for 28 methods of suicide for 4,117 cases of completed suicide in Los Angeles County in the period 1988-1991.

They were asked to rate each method as follows:

\begin{verse}
Lethality: How likely is the method to cause death (where 0\% is no chance, and 100\% is absolute certainty)

Time: An opinion on the length of time the method will require to produce death

Agony: The amount of pain and discomfort you would expect from the use of the particular method (ranked on scale of 0 to 100 where 0 is no pain/discomfort and 100 is the most pain/discomfort possible)
\end{verse}

The outcome of the study is presented below, ranked by order of lethality from just the pathologists who participated in the study (the lay persons tended to drastically overestimate the lethality of methods).

Anyone reading this table to identify a suitable method of suicide is advised to read carefully the information on the dangers of the suicide methods mentioned on this site, and visit the section Help me, because statistically you are much more likely to hurt yourself by attempting suicide than to succeed killing yourself.



\end{document}  